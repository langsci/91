%%%%%%%%%%%%%%%%%%%%%%%%%%%%%%%%%%%%%%%%%%%%%%%%%%%%
%%%                                              %%%
%%%                 Metadata                     %%%
%%%          fill in as appropriate              %%%
%%%                                              %%%
%%%%%%%%%%%%%%%%%%%%%%%%%%%%%%%%%%%%%%%%%%%%%%%%%%%%

\renewcommand{\title}{Einführung in die grammatische Beschreibung des Deutschen}  %look no further, you can change those things right here.
\newcommand{\subtitle}{}
\newcommand{\BackTitle}{\title} % Change if BackTitle != Title
\newcommand{\BackBody}{\emph{Einführung in die grammatische Beschreibung des Deutschen} ist eine Einführung in die deskriptive Grammatik am Beispiel der Grammatik des gegenwärtigen Deutschen, vor allem für Studierende der Germanistik bzw. deutschen Philologie. Inhaltlich wird auf typologische Breite und theoretische Spezialitäten zugunsten einer konsistenten und theorieneutralen Beschreibung eines einzigen grammatischen Systems verzichtet. Das Buch versteht sich wegen seiner didaktischen Ausrichtung und der damit einhergehenden Unvollständigkeit nicht selbst als Grammatik des Deutschen, obwohl einige Themen (vor allem die Nominal- und Verbalmorphologie) vergleichsweise vollständig behandelt werden.


Im Buch wird kein spezieller theoretischer Rahmen angenommen, aber nach der Lektüre sollten Leser/innen in der Lage sein, sowohl einfache Forschungsartikel zur Grammatik des Deutschen als auch theorie-orientierte Einführungen lesen zu können. Trotz seiner fast 500 Seiten ist das Buch für den Unterricht in BA-Studiengängen geeignet, unter anderem durch eine Trennung von grundlegenden und fortgeschrittenen Stoffanteilen. Die meisten Kapitel schließen mit Übungsaufgaben ab, die im Anhang vollständig gelöst werden.}
%\dedication{Change dedication in localmetadata.tex}
%\typesetter{Change typesetter in localmetadata.tex}
%\proofreader{Change proofreaders in localmetadata.tex}
\renewcommand{\author}{Roland Schäfer}
\newlength{\csspine} 
\newlength{\bodspine}
\setlength{\csspine}{28.4287976mm} % Please calculate: Total Page Number (excluding cover, usually (Total Page - 3)) * 0.0572008 mm for Createspace
\setlength{\bodspine}{40mm} % Please calculate: Total Page Number (excluding cover) * BODFACTOR + BODABS for Books on Demand
\newcommand{\lsISBN}{978-3-944675-53-4}                     
\newcommand{\lsSeries}{tbls} % use lowercase acronym, e.g. sidl, eotms, tgdi
\newcommand{\lsSeriesNumber}{2} %will be assigned when the book enters the proofreading stage
\newcommand{\lsURL}{http://langsci-press.org/catalog/book/51} % contact the coordinator for the right number