%%%%%%%%%%%%%%%%%%%%%%%%%%%%%%%%%%%%%%%%%%%%%%%%%%%%
%%%                                              %%%
%%%                 Metadata                     %%%
%%%          fill in as appropriate              %%%
%%%                                              %%%
%%%%%%%%%%%%%%%%%%%%%%%%%%%%%%%%%%%%%%%%%%%%%%%%%%%%

\renewcommand{\title}{Syntax und Valenz}  %look no further, you can change those things right here.
\newcommand{\subtitle}{Zur Modellierung kohärenter~und elliptischer Strukturen mit Baumadjunktionsgrammatiken}
\newcommand{\BackTitle}{\title} % Change if BackTitle != Title
\newcommand{\BackBody}{Diese Arbeit untersucht das Verhältnis zwischen Syntaxmodell und lexikalischen Valenzeigenschaften anhand der Familie der Baumadjunktionsgrammatiken (\textsc{tag}) und anhand der Phänomenbereiche Kohärenz und Ellipse. Wie die meisten prominenten Syntaxmodelle betreibt \textsc{tag} eine Amalagamierung von Syntax und Valenz, die oft zu Realisierungsidealisierungen führt. Es wird jedoch gezeigt,
\begin{itemize}
    \item dass \textsc{tag} dabei gewisse Realisierungsidealisierungen vermeidet und Diskontinuität bei Kohärenz direkt repräsentieren kann;
    \item dass \textsc{tag} trotzdem und trotz der im Vergleich zu \textsc{gb}, \textsc{lfg} und \textsc{hpsg} wesentlich eingeschränkten Ausdrucksstärke zu einer linguistisch sinnvollen Analyse kohärenter Konstruktionen herangezogen werden kann;
    \item dass der \textsc{tag}-Ableitungsbaum für die indirekte Gapping-Modellierung eine ausreichend informative Bezugsgröße darstellt.
\end{itemize}
Für  die direkte Repräsentation von Gapping-Strukturen wird schließlich ein baumbasiertes Syntaxmodell, \textsc{stug}, vorgeschlagen, in dem Syntax und Valenz getrennt, aber verlinkt sind. }
%\dedication{Change dedication in localmetadata.tex}
%\typesetter{Change typesetter in localmetadata.tex}
%\proofreader{Change proofreaders in localmetadata.tex}
\renewcommand{\author}{Timm Lichte}
\newlength{\csspine} 
\newlength{\bodspine}
\setlength{\csspine}{22.9947216mm} % Please calculate: Total Page Number (excluding cover, usually (Total Page - 3)) * 0.0572008 mm for Createspace
\setlength{\bodspine}{40mm} % Please calculate: Total Page Number (excluding cover) * BODFACTOR + BODABS for Books on Demand
\newcommand{\lsISBN}{978-3-944675-51-0}                     
\newcommand{\lsSeries}{eotms} % use lowercase acronym, e.g. sidl, eotms, tgdi
\newcommand{\lsSeriesNumber}{1} %will be assigned when the book enters the proofreading stage
\newcommand{\lsURL}{http://langsci-press.org/catalog/book/51} % contact the coordinator for the right number