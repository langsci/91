\addchap{Preface to the 2016 edition}

Almost 35 years have elapsed since \textit{Roots of Language} first appeared. It is therefore surprising how little needs to be changed. Despite repeated attempts to refute them (and, of course, unfounded claims that this work or that \textit{has} successfully refuted them) there is no need to change the central contentions of the original book, e.g. that creole languages arise in a single generation, and are created from an original, virtually structureless pidgin by children, who have an access to universal grammar unavailable to their elders, with minimal reference to the (substrate) languages spoken by their parents. However, it would be even more amazing if after so many years those contentions did not need spelling out much more clearly, and if a number of ancillary assumptions did not require correction or replacement. There will not be time or space here to do more than summarize these materials, but they are presented in considerable detail in Chapter 8 of \citet{Bickerton2014}, to which interested readers are referred.

Probably the greatest weakness of \textit{Roots of Language} was its failure to properly specify the group of languages to which it referred. The languages that have been described as creoles do not form a natural class. Their differences have to do with the very different nature and extent of contacts between the participants involved. It is therefore a complete waste of time to look for “a theory of creolization”, if by that we mean a theory that will provide a single explanation for all languages that have been described as creoles. In contrast, plantation creoles do form a natural class, because the sociocultural circumstances that gave rise to them (see \citealt{Bickerton2006} for a full description) were unique, stereotypical, and different from those that gave rise to “fort” and “maritime” creoles (which, of course, differed equally from one another) and of course those circumstances, by determining the nature and extent of contact, in turn determine the properties of the resulting language. For convenience sake I continue to use the term “creole”, but this should be understood as “plantation creole” in all that follows.

Members of this natural class show a much greater homogeneity in their grammars than the amorphous class of “all creoles”, and the significance of such differences as remain can be easily understood once we grasp the notion of a continuum of creoles. The continuum of creoles (not to be confused with the “creole continuum”, which applies within rather than between creoles) arose inevitably because of demographic and historical differences between different creole-forming locations -- differences that caused, in a few cases, more influence from the substrate, and, in a much larger number of cases, more influence from the superstrate. However, just as with lects in the creole continuum, languages within the continuum of creoles can be ranked on an implicational scale on which creoles with least outside influence can be placed at one end and creoles with most such influence at the other. In other words, just as with the creole continuum, the continuum of creoles will contain a basilect, a mesolect and an acrolect (think Sranan, Jamaican Creole and Bajan). If one is most interested in what creoles can tell us about the faculty of language, it is obvious that, in both continuums, the basilectal class will be of greatest interest.

Perhaps the most widely challenged claim of the original book was that children rather than adults are the creators of creole languages. This claim should have been unconditionally confirmed by \citet{Roberts1998}, which showed that children were responsible for the grammatical structures found in Hawaiian Creole but not in the pidgin that immediately preceded it. Subsequent claims by Roberts (\citeyear{Roberts2000}; \citeyear{Roberts2004}; see \citealt{Bickerton2014} for detailed discussion) that the same structures have substrate origins are directly contradicted by the very sources Roberts cites, which show unequivocally that the substrate knowledge of the creole-forming children was too limited for them to be even acquainted with (let alone control) the substrate features involved in those structures. Corroborating evidence never previously published, but again detailed in \citet{Bickerton2014}, comes from comparing the lexicons of Sranan and Saramaccan. The 50\% difference between these, extending even to grammatical items and showing even higher figures among bimorphemic compounds (novelties by definition), together with the unprecedentedly macaronic nature of both lexicons, shows that the standard explanation for the virtual identity between Sranan and Saramaccan grammars -- that both languages descend from a single proto-Surinamese Creole -- is unsustainable, that only an early-stage pidgin (or several such pidgins) could have existed prior to 1690, and that therefore the similarity of the two grammars can only be explained by universalist assumptions. 

One shortcoming of the book is that it never specified the rules and/or principles of universal grammar that were instantiated in creoles. It can now be hypothesized that this grammar has little if any syntactic apparatus beyond the Minimalist Program’s “Merge” (better described, since it is unidirectional, as “Attach”). Of course to operate this apparatus, the speaker has to know what can legitimately be attached to what. Though there are broad (and presumably innate) semantic guidelines for this knowledge, the properties enabling attachment are to some degree arbitrary and must therefore be learned inductively for each language. Pidgin speakers know what these are (for one language, at least) because they know the properties of the lexical items in their native language; unfortunately, that is not the language they now have to deal with. Creole speakers don’t know what these properties are (since they don’t exist yet) but presumably, being children, retain access to the innate semantic guidelines that inform them, inter alia, of a default list of semantic distinctions that should correlate with grammatical markers of some kind. They therefore seek in the lexical store of the pidgin for any items that might plausibly be interpreted as markers of those distinctions.

\textit{Roots of Language} claimed that some grammatical subsystems like nominal determiners and tense-aspect-mode (TMA) markers were innate, without precisely specifying what this innatism might consist of. It can now be argued that the classical creole TMA system is not innate per se but is an emergent property arising from a combination of default categories and principles of economy. Assume that default categories include +/– past, +/– unreal and +/– non-single (repeated or continuous, as opposed to single, unitary events/actions). The system that best minimizes the number of morphemes while maximizing distinguishable event types is a system with three overt markers and an unmarked form which, if permitted to combine freely with one another (subject to ordering constraints) yields a possible eight combinations, and it is this optimal system that, with slight additions or modifications, almost all creoles adopt.

The book was the first work to propose that the grammar underlying creoles~-- the “language bioprogram” as it came to be called -- must also be both what enabled children to acquire language on a limited exposure to it and the form in which language originally evolved. The research of the thirty-odd years that followed its publication has uncovered no evidence to challenge that relationship. New evidence supporting its function is again found in \citet[Chapter 7]{Bickerton2014}, where those aspects of French and English that take the longest time for children to acquire are shown to be precisely those aspects that most clearly and directly conflict with bioprogram specifications.

Though it is, of course, impossible to say what the earliest true languages of humans looked like, that they looked remarkably like creoles is consistent with all we know about evolution, prehistory, and the faculty of language. That said, it must be admitted that \chapref{ch:5} of \textit{Roots of Language} is the weakest part of the book. It could hardly have been otherwise. All but a handful of linguists still labored under the Linguistic Society of Paris’s ban on discussing language evolution. Ignorance of evolutionary biology was universal among those who defied the ban. Consequently, the chapter comes across today as naïve, and is of course superseded by much subsequent work, especially \citet{Bickerton2009,Bickerton2014}. Still, it remains as the first work to suggest that creoles could constitute a window on the earliest stages of language.