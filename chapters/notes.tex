\addchap{Notes}

\textbf{CHAPTER} \textbf{1}

%\setcounter{itemize}{0}
\begin{itemize}
\item I have not found any evidence for rule-ordering in either Creole English or Guyanese Creole. It would seem that rules apply wherever their structural description is met. It may be that below level of linguistic complexity, rule-ordering is not required. The topic merits further study.

%\setcounter{itemize}{1}
\begin{itemize}
\item The appropriate response in Hawaiian would have been \textit{ka} \textit{ilio} `\textsc{the} dog'; \textit{ilio} alone is quite ungrammatical.
\end{itemize}



%\setcounter{itemize}{2}
\begin{itemize}
\item In fact, it is very difficult to answer substratomaniac argu\-ments because of the profound vagueness in which they are invariably couched. For instance, \citet{Alleyne1979} states: ``In dealing with the [substratal] input source, we have \textit{to make allowances for plausible processes of change} analogous to what in anthropology are called reinterpretations \ldots  It is the \textit{failure to make such allowances} that reduces the merit of those statements that seek to refute the derivation by `substratomaniacs' of Atlantic creole verbal systems from \textit{generalized West African verbal systems}, because the two do not match up exactly point by point'' (emphasis added). Since nowhere are we told kind of allowances to make or what is or is not plausible, this simply amounts to a plea to swallow anything that fits the substratomaniac case--even such an absurdity as the existence of ``generalized West African verbal systems'' (if you want to flavor the condescension implicit in that concept, substitute ``generalized European verbal systems''), or the greater absurdity that real-world speakers could derive anything from such a chimera. In the case under discussion, if we took the semantic range of the Japanese form, half the syntax of the English form, and the HPE indifference to tense, we \textsc{might} wind up with something approximating HCE \textit{stei V}--but would anyone seriously propose that you can construct a language in this way? Moreover, if anyone did, the burden would be on that person to show why that particular mix of features from those particular languages, rather than dozens of other possible mixes from the dozen or so lan\-guages in contact, happened to get chosen. Until substratomaniacs are prepared to deal with problems of this nature, there is really nothing to argue against.

%\setcounter{itemize}{3}
\begin{itemize}
\item I am aware, of course, of the research that shows that English does make realized-unrealized distinctions, although in a much more oblique and clumsy fashion: e.g., the contrast between \textit{I believed that John was guilty, but he wasn't} and \textit{*I realized that John was guilty,  but he wasn't}. But (a) this distinction is made in \textit{that}-complement sentences rather than \textit{for-to} complement sentences; and (b) it is not surface-marked in the form of complementizer differences, but rather has to be inferred from the semantics of individual verbs. Again, it is true that \textit{-ing} complementation is in general ``more £active'' than \textit{for-to} complementation, but many cases go the opposite way, e.g., \textit{Bill managed to see Mary} (entails \textit{Bill saw Mary}) versus \textit{Bill dreaded seeing Mary} (does not entail \textit{Bill saw Mary}). For more relevant exam\-ples, see \chapref{ch:2}, examples \REF{ex:2.31}--\REF{ex:2.34}.

%\setcounter{itemize}{4}
\begin{itemize}
\item \citet{Alleyne1979} uses this fact to argue that there never were antecedent pidgins--if there had been, he claims, they should have left traces in contemporary creoles, but he denies the existence of such traces. This argument will be dealt with further in \chapref{ch:2}. Meanwhile, the reader may well wonder how much pidgin structure one could legitimately expect to be left in creoles, given the relation\-ship between the rules of HPE and HCE illustrated in \REF{ex:86}--\REF{ex:111} above.

\textsubscript{{\textbackslash} }\textsuperscript{NOTES} 305

\textbf{CHAPTER} \textbf{2}

%\setcounter{itemize}{0}
\begin{itemize}
\item %\setcounter{itemize}{0}
\begin{itemize}
\item Very few writers on creoles seem to have much background or experience in variation study, and on all the numerous occasions on which writers have used historical citations to make claims about earlier stages of creoles, I cannot recall a single one where the possi\-bility of codeswitching was even mentioned. It may well be that the average fieldhand was monolectal, but the slaves whose speech was most likely to be cited by Europeans were precisely the domestics and artisans who had the most access to superstrate models and who would therefore be the likeliest to be able and willing to adapt their speech in a superstrate direction when interacting with superstrate speakers. Historical citations should therefore be handled with great care, especially when they suggest earlier stages of a creole which would show a heavier superstrate influence than is found in the con\-temporary basilect of that creole.
\end{itemize}
\end{itemize}
%\setcounter{itemize}{1}
\begin{itemize}
\item It is at least highly questionable whether even an absolute majority of speakers of a single substrate language can influence the formation of a creole. Just after the tum of the century, when creoli'' zation must have been actively in progress, the Japanese constituted 50 percent of the population of Hawaii, yet there is virtually no trace of Japanese influence on HCE. It would be interesting to hear the substratomaniac explanation for this fact, but dealing with counter\-evidence has never been the strong point of that particular approach.
\item ``It is clear that R(eunion) C(reole) is, to quite a large degree,
\end{itemize}

. a different aninlal from M(auritian) C(reole), Ro(drigues) C(reole), and S(eychelles\} C(reole) . . . . There can be no doubt that RC shares many features in common with MC, RoC and SC . . . . The usual explanation

. . . is that RC is a 'decreollzed' version of proto-l(ndian) 0(cean) C(reole) . . . . Another, and perhaps more plausible explanation, is that RC is, on the contrary , a modified version of a variety of \textit{French} (original emphasis) . . . . The modification of this \textit{lete} \textit{ki} French may be seen in terms of convergence . . .'' Come is led to conclude that Bourbonnais (the conventional term for proto-IOC\} did not originate on the Ile de Bourbon (the old name for Reunion), but he is unable

%\originalpage{306}

to say where it did originate, or to commit himself as to whether there was or was not a true proto-IOC. In fact, only au analysis along the lines of Bickerton (197 5) can hope to make sense of RC history; but so far, no such analysis has been attempted.

%\setcounter{itemize}{1}
\begin{itemize}
\item Note that \textit{fak} \textit{ter} 'postman' also lacks au article, although the
\end{itemize}

defmite article is required in English. But in fact, the NP here is as nonspecific as let. 'THE postman', 'THE doctor', 'THE cashier', etc., are really role titles. Postmen often change routes and schedules, and there is no indication in the sentence that one particular postman might have brought the letter, that either the speaker or the listener could have answered the question ``WHICH postman?'' or that the identity of the postman had the slightest relevance to the topic.

%\setcounter{itemize}{1}
\begin{itemize}
\item The anterior-nonauterior distinction is not an easy one for the naive speaker (i.e., anyone who does not speak a creole) to under\-stand, as I have found in trying to teach it to several classes of graduate students. The reader who wishes to understand this is strongly recom\-mended to read the account in Bickerton (1975:Chapter 2).
\item Jansen et al. have a different (and much more complex) explanation involving logical form, propositional islands, truth values, etc. Although they cite \citet{Roberts1975} in another context, they appear to be unaware of the JC examples in that paper, cited above as /27 / and
\end{itemize}

/28/, as.well as of the other parallels cited here.

%\setcounter{itemize}{1}
\begin{itemize}
\item There is the possibility that an African source may also be involved. Yoruba, for instance, has both \textit{fi} and \textit{fan }(final nasals in
\end{itemize}

Yoruba orthography mean that the preceding vowel is nasalized, and do not indicate the presence of a nasal consonant). Both verbs have a number of functions, but perhaps the most relevant for creoles are those found in sentences like \textit{6} \textit{fi} \textit{ow6 naa} \textit{fun} \textit{mi,} lit. 'He take money the give me', or 'He gave me the money'. The similarity to creole

instrumentals is obvious, but if Yoruba \textit{fi} is the source for JC \textit{ft,} the

shift in meaning is baffling. \textit{Fun} is puzzling in a slightly different way. \citet{Rowlands1969} notes that ``Bilingual Yorubas tend to use \textit{fen} rather indiscriminately to translate 'for','' making a joint source for GC \textit{fu,} SR \textit{foe }(phonetically /fu/) sound ver;r plausible. Also many creoles use

%\originalpage{307}

verbs meaning 'give' to introduce dative and/or benefactive cases (e.g., HC \textit{bay,} ST \textit{da,} etc.). But if SR \textit{foe} is derived from Yoruba \textit{[Un,} why did SR select \textit{gi} (from Eng. \textit{give}\textit{)} to mark oblique cases and use \textit{foe} as a complementizer? Moreover, HCE uses \textit{fo} as a complementizer without the benefit of any Yoruba model, and French and Portuguese creoles turn Fr. \textit{pour} 'for' and Pg. \textit{para} 'for' into complementizers even though no one, to my knowledge, has suggested any verb with the form \textit{pu} or \textit{pa} in Yoruba or any West African language that could have served as a model. The question is by no means closed, however ; it merely underlines the fact that we need to know a lot more both about different West African grammars and about what African lan\-guages were spoken in which creole areas.

%\setcounter{itemize}{1}
\begin{itemize}
\item Both \citet{Christie1976} for LAC and \citet{Corne1981} for SC propose a tripartite division of verbs into Action, State, and Process. As far as I can tell (neither treatment is particularly rigorous), this proposal arises from a confusion of syntactic rules with semantic interpretation. For instance, it is not syntactic rules that (normally)
\end{itemize}

bar co-occurrence between stative verbs and nonpunctual markers, as is shown in the discussion of the sentence I \textit{bina} \textit{waan} \textit{ju} \textit{no} in \citet[38]{Bickerton1975}, which shows that pragmatic factors can also be involved.

%\setcounter{itemize}{1}
\begin{itemize}
\item A problem not faced by those who call for the examination of non··European creoles is that it is far from clear that there are any.
\end{itemize}

.The only languages without a European superstrate which might qualify under the conditions specified in Chapter 1, above, are Ki-Nubi and Juba Arabic. Although the data that have emerged on these lan\-guages so far are scanty and unclear (and for this reason I have refrained from citing them in the present volume), most of what is available suggests that they follow the creole pattern described here. But even these languages do not have a third condition which may be necessary to qualify for true creolehood: their populations were not, in general, displaced from their native homelands. It is a historical fact that it was only Europeans who uprooted people from their cultures and carried them across thousands of miles of ocean in order to exploit them;

%\originalpage{308}

therefore, it is only in European colonies that one would expect to fmd the massive disruption of normal language continuity which would permit the emergence of innate faculties.

%\setcounter{itemize}{1}
\begin{itemize}
\item However, anyone wishing to use Quow as a historical source should be warned that the above remarks apply only to his rendering of basilectal speakers. Like many whites, he did not feel threatened by illiterate blacks, and could therefore treat them objectively; but he did feel threatened by literate blacks, and in consequence, his ren\-derings of \textit{their} speech are spoiled by facetiousness and condescension.
\item There have been some nonserious nonchallenges, of course. \citet{Christie1976} produced an analysis of LAC which showed it to be not far short of identity with GC but insisted on preserving traditional terms, obvious though it was that these did not fit (getting the distri\-bution of anterior correct and then calling it past is, to me at least, a quite incomprehensible maneuver). \citet{Seuren1980} endorsed the analysis of Voorhoeve (1957 ), shown in \citet{Bickerton1975} to be intern\-ally incoherent, and neatly avoided having to consider the latter analy\-sis by calling it ``sociolinguistic'' \textit{[} \textit{sic!} \textit{]} \textit{.} But no one has systematically attempted to criticize my analyses of GC, SR, HC, and HCE, for the obvious reasons.
\item It is perhaps worth observing that no account of Papiamentu that I know of translates \textit{I} \textit{had} \textit{worked} , so that the PP TMA system may not, in fact, differ as much from the classic system as those ac\-counts might suggest. In general, not only are most analyses of TMA systems incorrect, nine out of ten of them are simply incomplete, lacking the critical information which would make it possible to deter\-mine how they work. Yet, since these defective analyses buttress Euro\-centric prejudices, they are hardly ever questioned, let alone criticized.
\item When I wrote this paragraph, I was quite unaware that Baker had produced an extremely interesting account of the historical de\-velopment of MC, based in part on an analysis of all currently known historical citations (Baker 197 6), which provides a striking piece of independent support for this analysis. While \textit{fini} is recorded as a pre\-verbal marker in 1780, \textit{ti }is not recorded until 1818; but the \textit{ti} \textit{va}
\end{itemize}

{\textbackslash}

%\originalpage{309}

combination is recorded in 1828, while the \textit{ti} \textit{fin} combination is not recorded until 1867! Granted that these dates are probably al!late\-nonstandard speech phenomena tend to have a long and lively life before they tickle the bourgeoisie, cf. \textit{olelo} \textit{pa'i'ai} (see Chapter 1) which blushed unseen in Hawaii for nearly a century-there is no need to doubt that their order and spacing are substantially correct. Baker seems not to realize, however, that the 17 80 source derives, on both intemal and external evidence, from a pidgin and not a creole speaker.

%\setcounter{itemize}{1}
\begin{itemize}
\item \citet{Corne1981} observes that ``with state 'Verbals' \textit{fin} does not occur, since a state has by definition already been attained.'' Thus, the failure of \textit{fin} to take over anterior marking in statives is a principled one, and not some inexplicable accident.
\item Here Corne falls victim to the First Law of Creole Studies, since he himself stated five pages earlier (1977:103) that \textit{ti} is omitted from subordinate clauses. But I suspect that he was mostly right on this occasion and that h.e had not made allowances for the nonhomogeneiry of SC. l would be prepared to bet that /110/ came from a higher-class, more decreolized consultant.
\item lf you believe in raising. If you don't, substitute ``whatever rule marks the second NP as object of the first V.{\textquotedbl}
\item As mentioned earlier in this chapter, it seems likely that in reality GC does not have VP as a constituent at the basilectal level. The contrary is assumed here merely in order to simplify the com\-parison between the English and GC \textit{processes,} and is not meant to imply any substantive claim about GC structure.
\item It is interesting to note that while ft-clauses in complement position can refer to one-time actions (as in /210/), and in consequence the higher verb can take punctual marking, preposed ft-clauses can refer only to habitual actions, and in consequence the higher verb must take nonpunctual marking. At the moment I have no idea why this is so.
\item Washabaugh's analysis of \textit{fi} differs radically from that made in the present chapter, although there is no reason to suppose that the facts of PIC differ significantly from those of GC. However, since I
\end{itemize}

%\originalpage{310}

have dealt with that analysis in \citet{Bickerton1980}, I will not repeat my criticisms of it here.

%\setcounter{itemize}{1}
\begin{itemize}
\item It would seem highly likely, indeed, that the inadequacies
\end{itemize}

of existing creole descriptions, often referred to in this volume, have served to diminish, rather than exaggerate, the degree of creole simi\-larity. To give just one very recent instance, it was long held that the verb-focusing rule discussed earlier in this chapter was not found in the grammars of any of the Indian Ocean creoles. Substratomaniacs could point to the nature of the substratum-Eastern Bantu, Malagasy, and Indian languages- as an explanation of this. Now Corne (p.c.) reports the finding of verb-focusing structures with a copied verb identical to those discussed in this chapter. Substratomaniacs will now doubtless seize on the claim by \citet{Baker1976} that in 1735, 60 percent ·of the nonwhite population of Mauritius was from West Africa. However, this finding is strongly challenged by \citet{Chaudenson1979} on .the basis of historical documents which he claims Baker did not examine; according to Chaudenson, the percentage of West Africans never rose much above 33.

In fact, the outcome of the disagreement is rather irrelevant to

the real issue. Baker's ``60 percent'' contained 66 percent of speakers from Guinea, and Guinean languages differ markedly in structure from the Kwa languages which are usually claimed as the source of creole structures. On Baker's own figures, the Kwa speakers in Mauritius in 1735 must have amounted to about 130! Within a few years, the population of Mauritius topped the 10,000 mark, swelled by recruits from India and Madagascar (Baker admits that hardly any Kwa speakers arrived after 1735). The question that substratomaniacs have to answer is: how did 130 people manage to impose their grammar (assuming they had a common one, which is a big assumption) upon a population in which they were outnumbered 100 to 1?

%\setcounter{itemize}{1}
\begin{itemize}
\item I am only too well aware that Piaget draws conclusions from
\end{itemize}

his studies quite contrary to those drawn here. That he does so, how\-ever, has always seemed to me baffling in light of the fact that the developmental stages he posits bear a nativistic explanation much more

'

%\originalpage{311}

easily than they do an experiential one. But there is not space here to attempt a reinterpretation of Piagetian findings, desirable though such an activity might seem. We. will see in the next chapter, however, that some linguistic findings of Piaget's disciples can very easily (and very fruitfully ) be reinterpreted in a nativistic manner (see especially the discussion of Bronckart and Sinclair [1973] ) ,

\textbf{CHAPTER} \textbf{S}

%\setcounter{itemize}{0}
\begin{itemize}
\item Even today, I know of no study of child language acquisition in any language which follows the simple and obvious procedure of noting the very first emergence of a given form or structure in a child's speech, then following the development of that feature until Brown's ``criterion'' is reached-meanwhile noting what that form or structure alternated with in those contexts where it was inappropriate, as well as those where it was appropriate, with the aim of figuring out why variation occurred and what the form or structure might mean to the child. Normally, second-language acquisition trots along obediently in the footsteps of first-language acquisition, but here roles are reversed, as my student, Tom Huebner, is about to complete a dissertation which applies the above approach to the acquisition of English by an immi\-grant Hmong speaker (see also Huebner 1979). The field is wide open for similar first-language studies, which should help to revolutionize our understanding of acquisition.
\item In fact, rather than such a conflict, the present theory entails a division of labor. The innate component is necessary in order to get the child into a position where he can learn any human language, for as \citet{Fodor1975} argues (see below), it is impossible to learn a language unless you already know a language. Some other kind of component is necessary to get the child from the innate creole-like grammar to the idiosyncratic grammars of Italian, Yoruba, Akawaio, Walbiri, or what\-ever language that particular child is going to have to learn as part of his socialization. Because I have not discussed this second component in the present volume, the reader should not conclude that I deny its
\end{itemize}

%\originalpage{312}

importance. My failure to say anything about it is, as I said, strategic; until we know where the innate component stops, we cannot know where any other devices start.

%\setcounter{itemize}{2}
\begin{itemize}
\item Or at least it is implausible to suppose that he could utilize them if he did not have some overall conceptual framework in which past tense (punctual, in our treatment) was associated with unique events and present tense (nonpunctual, in our treatment) was associ\-ated with generic events. How such an arbitrary framework could be derived from experience is totally opaque to me. But it might be derivable from species-specific or even genus-specific neural wiring, along the lines suggested in Chapter 4.
\item Students of the acquisition of Turkish please note: it would be most revealing to analyze 43 hours of a single child's speech (one hour at three-week intervals from 2:0 to 4:6) in order to determine exactly how he moves from a state-process to a direct-indirect analysis, along the lines indicated in Note 1above.
\item One of these exceptions is \citet{Miller1978}. In a brilliant flash of insight, Miller suggests that ``perhaps the difference between \textit{go} and \textit{went} is used to mark something else, like momentary happen\-ings as opposed to persisting states{\textquotedbl}; and, in discussing forms like \textit{wented} , adds that ``if they did not understand \textit{went} as incorporating a concept of pastness, then adding pastness with \textit{{}-ed} would not seem redundant.'' However, a stiff dose of Reichenbach and formal logic enables him to climb back into the sheepfold of the conventional wisdom. It should be noted, however, that one of his presuppositions\-that forms like \textit{wented} are quite uncommon in child speech-fails to take into account forms like \textit{did} \textit{he} \textit{went?,} \textit{he} \textit{didn't} \textit{went,} etc., which are semantically identical and much more common. These forms are discussed in Hurford (197 5), \citet{Kuczaj1976}, \citet{Fay1978}, Maratsos and \citet{Kuczaj1978}, and \citet{ErreichEtAl1980}; but unfortunately, it seems not to have occurred to any of these writers to look at the sentences with ``double pasts'' and the sentences with ``single pasts'' in their appropriate contexts and see whether, semantically or pragmatically, there are any differences betwn them. This is the first thing rhat
\end{itemize}

%\originalpage{313}

an investigator should do, as a matter of simple routine, whenever he is confronted by variable data of this kind.

%\setcounter{itemize}{2}
\begin{itemize}
\item The question is the more interesting in that the form auxiliary
\end{itemize}

+ past participle- the first to be acquired by French and Italian learners

{}-is among the last to be acquired by English learners. \citet{Maratsos1979} observes of the latter that ``its late acquisition, coming after children hear it used around them for years, probably stems from its subtle meaning,'' and indeed it is surely the case that the meaning of the ``composite past'' in French or Italian (a punctual meaning) is easier for the child to grasp than the meaning of the English perfect (a com\-pletive meaning). But this only opens up a host of other issues. For instance, if the meaning of English perfect is ``relevance to present state,'' and if, as Antinucci and Miller suggest, the child assigns his early past marking on the basis of ``relevance to present state,'' why should the meaning of perfect be so ``subtle'' in the child's view, and why should it not be the first, rather than the last, verb form to be acquired? Further, is it a matter of mere coincidence that perfect should be the last form to be acquired by both children learning English and speakers of an English creole in the course of decreolization (see

.Bickerton [197 5:126ff.] for details on the latter process)? If, as suggested later in this chapter, decreolization and the later stages of acquisition are processes which show a principled relationship, then there is no coincidence, but rather a joint reflection of one of the difficulties involved in getting from the bioprogram to English.

%\setcounter{itemize}{2}
\begin{itemize}
\item For instance, ``double pasts'' of the kind discussed in Note 5 above a:re assumed in orthodox generative accounts (e.g., Hurford 197 5) to stem from a process which copies the past-tense marker in Aux onto the verb-stem, as in the familiar ``Aux-Hopping'' rules, but then fails to delete the original occurrence of past tense under the Aux node (but see Matatsos and Kuczaj [1978] for criticism of this proposal). The fact that ``double pasts'' occur so frequently while ``double-WHs'' don't occur at all casts strong doubt on the assumption that children's mistakes stem from incomplete applications of standard transformational processes.
\end{itemize}

%\originalpage{3}

8, In other words, creolization and decreolization correspond to the two (overlapping) halves of the acquisition process proposed at the beginning of this chapter. The first half, dominated by the bio\-program, corresponds to creolization, but the second half, dominated by other components, in which the child bridges the gap between bioprogram and target language, corresponds to decreolization. The only significant difference would seem to be that creolization and decreolization cannot overlap, while the evolution of the bioprogram and the pressure from the target language can, do, and indeed must overlap. However, since this difference stems directly from purely pragmatic differences between the circumstances of the ``normal'' child and the circumstances of the creole-creating child, it can in no way invalidate the correspondence.

%\setcounter{itemize}{8}
\begin{itemize}
\item See also the theoretical discussion of this process in \citet{Bickerton1980}.
\item Why children don't do what they don't is often even more
\end{itemize}

mysterious (for the conventional wisdom) than why they do what they do, so that questions such as the one at the beginning of this paragraph are studiously avoided. However, there is no need to avoid such questions with the present model; why they don't do what they don't is in fact loaded with clues as to why they do do what they do.

1L It should not need to be emphasized that, fast, there is no evidence for ``hyperstrategic'' devices as such, beyond the problems whose solution might seem to call for them, and second, that if they did exist, they would constitute an innate component no less surely (although with far less justification) than does the bioprogram proposed here.

\textbf{CHAPTER} \textbf{4}

%\setcounter{itemize}{0}
\begin{itemize}
\item The exchange, which took place at the New York Academy of Sciences Conference on Language Origins in 1975, should be quoted at length; it demonstrates the orthogonal approaches and seemingly
\end{itemize}

invincible mutual incomprehensibility that have bedeviled glottogenetic

{\textbackslash}

%\originalpage{315}

studies better than could countless pages of exegesis:

\textit{Hamad:} Let me just ask a question which everyone else who has been faithfully attending these sessions is surely burning to ask. If some rules you have described constitute universal constraints on all languages, yet they are not learned, nor are they somehow logically necessary \textit{a} \textit{priori,} how did language get that way?

\textit{Chomsky:} Well, it seems to me that would be like asking the question how does the heart get that way? I mean, we don't learn to have a heart, we don't learn to have arms rather than wings. What is interesting to me is that the question should be asked. It seems to be a natural question; everyone asks it. And I think we should ask why people ask it.

The question ``Why do you ask that question?'' is of course a stalling ploy familiar to psychoanalysts; indeed, it was programmed into the ``robot psychiatrist'' with which some ingenious psychologists were able to simulate, with surprising plausibility, a therapeutic session, The present writer believes, as firmly as Chomsky, that we get language like

we get a heart and arms, yet I entirely fail to see why Harnad's question

was an illegitimate one or why it does not deserve, or rather demand, an answer. How we first got arms or a heart are questions so phylo\-genetically remote and so unrelated to the mental life of our species that Chomsky is right to dismiss them as not worth asking (except, presumably, for those whose professional specialism they are). But the evolution of language is so recent that we may reasonably suppose that its present nature is still conditioned by those origins, and its crucial role in distinguishing between us and other species (while any number of other species have arms and hearts) ls such that it must strongly influence, even if it does not wholly determine, all that we think and do. Thus, to put the determination of its origins on a par

with the determination of the origins of physical organs seems to me a piece of evasive perversity,

%\originalpage{316}

%\setcounter{itemize}{1}
\begin{itemize}
\item \citet{Hewes1975} provides a fairly exhaustive account of these theories.
\item In the Hockett and Ascher ``Flintstone,'' the key development
\end{itemize}

is a hominid who, in encountering food and danger at the same time, gives half the call for food and half the call for danger. Not one shred of even the most oblique evidence from ethological or other studies, or even the authors' own ratiocinations, is adduced in support of this inherently unlikely development, beyond their admission that they can't think of any other way language could have begun.

%\setcounter{itemize}{3}
\begin{itemize}
\item However, I have some (admittedly anecdotal) evidence that
\end{itemize}

dogs use cognitive mapping in recognition. Our dog, Rufus, will rush from the opposite end of the apartment to greet my wife when she comes home, but on meeting her on campus he ignores or even recoils from her until she is just a couple of feet from him, whereupon he performs his usual acts of greeting. It is not easy to account for such behavior unless (as is the case with us) part of the way he recognizes people has to do with a network of particular associations. He recog\-nizes her where he expects her to be, and fails to recognize her else\-where, in the same way (and why not for the same reason? ) that we fail to recognize, on the beach or in a restaurant, the clerk or cashier we may have met dozens of times in a work setting.

%\setcounter{itemize}{4}
\begin{itemize}
\item Nothing Blake ever wrote should be taken lightly. In the
\end{itemize}

broad brush-strokes with which we have to draw our cognitive maps,

is worse, we locked into stereo-

typic \textit{(} \textit{k} \textit{ike,} \textit{freak} , \textit{faggot} are some pernicious examples) which lead us to deny one another's individuality. A creature that could compute from percepts rather than concepts would out\-shine us as the sun outshines the moon (more on this in \textit{Language and}

\textit{S}\textit{pecies} \textit{).}

%\setcounter{itemize}{4}
\begin{itemize}
\item Some scholars remain unimpressed by the evidence that apes have concepts. For instance, Seidenberg and \citet{Petitto1979} seem tp
\end{itemize}

need reassurance that before and after Washoe signed \textit{water-bird} he did not also sign \textit{banana-bird} , \textit{water-berry,} \textit{banana-berry-in} other words, they at least envisage the possibility that signing apes proceed like

%\originalpage{317}

demented computers, throwing off random strings of signs (they have, after all, been reinforced for signing) from which biased experimenters simply pick out the rare one which happens, by pure chance, to be contextually appropriate. Leaving aside the unmerited slur which this casts on the morals and/or wide-awakeness of many dedicated re\-searchers, the approach adds a Cartesian twist to the old behaviorist\-nativist controversy: scholars who. are behaviorists with regard to animals and nativists with regard to people. It is more parsimonious as well as more fruitful to suppose that when animals similar to our\-selves evince behavior like ours, similar mechanisms underlie both sets of phenomena.

%\setcounter{itemize}{4}
\begin{itemize}
\item That the nature of linguistic facts can be determined by the order in which they necessarily occur and/or originally occurred has already been suggested in the contrast between the development of tense that takes place in learners of English and that which takes place in learners of Italian. Those who continue to believe (see Note 1, this chapter) that there is nothing to be learned from learning how language developed should read and compare these two cases and then ask themselves whether their attitude is not one of simple obscurantism.
\item This is not, of course, to say that older structures do not undergo changes, adaptations, and linkages. The neural dysfunction known as Gilles de la Tourette's syndrome is one that affects the limbic area, yet its victims shout lexical obscenities as well as more animal-like cries, ln general, lexical utterances are under cortical con trol, but in the case of those which express strong emotion, like non\-verbal vocal utterances, linkage between the speech areas of the neo- cortex and the limbic area must have been forged at some stage sub\-sequent to the farmer's development.
\item In fact, discussion of semantics would be clearer if \textit{semantic} \textit{prime} were reserved exclusively for category distinctions of potentially universal application (like the SNSD, the PNPD, etc.) and if what are sometimes referred to as ``semantic primes'' were referred to as \textit{primi\-} \textit{tive concepts.} However, note that primitive concepts are not necessarily constructed out of semantic primes.
\end{itemize}

%\originalpage{318}

%\setcounter{itemize}{4}
\begin{itemize}
\item The reading ``The answer is \textit{ther}\textit{e}\textit{/{\textquotedbl}} is of course not intended.
\item It is an open question whether any language could make the past-present-future distinction before the culture that used it produced any kind of time-measuring device. The fact that time-enslaved linguists may have analyzed preliterate languages as having such a distinction is, of course, no proof of anything-they have consistently done the same for creoles and they have been consistently wrong in so doing. In fact, there already exist more careful studies of such languages (e.g., Arnott 1970, Welmers 1973) which explicitly recognize the absence of the characteristic Western temporal framework. Analysis of TMA systems is too subtle to be left to logicians.
\end{itemize}
%\setcounter{itemize}{11}
\begin{itemize}
\item In Bickerton 1974.
\item Another common source is (phonologically salient) auxiliary verb forms in the superstrate. However, since there could not have been auxiliaries before there were auxiliaries, the situations of creole and primordial languages will differ in at least this respect.
\item Order in terms of distance from the verb is of course intended,
\end{itemize} 

and not the left-to-right ordering of surface constituents.

%\setcounter{itemize}{11}
\begin{itemize}
\item It was observed in Chapter 2 that the similarities between creole languages were in many cases closer and more consistent in the semantic component than they were in the syntactic component. This result would issue very naturally if the semantics of language depended on relatively old neural structnres while syntax depended
\end{itemize}

·partly on relatively new neural structures but also partly on extraneural factors intrinsic to the task of building a linear vocal language. These latter factors might in a number of cases permit more than one possible solution to a given structural problem, whereas with semantic struc\-tures, single solutions would be imposed in almost all cases.

\textbf{CHAPTER} \textbf{S}

\textbf{1.} Indeed, one objection to the hypothetical history oflanguage given in the preceding chapter might be that many essential prerequi\-sites of language, such as the development of the neural and physio·

{\textbackslash}

%\originalpage{319}

logical mechanisms required for vocalization , the lateralization of the brain, and the growth of auditory processing mechanisms. or ``tem\-plates'' which , as suggested in some fascinating work by Marler and associates (Marler 1977, 1980; Marler and Peters 1979, etc.), show striking parallels to those of avian species, have simply been ignored. However, these omissions in no way reflect my estimate of the impor\-tance of such developments. The reasons for them are threefold. First, reasons of space (and the overall purpose of this volume) pre\-vented me from describing everything that went into the makeup of language; second, these other developments have been excellently treated elsewhere; and third, I wanted to deal precisely with those aspects of language development which have been most systematically ignored or misunderstood. Certainly, such omissions were not for the purpose of strengthening my case since all the omitted developments are much more obviously the product of the genetic code than the developments discussed in this volume.

%\setcounter{itemize}{1}
\begin{itemize}
\item I certainly do not wish to suggest by this that no sooner had language reached the creole level than agriculture began. There may well have been an interval of tens of thousands of years between these two even ts, years during which cognitive maps became only gradually more complex; or the interval may have been quite short. There is no way, at present, that we can choose between these alternatives-or even prove that language in its present form did not exist two million years ago, although the latter possibility seems intrinsically unlikely.
\item I write ``seem to be'' because only empirical investigation will reveal whether such languages are indeed as far from the bio\-program as our intuitions would suggest. One test will be the time taken by children to acquire the main grammatical structures of given
\end{itemize}

languages. rt was often claimed (at a time when acquisition had hardly

<<<<<<< HEAD
been studied!) that all languages were equally easy for children to learn. This belief was, of course, simply deduced from the ``all-languages-are\-developmentally-equal'' dogma. Work by Slobin and his associates already suggests this may be quite far from the truth.





  


  
 
=======
been studied!) that all languages were equally easy for children to learn. This belief was, of course, simply deduced from the ``all-languages-are\-developmentally-equal'' dogma. Work by Slobin and his associates already suggests this may be quite far from the truth.
>>>>>>> f5fb59f7df6d1a4e606191ed080ff930faeb57af
