\chapter{Creole} \label{ch:2}

Although similarities among creoles have been known to exist at least since the pioneering work of Schuchardt and others in the latter half of the 19th century, it was not until the middle of the present century that articles by Taylor (\citeyear{Taylor1960}; \citeyear{Taylor1963}; etc.), \citet{Thompson1961}, \citet{Whinnom1956, Whinnom1965}, and others began to spell out these similarities in any detail. Curiously, their pioneering efforts were not systematically developed; in general, creolists continued to describe individual creoles, or (much more rarely) groups of creoles with a common superstratum (e.g., \citealt{Goodman1964}, \citealt{Hancock1970}, \citealt{Alleyne1980}), or else simply used already existing data in long-drawn-out and essentially fruitless debates on issues such as monogenesis versus poly\-genesis, or substrate versus superstrate influence (\citealt{Bickerton1976} provides a brief summary of these).

While the profession badly needs a volume that would systematically compare all the well-known creoles, such a task lies beyond the scope of the present volume. Instead, I shall look at general creole patterns in the five areas covered in the last chapter, plus some other areas, to give a rough general picture which should enable us to deter\-mine how far they, and HCE\il{Hawaiian Creole English}, resemble one another; and I shall then
%\originalpage{4}
explore in greater depth two areas -- verb-phrase complementation and the syntax and semantics of TMA systems -- which have already been treated by various writers more extensively than other areas. We should then be in a position to answer the questions posed at the end of the previous chapter. 

Before embarking on this task, however, it is necessary to say a few words about some of the peculiar problems it involves. One set of problems arises from the limitations of many existing descriptions of creoles.\is{creole!description of} No creole language has yet been provided the kind of com\-prehensive and detailed reference grammar that is taken for granted in most areal fields. With too few exceptions, creole grammars tend to stop where the syntax gets interesting, e.g., ``complex sentences'' are often dismissed with a page or two of unanalyzed examples. For many creoles, only outline sketches are available. Moreover, some descriptions may be based on incorrect data or contain incorrect analyses. As for the two creoles that I know best -- \ili{Guyanese Creole} and Hawaiian Creole English -- I must regretfully state that I find all previous descriptions deficient\enlargethispage{1\baselineskip} or misleading in a number of respects.\is{creole!description of}

It might be argued here that it is premature to begin any general or theoretical work, especially one of a novel or controversial nature, until these lacunae have been filled and these errors amended. For instance, \citet[2]{Corne1977} states: ``Questions about the `genesis' of the creole languages, their genetic relations with each other and with their source language(s), the processes of creolisation (and pidginisation) cannot be approached seriously unless we know something about the object being talked about, and that we shall not know (in sufficient detail) until a lot more of the unglamourous drudgery of careful descriptive work\is{creole!description of} has been completed.'' This statement shows a profound misunderstanding of the ways in which science is developed and knowledge increases. Empirical knowledge is no guarantee of certitude, and its absence no barrier to insight; I would oppose, to Corne, the following statement by \citet[3]{Dingwall1979}: ``Relying on logical argument alone, Leucippus was able to develop the atomic theory, while Aristotle\ia{Aristotle}, able to rely on the results of numerous dis%
%\originalpage{45}
sections, failed to discover the correct function of the brain, imagining it to be the cooling system of the body."

The view that theorists are mere grandstanding prima donnas, while the real work of the trade is done by the modest empirical plodder, is a widespread misconception in creole studies that merely underlines the immaturity of the field.\is{theory, role of} In the real world, unglamorous drudges never arrive at that moment of revelation which is always, like the rainbow, just beyond the next bend. For them, it's always ``a little too early to judge{\textquotedbl}; the data are ``not yet all in.'' They bequeath to their successors no more than mountains of fact, which may or may not contain the nuggets that would genuinely enrich us; more often, I suspect, the latter, since the facts one can gather about any language are infinite in number, and by no means all of equal value. What is needed is not dogged fact-gathering (with or without moral sermons) but the capacity to distinguish between the trivial and the nontrivial. The task of the theorist is to tell the field worker where to look and what to look for, and if the latter chooses to reject such aid, he has about as much brain as the man who throws away his metal detector and proceeds to dig by hand the three-acre field where he thinks treasure lies buried.\\\\

Another problem in creole studies is the question of how to interpret differences among creoles, where they genuinely exist. Are we to assume that any and every difference must be given equal weight? Such an assumption would be naive, as I shall try to show.\is{creole!description of}

\newpage Creoles are the nearest thing one can find to ab ovo creations of language, but they are not and cannot be purely ab ovo creations. At the very least, pidgins provide some input to them, and this, even if deficient, even if sometimes rejected, as we saw, is still input. Since pidgins show clear substratum differences\is{substratum influence}, and since the composition of substrata differs from place to place, that input must also be a variable, which must somehow be factored out if we are to determine the extent to which creoles are genuinely creative.

%\originalpage{46}

But there is another variable in pidgins that may have more far-reaching consequences than differences in the pidgin substrata: that is, the extent of superstrate influence\is{superstrate influence} on the pidgin. This in turn will depend on ratios of superstrate to nonsuperstrate speakers; if the former are numerous, there will be more superstrate features avail\-able to the first creole generation. We have already noted the case of Réunion\il{Réunion Creole}. But there are some areas in which \isi{population ratios} during the pidginization stage are unknown; hence, even if we exclude known cases like Réunion, this variable cannot be entirely eliminated. More\-over, factors other than demographic may influence it. Population ratios in \isi{Hawaii} differed little from those in the Caribbean\is{Caribbean}, but the (relative) freedom of an indentured as opposed to a slave society must have had some effects on the quantity and quality of linguistic inter\-action, and may well explain why we find more English features in both HPE\il{Hawaiian Pidgin English} and HCE\il{Hawaiian Creole English} than we do in a creole like Sranan, or even in the basilectal varieties of Guyanese\il{Guyanese Creole} or \ili{Jamaican Creole}.

Other problems arise from the operation of linguistic change\is{linguistic change} processes, be those processes internal or contact produced.

Internal changes affect languages regardless of their ancestry, and one would imagine that creoles, by the very recency of their emergence from rudimentary pre-pidgins, would be more, rather than less, subject to such changes than more developed languages. The oldest creoles have a time depth approaching five hundred years, which is certainly adequate for a number of significant changes to have taken place; but in most cases the absence of written records from earlier periods, and the unreliability of such records where they do exist -- not necessarily the fault of the witnesses, since these were Europeans, and code-switching presumably existed in the 17th century as it does today -- makes it difficult indeed to estimate the extent and nature of such changes.\footnote{Very few writers on creoles seem to have much background or experience in variation study, and on all the numerous occasions on which writers have used historical citations to make claims about earlier stages of creoles, I cannot recall a single one where the possi\-bility of codeswitching was even mentioned. It may well be that the average fieldhand was monolectal, but the slaves whose speech was most likely to be cited by Europeans were precisely the domestics and artisans who had the most access to superstrate models and who would therefore be the likeliest to be able and willing to adapt their speech in a superstrate direction when interacting with superstrate speakers. Historical citations should therefore be handled with great care, especially when they suggest earlier stages of a creole which would show a heavier superstrate influence than is found in the con\-temporary basilect of that creole.}

However, in addition to internal change there is the contact-stimulated type of change known as \isi{decreolization}. This can affect any creole which has remained in contact with its superstrate, as most have. Decreolization is well documented for some English creoles
%\originalpage{47}
(\citealt{DeCamp1971}, \citealt{Bickerton1973a,Bickerton1975}), but has been largely ignored in studies of other creoles. \citet{Valdman1973} suggests that it is equally widespread among French varieties, and its presence in \ili{Cabo Verdiense} and some other \ili{Portuguese creoles} is quite apparent. The result of \isi{decreolization} is to create a continuum of intermediate varieties be\-tween creole and superstrate. If this process is sufficiently long and intense, the continuum may be progressively eroded at its creole end. The result may be a synchronic state in which the most conservative variety recoverable is already considerably different from (and con\-siderably closer to the superstrate than) the original creole; this is obvious in some cases, e.g., \isi{Trinidad}, but may be less so elsewhere.

Again, in some cases where truly conservative varieties are recoverable, researchers may have failed to unearth them (the obser\-vations of \citet{Bailey1971} on the texts in \citet{LePageEtAl1960} are very relevant here). To compare a partially decreolized creole with a nondecreolized one can only produce an appearance of difference which might not have existed had it been possible to compare the two languages in their pristine condition. Yet, given present uncertainties as to which creoles have decreolized, and how much, this trap is one into which the most careful scholar might inadvertently fall.

A field so fraught with possible sources of error might seem to provide the comparativist with an inexhaustible source of alibis. Faced with any apparent difference, he could say: ``Well, this must be due to one or another of these interfering factors, so let's just forget it!'' Any such procedure would turn the inquiry into a farce, and yet, in light of the foregoing paragraphs, it would be equally irresponsible to take every difference at its face value and accord equal weight to each. If, as indicated in \chapref{ch:1}, there is some unique, creative force at work in the formation of creoles, we must try to distinguish this from other forces that might interact with it and serve to mask it. But unless we can show precisely \textit{which} factor is involved, \textit{why} it should have taken effect, and \textit{how} it could have worked to provide the observed results, our efforts will be valueless.

%\originalpage{48}

A final problem concerns the weighing of evidence and the criteria for making judgments. This is of particular importance when we come\enlargethispage{1\baselineskip} to deal with apparent cases of substratum influence.

Claims of substratum influence\is{substratum influence|(} still persist strongly in creole studies and are made in such recent works as \citet{JansenEtAl1978}, \citet{Alleyne1979, Alleyne1980}, etc. However, substrato\-maniacs, if I may give them their convenient and traditional name, seem to be satisfied with selecting particular structures in one or more creole languages and showing that superficially similar structures can be found in one or more West African languages\is{African languages} (at least one careful study, \citealt{Huttar1975}, has shown that such structures are not always as similar as they might appear at first glance). This may be just as well; if they pursued their inquiries any further, they would find that not only would they have to confront some rather serious diffi\-culties, but that even if they overcame these, they would, perforce, wind up in a position which is only a step away from that which is proposed here.

Let us suppose that a very common structure in \ili{Caribbean creoles} is also attested for \ili{Yoruba} and perhaps one or two other rela\-tively minor languages (this case is not hypothetical; we shall meet with it in the very next section). To most substratomaniacs, the mere existence of such similarities constitutes self-evident proof of the connection. They seldom even consider the problem of transmission\is{transmission problem}. How does a rule get from \ili{Yoruba} into a creole?

Theoretically, there are several possibilities. One at least -- some kind of monogenetic ancestor which would have taken structure from \ili{Yoruba} and other languages and passed it on to a wide range of descen\-dents -- has been proposed many times, but no body of evidence (save for just those creole similarities it purports to explain!) has ever been presented for such a language, and until one is, we can safely ignore it. Consequently, we must assume that our rule, and perhaps others, passed from \ili{Yoruba} into the antecedent pidgins of a number of creoles, and thence into those creoles. For this to have happened, a substantial number of \ili{Yoruba} speakers must have been present during the pidgin
%\originalpage{49}
phase in each area, or at least no later than the earliest phase of creoli\-zation. If not, if a substantial number did not arrive in a given area until after the creole had been formed, then previous speakers would hardly abandon the rules they themselves had arrived at and replace them with new rules, unless the number of \ili{Yoruba} was so great as to constitute an absolute majority -- and that, to the best of present knowledge, was never true at any time for any Caribbean\is{Caribbean} territory.\footnote{It is at least highly questionable whether even an absolute majority of speakers of a single substrate language can influence the formation of a creole. Just after the turn of the century, when creolization must have been actively in progress, the Japanese constituted 50 percent of the population of Hawaii, yet there is virtually no trace of Japanese influence on HCE. It would be interesting to hear the substratomaniac explanation for this fact, but dealing with counter-evidence has never been the strong point of that particular approach.}%\\\\

Now, while it would be difficult, if not impossible, to prove that there were \textit{no} \ili{Yoruba}s in any given area at the time of creolization, there are a number of areas where their presence must have been heavily outweighed by members of other groups. If we take Sara\-maccan, for instance, generally regarded as the most African-like of creoles, we find very few lexical survivals even from the whole Kwa group\il{Kwa languages} (of which \ili{Yoruba} is a member) but very many from Bantu lan\-guages\il{Bantu languages}, in particular \ili{Kikongo} \citep{Daeleman1972}. One would think that the first task in constructing any substratum theory would be to show that the necessary groups were in the necessary places at the necessary times. But this has simply not been done.

There are linguistic as well as historical problems to be faced by any serious substratum theory. As things stand, we are asked to believe that different African languages\is{African languages} contributed different rules and features to particular creoles. To accept that this is possible is to accept what \citet{Dillard1970}, in a slightly different context, aptly termed the ``Cafe\-teria Principle.''\is{cafeteria principle@``cafeteria principle''} Dillard was arguing against the once widespread belief that creoles were mixtures of rules and features from various regional dialects of the British Isles. But if it is absurd to suppose that a creole could mix fragments of Irish, Wessex, Norfolk, and Yorkshire dialects it is at least as absurd to suppose that a creole could mix fragments of \ili{Yoruba}, Akan, Igbo, Mandinka, and Wolof -- to mention some of the African languages\is{African languages} which substratomaniacs most frequently invoke.

Let us suppose, however, that such miracles were possible, and that \ili{Yoruba} speakers were indeed distributed in such a way that the
%\originalpage{50}
requisite input could be provided. Nobody can deny that, in every case, there were many other African languages involved in each area, and nobody who knows anything about African languages\is{African languages} can deny that, even within the Kwa\il{Kwa languages} group -- and a fortiori outside it -- there are wide differences in rules and rule systems\is{rules!types of|(}. What could be so special about a particular \ili{Yoruba} rule (such as the one for verb-focusing which we will shortly discuss) that would cause it to be accepted over all compe\-titors in a number of different and quite separate groups?

This question has been raised with respect to one feature which is by no means limited to \ili{Yoruba}: verb-serialization. In their analysis of this phenomenon, \citet{JansenEtAl1978}, while accepting the standard substratum explanation, wonder why it is that creoles, with their clear preference for features that are unmarked in the Jakobsonian sense, should have selected one which is quite rare among the world's languages, and highly unstable (subject to rapid change) even in those that have it. They are unable to provide an answer, although I shall suggest one in the latter part of this chapter.

In a general sense, we can claim that the only possible factor that could lead a group to accept a particular rule out of a set of alternatives must have to do with the emerging system of the language which that group is engaged in developing. It could only be that, at any given stage in that development, the language could only incorpor\-ate rules of a certain type, and would have to reject others. Although we still know far too little about dynamic processes in language to be able to say what such constraints on development might be like, we can be reasonably certain that they exist. Languages, even creoles, are systems, systems have structure, and things incompatible with that structure cannot be borrowed; SVO languages cannot borrow a set of post positions, to take an extreme and obvious case. If a marked struc\-ture is incorporated (and if verb-serialization is highly marked, then verb-focusing is super-highly marked), it can only be because the language, at that particular stage of its development, has to have some such rule.

That a creole language has to have certain types of rules is
%\originalpage{51}
exactly what the present study is designed to prove. If such rules happen to be present in the input in certain cases, that is in no way counter to the theory expressed here; the creole will acquire such rules, not because they are in the input, for many conflicting rules must be there also, but because such a rule is required by the struc\-ture of the emerging language. Indeed, presence in the input may not even be a necessary, let alone a sufficient, condition since the first creole generation could well have devised such a rule for itself; we saw in the first chapter that that generation can and does invent rules without benefit of experience. But even if we accept the entire sub\-stratum case, the situation is not substantively changed; the first creole generation has merely acquired the kind of rule that it was programmed to acquire, and saved itself the trouble, so to speak, of having to invent something equivalent. Thus, when taken to their logical conclusions, substratum arguments\is{substratum influence|)} only bring us back to the question this book will try to solve: why do creole speakers acquire some types of rule, but not others?\\\\

With these points clarified, we can now survey some key areas of grammar and see something of the range and extent of the similarities which any creole theory must somehow account for.\is{rules!types of|)}

\section{Movement Rules}\is{Guyanese Creole!movement rules|(}\is{movement rules!in GC|(}

HCE, as we have seen, moved focused constituents to sentence-initial position. The same procedure, with some modifications which I shall discuss in a moment, is followed by all other creoles.

It is sometimes suggested that there is nothing very remarkable about this fact, since many languages have similar processes. But it is also true that many languages have also, or instead, other methods of marking focus, such as changes in stress or tone patterns, or the use of focusing particles. The fact that creoles have adopted none of these alternative strategies cannot be without significance.

%\originalpage{52}

However, there are certain differences between HCE and the \ili{Caribbean creoles} in the ways in which this general strategy is imple\-mented. I shall illustrate the Caribbean strategy from \ili{Guyanese Creole} (GC), since this language seems to be typical in all respects.

Let us start with a simple declarative sentence such as \REF{ex:2:1}:

\ea\label{ex:2:1}
 Jan bin sii wan uman \\
\glt `John had seen a woman'
\z %remove the blank line between \z and the text to prevent indents
The subject can be focused by adjoining the equative \isi{copula} \textit{a} to the first NP:

\ea\label{ex:2:2}
 a Jan bin sii wan uman\\
\glt `It was John who had seen a woman'
\z
The object can be focused by moving the NP to sentence-initial position and again adjoining \textit{a}:\is{object-fronting}

\ea\label{ex:2:3}
 a wan uman Jan bin sii \\
\glt `It was a woman that John had seen'
\z
Other VP constituents such as oblique-case NPs and adverbials can be focused in an identical manner. However, the verb\is{verb phrase|(} can also be focused by a rather different procedure; it is again preposed and \textit{a} is adjoined to it, but a copy is obligatorily left at the extraction site:

\ea\label{ex:2:4}
 a sii Jan bin sii wan uman
\z
There is no exact equivalent to \REF{ex:2:4} in English; it is roughly equivalent to `John had \textit{seen} a woman' or `John had \textit{really} seen a woman' or `Seen a woman, that's what John had done'. In English, it is impossible to apply a movement rule to V alone; English movement rules\is{English!movement rules} apply to major categories, and major categories in English are NP and VP.

This fact must immediately raise doubts about the status of
%\originalpage{53}
VP in GC, for while NP and V can be moved freely, VP cannot:

\ea[*]{a sii wan uman Jan bin}\label{ex:2:5}\z

\ea[*]{a bin sii wan uman Jan}\label{ex:2:6}\z

\ea[*]{a sii wan uman Jan bin sii}\label{ex:2:7}\z

\ea[*]{a bin sii wan uman Jan bin sii}\label{ex:2:8}\z
Note that \REF{ex:2:6} without \textit{a} and with appropriate lexical and phonological changes would be grammatical in HCE:

\ea\label{ex:2:9}
(i) bin si wan wahini, Jan.
\z

One difference between GC and HCE\is{movement rules!in HCE|(} could then be due to the fact that the latter has the category VP while the former does not. VP has always been a problem for generative grammar; many scholars have been unwilling to accept it as a universal category since (among other things) it is hard to posit for VSO languages where it would be a discontinuous constituent in deep structure. I know of no rule in GC for which VP has to be specified in the structural description (GC does not have the equivalent of English VP deletion, for example).

However, this seems like a pretty massive difference to begin our list of similarities with. If creoles are constrained by a genetic program, how could things like this possibly come about?

If, as will be claimed in \chapref{ch:4}, the original building blocks of language are just NPs and Vs, then VP is not a primitive constituent, but V is; thus, in the earliest stages of a creole, I would predict that V, but not VP, would be a category. However, either as a result of \isi{decreolization}, involving contact with a language which already has VP as a category, or of internal change, a creole can develop VP.\\\\

Previously, we established that superstrate influence was one of the factors which would disrupt natural creole development, whether that influence came during pidginization or, much later, through
%\originalpage{54}
\isi{decreolization}. That the results of influence at these two points can be virtually identical and impossible to disentangle is testified by the eloquent bafflement of Corne's comments on the status of \ili{Réunion Creole} \citep[223--224]{Corne1977}.\footnote{``It is clear that R(éunion) C(reole) is, to quite a large degree, a different animal from M(auritian) C(reole), Ro(drigues) C(reole), and S(eychelles) C(reole) \ldots~There can be no doubt that RC shares many features in common with MC, RoC and SC \ldots~The usual explanation \ldots~is that RC is a `decreolized' version of proto-I(ndian) O(cean) C(reole) \ldots~Another, and perhaps more plausible explanation, is that RC is, on the contrary, a modified version of a variety of \textit{French} (original emphasis) \ldots~The modification of this \textit{lete} \textit{ki} French may be seen in terms of convergence~\ldots'' Corne is led to conclude that Bourbonnais (the conventional term for proto-IOC) did not originate on the Ile de Bourbon (the old name for Réunion), but he is unable to say where it did originate, or to commit himself as to whether there was or was not a true proto-IOC. In fact, only an analysis along the lines of \citet{Bickerton1975} can hope to make sense of RC history; but so far, no such analysis has been attempted.} For instance, as mesolectal varieties of GC come under English influence, they develop VP. Now, it seems plausible to suppose that HCE, which, as we have seen, was influenced by English more strongly than most other English creoles, acquired VP at birth, rather than two or three hundred years later (though I would agree that for the moment there is no obvious way to prove this). If this is so, then HCE would not be typical of the most natural creole development; but the overall theory would be unaffected, since Washabaugh's (\citeyear{Washabaugh1979}) claim that any genetically-programmed feature should appear universally in creoles, irrespective of other conflicting factors, is a blatant straw man.

However, we still have to show why GC copies the verb\is{verb-copying}. Here, the hypothetical case of the \ili{Yoruba} rule discussed in the preceding section becomes real, for \ili{Yoruba} does indeed have a rule that yields sentences very similar to, although not identical with \REF{ex:2:4}. At first sight this rule looks so weird that one thinks (I myself thought for several years) that direct borrowing must be the only possible source. However, consider for a moment what would happen if GC had a rule which said, ``Move all major categories'' (probably true of any human lan\-guage), plus a condition which specified that major categories were NP and V (which is highly probable based on the evidence), but this movement did \textsc{not} leave a copy of V at the extraction site.

Such a rule would separate verbs from their auxiliaries, and this would immediately cause severe processing problems for speakers of creoles. It is a condition on transformations generally that meaning be recoverable, but since a number of auxiliaries (e.g., GC \textit{go}) are homoph\-onous with full verbs or can modify zero copulas\is{copula}, and since many full verbs are homophonous with the nouns derived from them, sentences in which only V is fronted could wind up with meanings completely different from those they originally had. Take the following examples:

%\originalpage{55}

\ea[\hspaceThis{*}]{\label{ex:2:10}
 Jan bin go wok a haspital\\
\glt `John would have worked at the hospital'}
\z

\ea[*]{a wok \textit{Jan bin go a haspital}}\label{ex:2:11}\z
The italicized main clause in \REF{ex:2:11} constitutes a complete sentence with the meaning `John had gone to the hospital'. Since \textit{wok} can be noun or verb, and since nouns are fronted without copying, as in \REF{ex:2:2} and \REF{ex:2:3}, \REF{ex:2:11} could be, and almost certainly would be, interpreted as `It was work that John had gone to the hospital for'. Again, if V-fronting minus copying were applied to \REF{ex:2:1} above, it would yield:

\ea[*]{a sii Jan bin wan uman}\label{ex:2:12}\z
This could only be interpreted as a (slightly ungrammatical) version of `He (or I) saw that John was a woman!' Thus, if a copy is not left, meaning is irrecoverable. It would seem, therefore, that any language with movement rules that involve V only, rather than VP, \textsc{must} de\-velop a copying rule (or if, as has often been suggested in the literature, movement rules normally consist of two parts, one which Chomsky-adjoins a copy of the constituent to S and one which deletes the original constituent, it must then merely suppress the second half of the process). No borrowing from any other language would be required. Moreover, a claim that GC borrowed the rule from \ili{Yoruba} sets up an infinite regress: where did \ili{Yoruba} borrow it from? It is much more plausible to suppose that languages independently invent rules when these are demanded by the structure of the language plus func\-tional requirements.\is{Guyanese Creole!movement rules|)}

The other difference between GC and HCE rules involves the use of an equative \isi{copula}. HCE could not use such a \isi{copula} because it never developed one. Absence of an equative copula seems to be characteristic of those languages (e.g., HCE, \ili{Crioulo}, the \ili{Indian Ocean creoles} (IOC)) which show heavier superstrate influence\is{superstrate influence}, but as there is no plausible mechanism here to show \textsc{why} that influence should have this effect (positive influence is one thing, negative influence quite
%\originalpage{56}
another), we must note this as a potentially significant difference. In the absence of such a focus-marking device, some other morpheme must be recruited, and creoles seem to have no specific program for either source or position: \ili{Crioulo} \textit{ki} is drawn from a superstrate rela\-tivizer (Pg. \textit{que}) and postposed to the extracted constituent; Seychelles Creoles \textit{sa} is drawn from its own definite article, in turn derived from Fr. \textit{ça}, and optionally preposed; while HCE uses (for subject NP) the quasi-obligatory verb-predicate\is{verb phrase|)} marker fortuitously present in Filipino\is{Filipino pidgin speakers} versions of HPE\il{Hawaiian Pidgin English}, postposing it and adding number and gender. This diversity is in sharp contrast to the generality of left movement, and suggests (we will later provide abundant evidence, not just from creoles but from child language acquisition) that the genetic program which produces language in the species highly specifies some areas of language and leaves others undetermined; this is only to be expected, as a genetic blueprint which leaves no room for variation and development would freeze a species at a single developmental level.\is{movement rules!in GC|)}\is{movement rules!in HCE|)}

\section{Articles}

There seems, in contrast, to be hardly any variation at all in the way that creoles handle articles\is{articles}. Virtually all creoles have a system identical to that of HCE\is{Hawaiian Creole English!articles}: a definite article for presupposed-specific NP; an indefinite article for asserted-specific NP; and zero for nonspecific NP. GC provides the following examples:\is{Guyanese creole!articles}

\ea\label{ex:2:13}
Jan bai di buk\\
\glt `John bought the book (that you already know about)'
\z

\ea\label{ex:2:14}
 Jan bai wan buk\\
\glt `John bought a (particular) book'
\z


\ea\label{ex:2:15}
Jan bai buk\\
\glt `John bought a book or books'
\z

\ea\label{ex:2:16}
 buk dia fi tru\\
\glt `Books are really expensive!'
\z
% CREOLE 57
\ili{Papiamentu} (PP) provides the following examples\is{articles}:

\ea\label{ex:2:17}
mi tin e buki\\
\glt `I have the book'
\z

\ea\label{ex:2:18}
mi tin e bukinan \\
\glt `I have the books'
\z

\ea\label{ex:2:19}
mi tin un buki\\
\glt `I have a book'
\z

\ea\label{ex:2:20}
mi tin buki\\
\glt  'I have books'
\z

\ea\label{ex:2:21}
buki ta caru \\
\glt `Books are expensive'
\z
\ili{Seychelles Creole} (SC) provides the following examples:

\ea\label{ex:2:22}
 m\^o pe aste sa banan \\
\glt `I am buying the banana'
\z

\ea\label{ex:2:23}
m\^o pe aste ban banan\\
\glt `I am buying the bananas'
\z

\ea\label{ex:2:24}
 m\^o pe aste \^e banan\\
\glt `I am buying a banana'
\z
\citet[13]{Corne1977} follows the same Anglocentric route as \citet{Perlman1973} did for HCE when dealing with nonspecifics; he cites the following two examples, \REF{ex:2:25} and \REF{ex:2:26}, as ``zero form \ldots~in NP of the VP'' and ``Ind + plural,'' respectively:

\ea\label{ex:2:25}
\gll fakter i n amen let isi?\\
postman {\PM} {\COMP} bring letter here\\
\glt `Did the postman bring a letter (here)?'\footnote{Note that \textit{fakter} `postman' also lacks an article, although the definite article is required in English. But in fact, the NP here is as nonspecific as \textit{let}. `\textsc{the} postman', `\textsc{the} doctor', `\textsc{the} cashier', etc., are really role titles. Postmen often change routes and schedules, and there is no indication in the sentence that one particular postman might have brought the letter, that either the speaker or the listener could have answered the question ``\textsc{which} postman?'' or that the identity of the postman had the slightest relevance to the topic.}
\z

\ea\label{ex:2:26}
 \gll nu pu al pret zuti\\
we {\IRR} go borrow tool \\
\glt `We shall go and borrow some tools'
\z
%\originalpage{58}
Corne does not mention subject generics, but we can assume that these too are treated as nonspecifics.

Similar illustrations could be produced for almost any creole. This area of grammar seems to be highly specified in creoles; the dis\-tinction between specific and nonspecific\is{specific-nonspecific distinction (SNSD)} is particularly clear and consistent, and when we look at language acquisition in \chapref{ch:3}, we will find confirmatory evidence that it is probably innate.

\section{Tense-Modality-Aspect (TMA) Systems}

\is{tense-modality-aspect (TMA) systems|(}A majority of creoles, like HCE, express tense, modality, and aspect by means of three preverbal free morphemes, placed (if they co-occur) in that order. I have already discussed the typical creole system elsewhere (\citealt{Bickerton1974}, \citeyear[Chapter 2]{Bickerton1975}), so here I shall give only a brief outline, returning later in the chapter to go much more deeply into some apparent counterexamples which have been men\-tioned in the literature.

In the typical system -- which HCE\is{Hawaiian Creole English!TMA system|(} shares with GC\is{Guyanese Creole!TMA system}, \ili{Sranan|(} (SR), \ili{Saramaccan} (SA), \ili{Haitian Creole} (HC), and a number of other creoles -- ranges of meaning of the particles are identical: the tense particle ex\-presses [+Anterior]\is{anterior tense} (very roughly, past-before-past for action verbs and past for stative verbs);\footnote{The anterior-nonanterior distinction is not an easy one for the naive speaker (i.e., anyone who does not speak a creole) to under\-stand, as I have found in trying to teach it to several classes of graduate students. The reader who wishes to understand this is strongly recom\-mended to read the account in \citet[Chapter 2]{Bickerton1975}.} the modality particle expresses [+Irrealis] (which includes futures and conditionals); while the aspect particle expresses [+Nonpunctual] (progressive-durative plus habitual-iterative). The stem form in isolation expresses the unmarked term in these three oppositions, i.e., present statives and past nonstatives. In addition, there exist combined forms, some of which in some languages have been eroded (in GC\is{Guyanese Creole!TMA system} by phonological rules, in HCE by \isi{decreolization}), but of which the full set is attested for HC\il{Haitian Creole} \citep{Hall1953} and SR \citep{Voorhoeve1957}. Again, wherever combined forms are present, their meaning is the same: anterior plus irrealis, counterfactual conditions; anterior plus nonpunctual, past-before-past durative or habitual actions; irrealis\is{irrealis modality} plus nonpunctual, habitual or durative unrealized actions; anterior plus irrealis plus nonpunctual, counterfactuals which express duration or habituality.\is{nonpunctual aspect}

% CREOLE 59

Surface forms, of course, take a number of different shapes:\is{Guyanese Creole!TMA system} 
anterior,\is{anterior tense}
 GC \textit{bin}, 
 SA \textit{bi}, 
 SR \textit{hen}, 
 HC and \ili{Lesser Antillean Creole} (LAC) \textit{te}; 
irrealis,\is{irrealis modality} 
 GC \textit{sa/go}, 
 SA \textit{o},
 SR \textit{sa}, 
 HC \textit{ava}, 
 LAC \textit{ke};
nonpunctual, 
 GC \textit{a}, 
 SA \textit{ta}, 
 SR \textit{e}, 
 HC \textit{ape}, 
 LAC \textit{ka}.

HCE, with \textit{bin}, \textit{go}, and \textit{stei}, shares even two of the GC surface forms\is{Guyanese Creole!TMA system}, although the two languages are several thousand miles apart and their speakers have never been in contact. Combined forms have almost disappeared through \isi{decreolization}, but are retained by a few speakers \citep{Bickerton1974}, and/or are attested for earlier periods \citep{Reinecke1969,Tsuzaki1971}, although nowadays those who remember them are so unsure of what they once meant that one investigator \citep{Perlman1973} accused his consultants of making them up! (See discussion in \citealt[183ff]{Bickerton1980}.) Thus, we can again claim a highly programmed area, and many details of the ways in which children acquire quite different\enlargethispage{1\baselineskip} kinds of TMA systems (see \chapref{ch:3}, below) will serve to confirm this claim.\is{Hawaiian Creole English!TMA system|)}\is{tense-modality-aspect (TMA) systems|)}

\section{Realized and unrealized complements}

What work has so far been done on creole complementation has focused largely on \isi{verb serialization}, so data on this topic are extremely scarce. However, all the languages for which I have been able to find good data attest an identical structure to that of HCE, i.e., complementizers which are selected by the semantics of the embedded S.

\citet{Roberts1975} reports the following contrast from \ili{Jamaican Creole} (JC):

\ea[\hspaceThis{*}]{\label{ex:2:27}
im gaan fi bied, bot im duon bied\\
\glt `He went to wash, but he didn't wash'}
\z

\ea[*]{im gaan go bied, bot im duon bied}\label{ex:2:28}\z
Here, \textit{go} as complementizer cannot co-occur with a negative conjunct because its meaning expresses a realized action. However, \textit{fi} is fully
%\originalpage{60}
compatible with negative conjuncts since the actions it introduces are not (or, perhaps, are not necessarily) realized.

\citet{JansenEtAl1978} report an identical contrast in \ili{Sranan|)}:

\ea[\hspaceThis{*}]{\label{ex:2:29}
 a teki a nefi foe koti a brede, ma no koti en\\
\glt `He took the knife to cut the bread, but did not cut it'}
\z

\ea[*]{a teki a nefi koti a brede, ma no koti en}\label{ex:2:30}\z
Here, the contrast is between \textit{foe} and {\O} as complementizers, but the
semantic distinction is identical.\footnote{Jansen et al. have a different (and much more complex) explanation involving logical form, propositional islands, truth values, etc. Although they cite \citet{Roberts1975} in another context, they appear to be unaware of the JC examples in that paper, cited above as \REF{ex:2:27} and \REF{ex:2:28}, as well as of the other parallels cited here.}

The examples so far have all been from \ili{English} creoles although it is obvious that the distinction cannot have been derived from a lan\-guage which does not make it:

\ea\label{ex:2:31}
 {I} {managed} {to} {stop} {\rm (entails ``I stopped")}.
\z

\ea\label{ex:2:32}
 {I} {failed} {to} {stop} {\rm (entails ``I did not stop")}.
\z

\ea\label{ex:2:33}
I {went} {to} {see} {Mary} {and} {we} {talked} {about} {old} {times.}
\z

\ea\label{ex:2:34}
 {I} {went} {to} {see} {Mary} {but} {she} {wasn't} {home.}
\z
Fortunately for those who might still hypothesize some occult \ili{English} influence, the same contrast is found in \ili{Mauritian Creole} (MC). In one of the texts in \citet{Baker1972}, we find \REF{ex:2:35}:

\ea\label{ex:2:35}
\gll  li desid al met posoh ladah \\
she decide go put fish in-it\\
\glt `She decided to put a fish in (the pool)'
\z
A line or so later, \REF{ex:2:36} follows:
\ea\label{ex:2:36}
 \gll li don posoh-la en ti noh-gate\\
she give fish-the one small nickname\\
\glt `She gave the fish a little nickname'
\z
% {\textbackslash}
% 
% CREOLE\textsuperscript{ 61}
In other words, she had indeed done what she decided to do, i.e., put a fish in the pool. The \textit{al}-complement, therefore, indicates a realized action\is{realized--unrealized distinction}. However, in another text we find \REF{ex:2:37}:

\ea\label{ex:2:37}
 \gll li ti pe ale aswar pu al bril lakaz sa garsoh-la me lor sime ban dayin fin atake Ii\\
he {\TNS} {\MOD} go evening for go burn house that boy-the  but on path {\PL} witch {\COMP} attack him\\
\glt `He would have gone that evening to burn the boy's house, but on the way he was attacked by witches'
\z
Here, the subject of the sentence was prevented from carrying out his intention by the witches; accordingly, the complement is marked with \textit{pu} \textit{al}. Since Baker does not discuss this construction, we have no way of knowing if, as I suspect, \REF{ex:2:38} would be ungrammatical in that particular context:

\ea\label{ex:2:38}
 li ti pe ale aswar al bril lakaz \ldots 
\z
However, all realized complements in Baker's texts are marked with \textit{al} or {\O} , and all unrealized complements are marked with \textit{pu} or \textit{pu} \textit{al.}

These similarities, not previously pointed out in any published work, are particularly striking in that the structure looks like a highly marked one, being attested in few if any noncreole languages; and yet the identity is not merely semantic and syntactic; it extends even to the choice of lexical items -- for \textit{pu} derives from Fr. \textit{pour} `for', and Eng. \textit{for} is the source for HCE \textit{fo}, JC \textit{fi}, SR \textit{foe},\footnote{There is the possibility that an African source may also be involved. \ili{Yoruba}, for instance, has both \textit{fi} and \textit{f{\'u}n} (final nasals in Yoruba orthography mean that the preceding vowel is nasalized, and do not indicate the presence of a nasal consonant). Both verbs have a number of functions, but perhaps the most relevant for creoles are those found in sentences like \textit{{\'o} fi ow{\'o} n\'a\`a f\'un mi}, lit. `He take money the give me', or `He gave me the money'. The similarity to creole instrumentals is obvious, but if Yoruba \textit{fi} is the source for JC \textit{fi}, the shift in meaning is baffling. \textit{F\'un} is puzzling in a slightly different way. \citet{Rowlands1969} notes that ``Bilingual Yorubas tend to use \textit{f\'un} rather indiscriminately to translate `for'", making a joint source for GC \textit{fu}, SR \textit{foe} (phonetically /fu/) sound very plausible. Also many creoles use %\originalpage{307}
verbs meaning `give' to introduce dative and/or benefactive cases (e.g., HC \textit{bay}, ST \textit{da}, etc.). But if SR \textit{foe} is derived from Yoruba \textit{f\'un}, why did SR select \textit{gi} (from Eng. \textit{give}) to mark oblique cases and use \textit{foe} as a complementizer? Moreover, HCE uses \textit{fo} as a complementizer\is{Hawaiian Creole English!complementation} without the benefit of any Yoruba model, and \ili{French} and Portuguese creoles turn Fr. \textit{pour} `for' and Pg. \textit{para} `for' into complementizers even though no one, to my knowledge, has suggested any verb with the form \textit{pu} or \textit{pa} in Yoruba or any West African language that could have served as a model. The question is by no means closed, however; it merely underlines the fact that we need to know a lot more both about different West African grammars and about what African lan\-guages were spoken in which creole areas.} while MC \textit{al} `go' and MC {\O} parallel HCE \textit{go}, JC \textit{go}, and SR {\O}. While it is conceivable that JC and SR might have some kind of genetic connection (although no historical or systematic linguistic evidence has been advanced), there is no possi\-bility that either could have any connection with HCE, and it is, if that is possible, even less likely that there was ever any connection between these three and MC, which has a different superstrate \textit{and} different substratum language. It is impossible to imagine any other
%\originalpage{62}
explanation than one based on the possession, by speakers in all four areas, of some quite specific program for language-building.

\section{Relativization and subject-copying}\is{subject-copying|(}

In these areas there exist certain differences. Most creoles have relative pronouns, at least when the head-noun is also subject of the relative clause, but HCE does not. However, the time that most creoles have had to gain relative pronouns is little less than the time it took English to gain them in this position \citep{BeverEtAl1971}. If creoles were indeed born without surface relativizers, then the same processing problems that \citeauthor{BeverEtAl1971} discuss would have applied to them, and there would have been a similar pressure to borrow or adapt some feature that would serve to avoid such problems.\is{relativization|(}

However, any such speculation would be pure conjecture, if it were not for the fact that in a number of creoles there still exist conservative dialects or restricted sentence types in which relative pronouns are deletable in subject position -- or rather, more probably, were never inserted. In GC, for instance, this can happen when the head-noun of the relative clause is the object of the higher sentence and when the main verb of that sentence is an equivalent of \textit{have} or \textit{be} (the regular GC relative pronoun is \textit{we}):\is{Guyanese Creole!relativization}

\ea\label{ex:2:39}
 wan a dem a di man bin get di bam
\glt `One of them was the man who had the bomb'
\z

\ea\label{ex:2:40}
 shi get wan grandaata bina main
\glt `She had a grand-daughter who was being looked after (by her)'
\z
\citet[38]{Corne1977} gives some examples of relative clauses in SC\il{Seychelles Creole} where, also, the head-noun is object of the main clause, and where no relativizer is present on the surface:

\ea\label{ex:2:41}
\gll i ana Bom Lulu i d{\^a}se deor\\
{\PM} there-is good-guy wolf {\PM} dance outside \\
\glt `There is Old Wolf who is dancing outside'
\z



%\originalpage{63}

\ea\label{ex:2:42}
\gll zot truv sa pov drayver i {\^a}kor pe at{\^a} mem\\
they see the poor driver {\PM} still {\ASP} wait same\\
\glt `They see the poor driver who is \textit{still} waiting'
\z



Again, although in general the \ili{Portuguese creoles} of the Bight of Benin have relative pronouns, \citet[97]{Valkoff1966} reports a conserva\-tive dialect of Annobones which lacks them:

\ea\label{ex:2:43}
\gll  me mu gogo na-mina sa gavi \\
mother my like {\PL}-child be good\\
\glt `My mother likes the children who are good'
\z

Thus, although there is no proof that creoles started without relative pronouns, the possibility cannot be ruled out. Moreover, as we shall see later, a rather indirect argument based on grammatical sim\-plicity points in the same direction.

From what we have already seen of \isi{movement rules}, we would not expect to find much similarity in the area of subject-copying. This, at least initially; is some kind of focusing device, and we saw that other creoles have means for focusing which employ other features. However, there are at least two creoles, \ili{Crioulo} (CR) and SC\il{Seychelles Creole}, which have a form \textit{i} that characteristically occurs between subject and predicate, but is also the third person singular subject pronoun in both cases. Clearly, what\-ever function this form may originally have had, it has now become obligatory in certain contexts; for instance, it serves to mark present tense nonverbal predicates, i.e., it functions as a kind of \isi{copula}:

\ea\label{ex:2:44}
\ili{\langCR}\\
 \gll elis i amiigu\\
they {\PM} friend\\
\glt `They are friends'
\z

\ea\label{ex:2:45}
\ili{\langCR}\\
 \gll i amiigu\\
he friend\\
\glt `He is a friend'
\z
%\originalpage{64}

\ea\label{ex:2:46}
\ili{\langSC}\\
 \gll lerua i bet\\
king {\PM} stupid\\
\glt `The king is stupid'
\z

\ea\label{ex:2:47}
\ili{\langSC}\\
\gll i bet\\
he stupid\\
\glt `He is stupid'
\z
I shall not explore the meanings and functions of this particle since \citet{Wilson1962}, practically the only source for CR, does not provide adequate data, while at the other extreme, Corne (\citeyear{Corne1974-75} and \citeyear{Corne1977}) presents masses of data on the SC\il{Seychelles Creole} form and shows that its complexities will not yield easily to analysis. In any case, the principal point that was made with regard to HCE was not specifically to do with either relativization or subject-copying.\is{Hawaiian Creole English!subject-copying}

We saw in \chapref{ch:1} that when the subject of the higher clause is also subject of the relative clause, the subject-copy pronoun follows the relative clause rather than the subject noun, although elsewhere it directly follows the subject noun -- as \textit{i} does in \REF{ex:2:46}, for example. Now, in both CR and SC\il{Seychelles Creole}, in other words in the only two creoles in which we could look for an analogue of the HCE structure (since they are the only ones with anything like a subject copy), we find that in subject-subject relative-clause sentences, \textit{i} follows the relative clause and not the noun subject, i.e., it also obeys the A-over-A principle\is{A-over-A principle} or its equivalent (examples from \citealt[30]{Wilson1962}; \citealt[53]{Corne1977}):

\ea\label{ex:2:48}
\ili{\langCR}\\
 \gll ɔɔmi kə bay awɔnti i riba\\
man who go yesterday {\PM} return\\
\glt `The man who went yesterday has returned'
\z

\ea\label{ex:2:49}
\ili{\langSC}\\
 \gll sel abit{\^a} ki m{\^o} kapab al tr{\^o}p li i zis sa vie t{\^o}t{\^o}\\
 only farmer that I can go fool him {\PM} just that old man  \\
\glt `The only farmer that I can go and fool is just that old man'
\z

Moreover, these are not the only cases where the A-over-A principle\is{A-over-A principle} applies to creoles: the placement of HC articles in relative\-
% CREOLE 65
clause sentences is also affected. Normally, the HC\il{Haitian Creole} definite article \textit{la} immediately follows the noun: \textit{chwal-la} `the horse', \textit{kapt{\^e}n-na} `the captain'. However, if the noun is head-noun of a relative clause, the definite article follows that clause, i.e., it is adjoined to the higher rather than the lower NP:

\ea\label{ex:2:50}
\gll kapt{\^e}n ki té-arété-l-la t-ap-mété-l n{\^a}-bétiz\\
captain who \TNS-arrest-him-the \TNS-\ASP-put-him in-ridicule\\
\glt  `The captain who had arrested him was making fun of him'
\z

\ea\label{ex:2:51}
 \gll li voyé chwal yo pou-r{\^a}plasé sila ki mouri-a\\
he send horse {\PL} for-replace that which die-the\\
\glt `He was sending the horses to replace the one which had died'
\z

We have now surveyed the five areas which were discussed in \chapref{ch:1} and found that in three of them (articles, TMA markers, and realized/unrealized complements) the ``innovations'' made by the original speakers of HCE were identical with the equivalent forms and meanings in all or most creoles, while in the remaining two there were broad, general similarities along with some differences in detail. It is worth noting that the similarities are most striking where a combi\-nation of semantic and syntactic factors interact; where purely syntactic rules are involved, as with movement rules and relativization, there is a lesser degree of identity. Why this should be so will be explained in \chapref{ch:4}, fn. \ref{Ch4:Fn15}.\is{relativization|)}

I shall now, much more briefly, indicate some other areas in which strong creole resemblances can be found, before proceeding to a more thorough analysis of TMA systems and VP complementation.\is{subject-copying|)}

\section{Negation}\is{negation|(}

In creoles generally, nondefinite subjects as well as nondefinite VP constituents must be negated, as well as the verb, in negative sentences. Examples are from GC and \ili{Papia Kristang} (PK):\is{Guyanese Creole!negation}

%\originalpage{66}

\ea\label{ex:2:52}
 non dag na bait non kyat\\
\glt `No dog bit any cat'
\z

\ea\label{ex:2:53}
\gll ngka ng'koza nte mersimentu\\
 not no-thing not-have value \\
\glt `Nothing has any value'
\z
Sentences of this kind do occur occasionally in HCE, e.g.:\is{Hawaiian Creole English!negation}

\ea\label{ex:2:54}
 nowan no kaen bit diz gaiz \\
\glt `No one can beat these guys'
\z
However, while negated VP constituents are common, negated subjects with negative verb are rare, perhaps because of persecution in the schools.\is{negation|)}

\section{Existential and possessive}\is{Guyanese Creole!existential-possessive}\is{possession}

Over a wide range of creoles, the same lexical item is used to express \is{existential}existentials (``there is") and possessives (``have''), even though this is not true of any of the superstrates (it may be true of some substandard \ili{Portuguese} dialects of Brazil\is{Brazil}, but these may well be decreolized remains of an earlier creole). Examples are from GC, HC\il{Haitian Creole}, PP\il{Papiamentu}, and \ili{S\~ao Tomense} (ST), respectively:

\ea\label{ex:2:55}
 dem \textit{get} wan uman we \textit{get} gyal-pikni \\
\glt `There is a woman who has a daughter'
\z

\ea\label{ex:2:56}
\gll \emph{gê} you f{\^a}m ki gê you pitit-fi\\
have one woman who have one child-daughter\\
\glt `There is a woman who has a daughter'
\z

\ea\label{ex:2:57}
\gll \emph{tin} un muhe cu \emph{tin} un yiu-muhe\\
have a woman who have a child-woman \\
\glt `There is a woman who has a daughter'
\z

\ea\label{ex:2:58}
 \gll \emph{te} ua mwala ku \emph{te} ua mina-mosa \\
have a woman who have a child-girl\\
\glt `There is a woman who has a daughter'
\z
%\originalpage{67}
HCE follows an identical pattern:

\ea\label{ex:2:59}
 \textit{get} wan wahini \textit{shi} \textit{get} wan data\\
\glt `There is a woman who has a daughter' 
\z
We will refer to this area again in \chapref{ch:4}.

\section{Copula}\is{copula|(}

Practically all creoles show some similarities in this area. Adjec\-tives are surface verbs in creoles (see next section) and therefore require no copula. \is{locative verbs}Locatives are introduced by verbs which normally are limited to that role, i.e., do not extend to existential\is{existential} or prenominal environments. A split occurs over treatment of nominal complements: the more heavily superstrate-influenced creoles (HCE\is{Hawaiian Creole English!copula}, the \ili{Indian Ocean creoles}, some Asian \ili{Portuguese creoles}) tend to have zero copulas here also, although in some (SC\il{Seychelles Creole}, CR) the \textit{i} form appears here as a predicate marker; the less heavily superstrate-influenced creoles of the Caribbean generally have a distinct verb in these environments.

There are minor exceptions to these generalizations. GC locative \textit{de} can express existentials\is{existential}, which HCE \textit{stei}, for example, cannot:\is{Guyanese Creole!copula}

\ea[\hspaceThis{*}]{\label{ex:2:60}
 wok na de
\glt `There isn't any work'}
\z

\ea[*]{\label{ex:2:61}wok no stei}\z

\ea[\hspaceThis{*}]{\label{ex:2:62}
 nomo wok
\glt `There isn't any work'}
\z
HCE and SC\il{Seychelles Creole} have negative existentials\is{existential} -- \textit{nomo} and \textit{napa} -- a feature found in few if any other creoles which, in the HCE case at least, represents an inheritance from the antecedent pidgin. The locative\is{locative verbs} \textit{ye} in HC appears only sentence-finally, i.e., where it is stressed, which, together with its phonetic shape, suggests that it disappeared from medial position through phonological reduction processes.

%\originalpage{68}

Although some of these differences may arise from pidgin retentions or post-creolization changes, it would seem that the copula area is only moderately specified. There is a general tendency toward semantic transparency, i.e., having separate forms for each semantically distinct copula function (attribution, with adjectives; equation or class membership, with predicate nominals; locative\is{locative verbs}, with adverbials of place). However, since these semantic distinctions are unambiguously marked by predicate type, to mark them a second time with distinc\-tive copulas may seem redundant, and perhaps this accounts for copula variability within individual creoles as well as across the class.\is{copula|)}

\section{Adjectives as verbs}

In a number of creoles (e.g., JC\il{Jamaican Creole}, \citealt{Bailey1966}; GC, \citealt{Bickerton1973b}) the adjective has been analyzed as forming a subcategory of stative verbs. Evidence from GC\is{Guyanese creole!adjectives|(} is the identical behavior of verbs and adjectives under a number of rules:

\ea\label{ex:2:63}
 {i} {wok}\\
\glt `He worked'
\z

\ea\label{ex:2:64}
 i wiiri\\
\glt `He is tired'
\z

\ea\label{ex:2:65}
i a wok\\
\glt `He is working'
\z

\ea\label{ex:2:66}
ia {wiiri}\\
\glt `He is getting tired'
\z

\ea\label{ex:2:67}
au i wok!\\
\glt `How he works!'
\z

\ea\label{ex:2:68}
{au} {i} {wiiri!}\\
\glt `How tired he is!'
\z

\ea\label{ex:2:69}
 {a} {wok} {i} {wok}\\
\glt `Work, that's what he did'
\z

\ea\label{ex:2:70}
{a} {wiiri} {i} {wiiri}\\
\glt `Tired, that's what he is' 
\z
% CREOLE 69
Note that though syntactic rules apply identically, semantic interpre\-tation is often different in the two cases.\footnote{Both \citet{Christie1976} for LAC and \citet{Corne1981} for SC\il{Seychelles Creole} propose a tripartite division of verbs into Action, State, and Process. As far as I can tell (neither treatment is particularly rigorous), this proposal arises from a confusion of syntactic rules with semantic interpretation. For instance, it is not syntactic rules that (normally) bar co-occurrence between stative verbs and nonpunctual markers, as is shown in the discussion of the sentence \textit{i bina waan fu no} in \citet[38]{Bickerton1975}, which shows that pragmatic factors can also be involved.}

Originally, all writers on the \ili{Indian Ocean creoles} who dealt with this area (\citealt{Baker1972}; \citealt{Corne1973,Corne1977}; \citealt{Papen1975,Papen1978}; \citealt{Bollee1977}; etc.) treated verbs and adjectives as distinct classes and posited an underlying copula before predicate adjectives, which was subsequently deleted. However, in an insightful article, \citet{Corne1981} renounces his former analysis and sets up a class of ``Verbals,'' which would contain predicate adjectives as well as verbs and which would not require a copula in underlying structure; these ``verbals'' would then undergo at least some (Corne seems hesitant to push his argument too hard) of the processes which verbs undergo. It is worth noting that some of the evidence Corne surveys bears a striking resemblance to that found in GC, in particular the ``inchoative'' meaning of the nonpunctual marker \textit{pe} when applied to adjectives, as compared to the meaning of the GC nonpunctual marker \textit{a} when similarly acquired; compare the following with \REF{ex:2:66}:\is{Guyanese creole!adjectives|)}

\ea\label{ex:2:71}
 \gll li pe malad\\
he {\ASP} sick\\
\glt `He is getting sick'
\z

\ea\label{ex:2:72}
 li pe {\^a}-koler
\glt `He is getting angry'
\z

\ea\label{ex:2:73}
 m{\^o} pe laf{\^e}\\
\glt `I am getting hungry'
\z
This resemblance between creoles so widely separated in location and origin is quite striking. Moreover, I know of no creole where an alterna\-tive analysis of adjectives\is{adjectives} would be required. HCE\is{Hawaiian Creole English!TMA system}, not surprisingly, has a similar ``inchoative'' sense when nonpunctual and adjective are conjoined:

\ea\label{ex:2:74}
 ho, ai stei wail wid da meksikan gai\\
\glt `Wow, I was getting mad at the Mexican guy'
\z

%\originalpage{70}

\section{Questions}
No creole shows any difference in syntactic structure between questions and statements. Question-particles, where they occur, are sentence-final and optional:\is{questions!yes--no}

\ea\label{ex:2:75}
 \ili{\langGC}\\
 i bai di eg-dem\\
\glt `He bought the eggs'
\z

\ea\label{ex:2:76}
\ili{\langGC}\\
i bai di eg-dem?
\glt `Did he buy the eggs?'
\z

\ea\label{ex:2:77}
\ili{\langHC}{}{}\\
\gll yo pa-t-a-vlé m{\^e}n{\^e}-m lakay-li\\
they not-\TNS-\MOD-want take-me house-his\\
\glt `They wouldn't have wanted to take me to his house'
\z

\ea\label{ex:2:78}
\ili{\langHC}{}{}\\
yo pa-t-a-vlé m{\^e}n{\^e}-m lakay-li?
\glt `Wouldn't they have wanted to take me to his house?'
\z
\section{Question words}

\is{Guyanese Creole!questions|(}In WH-questions\is{English!questions}\is{questions!WH-}, the question-word is directly preposed to the declarative form of the sentence. The question-words themselves, if not clearly adapted from their superstrate equivalents, are always composed in the following manner: they are bimorphemic; the first morpheme is derived from a superstrate question-word -- English creole \textit{we}, \textit{wi}, or \textit{wa} from Eng. \textit{which} or \textit{what}, \ili{French} creole \textit{ki} from Fr. \textit{qui} `who' or \textit{que} `what', Portuguese creole \textit{ke} from \il{Portuguese}Pg. \textit{que} `what':

\ea\label{ex:2:79}
\ili{\langGC}{}{}\\
 \gll wisaid yu bin de?\\
{which side} you {\TNS} be-\LOC \\
\glt `Where have you been?'
\z

\newpage
\ea\label{ex:2:80}
\ili{\langHC}{}{}\\
\gll ki koté ou wè pwas{\^o}-a?\\
what side you see fish-the\\
\glt `Where did you see the fish?'
\z

\ea\label{ex:2:81}
\ili{\langST}{}{}\\
\gll ke situ e pe mi n-e-e?\\
what place he put maize in-it-\QP\\
\glt `Where did he put the maize?'
\z
%\originalpage{71}
Other forms in English creoles include Cameroons Creole \textit{wetin}, lit., `what thing', `what'; GC \textit{wa mek}, lit., `what makes', `why'.\is{Guyanese Creole!questions|)}\is{questions!WH-}

Very often a creole has doublets, a superstrate adaptation and a bimorphemic creole form. \citet[509]{Papen1978} gives the following sets for SC\il{Seychelles Creole} and RC:


\ea\label{ex:2:82} \begin{tabbing}	Lorem \= ~{\rm =}~ \= {\rm (iii)} \= Lorem Ipsum dolor \kill
					where \> {\rm =} \> {\rm (i)} \> (a)kot(e) {\rm (Fr. {\it à côté de} `at')}\\
					\> \> {\rm (ii)} \> ki ladrua {\rm (Fr. *{\it qui l'endroit}), `Which place?'}\\
					\> \> \> ki bor {\rm (Fr. *{\it qui bord}), `Which edge?'}
					\end{tabbing}\z

\ea\label{ex:2:83} \begin{tabbing}	Lorem \= ~{\rm =}~ \= {\rm (iii)} \= Lorem Ipsum dolor \kill
					how \> {\rm =} \> {\rm (i)} \> koma {\rm (Fr. {\it comment} `how')}\\
					\> \> {\rm (ii)} \> ki maner {\rm (Fr. *{\it qui manière}), `What way?'}\end{tabbing}\z

\ea\label{ex:2:84} \begin{tabbing}	Lorem \= ~{\rm =}~ \= {\rm (iii)} \= Lorem Ipsum dolor \kill
					why \> {\rm =} \> {\rm (i)} \> (l)akoz ki {\rm (Fr. {\it la cause que} `the reason that')}\\
					\> \> {\rm (ii)} \> ki fer {\rm (Fr. *{\it qui faire}), `What makes?'}\end{tabbing}\z

\ea\label{ex:2:85} \begin{tabbing}	Lorem \= ~{\rm =}~ \= {\rm (iii)} \= Lorem Ipsum dolor \kill
					when \> {\rm =} \> {\rm (i)} \> ka {\rm (Fr. {\it quand} `when')}\\
					\> \> {\rm (ii)} \> ki ler {\rm (Fr. *{\it qui l'heure}), `What hour?'}\end{tabbing}\z
Papen does not state whether, in his estimation, one set is older or more creole than the other (failure to make any serious attempt to sort variants is a grave weakness in the otherwise thorough work done recently on \ili{Indian Ocean creoles}), but we can be reasonably certain that the periphrastic forms represent the original creole; if the quasi-\ili{French} forms existed already, why should others have been invented?\is{questions!WH-}

Since HPE\il{Hawaiian Pidgin English} speakers acquired the full set of English question-words\is{English!questions} except for \textit{why} (HPE \textit{wasamata,} lit., `What's the matter?', which seems not to have been passed on to HCE)\is{Hawaiian Creole English!questions}, HCE was never required to develop a bimorphemic set. However, the similarities above are so close that we can predict that any creole which did not borrow directly from its superstrate would develop a set of forms along these lines.

\section{Passive equivalents}

Passive constructions in creoles are extremely rare, and those that exist (the \textit{wordu} and \textit{ser} passives in PP, cf. \citealt{MarkeyEtAl1980}; the \textit{gay} passive in MC\il{Mauritian Creole} and SC\il{Seychelles Creole}, cf. \citealt{Corne1977}; and the \textit{get} \isi{passive} in
%\originalpage{72}
GC) are either marginal to the language or relatively recent super\-strate borrowings, or both. The general pattern of creoles is described by \citet{MarkeyEtAl1980} as ``rampant lexical diathesis{\textquotedbl}; for any V-transitive, N V N will be interpreted as ``actor-action-patient,'' while any N V will be interpreted as ``patient-action{\textquotedbl}:\is{Guyanese Creole!passive@Guyanese Creole!``passive''}

\ea\label{ex:2:86}
 \ili{\langGC}{}{}\\
 dem a ponish abi\\
\glt `They are making us suffer'
\z

\ea\label{ex:2:87}
\ili{\langGC}{}{}\\
abi a ponish\\
\glt `We are suffering/being made to suffer'
\z

\ea\label{ex:2:88}
 \ili{\langJC}{}{}\\
  dem plaan di tri
\glt `They planted the tree'
\z

\ea\label{ex:2:89}
\ili{\langJC}{}{}\\
di tri plaan
\glt `The tree was planted'
\z

\ea\label{ex:2:90}
 \ili{\langHCE}{}{}\\
 dei wen teik foa bead
\glt `They took four boards'
\z

\ea\label{ex:2:91}
\ili{\langHCE}{}{}\\
foa boad wen teik
\glt `Four boards were taken'
\z
We shall return to structures of this type in \chapref{ch:3}.\\\\

We have now surveyed seven areas of the grammar in addition to the five already examined in greater depth. Of those seven, HCE\is{Hawaiian Creole English!passive@Hawaiian Creole English!``passive''} shows substantial identity with all other creoles in four (existential/possessive, adjective as verb, questions, and passive equivalents); substantial iden\-tity with a number of other creoles in one (copula); and little simi\-larity in two (negation, question-words). Thus, out of the twelve areas, HCE is identical with all or with a large percentage of creoles in eight, shows a fair degree of similarity in two, and differs sharply in two, one of which (negation) may well have followed the regular creole pattern before decreolization set in.

This degree of identity is quite remarkable when we consider that HCE shares none of the substratum languages of the other creoles -- %
%\originalpage{73}
except that a superstrate language for some creoles was a substrate language in HCE\il{Hawaiian Creole English}, i.e., \ili{Portuguese}! However, there is nothing in the grammar of HCE except perhaps \textit{stei} as locative that one can point to as having stemmed from Portuguese influence. The only thing HCE seems to have in common with other creoles (apart from the simi\-lar social conditions that gave birth to them) is that all have European superstrates, a fact which has been used to caution creolists against premature universalist claims \citep{Reinecke1977}.%
\footnote{A problem not faced by those who call for the examination of non-European creoles is that it is far from clear that there are any. The only languages without a European superstrate which might qualify under the conditions specified in \chapref{ch:1}, above, are \ili{Ki-Nubi} and \ili{Juba Arabic}. Although the data that have emerged on these lan\-guages so far are scanty and unclear (and for this reason I have refrained from citing them in the present volume), most of what is available suggests that they follow the creole pattern described here. But even these languages do not have a third condition which may be necessary to qualify for true creolehood: their populations were not, in general, displaced from their native homelands. It is a historical fact that it was only Europeans who uprooted people from their cultures and carried them across thousands of miles of ocean in order to exploit them; therefore, it is only in European colonies that one would expect to find the massive disruption of normal language continuity which would permit the emergence of innate faculties.} 
However, since practi\-cally all the common features of creoles are not only not shared by, but run dead counter to the structural tendencies of, Western Euro\-pean languages (the latter have well-established single copulas, well-established passives\is{passive}, use subject-verb or subject-auxiliary inversion in questions, etc.), no one could invoke this shared ancestry to explain creole similarities unless he were to propose that creoles, like naughty children, do everything the opposite of what their parents tell them to do!

However, an earlier work of mine \citep{Bickerton1974} that was limited to a discussion of TMA systems\is{tense-modality-aspect (TMA) systems|(} has been the subject of a number of criticisms, several to the effect that there were a number of exceptions to the generalizations made therein. I shall therefore deal with the issues raised in the most cogent and extensive of these criti\-cisms, namely, \citet{Muysken1981a}, before going on to show that all the genuine divergences from the classic TMA pattern can be accounted for by the impingement on that pattern of three factors. Two of these factors are quite extraneous and have already been discussed: influence of the antecedent pidgin and language change. A third will have to wait until \chapref{ch:4} for a full explanation; for the time being, let us call it ``indeterminacy in \isi{semantic space}.''

Muysken challenges my analysis of creole TMA systems by evidence drawn from six languages: \ili{Papiamentu}, \ili{Negerhollands}, \ili{Senegal Kriol}, Seychellois, \ili{Tok Pisin}, and \ili{S\~ao Tomense}. Data from two of these are quite irrelevant to the issues involved. Tok Pisin has already been ruled as having arisen under circumstances so vastly different from those of the classic creoles that the fact that it is now some people's %
%\originalpage{74}
native language -- hence, nominally a creole -- has no bearing on the present discussion. \ili{Senegal Kriol} is described by Muysken himself as an ``inter-tribal lingua franca which may have had native speakers in the past and which has some recent ones now in urban areas''; since he himself is forced to admit that this checkered history may have ``given it a very marked, deviant character,'' one wonders why he should have bothered to present data from it.

\ili{Negerhollands} is, or rather was, a genuine creole in the terms of this study, but there are at least two reasons why evidence drawn from it cannot stand up against evidence from languages which are still vital. First, the language is dead; one has to rely entirely on printed sources. This may not present a genuine obstacle to the writing of grammars of classical languages, but the case of creoles is quite different. If one takes the text\is{creole!texts|(} of a Hittite law or a Sanskrit prayer, one can be reasonably certain that it was written by a native speaker; if one takes any text of \ili{Negerhollands}, one can be certain that it was not written by a native speaker. As with virtually all other creoles, texts -- whether they take the form of fact or fiction, catechism or simulated dialogue -- were written by Europeans, with all the biases of their time and without any special linguistic skills or training. Many of the texts were written by missionaries, who are notorious for producing Europeanized varieties of pidgins and creoles wherever they go \citep{Voorhoeve1971}. This is not to say that a European, even a European missionary, could not on occasion accurately represent a creole. The problem is knowing when a creole \textit{is} being accurately represented.

For example, there is one excellent literary source for GC: \citet{Quow1877}. It is too excellent, if anything, because it gives several stylistic levels without the facts that might enable one to sort them out. There are also a number of other sources, of widely varying quality. If I had had to write a GC grammar from written sources only, there is no way that I could have learned to prefer Quow whenever he is in conflict with other evidence; that knowledge came from having four years of unrestricted access to native speakers.%
\footnote{However, anyone wishing to use Quow as a historical source should be warned that the above remarks apply only to his rendering of basilectal speakers. Like many whites, he did not feel threatened by illiterate blacks, and could therefore treat them objectively; but he did feel threatened by literate blacks, and in consequence, his ren\-derings of \textit{their} speech are spoiled by facetiousness and condescension.}
Consequently, my work would have seriously misrepresented the language.

%\originalpage{75}

The second reason against using Negerhollands as evidence for any general creole tendency is that although languages, like people, die, they do not, like some people, drop dead. On the contrary, like Charles {\sc ii}, they are an unconscionable time a-dying, and since we know that in language death languages become severely distorted, but do not know at what time the process started, there is no way in which we can be certain what any text\is{creole!texts|)} represents -- whether the full flush of the language, the early onset of decrepitude, or the final phases of decay, in which key forms are lost or, worse, replaced by forms from competing languages and dialects. For these reasons, we can dismiss the third of Muysken's six languages.

This leaves PP\il{Papiamentu}, SC\il{Seychelles Creole}, and ST. Muysken does not state where he acquired the data from \ili{S\~ao Tomense}. To the best of my knowledge, there are only two recent descriptions of the ST TMA system -- \citealt{Valkoff1966} and \citealt{Ferraz1979} -- although perhaps one should say that there are three, since Valkoff gives two different ones in the same chapter. His account is a somewhat tortuous one, and the exact status of these two descriptions is far from clear; he seems to suggest that the first is in some sense hypothetical, though whether intended as a reconstruc\-tion of some earlier phase of ST, or of proto-Bight-of-Benin, is far from clear. Be that as it may, one of the two forms he specifically stars as hypothetical turns up as real in Muysken's account, while four forms that appear in his second description do not appear in the first. Ferraz mentions Valkoff's work but does not discuss it; nor does he explain why, or even note, that his own account differs substantively from either of Valkoff's. Finally, Muysken's account bears scant resemblance to any of the previous three.

In \tabref{tab:2.1}%on the following page
, the various auxiliaries and combinations of auxiliaries claimed to occur in ST are arranged along the horizontal axis, and the four accounts (V\textsubscript{1} and V\textsubscript{2}, \citealt{Valkoff1966}; F, \citealt{Ferraz1979}; M, \citealt{Muysken1981a}) along the vertical. Pluses and minuses have the same values as in distinctive feature tables.
%\originalpage{76}

%%please move \begin{table} just above \begin{tabular
\begin{table}

\begin{tabularx}{.8\linewidth}{l*{9}{>{\centering\arraybackslash}p{.048\linewidth}}}
\lsptoprule
					& ka & tava & ta & ska & kia & te & sa & bi & za\\ \midrule
V\textsubscript{1} 	& + & -- & + & + & -- & -- & + & -- & -- \\
V\textsubscript{2} 	& + & + & -- & + & + & -- & -- & -- & + \\
F 					& + & + & -- & + & + & -- & -- & -- & -- \\
M					& + & + & -- & -- & -- & + & + & + & -- \\\midrule
\end{tabularx}
\bigskip
\begin{tabularx}{.8\linewidth}{l*{6}{>{\centering\arraybackslash}p{.09\linewidth}}}
					& tava ka & ta ka & sa ka & te di & ka bi & ka te \\ \midrule
V\textsubscript{1} 	& -- & + & -- & + & + & -- \\
V\textsubscript{2} 	& + & -- & -- & + & + & -- \\
F 					& + & -- & -- & -- & -- & -- \\
M					& + & -- & + & -- & + & + \\\lspbottomrule
\end{tabularx}
\caption{Four accounts of the ST TMA system}
\label{tab:2.1}
\end{table}

In addition, Muysken's account suggests four more forms (\textit{tava ka te}, \textit{tava ka bi}, \textit{sa ka te}, and \textit{sa ka bi}) which are not attested anywhere else, although to do him justice this impression may merely result from a faulty formalism. Even making allowances for this, he attests four forms that the other sources do not attest,\enlargethispage{1\baselineskip} and he fails to attest two that both the other writers attest, as \tabref{tab:2.1} shows.

If this picture seems confused, the reader had better not even attempt to follow the names which the various tenses, modes, and aspects are given by these three authors. I shall give a single example. The names of the \textit{ka + V} form are given, respectively, as: incompletive aorist, Valkoff\textsubscript{1}; habitual, Valkoff\textsubscript{2}; aorist, Ferraz; incompletive, Muysken. This pattern is followed throughout. If a tense, mode, or aspect is mentioned in two accounts, it has two names; if in three, three names; if in four, four names. Sometimes the differences in name merely disguise the semantic similarities of the accounts; sometimes they mark real conflicts; sometimes it is impossible to tell. In one case where there is a dear similarity between Valkoff's and Ferraz's accounts, Muysken is clearly wrong. Valkoff calls \textit{tava + V} ``completive
%\originalpage{77}
in the past'', Ferraz calls it ``pluperfect'', but it is obvious from their example sentences that whatever it is (and it looks like the anterior\is{anterior tense} of the present analysis), it is not a simple past -- which is what Muysken says it is. Here, of course, the evidence of Ferraz and Valkoff suports the position that Muysken is attacking.

Muysken's analysis is supported by two example sentences. The original form of the analysis he is attempting to undermine, in \citet[Chapter~2]{Bickerton1975}, is supported by ninety-eight example sentences. Further comment should be superfluous. Until someone is prepared to devote to the analysis of the ST system at least a fraction of the amount of careful work that went into the analysis of the GC\is{Guyanese Creole!TMA system} system, we can dismiss the fourth of Muysken's six counterexemplary systems.

The two remaining systems, those of PP\il{Papiamentu} and SC\il{Seychelles Creole}, have TMA sys\-tems too widely known to undergo much distortion, although even here Muysken's account is unsatisfactory in several respects. However, since the features of these and other systems which differ from my predictions have been mentioned by other writers (see \citealt{Hill1979}), I shall not comment further on Muysken's particular analysis, although I shall return to some broader aspects of his paper in \chapref{ch:4}.

The major and, if we were to eliminate sloppy scholarship, perhaps the only deviations from the regular creole TMA system are the following:

%\setcounter{itemize}{0}
\begin{itemize}\label{majordeviations}
\item[(A)] The presence in \ili{Crioulo} of an anterior marker, \textit{ba}, that follows rather than precedes the main verb.
\item[(B)] The presence in \ili{Papiamentu} of an irrealis marker, \textit{lo}, that may occur before rather than after the subject.
\item[(C)] The presence in certain creoles (e.g., \ili{Papiamentu}, \ili{Palenquero}, \ili{Papia Kristang}, and \ili{Negerhollands}) of tense markers that look more like +past than +anterior.
\item[(D)] The presence in \ili{Indian Ocean creoles} of two markers, \textit{ti} and \textit{(fi)n}, which compete for some kind of pastness, and two markers, \textit{pu} and \textit{a(va)}, which compete for some kind of irrealis.\is{irrealis modality}
%\originalpage{78}
\item[(E)] The merging of iteratives/habituals with either punctuals or irrealis, claimed to occur in a number of creoles (cf. \citealt{Taylor1971}), thus reducing the nonpunctual\is{nonpunctual aspect} category to no more than a progressive/durative.
\end{itemize}

The first two deviations involve only syntactic aspects of TMA systems, while the other three involve semantic aspects. It will be convenient if we take (A) and (B) together since both arise from the nature of antecedent pidgins.

\citet{Alleyne1979}, in arguing against the existence of a \isi{pidgin-creole cycle}, claims that no vestiges of pidgins can be found in creoles. This, if true, would be unsurprising -- as unsurprising as the fact that we find no trace of the caterpillar in the butterfly, and for similar reasons. In fact, the data now to be surveyed show some exceptions to the general irrecoverability of pre-creole pidgins.

As is widely known (but see \citet{Labov1971} for explicit discussion), pidgins express temporal relations by means of sentence adverbs, in clause-external position, which indicate the temporal sequence of events. HPE\il{Hawaiian Pidgin English} has two, \textit{baimbai} `then, later, afterward', and \textit{pau} `done; already, finished':

\ea\label{ex:2:92}
 {bambai} {mi} {waif} {hapai,} {bambai} {wan} {lil} {boi} {kam}
\glt `Then my wife got pregnant, and later a little boy was born'
\z

\ea\label{ex:2:93}
pau wrk fraidei, go daun kauai
\glt `After work on Friday, we went down to Kauai'
\z
Both \textit{baimbai }and \textit{pau }can occur clause-finally, although this is much more frequently the case with \textit{pau}; another speaker might well have begun \REF{ex:2:93} with \textit{fraidei, hanahana pau} \ldots\xspace `On Friday, when work was over \textellipsis'.
%\originalpage{79}

\il{English creoles|(}If creoles were, as they are popularly supposed to be, no more than ``expansions'' of pidgins, one would expect them to take markers of this kind, transmute them into obligatory markers of tense, modal\-ity, or aspect (the ``later'' sequence-marker into a future or irrealis\is{irrealis modality}, the ``earlier'' sequence-marker into a past, anterior\is{anterior tense}, or completive), and incorporate them into an Aux category. But this development is the creole exception rather than the creole rule.

When HCE developed out of HPE\il{Hawaiian Pidgin English}, neither \textit{pau} nor \textit{baimbai} underwent any change of meaning, nor were they incorporated into Aux. Two quite different forms, \textit{bin} and \textit{go}, were selected to express anterior and irrealis\is{irrealis modality}, respectively. \textit{Pau} and \textit{baimbai} are retained as optional, clause-external adverbs, but their frequency in HCE drops dramatically compared with their frequency in HPE (in the set of recordings which most accurately reflect basilectal HCE, \textit{bin} and \textit{go} occur a total of 433 times, while \textit{pau} and \textit{baimbai} occur a total of 38 times; \citealt[Tables 3.1, 3.6, 3.9]{Bickerton1977}).\is{Hawaiian Creole English!TMA system}

Good data on pidgins are even harder to come by than good data on creoles, and data of any kind on the antecedent pidgins of any creole but HCE are simply nonexistent; however, I still think that reconstruction is possible if we make the simple and reasonable assump\-tion that other pidgins resembled HPE in taking their ``later'' marker from some temporal adverb and their ``earlier'' marker from some verb with the general meaning of \textit{finish} (the meaning of \textit{pau} in \ili{Hawaiian}). We can then go on to show that while a majority of creoles decisively rejected ``later'' markers, most, if not all, accepted ``earlier'' markers with a marginal status, while some, at a later stage, allowed them, more or less grudgingly, to occupy positions within Aux. This is under\-standable since, if we are right, ``earlier'' markers have a verbal source, while ``later'' ones have a nonverbal source.

Most creole irrealis markers are derived from verbs or auxiliaries. English creole \textit{go} is an obvious case; SR \textit{sa} is usually (I am not sure if correctly) attributed to Eng. \textit{shall}, JC\il{Jamaican Creole} \textit{wi} to Eng. \textit{will}. The form that underlies most French creole\il{French creoles} irrealis\is{irrealis modality} markers is Fr.\il{French} \textit{va} `(3rd pers. sing.) go', yielding \textit{ava}, reduced to \textit{a}. LAC\il{Lesser Antillean Creole} \textit{ke} and ST\il{S\~ao Tomense} \textit{ka} remain mysterious; for the latter, \citet{Ferraz1979} suggests two possible sources -- \ili{Bini} \textit{ya}, an irrealis\is{irrealis modality} marker, and \ili{Twi} \textit{ka} `to be usual' -- while another possible source is Pg. \textit{ficar} `remain'. Only a few \ili{Portuguese creoles} show a different tendency, e.g., PK \textit{logo}, derived from the adverbial \il{Portuguese}Pg. \textit{logo} `next,
%\originalpage{80}
soon' and reducible to \textit{lo}. And \textit{lo}, as we have seen, is the Papiamentu form which deviates from the regular model.

We will return to \textit{lo} in a moment. First, let us look at the prove\-nance of \textit{ba}. Most creoles have an ``earlier'' form which is derived from a verb with the meaning `finish'; in addition to \textit{pau}, we find IOC\il{Indian Ocean creoles} \textit{(fi)n} from Fr. \textit{fini} `finished (p. part.)', English creole \textit{don} from another past participle, Eng. \textit{done}, and Portuguese creole \textit{(ka)ba} from Pg. \textit{acabar} `finish' (\textit{kaba} is found in SR also). Looking over the range of creoles, it would seem that such markers can have three distinct distributions.

First, they may remain as marginal particles, occurring option\-ally in clause-final position. This state is exemplified by SR, in which \textit{kaba} can only occur clause-finally and is never incorporated into Aux. The same is true of PP\il{Papiamentu} \textit{caba}. In basilectal GC, \textit{don} often occurs clause-finally (cf. \citealt[Examples 2.65--67]{Bickerton1975}).\is{Guyanese Creole!completive}

Second, they may be incorporated into Aux but without its being possible to combine them with other Aux constituents. This state is exemplified by mesolectal GC \textit{don} and possibly also JC\il{Jamaican Creole} and other Caribbean \textit{don} and by HC\il{Haitian Creole} \textit{fin}.

Third, they may be incorporated into Aux where they may combine with other Aux constituents quite freely. This state is exem\-plified by \ili{Krio} (KR) \textit{don}, and IOC \textit{(fi)n}, among others.

If we were working with a static-synchronic model, we would have to stop with this statement. However, since we have to work with a dynamic model in order to account for creole development, we can next propose that these three ``states'' in fact constitute stages in a diachronic development and exemplify a gradual process of incorpora\-tion which is well advanced in some creoles and has not begun in others. In order to prove that states in different languages show differ\-ent stages of the same process, it is desirable to be able to point to languages in which two stages co-exist synchronically. Basilectal GC has both postclausal and preverbal \textit{don}, the latter becoming obligatory in the mesolectal varieties; thus GC represents the transition between states one and two. Evidence for IOC is conflicting, but by at least\il{English creoles|)}
%\originalpage{81}
some accounts, stages intermediate between noncombinability and free combinability (states two and three) are to be found there.

\ili{Crioulo} \textit{ba} clearly derives from \textit{kaba}, which in accordance with its Portuguese etymon is stressed on the final syllable. Papiamentu \textit{lo} equally clearly derives from Pg. \textit{logo}. We can assume that in Portuguese pre-creole pidgins\il{Portuguese pidgins} generally \textit{logo} and \textit{kaba} were, respectively, the ``later'' and ``earlier'' forms that corresponded to HPE\il{Hawaiian Pidgin English} \textit{baimbai} and \textit{pau}. Papiamentu retained both; \ili{Crioulo} (as far as we can tell with present, inadequate data) retained only the second; and both PP\il{Papiamentu} \textit{lo} and CR \textit{ba} were incorporated \textit{semantically} into the TMA system (i.e., were allotted the expected meanings of irrealis\is{irrealis modality} and anterior\is{anterior tense}) while remaining \textit{syntactically} outside it (i.e., retained clause-external position, obligatory in the case of \textit{ba}, co-varying with subject type in the case of \textit{lo}).

It is one thing to show that deviations (A) and (B) (Page~\pageref{majordeviations}) could have arisen from pidgin features; it is quite another to explain why in these two cases, but not in others, pidgin characteristics should have been able to override creole ones. However, we can make what is at least a very plausible conjecture.

The only other member of the pidgin-creole\is{pidgin-creole cycle} family in which pidgin sequence-markers have graduated to creole auxiliaries is \ili{Tok Pisin} (TP). Here, ``later''-marker \textit{baimbai} reduced to \textit{bai}, acquired irrealis meaning, and is in the process of being incorporated into the auxiliary by native speakers \citep{SankoffEtAl1974}; ``earlier''-marker \textit{pinis} (from Eng. \textit{finish}) has followed a similar course except that it continues to occur only postverbally. I have consistently claimed that differences between TP and classic creoles would result from differences in their histories, in particular the period of several genera\-tions which TP passed as a pidgin prior to creolization. Such a period would allow time for the original sequence-markers to become firmly established in the language and to take on more tense-like and modal-like meanings through the operations of natural change. By the time TP creolized, therefore, it had already developed a complex auxiliary system, without any of the catastrophic suddenness which, as we saw
%\originalpage{82}
in \chapref{ch:1}, characterizes true creoles. The first creole generation in TP was therefore presented with a fait accompli; all it could do was accept the markers bequeathed to it and carry out some minor cosmetic operations on \textit{baimbai}, phonologically reducing it and shifting it to a more ``appropriate'' position.

The gap between TP and true creoles is not, of course, an abso\-lute one. Some of the features that distinguish TP (prolonged growth period, sustained bilingualism, etc.) could be shared to a lesser extent with some of the true creoles. In the case of \ili{Crioulo}, evidence is flatly contradictory: ``\ili{Crioulo} \ldots~has no native speakers'' \citep{Alleyne1979}; ``\ili{Crioulo} \ldots~[is] the first language of many who are born and bred in the main towns'' \citep[vii]{Wilson1962}. A plausible compromise would seem to be late creolization followed by the persistence of a small native-speaker core within a wide lingua-franca penumbra. Under such circumstances, a more gradual transition from pidgin to creole, with concomitant retention of more pidgin features, is certainly a possibility. 

\isi{Curaçao}, home of  \ili{Papiamentu}, might at first sight look very different from the Guinea of \ili{Crioulo} -- a Caribbean island where sustained bilingualism would have been impossible. However, Curaçao differs from most Caribbean islands\is{Caribbean} in that it is extremely dry and infertile. For over a century, before the Dutch seized it, and indeed to some extent thereafter, it served as a staging post in the slave trade, a place where slaves were held and seasoned while awaiting trans\-portation to other points in the Caribbean or Latin America. With a constant turnover in the population, and transients always heavily outnumbering the minority who remained, it may well be that a pidgin stage persisted here much longer than it did elsewhere in the Caribbean\is{Caribbean}, or at least long enough for more pidgin features to establish themselves. Clearly, in both cases, more historical study is needed, but the hypothe\-sis of somewhat delayed creolization would both explain the phe\-nomena involved and accord with our present knowledge of social history.

%\originalpage{83}

Let us turn now to Deviation C (Page~\pageref{majordeviations}), the past versus anterior\is{anterior tense} issue. In the first place, it must be made clear that GC, SR, and HC\il{Haitian Creole} were generally claimed to have past-tense markers, prior to my re\-analysis of their TMA systems\is{Guyanese Creole!TMA system} (see, e.g., \citet{Hall1953} for HC, \citet{Voorhoeve1957} for SR, etc.). However, that reanalysis has not been seriously challenged.\footnote{There have been some nonserious nonchallenges, of course. \citet{Christie1976} produced an analysis of LAC which showed it to be not far short of identity with GC but insisted on preserving traditional terms, obvious though it was that these did not fit (getting the distri\-bution of anterior correct and then calling it past is, to me at least, a quite incomprehensible maneuver). \citet{Seuren1980} endorsed the analysis of \citet{Voorhoeve1957}, shown in \citet{Bickerton1975} to be intern\-ally incoherent, and neatly avoided having to consider the latter analy\-sis by calling it ``sociolinguistic'' [\textit{sic!}]. But no one has systematically attempted to criticize my analyses of GC, SR, HC\il{Haitian Creole}, and HCE, for the obvious reasons.} It seems reasonable, therefore, to suppose that in a number of other creoles which I did not specifically examine, markers are still being described as ``past'' which in reality are +anterior\is{anterior tense}.

Let us examine a language, \ili{Seychelles Creole}, of which this claim is frequently made. According to \citet[102]{Corne1977}, ``the marker \textit{ti} defines the past, both simple and habitual{\textquotedbl}; according to \citet[55]{Bollee1977}, ``\textit{ti} expresses the past, definite or indefinite; it is comparable to the \isi{past tense|(} in \ili{English}''\is{English!tenses}. However, these confident and sweeping statements are immediately modified by both parties. Corne observes that ``once past time has been established in a given situation, \textit{ti} is frequently omitted'', especially in narratives ``where, after an initial use (or uses) of \textit{ti}, much of the remainder of the story may be told with verb forms unmarked for Past (i.e., as a sort of ``historical present'')''. This ``historical present'' also crops up in Bollée's second thoughts. While, according to her, the zero or stem form of the verb ``has the value of the \ili{French} present tense'' -- the reader will note the Euro\-centrism that unites these accounts -- it also expresses the ``historical present'' which is ``above all others the narrative tense \ldots~. After a brief introduction in the past, the rest of the story is told in the pres\-ent.'' However, she immediately adds a second thought: ``The above is not quite correct; the past often reappears at the opening of a new paragraph.''

Accounts of this nature inevitably arouse one's suspicions, especially as the \isi{First Law of Creole Studies} states: ``Every creolist's analysis can be directly contradicted by that creolist's own texts and citations.'' To demonstrate this law, I will analyze the middle portion of the second paragraph of the story \textit{Sabotaz at de ser} \citep[166]{Bollee1977}:

%\originalpage{84}
\ea\label{ex:2:94}
\gll {Biro} {leformasjo} {i} \emph{resevwar} {e} {let} {sorta} {lafras,} {avoje} {par} {e} {garso} {nome} {M}{sje} {Lezen} {ki} \emph{ti} \emph{ana} {trat-a,} {ki} \emph{ti} \emph{ule} {gaj} {portre} {e} {fij} {seselwas} {\ldots } {Me} {kom} {sa} {zofisje} \emph{ti} \emph{kon} {b}{je} {Msje} {ek} {M}{adam} {Lamur} {\ldots} {alor} i \emph{gaj} {e} {konsiderasjo} \\
Bureau information {\PM} receive a letter leaving France sent by a fellow name M. Lezen who {\TNS} have {thirty years} who {\TNS} want get portrait a girl Seychellois { } but as the agent {\TNS} know well M. and Mme. Lamur { } then he take into consideration\\
\glt `The Bureau of Information \textbf{received} a letter from France, sent by someone called Mr. Lezen, who \textbf{was} thirty years old, and who \textbf{wanted} to obtain a portrait of a Seychellois girl \ldots~. But as the agent \textbf{knew} Mr. and Mrs. Lamur well \ldots~he \textbf{took} into consideration \ldots '
\z

Here, as in many other places in Bollée's texts, the narrative switches from ``historical present'' into ``past'' and back again, right in mid-paragraph. Even Bollée's final disclaimer, therefore, will not work here. What is the explanation?

The alert reader will perhaps have noticed that the ``historical present'' verbs that immediately precede and follow the switch into ``past'' are both nonstatives -- \textit{resevwar} `receive', \textit{gaj(e)} `obtain, take' -- while the three verbs marked with \textit{ti} are all statives -- ana `have', \textit{ule} `want', and \textit{kone} `know'. If we refer back to the story openings that both Corne and Bollée mention, we find that the verbs marked there by \textit{ti} are also statives.\is{stative} Folktales\is{folktales} almost invariably begin with one or several of these: ``Once upon a time there \textit{was} a girl \ldots~ she \textit{was called} such and such~\ldots~ she \textit{had} two sisters \ldots '' It is this simple coincidence that has given rise to the hard-dying creole myth about ``narrative tenses'' and ``historical presents''.

In fact, in systems which have the feature anterior, past-reference nonstatives are unmarked, while past-reference statives receive anterior
%\originalpage{85}
marking (see \citet[Chapter 2]{Bickerton1975} for an explanation of why this is so). Thus, the distribution of \textit{ti} and zero in SC\il{Seychelles Creole} texts follows exactly the same rule of anterior\is{anterior tense} marking that affects stative and nonstative pasts in GC, HC\il{Haitian Creole}, SR, etc.\is{Guyanese Creole!TMA system}
Bollée and Corne\ia{Bollée, Annegret}\ia{Corne, Chris} cannot be blamed too heavily for this faulty analysis since SC is not a pure anterior system but one which under\-went certain changes when the completive \textit{(fi)n} was incorporated into Aux and permitted to combine with other markers. We shall see the consequences of this when we return to SC\il{Seychelles Creole} in the discussion of Deviation D.

However, there are creoles in which the presence of the category past cannot be attributed to faulty analyses. Papiamentu is perhaps the best attested of these so, pending adequate data on the few creoles that seem to resemble it, we may take the PP\il{Papiamentu} model as typical. I pro\-pose to claim that wherever this deviance is attested, it is the result of either heavy superstrate influence on the pidgin stage, or (more prob\-able in the majority of cases) subsequent \isi{decreolization}.

In the first place, in both HCE and GC\is{Hawaiian Creole English!TMA system}, where \isi{decreolization} phenomena are clear and well understood, the shift from anterior mark\-ing to past marking represents one of the earliest superstrate-influenced changes \citep{Bickerton1975,Bickerton1977}. Even in \ili{Sranan}, no longer in contact with its superstrate, a similar change is taking place at least among literate Sranan-Dutch bilinguals as can be seen if we compare the most recent texts with earlier ones in, e.g., \citet{Voorhoeve1976}.

However, a problem arises in the case of languages for which, unlike GC or SR\is{Guyanese Creole!TMA system}, no prior anterior stage is attested. Can we reconstruct such a stage from synchronic evidence?

There is a good likelihood that we can. In \citet{Bickerton1980} I showed that we could differentiate between \isi{decreolization} stages and natural changes: the former changed forms first and functions later; while the latter preserved old forms and gave them new functions. If the changes in PP\il{Papiamentu|(} are due to decreolization, and if there was an original anterior marker, then it follows that whatever is the past
%\originalpage{86}
marker now could not have been the anterior\is{anterior tense} marker then; in decreoli\-zation\is{decreolization|(}, instead of the original marker changing its function, a new marker is first adopted alongside of it, originally with an identical meaning (\textit{did} alongside GC \textit{bin}, \textit{wen} alongside HCE \textit{bin}), and then gradually changes its meaning to +past while the original anterior marker disappears or, if we are lucky, remains fossilized in some social or grammatical corner of the language. So, if we are both correct and lucky, we should be able to find in PP both a synchronic past marker and some vestige of the original anterior marker it displaced.

The PP past marker is \textit{a}, presumably cognate with PQ\il{Palenquero} \textit{a}, PK\il{Papia Kristang} \textit{ya} -- all of which most probably derive from Pg. \textit{ja} `already'. Adverbs, as we have seen, are not a good source for creole TMA markers. Anterior markers are most often recruited from a past copula form: Fr. \textit{été} yielding \ili{French} creole \textit{ti}, \textit{te}, etc.; Eng. \textit{been} yielding \ili{English} creole \textit{bin}, \textit{ben}, etc.; and \il{Portuguese}Pg. \textit{estava} yielding ST (and other Bight-of-Benin creole) \textit{tava}. \textit{Tava/taba} is therefore what one would predict as an anterior marker in the original stage of any Portuguese creole\il{Portuguese creoles}.

\textit{Taha} is of course well attested for PP, and its distribution is most interesting. Unlike other auxiliaries, it cannot occur alone before a verb, but only in conjunction with nonpunctual \textit{ta}:

\ea[*]{mi taba lesa}\label{ex:2:95}\z

\ea[ ]{
mi a lesa\\
\glt `I read (past)'}
\label{ex:2:96}\z



\ea[ ]{
mi ta lesa\\
\glt `I am reading'}
\label{ex:2:97}\z


\ea[ ]{
mi tabata lesa\\
\glt `I was reading'}
\label{ex:2:98}\z

The fusion of \textit{taba} and \textit{ta} clearly recalls GC \textit{bina}, of similar meaning and origin (i.e., the conjunction of anterior and nonpunctual attested, I believe, for every creole without exception). If \textit{a} had formed part of the original set of auxiliaries, whether with past or anterior\is{anterior tense} meaning, we would have expected to find the form *\textit{ata} for past-progressive sentences such as \REF{ex:2:98}.
%\originalpage{87}

We may therefore propose the following scenario for Papiamentu. Originally, it had \textit{taba} anterior and \textit{ta} nonpunctual, permitting the formation of \textit{tabata} (it is hard to think of any other way in which this form could have been derived). Decreolization then began, via contact between PP and the \ili{Spanish} of the Venezuelan mainland only a few miles away. Spanish \textit{ya} could have been the model as easily as Pg. \textit{ja}; indeed, for PP-Spanish bilinguals, \textit{ya} and \textit{ha}, the Spanish 3rd person perfective marker, could have easily reinforced one another to merge in \textit{a}. The result of borrowing \textit{a} as a past marker\is{past tense|)} would have been to bring Papiamentu, phonologically as well as semantically, more in line with its prestigious neighbor. But by the time \textit{a} entered the language, \textit{tabata} would already have come to be perceived as a single unit (as its modern orthography suggests) and would thus have survived the subsequent disappearance of \textit{taba}. Similarly, \textit{ti pe}, an SC\il{Seychelles Creole} form of comparable meaning, was retained even when some anterior functions of \textit{ti} were taken over by \textit{(fi)n}.\is{decreolization|)}

A further argument for an original anterior-nonanterior distinc\-tion in Papiamentu comes from the synchronic distribution of zero forms. For example, \citet[107]{Goilo1953} observes that stem forms express the present indicative for verbs such as \textit{gusta} `like', \textit{quier} `love', \textit{jama} `be called', etc. -- i.e., statives -- while all other verbs form the same tense with \textit{ta}. It should be made quite clear that this is not a parallel to the English habitual-progressive distinction. In English\is{English!tenses}, we can have \textit{I write} as well as \textit{I want}, and there exists the opposition \textit{I write/I am writing}. Papiamentu presents quite a different picture:

\ea[ ]{\label{ex:2:99}
mi quier esaquinan \\
\glt `I want these'}
\z
\exewidth{(234)}

\ea[*]{\label{ex:2:100}
mi ta quier esaquinan}
\z

\ea[*]{\label{ex:2:101}
mi skirbi buki}
\z

\ea[ ]{mi ta skirbi buki \\
\glt `I write books/am writing a book'}\label{ex:2:102}\z

If Papiamentu started life with a simple past-present opposition,
%\originalpage{88}
expressed by \textit{a} versus \textit{ta}, then the fact that statives in the present cannot take \textit{ta}, while nonstatives must, becomes merely a mysterious anomaly. However, if it began with an anterior-nonanterior\is{anterior tense} opposition, the fact is well motivated. In all such systems, present-reference statives are unmarked (since by definition they cannot take nonpunctual mark\-ing), and present-reference nonstatives are obligatorily marked with the nonpunctual morpheme (since any event or action that is ongoing in the present must be either durative or part of a series, whereas, con\-versely, any punctual event or action must be over, i.e., in the past, by the time it can be referred to!). The distribution of zeros and nonpunctuals in PP is identical to that of synchronic SR or basilectal GC\is{Guyanese Creole!TMA system}, except for one thing: in the latter, past punctuals, as well as present statives, are zero-marked (again, \citet[Chapter 2]{Bickerton1975} explains why this should be \textit{so}). In Papiamentu, \textit{a} has moved in to fill the ``vacuum'' created by zero-marked past-reference nonstatives\is{stative}, thereby bringing the PP\il{Papiamentu|)} TMA system closer to European models, but, again, as with \textit{tabata}, leaving clear traces of the more creole system that must have existed at an earlier stage.\footnote{It is perhaps worth observing that no account of Papiamentu that I know of translates \textit{I had worked}, so that the PP TMA system may not, in fact, differ as much from the classic system as those ac\-counts might suggest. In general, not only are most analyses of TMA systems incorrect, nine out of ten of them are simply incomplete, lacking the critical information which would make it possible to deter\-mine how they work. Yet, since these defective analyses buttress Euro-centric prejudices, they are hardly ever questioned, let alone criticized.}

We can now turn to Deviation D: the fact that MC\il{Mauritian Creole} and SC\il{Seychelles Creole} contain both \textit{ti} and \textit{(fi)n}, \textit{a} and \textit{pu}.

First, I shall have to comment on the state-of-the-art in Indian Ocean creoles\il{Indian Ocean creoles}. In MC and SC\il{Seychelles Creole}, and a fortiori in \ili{Réunion Creole}, that state is perhaps best exemplified by the following pessimistic remarks of \citet[94--95]{Corne1977}:

\begin{quote}
A close study of SC preverbal markers has been made by Bollée, by Papen and by myself, and the results of our efforts do not always coincide \ldots~. The sociolinguistic background of our informants is to some extent different, and this alone is quite possibly the source of conflicting data. It seems likely that some speakers categorise given markers differently from other speakers.
\end{quote}
% {\textbackslash}
%\originalpage{89}

Methods which make it possible, without invoking extralinguistic data, to systematically order and account for all variations in such systems were publicly presented by DeCamp in 1968 \citep{DeCamp1971}, refined and extended by C.-J. \citeauthor{Bailey1973} (\citeyear{Bailey1973}, etc.), and applied to the analysis of a seemingly far worse preverbal chaos in \citet{Bickerton1975}, after they had been proven effective in a number of other areas where creoles showed similar variation\is{variation, linguistic} (\citealt{Bickerton1971,Bickerton1973a,Bickerton1973b}, etc.). The bibliographies of Baker, Bollée, Corne, and other IOC\il{Indian Ocean creoles} scholars (\citealt{CarayolEtAl1977} is a distinguished exception) betray no awareness of any of this work, which I suppose merely reflects the parochialism that afflicts the field. Given this methodological time lag, anything one says about IOC TMA systems must be treated as provisional.

Combinations of markers form the liveliest areas of dispute. According to \citet{Baker1972} and \citet{Valdman1980}, \textit{(fi)n} will combine only with \textit{ti} in MC\il{Mauritian Creole}; according to \citet{Moorghen1975}, it will also combine with \textit{a} and \textit{pu}. According to \citet{Bollee1977}, \textit{(fi)n} will combine with \textit{a} and \textit{pu} in SC\il{Seychelles Creole}, but not with \textit{pe}; according to \citet{Corne1977}, \textit{n pe} com\-binations are found, in addition to the others.

A dynamic analysis can easily reconcile all these apparent contradictions. It was suggested earlier that between the second stage of \textit{(fi)n} incorporation -- movement into Aux -- and the third stage of \textit{(fi)n} incorporation -- free combinability with other preverbal markers -- we would expect to find intermediate stages, and the data in the previous paragraph suggest just such stages, preserved synchronically in the Indian Ocean population either by different groups, or at different stylistic levels, or both. If those data are correct, it would appear that the combinability of \textit{(fi)n} began with \textit{ti} in SC, spread to MC, while in SC it extended also to \textit{pu} and \textit{a}, and (for at least some speakers) spread to \textit{pu} and \textit{a} in MC\il{Mauritian Creole} while it was extending to \textit{pe} in SC\il{Seychelles Creole} -- a classic demonstration of Baileyan \isi{wave theory}.\footnote{When I wrote this paragraph, I was quite unaware that Baker\ia{Baker, Philip} had produced an extremely interesting account of the historical de\-velopment of MC, based in part on an analysis of all currently known historical citations \citep{Baker1976}, which provides a striking piece of independent support for this analysis. While \textit{fini} is recorded as a pre\-verbal marker in 1780, \textit{ti} is not recorded until 1818; but the \textit{ti va} combination is recorded in 1828, while the \textit{ti fin} combination is not recorded until 1867! Granted that these dates are probably all late -- nonstandard speech phenomena tend to have a long and lively life before they tickle the bourgeoisie, cf. \textit{olelo pa'i'ai} (see \chapref{ch:1}) which blushed unseen in Hawaii for nearly a century -- there is no need to doubt that their order and spacing are substantially correct. Baker seems not to realize, however, that the 1780 source derives, on both internal and external evidence, from a pidgin and not a creole speaker.} This proposal is made all the more plausible by the fact that combinability proceeds from tense, the leftmost Aux constituent, to modal, the second leftmost, to aspect, the rightmost (which, perhaps coincidentally, perhaps
%\originalpage{90}
not, is a conjunction of two members of the same class, \textit{(fi)n} being a \isi{completive}, therefore aspectual, and therefore, initially perhaps, being in the same slot as \textit{pe} and thus barred from co-occurring with it).\\\\

We are now almost ready to consider what would have been the repercussions of an invasion by \textit{(fi)n} of a classic creole system. First, however, we must note a particular characteristic of TMA systems which, though seemingly obvious, has been ignored by virtually all work up to and including \citegenp{Comrie1976} influential study of aspect. Meillet's famous observation that ``language is a system in which every\-thing keeps its place'' has the corollary that if a new element intrudes, everything must shift its place somewhat; while the latter statement may not be true of languages considered as wholes, it is certainly true for tight little grammatical subsystems like those of TMA. A TMA system may be compared to a cake, a cake that is always the same size, for TMA systems, whether simple or complex, all have to cover the same semantic area: every verb has to have some tense, mood, aspect, or combination of these applied to it, for there are (pace some creolists) no such things as ``TMA-neutral'' sentences.

But a cake may be split up into five, or eight, or ten slices, just as a TMA system can divide its semantic area among five, or eight, or ten TMA markers and/or combinations of markers. If a cake is divided into five slices, while another identical to it is divided into eight slices, there is no way in which each of the slices in Cake A can contain exactly the same amount of material as each of the slices in Cake B. In other words, how much, and exactly what, is contained in each slice will be largely determined by the number of slices. This is exactly the state of affairs in TMA systems throughout language; what each marker of modality, tense, or aspect means will be largely determined by how many markers of these things there are in the system and by what each of the others mean. Facts such as these are, however, ignored by most scholars in the field, who strive to fit all phenomena into the same conceptual straitjacket, and who, when
%\originalpage{91}
this fails, as fail it must, then seek, like Comrie, some kind of ideal type of the ``Progressive'' or the ``Perfective''.

The main point to be grasped here is that if you mark out a cake to be cut into \textit{n} slices, then change your mind and decide to cut \textit{n + 1}, you can only get your extra slice at the expense of one or more of the originals. Thus, if \textit{(fi)n} were introduced as a ninth term into the classic eight-term creole system, it could only be accommodated by robbing the semantic domain of one or more existing markers.

Since \textit{(fi)n} conjoined first with \textit{ti}, it is not surprising that \textit{ti} was its main victim. Admittedly, \textit{ti} held its ground with statives, as we saw in \REF{ex:2:94},\footnote{\citet{Corne1981} observes that ``with state ``Verbals'' \textit{fin} does not occur, since a state has by definition already been attained.'' Thus, the failure of \textit{fin} to take over anterior\is{anterior tense} marking in statives is a principled one, and not some inexplicable accident.} and in the nonpunctual \textit{ti pe} structure; the picture with punctual nonstatives is more complex. To clarify it, we need to refine the concept of anterior, which we can provisionally define as ``prior to the current focus of discourse''. But current focus may be explicit (where the times of an earlier and later event are directly contrasted), or implicit (where the relationship between the earlier and later events is simply assumed), or there may be nothing prior to current focus. The situation will become clearer with the following examples:\is{Guyanese Creole!TMA system|(}

%\todo{These examples are set in rm in the original}
%\ea\label{ex:2:103}
%{Current focus, nothing prior:}{}{}\\
%\begin{tabular}{lll}
%& Eng.: & Bill has come/came to see you\\
%& GC: &  Bil (don) kom fi sii yu	
%\end{tabular}
%\z

\ea\label{ex:2:103}
{Current focus, nothing prior:}\\
\ea 
\ili{\langEng}{}{}\\
Bill has come/came to see you\\
\ex 
\ili{\langGC}{}{}\\
Bil (don) kom fi sii yu\\
\z
\z

\ea\label{ex:2:104}
{Prior event, current focus implicit:}{}\\
\ea 
\ili{\langEng}{}{}\\
Bill came/*has/*had come to see you yesterday, too\\
\ex 
\ili{\langGC}{}{}\\ 
Bil bin kom/*don kom/*kom fi sii yu yestide an aal\\
\z
\z


\ea\label{ex:2:105}
{Prior event, current focus explicit:}{}\\
\ea 
\ili{\langEng}{}{}\\
When I got here, Bill had come/*has come/*came already\\
\ex 
\ili{\langGC}{}{}\\
 wen mi riich, bil bin kom/*don kom/*kom aredi\\
\z
\z


In \REF{ex:2:104} current focus is on the present, second visit of Bill implied by \textit{too}; this, English\is{English!tenses} can handle by one of the means available for \REF{ex:2:103}, but the anterior system of GC cannot. Example \REF{ex:2:104} has to be treated exactly like \REF{ex:2:105} in GC; \REF{ex:2:105} must be treated differ%
%\originalpage{92}
ently from \REF{ex:2:104} in English. This illustrates just one of the many differences between past-nonpast\is{past tense} and anterior-nonanterior\is{anterior tense} systems.

\citet[107]{Corne1977} has an illuminating minimal pair which shows that SC\il{Seychelles Creole} behaves much more like GC than like English\is{English!tenses} on ``current focus, nothing prior'' cases like \REF{ex:2:106} and ``prior event, current focus implicit'' cases like \REF{ex:2:108}; GC translations follow each example:

\ea\label{ex:2:106}
\gll mô \emph{n} \emph{vin} isi pur eksplik \ldots\\
I {\COMP} come here for explain\\
\glt `I have come here to explain (and I'm still here)'
\z

\ea\label{ex:2:107}
\ili{\langGC}{}{}\\
mi \textit{(don)} \textit{kom} ya fi ekspleen \ldots
\z

\ea\label{ex:2:108}
mô ti vin isi pur eksplik \ldots \\
\glt `I came (on a previous occasion) to explain (and then went away again)'
\z

\ea\label{ex:2:109}
 mi \textit{bin} \textit{kom} ya fi ekspleen \ldots
\glt
\z

\noindent The implicit current focus is of course the speaker's most recent arrival, since he could not say \textit{I came} unless he were here again.

However, when current focus is explicit, SC\il{Seychelles Creole} and GC part company (SC example from \citealt[108]{Corne1977}):

\ea\label{ex:2:110}
\gll letâ mô ti âtre dâ lasam, i \emph{ti} \emph{n} \emph{fini} mâz sô banan\footnotemark \\
time I {\TNS} enter in room, he {\TNS} {\COMP} finish eat his banana\\
\glt `When I entered the room, he had finished eating his banana'
\z
\footnotetext{Here Corne falls victim to the \isi{First Law of Creole Studies}, since he himself stated five pages earlier (\citeyear[103]{Corne1977}) that \textit{ti} is omitted from subordinate clauses. But I suspect that he was mostly right on this occasion and that he had not made allowances for the nonhomogeneity of SC. I would be prepared to bet that \REF{ex:2:110} came from a higher-class, more decreolized consultant.}

\ea\label{ex:2:111}
 wen mi kom iin di ruum, i \textit{bin} \textit{finish }(*bin don finish) nyam i banana\\
\z
In other words, it is only where there is prior reference with explicit current focus  --  i.e., when two past events have to be explicitly ordered with respect to one another  --  that \textit{ti n} encroaches on the domain of anterior \textit{ti}. However, reference of this kind is probably the most perceptually obvious of anterior functions (certainly it is the easiest to teach in creole courses), and its loss to a \isi{completive} cannot but serve to erode an anterior-based system and tilt it in the direction of a past-based system.

%\originalpage{93}

Anterior\is{anterior tense} is further eroded once one begins to get \textit{(ti) a n} and \textit{(ti) pu n} constructions. In the classic system, irrealis handles condi\-tionals and there is no distinction between probable and improbable conditions, so long as they are not counterfactual conditions:

\ea\label{ex:2:112}
\ili{\langGC}{}{}\\
mi go tel am if mi sii am
\glt `I'll tell him if I see him' or `I would tell him if I saw him'
\z
Counterfactuals\is{counterfactuals} are expressed by a combination of anterior\is{anterior tense} and irrealis: anterior because current focus in such cases is always on the conse\-quences of not having done whatever one didn't do; and irrealis\is{irrealis modality} because the action or event in question is an imaginary one:

\ea\label{ex:2:113}
\ili{\langGC}{}{}\\
if mi bin sii am mi bin go tel am
\glt `If I had seen him I would have told him'
\z

In SC\il{Seychelles Creole}, \textit{ti a n} (and less commonly, \textit{ti pit n}) naturally takes over from a prior \textit{ti a}, the ``pastest'' among conditions: counterfactuals such as \REF{ex:2:113}, yielding sentences such as \REF{ex:2:114} (example from \cite[109]{Corne1977}):

\ea\label{ex:2:114}
\gll mô \emph{ti} \emph{a} \emph{n} marie, si mô pa ti mizer\\
I {\TNS} {\MOD} {\COMP} marry, if I not {\TNS} poor\\
\glt `I would have gotten married if I weren't poor' 
\z

\ea\label{ex:2:115}
\ili{\langGC}{}{}\\
mi \textit{bin} \textit{go} mari if mi na bin puur
\z
The result of this development is another change in the system. The coming into existence of \textit{ti a n} does not automatically remove the former expression of counterfactuals, \textit{ti a}; \textit{ti a} remains in the lan\-guage, and what remains in the language has to mean something. The consequence of the \textit{ti a n} invasion is therefore the shifting of \textit{ti a} one step down the hierarchy of conditions, from impossible to merely improbable (\textit{if X, I would Y})  --  again, the example is from \citet[106]{Corne1977}:

%\originalpage{94}
\ea\label{ex:2:116}
\gll si u ti aste lavian, i \emph{ti} \emph{a} mâze\\
if you {\TNS} buy meat, he {\TNS} {\MOD} eat\\
\glt `If you bought some meat, he would eat it'
\z

\ea\label{ex:2:117}
\ili{\langGC}{}{}\\
if yu bai miit, i \textit{go} nyam am
\z
A further erosion of anterior terrain is indicated by the structure of the subordinate clauses in these examples. Note that the subordinate clause in counterfactual \REF{ex:2:115} requires anterior marking but that the subordinate clause in the merely improbable \REF{ex:2:117} does not; this is classic anterior marking. However, the subordinate clauses in both counterfactual \REF{ex:2:114} and merely improbable \REF{ex:2:116} are marked with \textit{ti}, presumably because subordinate clause marking is dragged down, so to speak, by the shift of \textit{ti a} main-clause marking from counterfactuals to improbables.

In other words, once you turn a \isi{completive} loose in a classic creole TMA system, the only consequence must be a drastic remodeling of that system. Some creoles (SR\il{Sranan}, basilectal GC, HC\il{Haitian Creole}) have kept their completives\is{completive} under control either by keeping them out of Aux al\-together or by allowing them in but not letting them combine with other auxiliaries. It is not coincidence that these creoles are ones which have kept the classic TMA system virtually intact. On the other hand, creoles that have let the \isi{completive} have the run of the house -- such as SC\il{Seychelles Creole}, MC\il{Mauritian Creole}, and \ili{Krio} -- have, in consequence, had to change their TMA systems to a point at which reconstruction of the original system becomes quite difficult, although not -- thanks to the careful work of Corne and others -- impossible.

The curious reader may well ask, ``Why is it that some systems have let loose their completives\is{completive}, while others have not?'' I have no answer to that, at present. Suffice it to say that doing so must remain an option within any theory of \isi{linguistic change}; if the option is taken, certain results must follow, as night, day; if it is not, they will not. It would be interesting to know why, but the fact that we do not can in no way affect the validity of the foregoing analysis.

% CREOLE 95

The other main divergence of IOC\il{Indian Ocean creoles} from the classic model is the presence of two ``future'' forms, \textit{a} and \textit{pu}. Several things are at issue here. One of them is why either form should be limited to future, rather than being a true irrealis\is{irrealis modality}. In fact, on the evidence of \citet[103]{Corne1977}, \textit{pu} retains conditional meaning, if only in subordinate clauses, and thus still covers a large part of the semantic area of ir\-realis:

\ea\label{ex:2:118}
\gll ipa ti kone ki i \emph{pu} fer\\
he not {\TNS} know what he {\MOD} do \\
\glt `He didn't know what he would do'
\z

\ea\label{ex:2:119}
\ili{\langGC}{}{}\\
i na (bin) no wa i go du/wa fe du
\z

The GC translation is instructive in several respects. First, the optionality of \textit{bin} serves to underscore another characteristic of ante\-rior as opposed to past marking. Expressions like \textit{he didn't know} are ambiguous between `he didn't know, but he knows now' and `he didn't know then and he still doesn't know'. In the first reading, \textit{bin} is obli\-gatorily present since this reading represents another instance of prior event (or rather, prior state, in this case) with implicit current focus, i.e., upon the change in state of the person referred to. In the second reading, \textit{bin} is obligatorily absent since the state of not knowing is a continuing one, and there is therefore nothing prior to refer to.

Second, we see again that the range of a true irrealis\is{irrealis modality} (GC \textit{go}) parallels at least part of the range of an SC\il{Seychelles Creole} marker. But the third point is perhaps the most interesting. \textit{He did not know what he would do} is semantically close to, if not quite synonymous with, \textit{he did not know what to do}. The GC sentence \textit{i na no wa fi du} more accurately translates the second of these sentences.

\textit{Fi} (which often takes the form \textit{fu}) derives from \il{English}Eng. \textit{for}, while \textit{pu} derives from Fr. \textit{pour} `for'. \textit{Fi} can also be a complementizer, as can \textit{pu}. \textit{Fi} can occur as an auxiliary in its own right:

\ea\label{ex:2:120}
mi fi go\\
\glt `I should/ought to go'
\z

%\originalpage{96}

Note that \textit{fi} is narrowly restricted to a meaning of \isi{obligation}, while it is a verbal auxiliary. When it is a complementizer, however, it and its cognates in other creoles express irrealis\is{irrealis modality} meaning just as \textit{pu} does, see examples \REF{ex:2:27}--\REF{ex:2:30} and \REF{ex:2:35}--\REF{ex:2:37} above.

\textit{Fi} in general does not combine with other auxiliaries, but in GC it does occasionally occur with \textit{bin}, and it is tempting to claim that it is starting to do just what \textit{(fi)n} did in IOC, i.e., combine first of all with the anterior marker. Thus one can have GC \REF{ex:2:121}:

\ea\label{ex:2:121}
mi bin fu nak am
\glt `I should have hit him' or `I was about to hit him'
\z

The construction is not common in GC, and native speakers are more or less evenly divided as to which gloss is the more appropriate.

Thus, in basilectal GC, with a classic TMA system, only a slight increase in the semantic range of \textit{fi} and in the syntactic privileges of occurrence is needed for the situation to begin to approximate that of SC\il{Seychelles Creole} and MC\il{Mauritian Creole}. All we have to assume is that a process which is beginning in GC (and which must be latent in any creole since all creoles, presumably -- the most drudgingly comprehensive grammars are all too often silent on this score -- have an auxiliary of obligation which is ipso facto +irrealis) has been taken a stage or two further in IOC, languages which we already know have a predilection for expanding and complicating Aux.

The rest of the story is simple. As \textit{pu} was graduating as a full-fledged competitor to the original \is{irrealis modality}irrealis \textit{a}, \textit{(fi)n} was distorting the classic system in such a way that the irrealis scope of both markers, \textit{a} and \textit{pu}, was losing some of its conditional functions and thus was getting closer to a simple future. In any case, when any two mor\-phemes divide the semantic terrain of ``future'', it is a highly natural development that they should mark out their boundaries in some way, and that those boundaries, while often vague, should generally dis\-tinguish relatively likely from relatively unlikely events (cf. the discus\-sion of \textit{go} versus \textit{gon}, HCE's first mesolectal replacement, \citealt[23ff., 181ff.]{Bickerton1977}).
%\originalpage{97}
In fact, the IOC\il{Indian Ocean creoles} position is far from clear-cut, and MC and SC\il{Seychelles Creole} seem to have developed rather differently. According to \citet{Corne1977}, \textit{pu} in MC marks a more definite future. In SC\il{Seychelles Creole}, on the other hand, the precise roles of \textit{pu} and \textit{a} are more vague, although the fact that only \textit{pu} can occur in the scope of negation suggests that here, \textit{pu} may be becoming the less definite of the two.\\\\

We have now seen how two very natural developments could have turned a classic creole TMA system into the kind of system we see in IOC today. IOC scholars will doubtless object that the account I have given is a purely conjectural one. That may be; but if it is con\-jectural, that is only because those scholars have not done the job of tracing the diachronic development of IOC through synchronic resi\-dues, as was done in \citet{Bickerton1975} for GC.\is{Guyanese Creole!TMA system|)} The most anyone can do who does not have direct and unlimited access to the IOC commun\-ity is to show that similar developments have occurred or are occurring elsewhere in other creoles, which I have done. There is thus a prima facie case for the scenario outlined above; conclusive evidence can only come through the patient sifting of the highly variable data about which all IOC scholars have complained, but which none of them have yet exploited.\\\\

We can now turn to our final deviation -- the split between habi\-tuals\is{habitual} and progressives which, according to \citet{Taylor1971}, conflates the former with the ``completive'' (in the present terminology, zero-marked past punctuals) in JC\il{Jamaican Creole} and HC\il{Haitian Creole} and with the ``future'' (in the present terminology, irrealis\is{irrealis modality}) in CR and ST\ili{S\~ao Tomense}. Again, as with anterior versus past marking, we must first ask ourselves if the data on which such claims are made are valid. Again, we must answer that at least sometimes they are not.

For example, \citet[31]{Hall1953} describes the HC aspect marker \textit{ap(e)} as ``indicating action which is continuing, not yet complete, or future'' -- in other words, \textit{ap(e)} does not include habituals. This, if true, would indicate that the HC\il{Haitian Creole} system is not a classic TMA system since in that system the nonpunctual category embraces both con%
%\originalpage{98}
tinuing and \isi{habitual} actions. However, the \isi{First Law of Creole Studies} enables us to find, in Hall's own texts, numerous sentences in which \textit{ap(e)} marks habituals, past as well as nonpast:

\ea\label{ex:2:122}
\gll sa \emph{k-ap-fè} mâjé, \emph{ap-fè} mâjé pou-apézé lwa yo\\
that which-\ASP-make eat, \ASP-make eat {for appease} \textit{loa} {\PL}\\
\glt `Whatever they \textit{used} \textit{to} \textit{give} it to eat, they \textit{used} \textit{to} \textit{give} it to eat to appease the African gods'
\z

\ea\label{ex:2:123}
 \gll tout gasô chak mèkrédi \emph{ap-prâ} \emph{bou-t-makak} yo\\
all fellow each Wednesday \ASP-take stick {\PL}\\
\glt `Each Wednesday, all the fellows \textit{take} their sticks'
\z

\ea\label{ex:2:124}
\gll tout moun sou-latè \emph{ap-chaché} pou-yo viv avèk êtélijâs\\
all person on-earth \ASP-seek for-they live with intelligence\\
\glt `Everyone on earth \textit{tries} to live with a head well filled (with knowledge)'
\z

Although some claims may be disposed of in this way, there remains, as with anteriors, a residue of cases where genuine problems need to be resolved. In this particular case, a new factor enters the system: areas of relative indeterminacy in \isi{semantic space}. In \chapref{ch:4}, where we will try to extract the very roots of semantics, it will become apparent that Deviation E represents an inherent point of weakness in the semantic infrastructure of the TMA system\is{tense-modality-aspect (TMA) systems|)}; we will see also that such points \textsc{must} exist if language is to change and develop. However, since a full account of Deviation E depends on a prior analysis of the nature of \isi{semantic space}, it will have to be postponed until that chapter.

For the present, then, we can conclude that the bulk of so-called ``counterexamples'' to our analysis of creole TMA systems arises from one of the three following sources:

%\setcounter{itemize}{0}
\begin{enumerate}
\item Inadequate data-gathering and/or acceptance of inaccurate data and/or faulty analysis of data.
\item A slightly longer than normal antecedent period of \isi{pidginization}, allowing pidgin features to become fixed.
% CREOLE 99
\item Linguistic change, internal or contact-stimulated, subsequent to creolization.
\end{enumerate}
In one or two cases, 2. would distort the normal process of creolization, although we must note that only syntactic, and not semantic, aspects of that process are affected: PP \textit{lo} and CR \textit{ba} retain their predicted meanings, even though they do not assume their predicted place in sentence structure. Being a subsequent development, 3. cannot have any relevance to the process of creolization itself. As for 1., one can only hope that this will disappear as the field continues to develop.\\\\

Finally, I shall examine three types of complementation in creoles: complements of perception verbs\is{complementation!of perception verbs|(}; factive\is{complementation!factive}, nonfactive, and related complement structures; and ``serial verb'' structures. My aim in doing so will be twofold. First, as in the previous sections of this chapter, I shall seek to show that substantial identities of structure exist throughout creoles, even where these may be masked by ongoing change processes or other factors. But I also want to establish certain facts about the nature of creole syntax -- facts which will assume a greater significance when we meet them again in Chapters \ref{ch:3} and \ref{ch:4}.

In English, the complements\is{English!complements} of perception verbs consist of nonfinite sentences from which aspectual markers are excluded, and the subjects of which have undergone raising:\footnote{If you believe in raising. If you don't, substitute ``whatever rule marks the second NP as object of the first V.''}

\ea[ ]{\label{ex:2:125}I saw him leaving the building.}\z

\ea[ ]{\label{ex:2:126}We can hear them play trombones.}\z

\ea[*]{\label{ex:2:127}I saw he leaving the building.}\z

\ea[*]{\label{ex:2:128}I saw him was leaving the building.}\z

\ea[*]{\label{ex:2:129}We can hear them have played trombones.}\z
While a superficially similar sentence, \REF{ex:2:130}, is grammatical, it does not contain a perception-verb complement, but a factive complement\is{complementation!factive} that has undergone complementizer deletion:
%\originalpage{100}

\ea\label{ex:2:130}I saw he was leaving the building.\z

\ea\label{ex:2:131}I saw \textit{that} he was leaving the building.\z

In creoles, perception-verb complements\is{Guyanese Creole!complementation|(} are finite, can contain aspect markers, and have subjects which do not undergo raising:

\ea\label{ex:2:132}
\ili{\langGC}{}{}\\
\gll mi hia drom a nak\\
I hear drum {\ASP} beat\\
\glt `I heard drums beating'
\z

\ea\label{ex:2:133}
\ili{\langGC}{}{}\\
dem sii i kom
\glt `They saw him come'
\z
Although one might be tempted to gloss \REF{ex:2:132} as `I heard \textit{that} drums were beating' -- along the lines of \REF{ex:2:130} -- such a gloss would be incorrect; factive complements\is{complementation!factive} are introduced by the obligatory particle \textit{se}, which we shall return to later:

\ea\label{ex:2:134}
\ili{\langGC}{}{}\\
mi hia \textit{se} drom a nak
\glt `I heard {\scshape that} drums were beating'
\z
In \REF{ex:2:133}, the nominative case of the 3rd person singular pronoun is obligatory; the accusative case is ungrammatical:

\ea\label{ex:2:135}
\ili{\langGC}{}{}\\
\textnormal{*} dem sii am kom
\z
Free occurrence of nonpunctual-aspect\is{nonpunctual aspect} \textit{a} and the ungrammaticality of an accusative form such as would result from raising indicate that the embedded sentences in \REF{ex:2:132} and \REF{ex:2:133} are finite. At least two further arguments point in a similar direction.

In English perception-verb complements\is{English!complements}, it is possible not merely to raise the subject of the embedded S, but also to delete it. Thus, alongside \REF{ex:2:136} we can have \REF{ex:2:137}:

\ea\label{ex:2:136}I heard Bill singing.\z

%\originalpage{101}
\ea\label{ex:2:137}I heard singing.\z
GC will allow the equivalent of \REF{ex:2:136}, but not of \REF{ex:2:137}:

\ea[ ]{\label{ex:2:138}mi hia bil a sing}\z

\ea[*]{\label{ex:2:139}mi hia a sing}\z
It is characteristic of languages in general that while they may allow zero subjects\is{zero subject} in nonfinites, they cannot freely delete subjects in finite clauses, except of course under identity, which does not apply here.

A second argument involves the \isi{Propositional Island Constraint (PIC)} as proposed by \citet{Chomsky1977}. The PIC affects structure of the form:

\ea\label{ex:2:140}\ldots~X \ldots~$_{\alpha}$[\ldots~Y \ldots] \ldots~X \ldots \z
and prevents any rule from moving a constituent from position Y to either position X just in case α marks a finite clause. Let us assume that, ignoring some nonrelevant details, the structure underlying both \REF{ex:2:132} and its English equivalent is something like \REF{ex:2:141}.\footnote{As mentioned earlier in this chapter, it seems likely that in reality GC does not have VP as a constituent at the basilectal level. The contrary is assumed here merely in order to simplify the comparison between the English and GC \textit{processes}, and is not meant to imply any substantive claim about GC structure.} In the case of the English version of \REF{ex:2:132}, this would yield a derived structure something like \REF{ex:2:142}. However, in the case of the GC version of \REF{ex:2:132}, \REF{ex:2:141} would represent the superficial as well as the underlying structure.

\begin{exe}
\ex\label{ex:2:141}
%\resizebox{.9\textwidth}{!}{
	{\normalfont
	\begin{tikzpicture}[baseline]
	% Levels 1 & 2
	\node at (0,0) (1S) {S};
	\node [below left=\baselineskip and 18mm of 1S] (2NP) {NP};
	\node [below=\baselineskip of 1S] (2AUX) {AUX};
	\node [below right=\baselineskip and 30mm of 1S] (2VP) {VP};
	
	% Level 3
	\node [rectangle split, rectangle split parts=2, below=\baselineskip of 2NP] (3I) {I\nodepart{two}mi};
	\node [below=\baselineskip of 2AUX] (3Past) {Past};
	\node [below left=\baselineskip and 12mm of 2VP] (3V) {V};
	\node [below right=\baselineskip and 12mm of 2VP] (3S) {S};
	
	% Level 4
	%	\node [below=.1\baselineskip of 3I] (4mi) {mi};
	\node [rectangle split, rectangle split parts=2, below=\baselineskip of 3V] (4hear) {hear\nodepart{two}hia};
	\node [circle, draw, below left=\baselineskip and 12mm of 3S] (4NP) {NP};
	\node [below=\baselineskip of 3S] (4AUX) {AUX};
	\node [below right=\baselineskip and 12mm of 3S] (4VP) {VP};
	
	% Levels 5 and 6
	\node [rectangle split, rectangle split parts=2, below=\baselineskip of 4NP] (5drums) {drums\nodepart{two}drom};
	\node [below=2\baselineskip of 4AUX] (5ASP) {ASP};
	\node [below=2\baselineskip of 4VP] (5V) {V};
	\node [rectangle split, rectangle split parts=2, below=\baselineskip of 5V] (6beat) {beat\nodepart{two}nak};
	
	
	% Connections
	\draw (node cs:name=1S, anchor=south) -- (node cs:name=2NP);
	\draw (node cs:name=1S, anchor=south) -- (node cs:name=2AUX);
	\draw (node cs:name=1S, anchor=south) -- (node cs:name=2VP);	
	\draw (node cs:name=2NP, anchor=south) -- (node cs:name=3I);
	\draw (node cs:name=2AUX, anchor=south) -- (node cs:name=3Past);
	\draw (node cs:name=2VP, anchor=south) -- (node cs:name=3V);
	\draw (node cs:name=2VP, anchor=south) -- (node cs:name=3S);	
	\draw (node cs:name=3V, anchor=south) -- (node cs:name=4hear);
	\draw (node cs:name=3S, anchor=south) -- (node cs:name=4NP);
	\draw (node cs:name=3S, anchor=south) -- (node cs:name=4AUX);
	\draw (node cs:name=3S, anchor=south) -- (node cs:name=4VP);
	\draw (node cs:name=4NP, anchor=south) -- (node cs:name=5drums);
	\draw (node cs:name=4AUX, anchor=south) -- (node cs:name=5ASP);
	\draw (node cs:name=4VP, anchor=south) -- (node cs:name=5V);
	\draw (node cs:name=5V, anchor=south) -- (node cs:name=6beat);
	\end{tikzpicture}
}
%}
\end{exe}
%\originalpage{102}

\begin{exe}
\ex \label{ex:2:142} %\resizebox{.8\textwidth}{!}{
	{\normalfont
	\begin{tikzpicture}[baseline]
	% Levels 1 & 2
	\node at (0,0) (1S) {S};
	\node [below left=\baselineskip and 12mm of 1S] (2NP) {NP};
	\node [below=\baselineskip of 1S] (2AUX) {AUX};
	\node [below right=\baselineskip and 24mm of 1S] (2VP) {VP};
	
	% Level 3
	\node [below=\baselineskip of 2NP] (3I) {I};
	\node [below=\baselineskip of 2AUX] (3Past) {Past};
	\node [below left=\baselineskip and 6mm of 2VP] (3V) {V};
	\node [circle, draw, inner sep=2pt, yshift=.25em, below=\baselineskip of 2VP] (3NP) {NP};
	\node [below right=\baselineskip and 6mm of 2VP] (3VP) {VP};
	
	% Level4
	\node [below=\baselineskip of 3V] (4hear) {\strut hear};
	\node [below=\baselineskip of 3NP, yshift=.3em] (4drums) {\strut drums}; % NOT PERFECT!!
	\node [below=\baselineskip of 3VP] (4beating) {\strut beating};
	
	% Connections
	\draw (node cs:name=1S, anchor=south) -- (node cs:name=2NP);
	\draw (node cs:name=1S, anchor=south) -- (node cs:name=2AUX);
	\draw (node cs:name=1S, anchor=south) -- (node cs:name=2VP);	
	\draw (node cs:name=2NP, anchor=south) -- (node cs:name=3I);
	\draw (node cs:name=2AUX, anchor=south) -- (node cs:name=3Past);
	\draw (node cs:name=2VP, anchor=south) -- (node cs:name=3V);
	\draw (node cs:name=2VP, anchor=south) -- (node cs:name=3NP);
	\draw (node cs:name=2VP, anchor=south) -- (node cs:name=3VP);
	\draw (node cs:name=3V, anchor=south) -- (node cs:name=4hear);
	\draw (node cs:name=3NP, anchor=south) -- (node cs:name=4drums);
	\draw (node cs:name=3VP, anchor=south) -- (node cs:name=4beating);
	
	\end{tikzpicture}
}
%}
\end{exe}

If we have analyzed these sentences correctly, then it should be possible to extract from the circled node in \REF{ex:2:142}, since such a move does not violate the PIC\is{Propositional Island Constraint (PIC)}, but impossible to extract from the circled node in \REF{ex:2:141}, which would constitute such a violation since the lower S dominates a finite clause. Extraction from \REF{ex:2:142} is fine:

\ea\label{ex:2:143}
 It was drums that I heard beating.
\z

\ea\label{ex:2:144}
 What did I hear beating?
\z

\ea\label{ex:2:145}
 The drums that I heard beating never stopped.
\z
Extraction from the circled node of \REF{ex:2:143}, however, yields only sentences which are ungrammatical in GC:

\ea[*]{\label{ex:2:146}a drom mi hia a nak}\z

\ea[*]{\label{ex:2:147}a wa mi hia a nak?}\z

\ea[*]{\label{ex:2:148}di drom-dem we mi hia a nak neva stap}\z
Since there is no other possible reason for the ungrammaticality of these sentences, we can only conclude that they are ungrammatical because they violate the PIC\is{Propositional Island Constraint (PIC)}, and that therefore the embedded sentence in \REF{ex:2:132} is a finite one.

%\originalpage{103}

Very few writers on creoles have discussed perception-verb com\-plements specifically, and fewer still have even attempted to analyze them. Thus, the most that can be done at present is to point to a wide range of superficially similar structures and hope that scholars in the various regions will determine whether they show the same constraints on subject deletion and extraction as did the GC examples. Similar structures are found in other English creoles, such as \ili{Belize Creole} (BC); in \ili{French creoles}, such as HC\il{Haitian Creole} and \ili{Guyanais} (GU); and in Portuguese creoles like ST\il{S\~ao Tomense}:

\ea\label{ex:2:149}
\ili{\langBC}{}{}\\
i onli si di tar a flo:t ina di bailing wata
\glt `He only saw the tar floating (lit., tar was floating) in the boiling water'
\z

\ea\label{ex:2:150}
\ili{\langGU}{}{}\\
 \gll mo we li ka briga\\
I see he {\ASP} fight\\
\glt `I saw him fighting (lit., he was fighting)'
\z

\ea\label{ex:2:151}
\ili{\langHC}{}{}\\
\gll li wè tèt Boukinèt ap-gadé li\\
he see head Bouquinette \ASP-watch him \\
\glt `He saw Bouquinette's head watching him (lit., head was watching)'
\z

\ea\label{ex:2:152}
\ili{\langST}{}{}\\
\gll e be i-ska landa\\
he see {I-\ASP} swim\\
\glt `He saw me swimming (lit., I was swimming)'
\z
Before leaving perception-verb complements, we should note that there are also some similar constructions which are nonfinite in English but clearly finite in at least one English creole, GC; for example, causative\is{causative} imperatives:

\ea[ ]{\label{ex:2:153}
mek i gowe\\
\glt `Make him leave'}
\z

\ea[*]{\label{ex:2:154}mek am gowe}\z

\ea[ ]{\label{ex:2:155}
 na mek i na wok\\
\glt `Don't prevent him from working'}
\z
%\originalpage{104}

\ea[*]{\label{ex:2:156}
na mek am na wok\\}
\z

\noindent Note the impossibility of clefting\is{clefting} such sentences in GC:\is{movement rules!in GC}


\ea[ ]{\label{ex:2:157}I prevented him from working.}\z

\ea[ ]{\label{ex:2:158}It was him that I prevented from working.}\z

\ea[ ]{\label{ex:2:159}mi mek i na wok}\z

\ea[*]{\label{ex:2:160}a i mi mek na wok}\z

Example \REF{ex:2:160} is so bad that it is almost unpronounceable. However, the restriction does not apply to clefting per se since the subject NP may undergo the process:

\ea a mi mek i na wok\\
\glt `It was I who prevented him from working'
\label{ex:2:161}
\z

The fact that complements of perception\is{complementation!of perception verbs|)} and causation verbs appear to constitute finite sentences in creoles suggests the possibility that there might be no such thing as a nonfinite structure in these languages. In fact, I doubt whether there is any creole extant for which such an extreme statement would be true. However, there is a good deal of evidence which suggests that at their earliest stage of develop\-ment creoles may not have had any nonfinite structures.\\\\

It should have become apparent by now that we are not going to get very far with the study of creoles -- or of child language acquisition, or of language origins -- if we allow ourselves to remain trapped within the static, antiprocessual framework which has dominated linguistics since de Saussure. The emergence of creole languages is a process; language acquisition is a process; the original growth and development of human language was assuredly a process. To apply to processes those methods expressly designed to handle static-synchronic systems is simply absurd; in order to do this, you have to pretend that a process
%\originalpage{105}
is a state, and ignore exactly those characteristics that render it distinc\-tive. Such a procedure is sometimes defended as an ``\isi{idealization}'', cf. \citet[Chapter 8]{ChomskyEtAl1968}, but the difference between ``idealization'' and ``convenient fiction'' seems not to be grasped by these authors. In fact, static generativism, the only kind we have had so far (although there is no a priori reason why there should not be a dynamic generativism), has ignored creoles, ignored language origins, and in the case of language acquisition\is{language!acquisition of} -- something it could hardly ignore since the mystery of language acquisition was what it was originally set up to explain -- it has intervened with the sole result of turning off 90 percent of the workers in the field, as we shall see in the next chapter.\\\\

\is{complementation!factive|(}If we are going to call a spade a spade and a process a process, we need to make some basic assumptions. One is that previous changes in any language inevitably leave their footprints behind them \citep{Givón1971}. Another is that diachronic changes\is{linguistic change} must be directly reflected in synchronic variation \citep{WeinreichEtAl1968,Bickerton1975,Bailey1973}. Equipped with these, we shall examine other types of complementation in creoles to determine whether the current state of affairs in perception-verb\is{complementation!of perception verbs} and causative\is{causative} constructions may at one time have been that of all complement types.

First let us look at a set of sentences which might appear to contain complementizers. In GC, there are three forms that might be taken for complementizers: \textit{se}, \textit{go}, and \textit{fu/fi.} We have already glanced at the second two in connection with the realized/nonrealized com\-plement distinction. The first, \textit{se}, introduces complements of verbs of reporting and ``psychological'' verbs:

\ea\label{ex:2:162}
i taak se i na si am \\
\glt `He said that he didn't see it'
\z

\ea\label{ex:2:163}
 i tel mi se i na si am\\
\glt `He told me that he didn't see it'
\z

\ea\label{ex:2:164}
mi no se i na si am\\
\glt `I know that he didn't see it'
\z
%\originalpage{106}

\noindent Clearly, the complements that \textit{se} introduces are finite Ss, just as are those introduced by Eng. \textit{that}. However, it does not follow from this that \textit{se} is a complementizer.

Doubt arises in the first place because \textit{se}, unlike complementizers in general, is nondeletable. A sentence like \textit{mi hia se i a kom} means `I heard (that) he was coming'; \textit{mi hia i a kom}, however, cannot be synonymous with this, but can only mean `I heard him earning'. In other cases, such as \textit{mi taak se i a kom} `I said that he was coming', deletion yields only ungrammatical sentences: *~\textit{mi taak i a kom.}

Further, there is the fact that {\itshape se}-clauses cannot be generated in subject position. In English\is{English!complements|(}, \textit{that}-clauses can be generated in subject position and then undergo optional rightward movement\is{movement rules!in GC} by a rule of \isi{extraposition}; thus, \REF{ex:2:165} would be assumed to be closer to its underlying structure than \REF{ex:2:166}, derived from the same underlying structure via extraposition:

\ea\label{ex:2:165}
That John has left isn't true.
\z

\ea\label{ex:2:166}
It isn't true that John has left.
\z
However, a similar generalization could not be true for GC since while there is grammatical equivalent for \REF{ex:2:166}, there is no grammatical equivalent for \REF{ex:2:165}:


\ea[*]{\label{ex:2:167}
se jan gaan na tru}
\z

\ea[\hspaceThis{*}]{\label{ex:2:168}
na tru se jan gaau}
\z

Not only can \textit{se}-clauses not be generated in subject position, they cannot be moved to sentence-initial position by any rule. There is no creole passive that would turn \textit{Everybody knows that he won} into \textit{That he won is known by everyone}. Clefting\is{clefting} and pseudoclefting will front simple NP objects of verbs like \textit{no} `know' but not \textit{se}-clause objects: 

\ea[ ]{\label{ex:2:169}
mi no dis\\
\glt `I know this'}
\z

\ea[ ]{\label{ex:2:170}
 a dis mi no \\
\glt `It's this that I know'}
\z

\ea[ ]{\label{ex:2:171}
 dis a wa mi no\\
\glt `This is what I know'}
\z

\ea[ ]{\label{ex:2:172}
 mi no se dem gaan\\
%\originalpage{107}
\glt `I heard (that) he was coming'}
\z

\ea[*]{\label{ex:2:173}
a se dem gaan mi no}\z

\ea[*]{\label{ex:2:174}se dem gaan a wa mi no}\z

\noindent True, neither clefting\is{clefting} nor pseudoclefting works in English either, unless there is a head noun:

\ea[*]{\label{ex:2:175}It's that they've left that worries Bill.}\z

\ea[ ]{\label{ex:2:176}
 It's the fact that they've left that worries Bill.} 
\z
But English can front via \isi{topicalization}:

\ea\label{ex:2:177}
I knew already that they'd left.
\z

\ea\label{ex:2:178}
That they'd left I knew already.
\z

\noindent GC cannot:

\ea[ ]{\label{ex:2:179}
mi no aredi se dem gaan}
\z

\ea[*]{\label{ex:2:180}
se dem gaan mi no aredi}
\z

Now, it is true that this datum, taken in isolation, says nothing directly about the status of \textit{se}. It merely suggests that \textit{se}-clauses cannot be dominated by an NP node since if they were, they would presumably be eligible for movement rules\is{English!movement rules} that affect NPs and would also constitute possible expansions of subject NPs. If we assumed that \textit{se}-clauses were generated under an \=S node which in turn was immediately dominated by VP (or S$_0$, if VP is not a constituent in GC grammar), all the above data would follow.\is{complementation!factive|)}

However, there are some facts that suggest the possibility of
%\originalpage{108}
an alternative analysis. In English, there are pairs of sentences such as:\is{English!complements|)}

\ea\label{ex:2:181}
I'm glad \textit{that} \textit{they've} \textit{left}.
\z

\ea\label{ex:2:182}
\textit{That} \textit{they've} \textit{left} makes me glad.
\z
These sentences are perhaps slightly less than synonymous, and they certainly would not be regarded as transformationally related; since we have already established that \textit{se}-clauses cannot be base-generated in subject position, it will come as no surprise that the GC equivalent of \REF{ex:2:181} is grammatical, while the GC equivalent of \REF{ex:2:182} is ungrammatical:

\ea[ ]{\label{ex:2:183}mi glad se dem gaan}\z

\ea[*]{\label{ex:2:184}se dem gaan mek ml glad }\z

\noindent Yet \REF{ex:2:185} is grammatical:

\ea\label{ex:2:185}
 dem gaan mek mi glad
\glt  lit., `They've left \textsc{cause} I glad'
\z

Example \REF{ex:2:185} cannot be derived from \REF{ex:2:184} by \textit{se}-deletion since, as we saw, \textit{se} does not delete. It could only be derived by embedding S under \textit{the} subject NP node.

Again, these facts, taken in isolation, might not seem to con\-stitute evidence against the status of \textit{se} as a complementizer. Since we have already suggested that \textit{se}-clauses could be introduced under \=S not dominated by NP, all we need in order to accommodate \REF{ex:2:185} is a rule that will expand NP as S, but not as \=S. However, the picture would change somewhat if we could show one or both of two things:

%\setcounter{itemize}{0}
\begin{enumerate}
\item That GC required a rule NP → \=S. \label{GCrequirementcondition}
\item That \textit{se}-clauses could not be generated under \=S. \label{GCrequirementcondition2}
\end{enumerate}

In order to examine these possibilities, let us look at another quasi-%
%\originalpage{109}
complementizer, \textit{fi/fu} (henceforth referred to as \textit{fi}, for the sake of convenience, since \textit{fi} is the more basilectal, if nowadays rarer, form). In the GC lexicon, \textit{fi} must be entered both as a preposition and as a modal auxiliary of \isi{obligation}:

\ea\label{ex:2:186}
mi du am fi meri, na fi ayu\\
\glt `I did it for Mary, not for you (pl.)'
\z

\ea\label{ex:2:187}
mi fi go tumara\\
\glt `I ought to go tomorrow'
\z
However, there are also sentences such as:

\ea\label{ex:2:188}
mi waan fi go\\
\glt `I want to go'
\z

\ea\label{ex:2:189}
 mi waan i fi go\\
\glt `I want him to go'
\z
In \REF{ex:2:188}, \textit{fi} looks like a complementizer, more or less the equivalent of Eng. \textit{to}. The likelihood that, unlike \textit{se}, \textit{fi} is a genuine complementizer is increased by the fact that \textit{fi} in \REF{ex:2:188} will delete without change of meaning:

\ea\label{ex:2:190}
 mi waan go \\
\glt `I want to go'
\z

Unfortunately, \REF{ex:2:189} seems at first sight to suggest a quite different analysis. Complementizers normally precede the sentences they introduce, but \REF{ex:2:191} is ungrammatical:

\ea[*]{\label{ex:2:191}
mi waan fi i go}
\z

Complementizers may follow subjects of embedded sentences if raising (or whatever you believe in if you don't believe in raising) has taken place, as in \textit{I want him to go}; however, as with perception-verb complements\is{complementation!of perception verbs}, a morpheme-for-morpheme translation of such sentences is ungrammatical:
% 11O 

\ea[*]{\label{ex:2:192}
mi waan am fi go}
\glt
\z

However, \textit{fi} in its \REF{ex:2:189} location is nondeletable:

\ea[*]{\label{ex:2:193}mi waan i go}
\z
This contrasts with the status of \textit{fi} in \REF{ex:2:188}, and suggests that while
\textit{fi} in \REF{ex:2:188} is a complementizer, \textit{fi} in \REF{ex:2:189} is a modal auxiliary.

There would seem to be two possible analyses of \REF{ex:2:188} and \REF{ex:2:189}. In the first, \textit{fi} is really a modal auxiliary in both cases -- in \REF{ex:2:189} for the reasons already given, and in \REF{ex:2:188} because \REF{ex:2:188} is simply derived from \REF{ex:2:194} by equi-deletion (obligatory since \REF{ex:2:194} is ungrammatical without it):

\ea[*]{\label{ex:2:194}mi waan mi fi go}\z

This first solution would be tempting but for \REF{ex:2:190}: modal auxiliaries do not normally delete without loss of meaning. We might then wish to choose the second analysis, which would derive \REF{ex:2:189} from \REF{ex:2:195} via obligatory complementizer deletion since without such deletion \REF{ex:2:195} is simply ungrammatical:

\ea[*]{\label{ex:2:195}mi waan fi i fi go}\z

However, there still lurks in the background the possibility that, despite \REF{ex:2:190} and \REF{ex:2:195}, the prepositional role of \textit{fi} might somehow be involved (cf. the claim by \citet{KoopmanEtAl1981} that HC\il{Haitian Creole} \textit{pu}, a close relative of \textit{fi} ``can introduce final complements, either infinitival \ldots~or tensed'').

To choose out of these three possibilities -- complementizer, modal verb, preposition -- we need to examine sentences in which
%\originalpage{111}
constituents are extracted from \textit{fi}-clauses. Let us look first of all at a sentence such as:

\ea\label{ex:2:196}
Where did he want to go? 
\z
This has the GC equivalent:

\ea\label{ex:2:197}
wisaid i waan fi go?
\z
We should note also that sentences like \REF{ex:2:198} have exact English equivalents:

\ea\label{ex:2:198}
wisaid i waan mi fi go?
\glt `Where did he want me to go?'
\z

In all three sentences \REF{ex:2:196}--\REF{ex:2:198}, a constituent, WH-place, is moved out\is{WH-movement|(} of an embedded S -- in the case of \REF{ex:2:198} presumably a tensed S.\is{movement rules!in GC|(}

If \textit{fi} in \REF{ex:2:198} is a modal verb, and no complementizer or preposition has been deleted, \REF{ex:2:198} must have an underlying structure (ignoring irrelevant detail) something like \REF{ex:2:199}:

\ea\label{ex:2:199}
\resizebox{.9\textwidth}{!}{
	{\normalfont
	\begin{tikzpicture}[baseline]
	% Levels 1 to 3
	\node at (0,0) (1S) {\=S};	
	\node [below left=\baselineskip and 6mm of 1S] (2COMP) {COMP};
	\node [below right=\baselineskip and 24mm of 1S] (2S) {S};
	\node [below left=\baselineskip and 12mm of 2S] (3NP) {NP};
	\node [below=\baselineskip of 2S] (3V) {V};
	\node [below right=\baselineskip and 24mm of 2S] (3S) {S};
	
	% Levels 4 and 5
	\node [below=\baselineskip of 3NP] (4i) {\strut i};
	\node [below=\baselineskip of 3V] (4waan) {\strut waan};
	\node [below left=\baselineskip and 6mm of 3S] (4NP) {\strut NP};
	\node [below=\baselineskip of 3S] (4Aux) {\strut Aux};
	\node [below right=\baselineskip and 6mm of 3S] (4V) {\strut V};
	\node [below right=\baselineskip and 24mm of 3S] (4NP2) {\strut NP};
	\node [below=\baselineskip of 4NP] (5mi) {\strut mi};
	\node [below=\baselineskip of 4Aux] (5fi) {\strut fi};
	\node [below=\baselineskip of 4V] (5go) {\strut go};
	\node [isosceles triangle, isosceles triangle apex angle=100, shape border rotate=90, minimum height=1.5em, draw, below=3mm of 4NP2.center] (5polygon) {};
	%	\path [on grid] let \p1 = (4NP2), \p2 = (5go.base) in node at (\x1,\y2) (6WH) {WH-place};
	\node [below=\baselineskip of 4NP2] (6WH) {\strut WH-place};	
	
	% Connections
	\draw (node cs:name=1S) -- (node cs:name=2COMP);
	\draw (node cs:name=1S) -- (node cs:name=2S);
	\draw (node cs:name=2S) -- (node cs:name=3NP);
	\draw (node cs:name=2S) -- (node cs:name=3V);
	\draw (node cs:name=2S) -- (node cs:name=3S);
	\draw (node cs:name=3NP) -- (node cs:name=4i);
	\draw (node cs:name=3V) -- (node cs:name=4waan);
	\draw (node cs:name=3S) -- (node cs:name=4NP);
	\draw (node cs:name=3S) -- (node cs:name=4Aux);
	\draw (node cs:name=3S) -- (node cs:name=4V);
	\draw (node cs:name=3S) -- (node cs:name=4NP2);
	\draw (node cs:name=4NP) -- (node cs:name=5mi);
	\draw (node cs:name=4Aux) -- (node cs:name=5fi);
	\draw (node cs:name=4V) -- (node cs:name=5go);
	
	\end{tikzpicture}
}
}
\z

%\originalpage{112}
\isi{WH-movement} would then move WH-place under the COMP node. However, such movement would violate the PIC\is{Propositional Island Constraint (PIC)}, since it moves WH out of a tensed S. Similarly, if a deleted prepositional \textit{fi} introduced \textit{mi fi go} in \REF{ex:2:198}, the latter sentence would have an underlying structure something like \REF{ex:2:200}. Example \REF{ex:2:200} would involve a rule which would expand PP as either P NP or P S; just such a rule is proposed for HC\il{Haitian Creole} by \citet{KoopmanEtAl1981}, on rather similar evidence, involving prepositional \textit{pu} and its tensed complements. However, movement of the WH-constituent from the right-hand NP node to COMP would again involve violation of the PIC.\is{Propositional Island Constraint (PIC)}

If, on the other hand, \textit{fi} is a complementizer, no violation need ensue. In this case, the underlying structure of \REF{ex:2:198} would be something like \REF{ex:2:201}.

\ea\label{ex:2:200}
\resizebox{.9\textwidth}{!}{
	{\normalfont
	\begin{tikzpicture}[baseline]
	% Levels 1 to 3
	\node at (0,0) (1S) {\=S};	
	\node [below left=\baselineskip and 12mm of 1S.south] (2COMP) {COMP};
	\node [below right=\baselineskip and 6mm of 1S.south] (2S) {S};
	\node [below left=\baselineskip and 12mm of 2S] (3NP) {NP};
	\node [below=\baselineskip of 2S] (3V) {V};
	\node [below right=\baselineskip and 12mm of 2S] (3PP) {PP};
	
	% Levels 4 and 5
	\node [below=\baselineskip of 3NP] (4i) {\strut i};
	\node [below=\baselineskip of 3V] (4waan) {\strut waan};
	\node [below left=4\baselineskip and 3mm of 3PP] (4P) {\strut P};
	\node [below right=4\baselineskip and 24mm of 3PP] (4S) {\strut S};
	\node [below=\baselineskip of 4P] (5fi) {\strut fi};
	\node [below left=4\baselineskip and 18mm of 4S] (5NP) {\strut NP};
	\node [below left=4\baselineskip and 2mm of 4S] (5Aux) {\strut Aux};
	\node [below right=4\baselineskip and 2mm of 4S] (5V) {\strut V};
	\node [below right=4\baselineskip and 18mm of 4S] (5NP2) {\strut NP};
	
	\node [below=\baselineskip of 5NP] (6mi) {\strut mi};
	\node [below=\baselineskip of 5Aux] (6fi) {\strut fi};
	\node [below=\baselineskip of 5V] (6go) {\strut go};
	\node [isosceles triangle, isosceles triangle apex angle=100, shape border rotate=90, minimum height=1.5em, draw, below=3mm of 5NP2.center] (6polygon) {};
	%	\path [on grid] let \p1 = (5NP2), \p2 = (6go.base) in node at (\x1,\y2) (7WH) {WH-place}; 
	\node [below=\baselineskip of 5NP2] (7WH) {\strut WH-place};	
	
	% Connections
	\draw (node cs:name=1S, anchor=south) -- (node cs:name=2COMP, anchor=north);
	\draw (node cs:name=1S, anchor=south) -- (node cs:name=2S, anchor=north);
	\draw (node cs:name=2S, anchor=south) -- (node cs:name=3NP);
	\draw (node cs:name=2S, anchor=south) -- (node cs:name=3V);
	\draw (node cs:name=2S, anchor=south) -- (node cs:name=3PP);
	\draw (node cs:name=3NP) -- (node cs:name=4i);
	\draw (node cs:name=3V) -- (node cs:name=4waan);
	\draw (3PP.south) -- (4P.north);
	\draw (3PP.south) -- (4S.north);
	\draw (4P.south) -- (5fi.north);
	\draw (4S.south) -- (node cs:name=5NP);
	\draw (4S.south) -- (node cs:name=5Aux);
	\draw (4S.south) -- (node cs:name=5V);
	\draw (4S.south) -- (node cs:name=5NP2);
	\draw (node cs:name=5NP) -- (node cs:name=6mi);
	\draw (node cs:name=5Aux) -- (node cs:name=6fi);
	\draw (node cs:name=5V) -- (node cs:name=6go);
	
	% \draw [red] (6mi.base) -- (7WH.base); % This line checks whether basline alignment is correct.
	\end{tikzpicture}
}}
\z

% CREOLE 113

\ea\label{ex:2:201}
\resizebox{.9\textwidth}{!}{
{\normalfont 
	\begin{tikzpicture}[baseline]
% Levels 1 to 3
\node at (0,0) (1S) {\=S};	
\node [below left=\baselineskip and 6mm of 1S] (2COMP) {\strut COMP};
\node [below right=\baselineskip and 24mm of 1S] (2S) {\strut S};
\node [below left=\baselineskip and 12mm of 2S] (3NP) {\strut NP};
\node [below=\baselineskip of 2S] (3V) {\strut V};
\node [below right=\baselineskip and 24mm of 2S] (3S) {\strut \=S};
\node [below=\baselineskip of 2COMP] (3space) {( )};

% Levels 4 and 5
\node [below=\baselineskip of 3NP] (4i) {\strut i};
\node [below=\baselineskip of 3V] (4waan) {\strut waan};
\node [below left=\baselineskip and 3mm of 3S] (4comp) {\strut COMP};
\node [below right=\baselineskip and 12mm of 3S] (4S) {\strut S};

\node [below=\baselineskip of 4comp] (5fi) {(fi)};
\node [below left=\baselineskip and 6mm of 4S] (5NP) {\strut NP};
\node [below=\baselineskip of 4S] (5Aux) {\strut Aux};
\node [below right=\baselineskip and 6mm of 4S] (5V) {\strut V};
\node [below right=\baselineskip and 24mm of 4S] (5NP2) {\strut NP};
\node [below=\baselineskip of 5NP] (6mi) {\strut mi};
\node [below=\baselineskip of 5Aux] (6fi) {\strut fi};
\node [below=\baselineskip of 5V] (6go) {\strut go};
\node [isosceles triangle, isosceles triangle apex angle=100, shape border rotate=90, minimum height=1.5em, draw, below=3mm of 5NP2.center] (6polygon) {};
%	\path [on grid] let \p1 = (5NP2), \p2 = (6go.base) in node at (\x1,\y2) (6WH) {WH-place};
\node [below=\baselineskip of 5NP2] (6WH) {\strut WH-place};	

% Connections
\draw (node cs:name=1S, anchor=south) -- (node cs:name=2COMP);
\draw (node cs:name=1S, anchor=south) -- (node cs:name=2S);
\draw (node cs:name=2S, anchor=south) -- (node cs:name=3NP);
\draw (node cs:name=2S, anchor=south) -- (node cs:name=3V);
\draw (node cs:name=2S, anchor=south) -- (node cs:name=3S);
\draw (node cs:name=3NP) -- (node cs:name=4i);
\draw (node cs:name=3V) -- (node cs:name=4waan);
\draw (node cs:name=3S, anchor=south) -- (node cs:name=4comp);
\draw (node cs:name=3S, anchor=south) -- (node cs:name=4S);
\draw (node cs:name=4S, anchor=south) -- (node cs:name=5NP);
\draw (node cs:name=4S, anchor=south) -- (node cs:name=5Aux);
\draw (node cs:name=4S, anchor=south) -- (node cs:name=5V);
\draw (node cs:name=4S, anchor=south) -- (node cs:name=5NP2);
\draw (node cs:name=5NP) -- (node cs:name=6mi);
\draw (node cs:name=5Aux) -- (node cs:name=6fi);
\draw (node cs:name=5V) -- (node cs:name=6go);
\draw (node cs:name=2COMP) -- (node cs:name=3space);
\draw (node cs:name=4comp) -- (node cs:name=5fi);

\draw (6WH.south) edge [ dashed, -{Stealth[]}, bend left=60]  (5fi.south);
\draw (5fi.south west) edge [ dashed, -{Stealth[]}, bend left=50] (3space.south);

\end{tikzpicture}}}
\z

Complementizer deletion, optional in \REF{ex:2:188}, is, as we have seen, obliga\-tory in \REF{ex:2:189} and \REF{ex:2:198}. However, once \textit{fi} is deleted, we have an empty COMP node (the one dominating the \textit{fi} in parentheses in \REF{ex:2:201}). \citet{Chomsky1977} has argued that WH-movement, being a cyclic rule, can move constituents from COMP to COMP, thus forming a ``bridge'' over the barrier of the PIC. In \REF{ex:2:201} -- but not in \REF{ex:2:199} or \REF{ex:2:200} -- there is a lower COMP node to which WH can be moved on the first cycle, allowing the second cycle to move it to the higher COMP node, as indicated by the dotted line in \REF{ex:2:201}. Thus, in contrast with \REF{ex:2:199} and \REF{ex:2:200}, WH-movement in \REF{ex:2:201} does not violate the PIC.\is{Propositional Island Constraint (PIC)}

If the foregoing analysis is correct, GC does contain an \=S structure in some complements. However, there was no motivation in \REF{ex:2:201} for assuming \=S to be dominated by NP, so we have yet to prove condi\-tion (\ref{GCrequirementcondition}) (Page~\pageref{GCrequirementcondition}).

In order to prove condition (\ref{GCrequirementcondition}), we need another set of \textit{fi}-sentences. Unlike perception-verb complements\is{complementation!of perception verbs}, which cannot have zero subjects (\textit{mi hia dem a sing} versus *~\textit{mi hia a sing}), \textit{fi}-clause complements can:

\ea\label{ex:2:202}
yu gafi kraas di riba fi miit tong
\glt `You have to cross the river in order to get to town'
\z

%\originalpage{114}

\ea\label{ex:2:203}
na bin iizi fi kech taiga
\glt `It wasn't easy to catch a jaguar'
\z
In both cases, the \textit{fi}-clause can be moved:

\ea\label{ex:2:204}
 fi miit tong yu gafi kraas di riba
\glt `To get to town you have to cross the river'
\z

\ea\label{ex:2:205}
fi kech taiga na bin iizi
\glt `To catch a jaguar wasn't easy'
\z
However, since both \citet{Woolford1979} and \citet{KoopmanEtAl1981} give arguments that \ili{Tok Pisin} and Haitian Creole, respectively, have homophonous pairs of complementizers and prepositions (TP \textit{long}, HC\il{Haitian Creole} \textit{pu}), we cannot automatically assume that \textit{fi} is a complementizer in \REF{ex:2:202}--\REF{ex:2:205} just because it was in \REF{ex:2:198}. To show this, we have to question the NP in \REF{ex:2:103}:

\ea\label{ex:2:206}
a wa na bin iizi fi kech ?
\glt `What was it that wasn't easy to catch?'
\z
If \textit{fi} were a preposition in \REF{ex:2:203}, \REF{ex:2:203} and \REF{ex:2:206} would have the underlying structure of \REF{ex:2:200}; but we saw in our analysis of \REF{ex:2:200} that if WH-movement were applied to that structure, a violation of the PIC\is{Propositional Island Constraint (PIC)} would result. To avoid such violation, \textit{fi} would have to be a complementizer, and \REF{ex:2:203} and \REF{ex:2:206} would have to have the underlying structure of \REF{ex:2:201}. Since \REF{ex:2:206} is grammatical, \textit{fi} must be a complementizer.

If this is the case, \REF{ex:2:205} must contain an \=S directly dominated by NP, as in \REF{ex:2:207}:

%\originalpage{115}
\ea\label{ex:2:207}
%\resizebox{.9\textwidth}{!}{
	{\normalfont
		\begin{tikzpicture}[baseline]
		
		\node at (0,0) (1S) {S};
		\node [below right=\baselineskip and 24mm of 1S] (2V) {V\textsubscript{adj}};
		\node [below=\baselineskip of 1S] (2Aux) {Aux};
		\node [below left=\baselineskip and 24mm of 1S] (2NP) {NP};
		
		\node [below=\baselineskip of 2NP] (3S) {\strut\=S};
		\node [below left=\baselineskip and .25mm of 2Aux] (3na) {\strut na};
		\node [below right=\baselineskip  and .25mm of 2Aux] (3bin) {\strut bin};
		\node [below=\baselineskip of 2V] (3iizi) {iizi};
		
		\node [below left=\baselineskip and 12mm of 3S] (4comp) {COMP};
		\node [below right=\baselineskip and 6mm of 3S] (4S) {S};
		
		\node [below=\baselineskip of 4comp] (5fi) {\strut fi};
		\node [below left=\baselineskip and 6mm of 4S] (5NP) {\strut NP};
		\node [below=\baselineskip of 4S] (5V) {\strut V};
		\node [below right=\baselineskip and 6mm of 4S] (5NP2) {\strut NP};
		
		\node [below=\baselineskip of 5NP] (6emptyset) {\strut $\varnothing$}; %from amssymb
		\node [below=\baselineskip of 5V] (6kech) {\strut kech};
		\node [below=\baselineskip of 5NP2] (6taiga) {\strut taiga};
		
		\draw (2NP.north) -- (1S.south) -- (2Aux.north); 
		\draw (1S.south) -- (2V.north); 
		\draw (3na.north) -- (2Aux.south) -- (3bin.north);
		\draw (2NP.south) -- (3S);
		\draw (2V.south) -- (3iizi);
		\draw (4comp.north) -- (3S.south) -- (4S.north);
		\draw (4comp.south) -- (5fi);
		\draw (5NP.north) -- (4S.south) -- (5NP2.north);
		\draw (4S.south) -- (5V.north);
		\draw (5NP.south) -- (6emptyset.north);
		\draw (5NP2.south) -- (6taiga.north);
		\draw (5V.south) -- (6kech.north);	
		
		\end{tikzpicture}
	}
%	}
	\z

\noindent Thus, the exclusion of \textit{se}-clauses from subject position, which we noted in discussing \REF{ex:2:184} above, cannot be due to the absence of a rule rewriting NP as \=S (COMP S). The ungrammaticality of \REF{ex:2:184} must, therefore, result from the fact that \textit{se} is not a complementizer, and consequently cannot be inserted in structures such as \REF{ex:2:207}.

We can now turn to condition (\ref{GCrequirementcondition2}) (Page~\pageref{GCrequirementcondition2}). The fact that \textit{se} cannot be a complementizer, suggested by the foregoing analysis, would of course also make it impossible for \textit{se}-clauses to be generated in \=S complements. But let us assume for the moment that \textit{se} is a complementizer. If this were so, a sentence such as \REF{ex:2:208} below would have a structure similar to that of \REF{ex:2:201}:

\ea\label{ex:2:208}
dem taak se i de a tong
\glt `They said that he was in town'
\z
In other words, the complement \=S would contain a COMP node which would permit COMP-to-COMP \isi{WH-movement|)} and hence permit questioning of the rightmost NP. We would then have to predict that \REF{ex:2:209}  would be grammatical:

%\originalpage{116}
\ea[*]{\label{ex:2:209}
wisaid dem taak se i de? \\
\glt `Where did they say he was?'}
\z
Unfortunately, it is not. We can therefore only conclude that \textit{se} is something other than a complementizer.

The third quasi-complementizer, \textit{go}, is even more restricted than \textit{se}. Like \textit{se}, but unlike \textit{fi}, it cannot be preposed:

\ea[ ]{\label{ex:2:210}
i tek i gon fi shuut taiga
\glt `He took his gun to shoot a jaguar (but did not necessarily do so)'}
\z

\ea[ ]{\label{ex:2:211}
i tek i gon go shuut taiga
\glt `He took his gun to shoot a jaguar (and did shoot one)'}
\z

\ea[ ]{\label{ex:2:212}
fi shuut taiga i a tek i gon
\glt `For shooting jaguars he used to take his gun'\footnotemark}
\z
\footnotetext{It is interesting to note that while \textit{fi}-clauses in complement position can refer to one-time actions (as in \REF{ex:2:210}), and in consequence the higher verb can take punctual marking, preposed \textit{fi}-clauses can refer only to habitual actions, and in consequence the higher verb must take nonpunctual marking. At the moment I have no idea why this is so.}

\ea[*]{\label{ex:2:213}
go shuut taiga i a tek i gon}
\z
Unlike both \textit{fi} and \textit{se}, \textit{go} cannot occur with adjectival verbs:

\ea[ ]{\label{ex:2:214}
 mi glad fi sii yu\\
\glt `I'm glad to see you'}
\z

\ea[ ]{\label{ex:2:215}
mi glad se yu kom \\
\glt `I'm glad you came'}
\z

\ea[*]{\label{ex:2:216}
mi glad go sii yu}
\z
As with \textit{se} (but not \textit{fi}), complement constituents cannot be extracted:


\ea[ ]{\label{ex:2:217}
i gaan a tong go sii dakta\\
\glt `He's gone to town to see the doctor'}
\z

\ea[*]{\label{ex:2:218}
a hu i gaan a tong go sii?\\
\glt `Who did he go to town to see?'}
\z

\ea[*]{\label{ex:2:219}
di dakta we i gaan a tong go sii de bad an aal \\
\glt `The doctor he went to town to see is sick too'}
\z
%\originalpage{117}
We can assume, as with \textit{se}, that extraction is blocked because COMP-to-COMP movement is impossible; therefore, \textit{go} is not a complementizer either.

The claim that \textit{se} is a serial verb\is{verb serialization|(} in Krio has been argued strongly by \citet{Larimore1976}, but since there are some minor differences between the grammars of Krio and GC, not all her arguments apply to the latter language. I shall assume without further argument that \textit{se} and \textit{go} are both serial verbs. If this assumption is correct, then \textit{se} and \textit{go}, if not \textit{fi}, really belong with the verbs that we will discuss in the next section on serialization. But because \textit{fi} may be a complementizer synchronically, it by no means necessarily follows that \textit{fi} always was a complementizer.

\citet[Example 9]{Washabaugh1979} cites the following sentence: \footnote{Washabaugh's analysis of \textit{fi} differs radically from that made in the present chapter, although there is no reason to suppose that the facts of PIC differ significantly from those of GC. However, since I have dealt with that analysis in \citet{Bickerton1980}, I will not repeat my criticisms of it here.}\is{movement rules!in GC|)}

\ea\label{ex:2:220}
ah waan di rien kom fi ah don go huoam
\glt `I want the rain to come so that I won't have to go home'
\z
This sentence, from \ili{Providence Island Creole} (PIC), a variety similar in many ways to GC, is of a type claimed by Washabaugh to be ``rare in most contemporary varieties of [Caribbean English creoles] , but \ldots~frequent enough in older texts.'' Washabaugh does not analyze this sentence, so we do not know whether the following sentence would be rejected by Providence Islanders:\is{Guyanese Creole!complementation}

\ea[?]{\label{ex:2:221}
wisaid ah waan di rien kom fi ah don go?}
\z
It would almost certainly be rejected by speakers of other Caribbean English creoles.

The most likely structure of \REF{ex:2:220} would be one similar to that of \REF{ex:2:185}, reproduced here for convenience: 

\begin{exe}
\exr{ex:2:185} \textit{dem gaan mek mi glad}
\end{exe}

\noindent That structure is illustrated in \REF{ex:2:222}. Here, \textit{mek} functions rather like the abstract verb \textsc{cause} once posited by generative semanticists. In \REF{ex:2:220}, \textit{fi} would have a meaning something like \textsc{should cause}, with \textit{di rien kom} as its subject and \textit{ah don go huoam} as its object. On present evidence we cannot determine for sure whether \textit{fi} was once exclusively a serial verb. However, it seems reasonable to suppose that in GC and other creoles, serial verbs may be turning into complementizers. Such a process certainly exists in some West African languages \citep{Lord1976}\is{African languages}, and we shall shortly examine evidence from \ili{Sranan} which indicates that serial verbs there may be undergoing a rather similar kind of reanalysis.

%\originalpage{118}

\ea\label{ex:2:222}
%\resizebox{.9\textwidth}{!}{
	{\normalfont
	\begin{tikzpicture}[baseline]
	
	\node at (0,0) (1S) {S};
	\node [below right=\baselineskip and 24mm of 1S] (2NP2) {NP};
	\node [below left=\baselineskip and 24mm of 1S] (2NP) {NP};
	\node [below=\baselineskip of 1S] (2V) {V};
	\node [below=\baselineskip of 2NP] (3S) {S};
	\node [below=\baselineskip of 2NP2] (3S2) {S};
	\node [below=\baselineskip of 2V] (3mek) {mek};
	\node [below left=\baselineskip and 6mm of 3S] (4NP) {NP}; \node [below right=\baselineskip and 6mm of 3S] (4V) {V};
	\node [below left=\baselineskip and 6mm of 3S2] (4NP2) {NP}; \node [below right=\baselineskip and 6mm of 3S2] (4V2) {V};
	\node [below=\baselineskip of 4NP] (5dem) {dem}; \node [below=\baselineskip of 4V] (5gaan) {gaan}; 
	\node [below=\baselineskip of 4NP2] (5mi) {mi}; \node [below=\baselineskip of 4V2] (5glad) {glad}; 
	\draw (2NP.north) -- (1S.south) -- (2V.north); \draw (1S.south) -- (2NP2.north);
	\draw (3S) -- (2NP); \draw (3S2) -- (2NP2); \draw (2V) -- (3mek);
	\draw (4NP) -- (3S.south) -- (4V); \draw (4NP2) -- (3S2.south) -- (4V2);
	\draw (5dem) -- (4NP); \draw (5gaan) -- (4V); \draw (5mi) -- (4NP2); \draw (5glad) -- (4V2);
	\end{tikzpicture}	
%}
}
\z

The boundaries of serial verb constructions are not easy to define, nor is it easy (or perhaps even desirable) to distinguish them from other superficially similar constructions such as ``verb chains'' \citep{Forman1972}. Here, I shall simply concern myself with those serial constructions which are equivalent to multi-case sentences, i.e., which mark oblique cases (dative, instrumental, etc.) with verbs rather than with prepositions or with other types of formal devices. Examples of such structures would include:

\ea\label{ex:2:223}
Directionals:\is{directionals}\\
\ili{\langSR}{}{}\\
\gll a \emph{waka} \emph{go} a wosu\\
he walk go to house \\
\glt `He walked home'
\z

%\originalpage{119}
\ea\label{ex:2:224}
Benefactives:\\
\ili{\langGU}{}{}\\
\gll li \emph{pote} sa \emph{bay} mo\\
he bring that give me\\
\glt `He brought that for me'
\z

\ea\label{ex:2:225}
Datives:\\
\ili{\langST}{}{}\\
\gll \emph{e} \emph{fa} \emph{da} ine \\
he talk give them \\
\glt `He talked to them'
\z

\ea\label{ex:2:226}
Instrumentals:\\
\ili{Djuka}{}{}\\
\gll a \emph{teke} nefi \emph{koti} a meti \\
he take knife cut the meat\\
\glt `He cut the meat with a knife'
\z

Sentences such as \REF{ex:2:223}--\REF{ex:2:226} are by no means always the only ways in which those creoles that have them can express case relations. Alongside \REF{ex:2:226}, \ili{Djuka} has \REF{ex:2:227}:

\ea\label{ex:2:227}
\gll a koti a meti anga nefi\\
he cut the meat with knife\\
\glt `He cut the meat with a knife' 
\z
According to \citet{Huttar1975}, sentences like \REF{ex:2:227} occur more fre\-quently in speech than sentences like \REF{ex:2:226}.

Which of such pairs represents the most conservative creole level? Serial verbs form a more marked means of expressing case rela\-tions than do prepositions. It is, therefore, relatively unlikely that a language which already had prepositions to mark case would develop serial verbs (except in certain circumstances which could hardly apply to creoles and which will be discussed later). On the other hand, it is relatively likely that a language which originally had only serial verbs as a case-marking device would subsequently develop prepositions, either by a type of reanalysis already attested for West African languages \citep{Lord1976}\is{African languages}, or by direct borrowing from a high-prestige language with which it was in contact (probably the case in any creole that has undergone even a relatively small amount of decreolization). We are therefore justified in assuming that serial verb constructions represent extremely conservative varieties of those creoles in which they are found.
%\originalpage{120}

Serial verbs are usually interpreted as the result of African substratum influence\is{substratum influence|(} on creoles, but creolists seldom if ever ask how those West African languages which have serial verbs (by no means all of them) happen to have come by them. Despite lip service to linguistic equality, a dual standard is still applied to creoles: if a creole has a feature, it must have borrowed it; but if a noncreole language has the same feature, it is assumed to be an independent innovation -- at least in the absence of clear evidence to the contrary. In fact, I would claim that creoles and West African languages invented verb serialization independently, but for slightly different reasons.

Wherever serial verbs are found outside creoles, a change in \isi{word order} is always involved (see \citet{LiEtAl1974} for Chinese; \citet{Givón1974} and \citet{Hyman1974} for West African languages\is{African languages}; \citet{Bradshaw1979} for Austronesian languages in New Guinea). Sometimes the change may be contact-influenced, as in New Guinea; sometimes it may come from purely language-internal developments, as with the \ili{Kwa languages} of West Africa. Precise explanations of why SOV-SVO (West Africa) or SVO-SOV (New Guinea) changes involve serialization are still controversial. \citet{Givón1974} suggests that serialization results from the decay of post positional case marking, an explanation chal\-lenged by \citet{Hyman1974}; \citet{Bradshaw1979} suggests serialization eases parsing problems in a period of transition by generating sentences that can be parsed as either SVO or SOV without any semantic confu\-sion (we shall return to this point at the end of \chapref{ch:4}). However, there seems to be no serious ground for doubting that serialization and word-order change are involved with one another in some kind of way.

Word-order change cannot have been a factor in creolization since most of the languages in contact, as well as the resultant creoles, have been SVO. However, the problem that word-order change creates --
%\originalpage{121}
that of unambiguously identifying case roles while the change is under way -- must have been a problem in creolization too, if we assume what must almost certainly have been the case in at least some pidgins, i.e., that the latter did not contain (or at least did not contain a full range of) prepositions. Without prepositions and without inflectional morphology, how else could oblique cases be distinguished if not by serial verbs?

More specific doubts about the viability of substratal accounts, as well as the seeds of an explanation as to why creoles differ so much in the extent to which they exhibit serialization, are suggested by the following data on the Surinam creoles (\ili{Djuka}, \ili{Sranan}, \ili{Saramaccan}). In these languages, instrumental constructions (as expressed via equiva\-lents of `He cut the meat with a knife') have the following range:

\ea\label{ex:2:228}
\ili{\langDJ}{}{}\\
a koti a meti anga nefi
\z

\ea\label{ex:2:229}
\ili{\langSR}{}{}\\
a koti a meti nanga nefi
\z

\ea\label{ex:2:230}
\ili{\langSA}{}{}\\
a koti di gbamba ku faka
\z

\ea\label{ex:2:231}
\ili{\langDJ}{}{}\\
a teke nefi koti a meti
\z

\ea\label{ex:2:232}
\ili{\langSR}{}{}\\
a teki nefi koti a meti
\z

\ea\label{ex:2:233}
\ili{\langSA}{}{}\\
\textnormal{?} a tei faka koti di gbamba
\z

\noindent A sentence similar to \REF{ex:2:233} -- \textit{a tei di pau naki en}, lit., `He took the stick hit it', i.e., `He hit it with the stick' -- is cited in \citet{GrimesEtAl1970} but footnoted to the effect that the authors have since become highly doubtful as to its status in SA. In \citet{Glock1972}, which deals explicitly with case phenomena, there is no mention of sentences like \REF{ex:2:233}, although sentences like \REF{ex:2:230}, as well as serialization of other cases, are cited and discussed; nor is SA credited with \textit{tei}-serialization in \citet{JansenEtAl1978}, although, again, there is no explicit discussion. It is thus impossible to tell whether Saramaccan has this kind of serialization, although the present balance of evidence seems to be against it.

Saramaccan is well known as being, among the three Surinam creoles (or, for that matter, among all the \ili{Caribbean creoles}), the one which best preserves African lexical and phonological characteristics
%\originalpage{122}
(note, in the preceding examples, \textit{gbamba} `meat', a word of presumably African origin which preserves the coarticulated and prenasalized stops characteristic of many West African\is{African languages} languages but of no other creoles, compared with DJ and SR\il{Sranan|(} \textit{meti} from Eng. \textit{meat}). This being so, and if serial constructions also reflect African influence, one would expect to find that SA had more of such constructions than DJ and SR, rather than the reverse.

But while there is no explanation for the pattern in terms of substrate influence\is{substratum influence|)}, an explanation can be provided in terms of interaction between the antecedent pidgin and its superstrate. It seems reasonable to assume that if a creole can acquire prepositions from its antecedent pidgin\is{pidgin-creole cycle|(} (as HCE did), it will not need to develop serial verbs for case marking. The only question is, why should some antecedent pidgins acquire prepositions while others do not?

Clearly, one factor is population balance, while another factor is the type of social structure; between them, these will determine the accessibility of the superstrate language and hence help to deter\-mine how many superstrate items the pidgin will absorb\is{superstrate influence}. However, these are by no means the only factors involved. Other things, including social conditions, being equal, structural differences between super\-strate features may determine whether a pidgin will or will not absorb these features.

\is{English!phonology|(}For a superstrate feature to be accessible to a pidgin, that feature must be more or less unambiguous with respect to meaning, more or less free from mutation with respect to phonological structure, and as close as possible to the canonical form of CV(CV). The superstrate prepositions of instrumentality available to the three languages were: for SR and DJ, Eng. \textit{with}; for SA, Eng. \textit{with} and Pg. \textit{com} -- phonetically, [k\~o] or [k\~u] in many contemporary dialects. The former, with its initial semivowel (a marked segment) and final labiodental fricative (an extremely marked segment), is remote from the canonical pattern; the latter, in the form in which it is perhaps most frequently realized, fits it exactly. The relative difficulty of acquiring \textit{com} and \textit{with} may best be pictured if the reader imagines that his linguistic competence
%\originalpage{123}
is limited to African languages\is{African languages} and then attempts to segment the following synonymous utterances:

\ea\label{ex:2:234}
vai com aquele homem
\z

\ea\label{ex:2:235}
go with that man
\z

Word boundaries in the Portuguese utterance are fairly un\-ambiguously marked; in the English one\is{English!phonology|)}, it would be hard to determine where the verb ended and the preposition began, or where the preposi\-tion ended and the demonstrative began -- or even to be sure that there was a preposition there at all. It is hardly surprising, therefore, that neither Sranan nor Djuka could absorb \textit{with}, but had to adopt a serializing device in order to express instrumentality, and that only later did they develop their own preposition, \textit{(n)anga}, of uncertain origin; whereas, on the other hand, Saramaccan, like all \ili{Portuguese creoles}, easily acquired \textit{ku} and thus did not need a serial construction for instrumentality.

The underlying structure of serial-verb constructions has been a subject of some controversy (see \citet{Williams1971,Williams1975}, \citet{Roberts1975}, \citet{Voorhoeve1975}, \citet{JansenEtAl1978} for some differing views on this subject). I suspect that varying analyses are due at least in part to inherent conflicts in the data, and that these conflicts, in turn, are due to ongoing developments in creoles which have the result of complicating the original creole syntax and intro\-ducing categories which formed no part of the original grammar. If we are to understand what creoles are, and make comparisons between particular creoles without allowing ourselves to be misled by sub\-sequent and irrelevant accretions, we must -- to cite the words of \citeauthor{KoopmanEtAl1981}, on which I could not hope to improve -- ``restrict the notion of syntactic expansion to changes leading to the acquisition of features that are part of core grammar up to the time of creolization and to consider the emergence of other features as regular cases of syntactic change''\is{linguistic change} \citep[218]{KoopmanEtAl1981}.

\citeauthor{KoopmanEtAl1981} assume, as I do, that pidgins begin with
%\originalpage{124}
nouns, verbs, and very little else. They assume that VP is a pidgin category, but do not defend it as such. For reasons given in the discussion of GC movement\is{movement rules!in Sranan|(} rules above, I do not think VP is a category in most early creoles, although of course creoles may acquire it later, either through decreolization or regular syntactic change (reanalysis). The only rule creoles would then require to generate any of the comple\-ment types, serial or other, discussed so far (with the exception of \textit{fi}-clauses) would be \REF{ex:2:236}:\is{rules!of early creoles}

\ea\label{ex:2:236}
 S → NP Aux V (NP) (S)
\z
If in fact VPs were developed initially, two rules would be required:

\ea\label{ex:2:237}
S → NP Aux VP
\z

\ea\label{ex:2:238}
 VP → V (NP) (S)
\z

Not a great deal depends on which analysis is correct; therefore, since crucial evidence will be taken from Sranan, and since Sranan scholars generally assume a VP (although, once again, without explicit discussion), I shall accept \REF{ex:2:237} and \REF{ex:2:238} as specifying the earliest and most basic level of creole syntax, while continuing to suspect that \REF{ex:2:236} may be a more accurate representation of it.\is{pidgin-creole cycle|)}

The evidence consists of judgments by native speakers of Sranan cited in \citet{JansenEtAl1978}. The parenthesized asterisks before certain sentences indicate that while some speakers found them grammatical, others did not.

\ea{\label{ex:2:239}
Kofi teki a nefi koti a brede
\glt `Kofi cut the bread with the knife'}
\z

\ea{\label{ex:2:240}
san Kofi teki koti a brede?
\glt `What did Kofi cut the bread with?'}
\z

\ea{\label{ex:2:241}%
(*)san Kofi teki a nefi koti?
\glt `What did Kofi cut with the knife?'}
\z

\ea{\label{ex:2:242}
na a nefi Kofi teki koti a brede.
\glt `It was the knife that Kofi cut the bread with'}
\z

\ea{\label{ex:2:243}%
(*)na a brede Kofi teki a nefi koti
\glt `It was the bread that Kofi cut with the knife'}
\z
% CREOLE 125

\ea{\label{ex:2:244}
na teki Kofi teki a nefi koti a brede\\
\glt `\textit{With} the knife, that's how Kofi cut the bread'}
\z

\ea{%
(*)na koti Kofi teki a nefi koti a brede
\glt `\textit{Cut} it, that's what Kofi did to the bread with the knife'}
\label{ex:2:245}\z
In \REF{ex:2:241} and \REF{ex:2:243}, some speakers can extract the most deeply embed\-ded NP, while others cannot. In \REF{ex:2:244}, a rule similar to the GC verb-focusing rule copies the higher of the two verbs, for all speakers; but in \REF{ex:2:245}, while some speakers can copy the more deeply embedded verb, others cannot.\is{Guyanese Creole!movement rules}

These facts can be accounted for if we assume that the two sets of speakers have different types of underlying structures for these sentences, as represented in \REF{ex:2:246} and \REF{ex:2:247}, respectively:

\ea\label{ex:2:246}
%\resizebox{.9\textwidth}{!}{
	{\normalfont
	\begin{tikzpicture}[baseline]
	
	\node at (0,0) (1S) {S};
	\node [below left=\baselineskip and 12mm of 1S] (2NP) {NP}; \node [below right=\baselineskip and 12mm of 1S] (2VP) {VP};
	
	\node [below=\baselineskip of 2NP] (3Kofi) {Kofi}; \node [below left=\baselineskip and 12mm of 2VP] (3V) {V};
	\node [below=\baselineskip of 2VP] (3NP) {NP}; \node [below right=\baselineskip and 24mm of 2VP] (3S) {S};
	\node [below=\baselineskip of 3V] (4teki) {teki}; \node [below=\baselineskip of 3NP] (4nefi) {a nefi};
	\node [below left=\baselineskip and 6mm of 3S] (4NP) {NP}; \node [below right=\baselineskip and 6mm of 3S] (4VP) {VP};
	
	\node [below=\baselineskip of 4NP] (5Kofi) {(Kofi)}; \node [below left=\baselineskip and 3mm of 4VP] (5V) {V}; \node [below right=\baselineskip and 3mm of 4VP] (5NP) {NP};
	\node [below=\baselineskip of 5V] (6koti) {koti}; \node [below=\baselineskip of 5NP] (6brede) {a brede};
	
	\draw (2NP) -- (1S.south) -- (2VP);
	\draw (3Kofi) -- (2NP); \draw (3V) -- (2VP.south) -- (3NP); \draw (2VP.south) -- (3S); \draw (4teki)--(3V); \draw (4nefi)--(3NP); \draw (4NP) -- (3S.south) -- (4VP); 
	\draw (5Kofi)--(4NP); \draw (5V) -- (4VP.south) -- (5NP); \draw (6koti) -- (5V); \draw (6brede) -- (5NP);
	
	\end{tikzpicture}
}
%}

\z
%\originalpage{126}

\ea\label{ex:2:247}
%\resizebox{.9\textwidth}{!}{
	{\normalfont
	\begin{tikzpicture}[baseline]
	\node at (0,0) (1S) {S};
	\node [below left=\baselineskip and 12mm of 1S] (2NP) {NP}; \node [below right=\baselineskip and 12mm of 1S] (2VP) {VP};
	\node [below=\baselineskip of 2NP] (3Kofi) {Kofi}; \node [below left=\baselineskip and 12mm of 2VP] (3V) {V};
	\node [below=\baselineskip of 2VP] (3NP) {NP}; \node [below right=\baselineskip and 24mm of 2VP] (3VP) {VP};
	\node [below=\baselineskip of 3V] (4teki) {teki}; \node [below=\baselineskip of 3NP] (4nefi) {a nefi};
	\node [below left=\baselineskip and 3mm of 3VP] (5V) {V}; \node [below right=\baselineskip and 3mm of 3VP] (5NP) {NP};
	\node [below=\baselineskip of 5V] (6koti) {koti}; \node [below=\baselineskip of 5NP] (6brede) {a brede};
	
	\draw (2NP) -- (1S.south) -- (2VP);
	\draw (3Kofi) -- (2NP); \draw (3V) -- (2VP.south) -- (3NP); \draw (2VP.south) -- (3S); \draw (4teki)--(3V); \draw (4nefi)--(3NP); \draw (4NP) -- (3S.south) -- (4VP); 
	\draw (6koti)--(5V); \draw (6brede)--(5NP);
	
	\end{tikzpicture}		
%}
}

\z

Speakers who rejected \REF{ex:2:241}, \REF{ex:2:243}, and \REF{ex:2:245} would have \REF{ex:2:246} as an underlying structure. Here, both the \is{Specified Subject Condition (SSC)}Specified Subject Condition (SSC: \citealt{Chomsky1973}) and the PIC would block movement out of sites dominated by the lower S. Speakers who accepted these three sentences would have \REF{ex:2:247} as an underlying structure. Since \REF{ex:2:247} contains no tensed S, specified subject, or bounding nodes that would block extraction, all constituents could be moved without violating the SSC or the PIC. The first set of speakers would have rules \REF{ex:2:237} and \REF{ex:2:238}; the second set would replace \REF{ex:2:238} with \REF{ex:2:248}:

\ea\label{ex:2:248}
VP → V (NP) (VP)
\z

Greg Lee (p.c.) has pointed out that there is an alternative solution which does not involve positing two different underlying structures: the two sets of speakers could have different rule orderings for extraction and Equi\is{Equi-NP deletion}. If the first set ordered extraction before Equi, the lower occurrence of \textit{Kofi} would still be undeleted when extraction applied, the S-node would remain unpruned, and the SSC and PIC would still apply. If the second set ordered Equi before ex\-traction, the offending S-node would be pruned to give \REF{ex:2:247} as a derived structure, and no constraints would then inhibit movement.

This is true, but it means that both sets of speakers would then have the more primitive phrase-structure rules\is{rules!phrase-structure} -- \REF{ex:2:237} and \REF{ex:2:238} -- that
%\originalpage{127}
would yield \REF{ex:2:246} as the underlying structure for \REF{ex:2:239}. I shall show in a moment that even if speakers did not have underlying structures like \REF{ex:2:247} for serial \textit{teki} sentences, they would require them for other types of serial-verb constructions. In any case, I know of no evidence that would point to differences in rule ordering among Sranan speakers; indeed, it is very hard, in creole grammars, to find any clear cases in which rule ordering is crucial.

A second set of native-speaker judgments concerns Sranan directional constructions, for example \REF{ex:2:223}, repeated here for conveni\-ence as \REF{ex:2:249}:

\ea\label{ex:2:249}
\gll a waka go a wosu\\
he walk go to house \\
\glt `He walked home'
\z
Here, in contrast with the previous examples, there are no disagreements; either verb can be fronted by the same verb-focusing rule that led to disputes over the status of \REF{ex:2:245};

\ea\label{ex:2:250}
na waka a waka go a wosu
\glt `He \textit{walked} to the house (rather than ran to it)'
\z

\ea\label{ex:2:251}
na go a waka go a wosu
\glt `He walked \textit{to} the house (rather than away from it)'
\z
Thus there is no possibility that speakers could assign to \REF{ex:2:249} a struc\-tural description like that of \REF{ex:2:246}, where the existence of a tensed S would prohibit extraction of the lower verb; rather, \REF{ex:2:249} must have a structure similar to \REF{ex:2:247}, as illustrated in \REF{ex:2:252}:% on the following page:

%\originalpage{128}
\is{dative-benefactive case|(}
\ea\label{ex:2:252}
%\resizebox{.9\textwidth}{!}{
	{\normalfont
	\begin{tikzpicture}[baseline]
	\node at (0,0) (1S) {S};
	\node [below left=\baselineskip and 12mm of 1S] (2NP) {NP}; \node [below right=\baselineskip and 12mm of 1S] (2VP) {VP};
	\node [below=\baselineskip of 2NP] (3a) {\strut a}; \node [below left=\baselineskip and 6mm of 2VP] (3V) {\strut V}; \node [below right=\baselineskip and 6mm of 2VP] (3VP) {\strut VP};
	\node [below=\baselineskip of 3V] (4waka) {\strut waka}; \node [below left=\baselineskip and 3mm of 3VP] (4V) {\strut V}; \node [below right=\baselineskip and 3mm of 3VP] (4PP) {\strut PP}; 
	\node [below=\baselineskip of 4V] (5go) {go}; \node [below=\baselineskip of 4PP] (5wosu) {a wosu};
	
	\draw (2NP) -- (1S.south) -- (2VP); \draw (3a)--(2NP); \draw (3V)--(2VP.south)--(3VP); \draw (3V)--(4waka); \draw (4V)--(3VP.south)--(4PP); \draw (4V)--(5go); 
	\node [isosceles triangle, isosceles triangle apex angle=100, shape border rotate=90, minimum height=1.25em, draw, below=3mm of 4PP.center] (5polygon) {};
	%\draw (5wosu.north east) -- (4PP.south) -- (5wosu.north west) -- cycle; % I don't like this, maybe some optimization is possible?
	
	\end{tikzpicture}	
}
%}
\z

Finally, \ili{Sranan} speakers disagree again over sentences involving datives and benefactives expressed with \textit{gi}, which has an independent existence as a main verb with the meaning `give'. Differences may be illustrated by the following sentences:

\ea{\label{ex:2:253}
\gll Meri tek watra gi den plantjes\\
Mary take water give the plants\\
\glt `Mary brought water for the plants'}
\z
\ea{\label{ex:2:254}
gi san Meri tek watra?\\
\glt `What did Mary bring water for?'}
\z
\ea{\label{ex:2:255}
(*)san Meri teki watra gi?
\glt `What did Mary bring water for?'}
\z
\ea{\label{ex:2:256}
Meri teki a buku gi mi\\
\glt `Mary gave me the book'}
\z
\ea{\label{ex:2:257}
(*)na mi Meri teki a buku gi\\
\glt `It was me Mary gave the book to'}
\z
\ea{\label{ex:2:258}
(*)na gi Meri teki a buku gi mi
\glt `Mary \textit{gave} the book to me'}
\z

Sranan does not strand prepositions, For those who find \REF{ex:2:255} and \REF{ex:2:257} ungrammatical, \textit{gi} must be a preposition, and they must have, instead of \REF{ex:2:248}, the rule \REF{ex:2:259}:\is{movement rules!in Sranan|)}

%\originalpage{129}
\ea\label{ex:2:259}
 VP → V (NP) (PP)
\glt
\z
For those who find the two sentences grammatical, \textit{gi} must be a verb, and such speakers must have rule \REF{ex:2:248}. The verb-focusing rule that applies in \REF{ex:2:258} cannot front prepositions, so, again, those who find \REF{ex:2:258} ungrammatical must consider \textit{gi} as a preposition, while those who find it grammatical must consider \textit{gi} a verb. Differences of this kind are only comprehensible if the set of speakers who find all three sentences grammatical assigns to \REF{ex:2:253} and \REF{ex:2:256} a structure similar to \REF{ex:2:252}, with VP expanded as in \REF{ex:2:260} below; while the set of speakers who find all three sentences ungrammatical (and therefore regard \textit{gi} as a preposition) analyzes the same VP as in \REF{ex:2:261} below:

\ea\label{ex:2:260}
%\resizebox{.9\textwidth}{!}{
	{\normalfont
	\begin{tikzpicture}[baseline]
	\node at (0,-.3) (1VP) {VP}; \node [above left=\baselineskip and 3mm of 1VP] (0inv) {};
	\draw [dashed] (1VP.north)--(0inv);
	\node [below left=\baselineskip and 6mm of 1VP] (2V) {V}; \node [below=\baselineskip of 1VP] (2NP) {NP}; \node [below right=\baselineskip and 18mm of 1VP] (2VP) {VP};
	\draw (2V)--(1VP.south)--(2NP); \draw (2VP)--(1VP.south);
	\node [below=\baselineskip of 2V] (3teki) {\strut teki}; \draw (2V)--(3teki);
	\node [rectangle split, rectangle split parts=2, below=\baselineskip of 2NP] (3watra) {watra\nodepart{two}buku}; \draw (2NP)--(3watra);
	\node [below left=\baselineskip and 6mm of 2VP] (3V) {\strut V}; \node [below right=\baselineskip and 6mm of 2VP] (3NP) {NP}; \draw (3V)--(2VP.south)--(3NP);
	\node  [rectangle split, rectangle split parts=2, below=\baselineskip of 3V] (4gi) {gi}; 
	\node [rectangle split, rectangle split parts=2, below=\baselineskip of 3NP] (4plantjes) {\strut plantjes\nodepart{two}mi};
	\draw (3V) -- (4gi); \draw (3NP) -- (4plantjes);
	\end{tikzpicture}	  
	}
%}
\z
 

\ea\label{ex:2:261}
%\resizebox{\textwidth}{!}{
	{\normalfont
	\begin{tikzpicture}[baseline]
	\node at (0,-.3) (1VP) {VP}; \node [above left=\baselineskip and 3mm of 1VP] (0inv) {};
	\draw [dashed] (1VP.north)--(0inv);
	\node [below left=\baselineskip and 6mm of 1VP] (2V) {V}; \node [below=\baselineskip of 1VP] (2NP) {NP}; \node [below right=\baselineskip and 18mm of 1VP] (2PP) {PP};
	\draw (2V)--(1VP.south)--(2NP); \draw (2PP)--(1VP.south);
	\node [below=\baselineskip of 2V] (3teki) {\strut teki}; \draw (2V)--(3teki);
	\node [rectangle split, rectangle split parts=2, below=\baselineskip of 2NP] (3watra) {watra\nodepart{two}buku}; \draw (2NP)--(3watra);
	\node [below left=\baselineskip and 6mm of 2PP] (3P) {\strut P}; \node [below right=\baselineskip and 6mm of 2VP] (3NP) {NP}; \draw (3V)--(2VP.south)--(3NP);
	\node  [rectangle split, rectangle split parts=2, below=\baselineskip of 3P] (4gi) {gi}; 
	\node [rectangle split, rectangle split parts=2, below=\baselineskip of 3NP] (4plantjes) {\strut plantjes\nodepart{two}mi};
	\draw (3P) -- (4gi); \draw (3NP) -- (4plantjes);
	\end{tikzpicture}
	
}
%}
\z
\is{dative-benefactive case|)}
%\originalpage{130}

In other words, there are three distinct ways in which Sranan speakers analyze serial-verb constructions:

\ea\label{ex:2:262}
\begin{tabular}[t]{@{}lp{8.5cm}}
VP → \ldots~(S) & (some speakers for \textit{teki} instrumentals)\\
VP → \ldots~(VP) & (some speakers for \textit{teki} instrumentals; all speakers for \textit{go} \isi{directionals}; some speakers for \textit{gi} dative/benefactives)\\
VP → \ldots~(PP)&(some speakers for \textit{gi} dative/benefactives)
\end{tabular}
\z

\noindent If the first of these stages represents the most primitive level of creole development, as we have given reason to believe, then the data shown here, drawn from a \textit{synchronic} analysis of Sranan verb serialization, represent at the same time the \textit{diachronic development} of Sranan, from an original state in which presumably all serial verbs were full verbs in tensed sentences to a stage in which these verbs are beginning to be reduced to mere prepositions. Note that this process serves to bring \ili{Sranan|)} structurally closer to the high-prestige language, Dutch, with which it has been in continuous contact for over three centuries.

Thus, there is good reason for claiming, across creole languages generally, that the vast majority of embedded sentences are finite and tensed, and that where exceptions to this generalization can be found, they constitute developments that have taken place subsequent to creolization. The second half of this claim would be hard to prove convincingly because of the inaccessibility of evidence; but I know of neither facts nor arguments that would point in an opposite direction.

With regard to the types of complementation featuring serial verbs, it would seem that the strongest constraint on such developments was the availability of superstrate prepositions for case-marking purposes. Where prepositions were available, even if African influence was strong (as with \ili{Saramaccan}), they would be chosen over serial models. In the absence of superstrate prepositions, serialization would always be chosen. I suspect that it was reinvented, rather than selected,
%\originalpage{131}
in most if not all cases; but if not, if it was indeed selected out of a range of substrate alternatives, the present theory would remain unaffected. This theory claims that verb serialization is the only answer to the problem of marking cases in languages which have only N and V as major categories. Thus, if such structures were selected from a substratum\is{substratum influence}, they were selected because they offered the only answer, not merely because they happened to be present in the substratum; and, in those cases where they were not present in the substratum (as may have been the case in creoles that drew heavily on Guinean or Bantu rather than \ili{Kwa languages}\il{Bantu languages}\il{Guinean languages}), readers of \chapref{ch:1} should have little doubt as to the power of creole children to invent such structures, should the language they were developing require these.

However, before leaving serial verbs, a word is in order on HCE. Substratomaniacs will naturally wish to attribute the absence of serial-verb constructions in HCE to the absence of an African substratum rather than to the presence of prepositions. In fact, though no true serial constructions have developed as a consistent part of the synchronic grammar, sporadic residues of such constructions are to be found both in synchronic speech and in the literature. For instance, instrumental and directional uses\is{Guyanese Creole!directionals} of verbs would occur occasionally in the speech of the very oldest HCE speakers:\is{Hawaiian Creole English!directionals}

\ea\label{ex:2:263}
dei wan get naif pok yu\\
\glt `They want to stab you with a knife'
\z

\ea\label{ex:2:264}
\gll dei wawk fit go skul\\
they walk feet go school\\
\glt `They went to school on foot'
\z
Moreover, \isi{decreolization} in Hawaii began so early and progressed so rapidly that there is good reason to believe that other similar forms were already lost by the early seventies. For instance, in basilectal GC, \textit{take it} and \textit{bring it} are regularly rendered as \textit{ker am go} (lit., `carry it go') and \textit{bing am kom}, a fact usually explained by pointing to a similar structure in \ili{Yoruba}. \citet{Smith1939}, listing the commonest ``mistakes''
%\originalpage{132}
made by children in Hawaii -- most early sources of HCE have this deplorable pedagogical bias -- mentions \textit{take om go} and \textit{bring om come}, only this time the origin is given as Chinese! (In fact, the majority of children listed as using it are non-Chinese.) However, similar forms did not occur in any of our recordings (although the children of Smith's study would only have been in their forties when those recordings were made), and they would seem today to have disappeared entirely. Yet their existence, for however brief a period, can leave little doubt that HCE could have and would have invented regular serial-verb\is{verb serialization|)} constructions if no other means of marking case had been available.%
\footnote{It would seem highly likely, indeed, that the inadequacies of existing creole descriptions, often referred to in this volume, have served to diminish, rather than exaggerate, the degree of creole similarity. To give just one very recent instance, it was long held that the verb-focusing rule discussed earlier in this chapter was not found in the grammars of any of the Indian Ocean creoles. Substratomaniacs could point to the nature of the substratum -- Eastern Bantu, Malagasy, and Indian languages -- as an explanation of this. Now \citet{Corne1977} reports the finding of verb-focusing structures with a copied verb\is{verb-copying} identical to those discussed in this chapter. Substratomaniacs will now doubtless seize on the claim by \citet{Baker1976} that in 1735, 60 percent of the nonwhite population of \isi{Mauritius} was from West Africa. However, this finding is strongly challenged by \citet{Chaudenson1979} on the basis of historical documents which he claims Baker did not examine; according to Chaudenson, the percentage of West Africans never rose much above 33.
	
In fact, the outcome of the disagreement is rather irrelevant to the real issue. Baker's ``60 percent'' contained 66 percent of speakers from Guinea, and \ili{Guinean languages} differ markedly in structure from the Kwa languages which are usually claimed as the source of creole structures. On Baker's own figures, the Kwa speakers in Mauritius in 1735 must have amounted to about 130! Within a few years, the population of Mauritius topped the 10,000 mark, swelled by recruits from India and Madagascar (Baker admits that hardly any Kwa speakers arrived after 1735). The question that substratomaniacs have to answer is: how did 130 people manage to impose their grammar (assuming they had a common one, which is a big assumption) upon a population in which they were outnumbered 100 to 1?}

We have now surveyed a wide range of creole structures across a number of unrelated creole languages. We have seen that even taking into account the, in some cases, several centuries of time that have elapsed since creolization, and the heavy pressures undergone by those creoles (a large majority) that are still in contact with their superstrates, these languages show similarities which go far beyond the possibility of coincidental resemblance, and which are not explicable in terms of conventional transmission processes such as diffusion or substratum influence (the ad hoc nature of the latter should be adequately demonstrated by the opportunism of those who attribute a structure to \ili{Yoruba} when it appears in the Caribbean and to Chinese when it appears in Hawaii). Moreover, we find that the more we strip creoles of their more recent developments, the more we factor out superficial and accidental features, the greater are the similarities that reveal themselves. Indeed, it would seem reasonable to suppose that the only differences among creoles at creolization were those due to differences in the nature of the antecedent pidgin, in particular to the extent to which superstrate features had been absorbed by that pidgin and were therefore directly accessible to the first creole generation in the outputs of their pidgin-speaking parents. Finally, the overall pattern of similarity which emerges from this chapter is entirely consonant with the process of building a language from the simplest constituents -- in many cases, no more than S, N, and V, the minimal constituents necessary for a pidgin.\is{pidgin-creole cycle}

%\originalpage{133}

In theory, given these basic constituents, there are perhaps not infinitely many but certainly a very large number of ways in which, one might suppose, a viable human language could be built -- at least as many ways as there are different kinds of human language. This would certainly be the conclusion to which any existing school of linguistic theory would lead one. It would, however, be an incorrect conclusion. The fact that there appears to be only one way of building up a language (with some, but relatively few and minor variations, of course) strongly suggests that when this problem was originally faced, whether thirty thousand years or thirty thousand centuries ago, it might have had to be\enlargethispage{1\baselineskip} solved in a very similar way, and we shall further explore this possibility in \chapref{ch:4}.

Our original aim in this chapter was to show that the ``inventions'' of HCE\il{Hawaiian Creole English} speakers illustrated in \chapref{ch:1} were not peculiar to them, but followed a regular pattern of ``invention'' which emerged wherever human beings had to manufacture an adequate language\is{language!origins of} in short order from inadequate materials. Now, if all children can indeed do this -- and it would be bizarre indeed if the capacity developed only when it was needed -- they can only do so as the result of the factor which is responsible for all species-specific behavior: genetic transmission of the bioprogram\is{language bioprogram|(} for the species.

The idea that there is a bioprogram for human (and other species) \textit{physical} development is wholly uncontroversial. No one supposes that human beings have to learn to breathe, eat, yell when they are hurt, stand upright, or flex the muscles of finger and thumb into what is the purely human, species-specific ``precision grip''. We speak of children ``learning to walk'', and we characteristically help them in their first stumbling efforts, hut no one seriously imagines that if we neglected to do this, the child would go crawling into maturity. The term ``learning'' is used here in a purely metaphorical sense.

Yet the idea that there is a bioprogram for human \textit{mental} development still meets with massive resistance, despite the fact that Piaget and his disciples have shown how human cognitive development unrolls in a series of predetermined and invariant stages,\footnote{I am only too well aware that Piaget draws conclusions from his studies quite contrary to those drawn here. That he does so, however, has always seemed to me baffling in light of the fact that the developmental stages he posits bear a nativistic explanation much more easily than they do an experiential one. But there is not space here to attempt a reinterpretation of Piagetian findings, desirable though such an activity might seem. We will see in the next chapter, however, that some linguistic findings of Piaget's disciples can very easily (and very fruitfully) be reinterpreted in a nativistic manner (see especially the discussion of \citealt{BrockartEtAl1973}).} and despite
%\originalpage{134}
the fact that, at an ever-increasing rate over the last few decades, experiences long believed to be due to some unanalyzable entity called ``mind'' -- if they were indeed more than subjective illusions -- have been shown to be conditioned and in some cases entirely determined by electrochemical events in the brain. The mind-body \isi{dualism} that has so long dominated Western thought is beginning to seem more and more like an artifact of armchair philosophers operating in blissful ignorance of the laws of reality; and yet the idea \textit{that there is an innate bioprogram that determines the form of human language} is still vigorously if often quite illogically resisted, threatening, as it seems to, \isi{free will}, mental improvement, and the whole galaxy of human dreams and desires.\\\\

I shall return to these fears in the final chapter. In the next chapter I want to pursue what would appear to be an inevitable corollary of the language bioprogram theory. If it is the case that the creole child's capacity to create language is due to such a bioprogram, then, as noted above, it would be absurd to suppose that this bioprogram functions only in the rare and unnatural circumstances in which the normal cultural transmission of language breaks down. Forces that are under genetic control simply cannot be turned on and off in this way. Therefore, if our theory is correct, it should be the case that the acquisition of language\is{language!acquisition of|(} under \textit{normal} circumstances should differ considerably from what has hitherto been supposed.

Briefly, the theory predicts that instead of merely processing linguistic input, the child will seek to actualize the blueprint for language with which his bioprogram provides him. We should note from the outset that there are numerous differences between the present theory and earlier Chomskyan theories of linguistic innateness\is{innateness (of language)|(}, although the latter are often so vague that such differences are not always clear. One point that should be made is that in the present theory, the child is not supposed to ``know'' the bioprogram language from birth -- whatever that might mean -- any more than we would suppose that a child at birth, or even at six months, ``knows'' how to walk.
%\originalpage{135}
Rather, the bioprogram language would unfold, just as a physical bioprogram unfolds; the language would grow just as the body grows, presenting the appropriate structures at the appropriate times and in the appropriate, preprogrammed sequences (I shall have more to say about the mechanisms by which this might be accomplished when we come to \chapref{ch:4}).

\is{hypothesis formation|(}However, the vast mass of human children are not growing up in even a partial linguistic vacuum. There will be a ready-made language which their elders will be determined that they should learn. Thus, almost (but not quite) from the earliest stages, the evolving bioprogram will interact with the target language. Sometimes features in the bioprogram will be very similar to features in the target language, in which case we will find extremely rapid, early, and apparently effortless learning. Sometimes the target language will have evolved away from the bioprogram, to a greater or lesser extent, and in these cases we will expect to find common or even systematic ``errors'' which, in orthodox learning theory, will be attributed to ``incorrect hypotheses'' formed by the child, but which, I shall claim, are simply the result of the child's ignoring (because he is not ready for it) the data presented by speakers of the target language and following out instead the instructions of his bioprogram.
 
Clearly, then, it should be possible to examine existing studies of child language acquisition and reinterpret them in light of the theory outlined above. If that theory is correct, we expect to find a wide variety of evidence that would arise directly from the interaction of bioprogram and target language, and hopefully, be able to account for phenomena of acquisition which have remained mysterious in all previous theories. Accordingly, the next chapter will present just such a survey of the existing literature on language acquisition.\is{language!acquisition of|)}\is{language bioprogram|)}