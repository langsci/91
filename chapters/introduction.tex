\addchap{Introduction}

Of all the fields of study to which human beings have devoted themselves, linguistics could lay claim to being the most conservative. Two thousand five hundred years ago, Panini began it by describing an individual human language, and describing individual languages is what the majority of linguists are still doing. Even during the last couple of decades, in which linguists have begun to be interested in some of the larger issues that language involves, the main thrust toward clarifying those issues has involved making more and more detailed and ingenious descriptions of currently existing natural languages. In consequence, little headway has been made toward answering the really important questions which language raises, such as: how is language acquired by the individual, and how was it acquired by the species?

The importance of these questions is, I think, impossible to exaggerate. Language has made our species what it is, and until we really understand it -- that is, understand what is necessary for it to be acquired and transmitted, and how it interacts with the rest of our cognitive apparatus -- we cannot hope to understand ourselves. And unless we can understand ourselves, we will continue to watch in helpless frustration while the world we have created slips further and further from our control.

The larger and, in a popular sense, more human issues which language involves lie outside the scope of the present work, and will be dealt with at length in a forthcoming volume, \textit{Language} \textit{and} \textit{Species.} First, there is a good deal of academic spadework to be done. In the chapters that follow, I shall try to develop a unified theory which will propose at least a partial answer to three questions, none of which has as yet been satisfactorily resolved:

%\setcounter{itemize}{0}
\begin{enumerate}
\item How did creole languages originate?
\item How do children acquire language?
\item How did human language originate?
\end{enumerate} 

Traditionally, these three questions, insofar as they have been treated at all, have been treated as wholly unrelated. None of the solutions offered for (1) have had any relevance to (2) or (3); none of the solu\-tions offered for (2) have had any relevance to (1) or (3); and none of the solutions offered for (3) have had any relevance to (1) or (2). It has even been explicitly denied, although without a shred of support\-ing evidence, that an answer to (1) could possibly be an answer to (3) \citep{Sankoff1979}. Here and there, a few insightful scholars have hinted at possible links between the problems, and such insights will be ac\-knowledged in subsequent pages. However, a single, unified treatment has never even been attempted, and this book, whatever its short\-comings, may therefore claim at least some measure of originality. Doubtless many of its details will need revision or replacement; the explorer is seldom the best cartographer. However, of one thing I am totally convinced: that the three questions are really one question, and that an answer to any one of them which does not at the same time answer the other two will be, ipso facto, a wrong answer.

I shall begin with the origin of creoles. To some, this may appear the least general and least interesting question of the three. However, as I shall show, creoles constitute the indispensable key to the two larger problems, and this should come as no surprise to those familiar with the history of science, in which, repeatedly, the sideshow of one generation has been the central arena of the next. In \chapref{ch:1}, I shall examine the relationship between the variety of Creole English spoken in Hawaii and the pidgin which immediately preceded it, and I shall show how several elements of that creole could not have been derived from its antecedent pidgin, or from any of the other languages that were in contact at the time of creole formation, and that therefore these elements must have been, in some sense, ``invented''. In \chapref{ch:2}, I shall discuss some (not all -- there would not be space for all) of the features which are shared by a wide range of creole languages and show some striking resemblances between the ``inventions'' of Hawaii and ``inventions'' of other regions which must have emerged quite independently; and I shall also try to probe more deeply into certain aspects of creole syntax and semantics which may prove signifi\-cant when we come to deal with the other two questions. In \chapref{ch:3}, which will deal with ``normal'' language acquisition in noncreole societies, I shall show that some of the things which children seem to acquire effortlessly, as well as some which they get consistently wrong -- both equally puzzling to previous accounts of ``language learning{\textquotedbl} -- follow naturally from the theory which was developed to account for creole origins: that all members of our species are born with a bio\-program for language which can function even in the absence of ade\-quate input. In \chapref{ch:4}, I shall try to show where this bioprogram comes from: partly from the species-specific structure of human perception and cognition, and partly from processes inherent in the expansion of a linear language. At the same time, we will be able to resolve the continuity paradox (``language is too different from animal communication systems to have ever evolved from them{\textquotedbl}; ``language, like any other adaptive mechanism, must have been derived by regular evolutionary processes{\textquotedbl}) which has lain like some huge roadblock across the study of language origins. In the final chapter, I shall briefly summarize and integrate the findings of previous chapters, and suggest answers to some of the criticisms which may be brought against the concept of a genetic program for human language.