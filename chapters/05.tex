\chapter{Conclusions}\label{ch:5}

The foregoing chapters have surveyed the three major areas of language development: development in the individual, development of new languages, and original development of language. Parsimony alone would suggest that these developmental processes might have much in common with one another, and the common pattern that emerges has an independent support that no other linguistic theory that I know of could claim: it is in accord with all we have so far learned about evolutionary processes and it is in accord with all we have so far learned about how processes in the brain determine the behavior of animate creatures. During the sixties and seventies, we heard a good deal about something called ``psychological reality,'' although what it was was never well defined; I would suggest that whatever the fate of the theory argued here, any future linguistic theory will have to be able to claim ``biological reality''\is{``biological reality''} if it is to be taken seriously. 

The theory argued here has claimed that many of the prerequisites for human language were laid down in the course of mammalian evolution, and that the most critical of those prerequisites -- for even things like vocal tract development were necessary, but in no sense sufficient requirements\footnote{Indeed, one objection to the hypothetical history of language given in the preceding chapter might be that many essential prerequi\-sites of language, such as the development of the neural and physiological mechanisms required for vocalization, the lateralization of the brain, and the growth of auditory processing mechanisms or ``templates'' which, as suggested in some fascinating work by Marler and associates (\citealt{Marler1977,Marler1980}; \citealt{MarlerEtAl1979}, etc.), show striking parallels to those of avian species, have simply been ignored. However, these omissions in no way reflect my estimate of the impor\-tance of such developments. The reasons for them are threefold. First, reasons of space (and the overall purpose of this volume) pre\-vented me from describing everything that went into the makeup of language; second, these other developments have been excellently treated elsewhere; and third, I wanted to deal precisely with those aspects of language development which have been most systematically ignored or misunderstood. Certainly, such omissions were not for the purpose of strengthening my case since all the omitted developments are much more obviously the product of the genetic code than the developments discussed in this volume.} -- was the capacity to construct quite elaborate mental representations of the external world in terms of concepts rather than percepts. In other words, something recognizable as thought (though clearly far more primitive than developed human thought) necessarily preceded the earliest forms of anything recognizable as language.

Circumstances still obscure enabled our (fairly remote) ancestors -- perhaps \textit{Homo erectus}, perhaps some other species -- to lexicalize concepts and construct a primitive form of language probably not too dissimilar to that achievable, under training, by modern apes. Language even at so primitive a level conferred a sharp, selective advantage to its users. Over a long period, language developed biologically in the following manner. In any group of any species, there is a certain amount of random variation which allows for variation in individual skill. Those individuals who had higher skills in the manipulation of language had those skills as a direct result of the fact that such random variation had produced, in their brains, mechanisms better adapted for converting preexisting mental representations into linguistic form by lexicalizing and grammaticizing the categories into which those representations were already sorted by neurological processes. Since language-skilled individuals possessed a higher potential for survival, they would produce more offspring than other individuals, and the capacities that had arisen in them by random variation would he preserved and transmitted intact to their descendants.

Note that there is nothing particularly novel about all of this; most people nowadays would agree without any hesitation that the giraffe's neck, the hummingbird' s bill, and all other adaptive developments \textsc{on a physical level} have originated in precisely this manner. It is merely the superstitious persistence of Cartesian dualism that makes people reluctant to admit that since mental characteristics have just as firm a physical foundation in neurological structures, the same processes of biological evolution must apply to them also.

If language arose in the way I have indicated, then what was passed on from generation to generation was not some vague, abstract ``general learning capacity'', or even some highly-specified ``language
%\originalpage{296}
learning capacity''. Biological evolution does not trade in nebulous concepts like these; it hands out concrete features, concrete capacities for specific operations. What was passed on was precisely the capacity to produce a particular, highly-specified language, given only some (perhaps quite minimal) triggering in the form of communal language use. This capacity had attained the level of contemporary creoles when the computational power it bestowed on its owners triggered the cultural explosion of the last ten millennia;\footnote{I certainly do not wish to suggest by this that no sooner had language reached the creole level than agriculture began. There may well have been an interval of tens of thousands of years between these two events, years during which cognitive maps became only gradually more complex; or the interval may have been quite short. There is no way, at present, that we can choose between these alternatives -- or even prove that language in its present form did not exist two million years ago, although the latter possibility seems intrinsically unlikely.} and since cultural evolu\-tion works far faster than biological evolution, and since it operates at a far more abstract level, the effects of cultural evolution on language could not be transferred to the gene pool. Therefore, biological language remained right where it was, while cultural language rode off in all directions. However, it was always there, under the surface, waiting to emerge whenever cultural language hit a bad patch, so to speak; and the worst patch that cultural language ever hit was the unprecedented, culture-shattering act of the European colonialists who set up the slave trade. But it is true that out of evil, good may come, and if they had not done this, we might never have found the one crucial clue to the history of our species.

However, even without such setbacks, cultural language could not expand away from the biological base indefinitely. Just as biology produced a floor below which human language could not fall, so it produced a ceiling above which human language could not rise. The realm of variability of any species has upper limits consisting of capa\-cities from which it is barred genetically from ever having. There can be little doubt that what we genetically have determines how far (and in what directions) we can go culturally in ways which, hopefully, will be major focal points of linguistics, philosophy, psychology, and anthropology in the decades to come. Thus, though languages may diversify and complexify, they can never become unlearnable -- or if they do, children will soon pull them back to earth again.

The child does not, initially, ``learn language''. As he develops, the genetic program for language which is his hominid inheritance unrolls exactly as does the genetic program that determines his increase
%\originalpage{297}
in size, muscular control, etc. ``Learning'' consists of adapting this program, revising it, adjusting it to fit the realities of the cultural language he happens to encounter. Without such a program, the simplest of cultural languages would presumably be quite unlearnable. But the learning process is not without its tensions -- the child tends to hang on to his innate grammar for as long as possible -- so that the ``learning trajectory'' of any human child will show traces of the bioprogram, and bioprogram rules and structures may make their way into adult speech whenever the model of the cultural language is weakened.

This, then, in outline is the unified theory of language acquisition, creole language origins, and general language origins for which the present volume has amassed numerous and diverse types of evidence. The question must now arise: how does this theory relate to existing linguistic theories, and what modifications in such theories does it appear to demand?

Generative theory has now survived for more than two decades as the leading theory in modern linguistics, despite attacks from diverse quarters. Although in the course of this book I have said some harsh words about some current generative stances, it should have been apparent, first, that the theory expressed here would probably have been impossible to frame if generative grammar had never existed, and second, that there is no hostility between the two theories on major issues. The present theory complements and amplifies generative theory. The latter has, in fact, ceded most of the former's territory. The leading figures in generative grammar have simply ignored creoles and shown a positive antipathy to the mere idea of language origins; as for acquisition, while they have theorized about it, they have not deigned to get their hands dirty by actually examining it.

In fact, bioprogram theory and Chomskyan formal universals fit rather well together, as illustrated in \figref{fig:5.1}. % on the following page. 
The bioprogram language would constitute a core structure for human language. Natural languages would be free to vary within the space between the outer limit of the bioprogram and the overall limit
%\originalpage{298}
imposed by formal universals, which represent neural limits --species-specific limits -- on human capacity to process language.

\begin{figure}
	\begin{center}
		\hyphenpenalty=100000000
		\resizebox{.75\textwidth}{!}{
		\begin{tikzpicture}
		
		\node at (0,0) [text width=4cm, align=center] (domain) {Domain of ``natural languages''};
		\node at (6,3) [text width=3cm, align=center] (limit) {Limit imposed by formal universals};
		
		\node at (0,-2.5) [text width=3cm, align=center, draw, thick, inner xsep=1cm, inner ysep=.6666cm] (bioprog) {Bioprogram language};
		\node (gr1) [draw, thick, inner xsep=1.5cm, inner ysep=1cm, fit = (domain) (bioprog)] {};
		
		\draw (limit.south) edge [very thick, draw, -{Stealth[length=15pt,inset=2pt]}, bend left]  (gr1.10);
		\end{tikzpicture}
		}
	\end{center}
	\caption{Relationship of bioprogram to formal universals}\label{fig:5.1}
\end{figure}

Note, however, that the bioprogram does not correspond directly to superficially similar concepts such as ``substantive universals'' or (in one of its several senses) ``universal grammar''. That is, it does not constitute a body of categories, rules, and structures that are necessarily shared by all languages. Indeed, above the trivial level on which all languages have nouns, verbs, oral vowels, etc., I would argue that such a body could not exist. Language systems are wholes, and earlier parts necessarily get mutated to accommodate later parts. Such a statement would be wholly uncontroversial save for the hostility to process that is shown, quite gratuitously, by generative grammar.
%\originalpage{9}

In fact, what linguistics will have to change is not generative theory, in its essential rather than accidental aspects, but a set of much more widely held beliefs, central to which is the belief that all existing languages are at the same level of development. Beliefs that have no empirical foundation generally stem from some kind of politi\-cal commitment, and I am sure that this one, often expressed as ``there are no primitive languages'', arose as a natural and indeed laudable reaction to the claim that thick lips and subhuman minds underlie the characteristics of both creole and tribal languages. According to 19th-century racists, languages and people alike were ranged along a scale of being from the primitive Bushman with his clicks, grunts, and shortage of artifacts, to the modern Western European with his high pale brow and plethora of gadgets. That was when everyone, racist or anti-racist, did believe that Western Man was superior; the only argument was about how nasty this superiority permitted him to be toward ``lesser'' breeds. Now that we are rapidly disabusing ourselves of this kind of mental garbage, it becomes possible to uncouple language from ``level of cultural attainment'' and look at it developmentally without any pejorative implications.

That there is indeed no simple connection between language development and cultural development should be obvious from just two facts. First, many peoples with hunting-and-gathering cultures have languages of horrendous complexity which seem to be a lot further from the bioprogram than ``rich cultural'' languages like English or Chinese.\footnote{I write ``seem to be'' because only empirical investigation will reveal whether such languages are indeed as far from the bioprogram as our intuitions would suggest. One test will be the time taken by children to acquire the main grammatical structures of given languages. It was often claimed (at a time when acquisition had hardly been studied!) that all languages were equally easy for children to learn. This belief was, of course, simply deduced from the ``all-languages-are-developmentally-equal'' dogma. Work by Slobin and his associates already suggests this may be quite far from the truth.} Second, creole languages originated in the most advanced cultures of their day. I do not mean that the strains of Mozart nightly pervaded the barracoons; I mean that it was in the slave colonies that the Western powers developed the industrial technology and systems of disciplined mass labor which later, with the aid of the capital amassed by so doing, they generously bestowed upon their own citizens. While creole speakers were working in organized bodies of hundreds or even thousands and operating complex mechanical processes, the leading technocrats of Western Europe were sitting in their own kitchens with their handlooms. So much for simplistic ``culture-and-language'' equations.

%\originalpage{300}

However, old beliefs die hard, and assuredly, no matter what I say, racists will pounce on the phrase ``developmental differences'' and use it to suggest that in some never-to-be-precisely-specified fashion my work ``proves'' that creoles, or their speakers, or both, are inferior to those who \textit{s} their third person singulars and cross their \textit{a}s, \textit{the}s, and zeros when they come to generics. Assuredly, too, progressives, rallying indiscriminately to the struggle, will feel obliged to include this theory in their denunciations, and to accuse me of having called creoles ``primitive languages'' and of having revived the despised ``baby-talk theory'' of creole origins. There is no prophylactic against ignorance. But to anyone who has read this book with even a minimum of care, it should be apparent that the theory presented here is at an opposite pole to those which sought to derive creoles from the babyish imitations of Europeans' condescending simplifications, and that creoles, far from being ``primitive'' in anything but the sense of ``primary'', give us access to the essential bedrock on which our humanity is founded; their re-creation, in the face of what the French sociologist Roger Bastide aptly termed the ``Cartesian savagery'' of colonialism, represents a triumph of the human spirit, and if it were necessary to justify them in such a fashion, I could show a dozen ways in which they are more lucid, more elegant, more logical, and less easy to lie in than English or other European languages. But I will let the dedication of this volume speak for itself.\\\\

The idea of language development is not, I would suspect, the only aspect of the present theory that is likely to arouse ideological rather than logical opposition. A great deal of human self-esteem is vested in the belief that there is a qualitative difference between ours and other species, and there is much in this volume that might be thought to weaken such a belief. Weakening such a belief, it is often claimed, may destroy ``the Dignity of Man'' and lead members of our species to treat other members as if they were no more than beasts.

One could ask a Tasmanian what he thought of this claim, if the advanced techniques of transplanted English foxhunters had
%\originalpage{301}
left any Tasmanians to be asked. Anyone who casts a candid eye down the perspective of human history must find it hard to explain how the idea that ``people are no more than animals'' could get people any worse treatment than they have gotten already. Moreover, as the discussion of Cartesian dualism at the beginning of \chapref{ch:4} made clear, the position of this theory is not ``Animals don't have souls, and we don't either''; rather, it is ``We have souls, and animals do too.'' The result, I should have thought, would have been to upgrade animals rather than downgrade ourselves.

Further hostility may arise from fears that the theory threatens free will and human perfectibility. If we speak what we are biologically programmed to speak, and if what we are biologically programmed to speak directly reflects the structure of our central nervous system, then the thoughts we think must be biologically programmed too.

If other reactions to the theory can be dismissed as knee-jerk alarmism, this one cannot. It is, I think, pretty likely that our thinking is species-specific, and therefore, almost by definition, incapable of providing adequate solutions to the problems we see ourselves facing or of answering the questions about the nature of the universe which we find so easy to ask. If this is so, it is so. If it is even possible that it could be so, then the appropriate reaction is not to hide behind a smokescreen of rhetoric, but to determine whether or not it is so. If it is not so, we have a green light to go ahead with human perfectibility, despite the unpromising auguries of our previous efforts in that direction. If it is so, then we have to learn either to live within our limits or to change those limits, if we can. For one thing is certain: if they exist, they cannot be talked away.

Although I am convinced that future research will show the scope of human freedom to be narrower than we had believed, and although there is no value that I personally rate above human freedom, I do not find myself in the least depressed by the prospect. Evolution has maintained a steady increase in the autonomy of its creatures without, so far as I am aware, a single retrogressive step. We as a species may lack the infinite capacities which some members of it have thought,
%\originalpage{302}
and continue to think, that we possess, but the range of options open to us is still infinitely greater than that available to any other species, and the peculiar powers we have inherited allow the possibility that we may one day transcend the limits of species. But we will not do this by laying claim to capacities that we do not possess. We will do it only by determining what the limits of our species are, and then deciding what we want to do about that knowledge.

We may decide that less is more, small is beautiful, and that we must live within our cognitive means, even if so living entails perpetuating the cycle of injustice, revolt, and more injustice which constitutes the major part of our history to date. But somehow I do not think that this will be our choice.
 
One recalls the TV game show in which the quizmaster asks, ``Will you take the money or open the box?'' ``Open the box, open the box!'' the studio audience roars. I think we would try to open the box of species that encloses us, even if we knew that it was an inside-out Pandora's Box, and that once we had broken free of it, all the terrors of the universe would rain down upon our heads.