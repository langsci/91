%%%%%%%%%%%%%%%%%%%%%%%%%%%%%%%%%%%%%%%%%%%%%%%%%%%%
%%%                                              %%%
%%%                 Metadata                     %%%
%%%          fill in as appropriate              %%%
%%%                                              %%%
%%%%%%%%%%%%%%%%%%%%%%%%%%%%%%%%%%%%%%%%%%%%%%%%%%%%

\title{Roots\newlineCover of language}  %look no further, you can change those things right here.
% \renewcommand{\subtitle}{History and typology}
\BackTitle{Roots of language} % Change if BackTitle != Title
\BackBody{{\itshape Roots of language} was originally published in 1981 by Karoma Press (Ann Arbor).\\It was the first work to systematically develop a theory first suggested by Coelho in the late nineteenth century: that the creation of creole languages somehow reflected universal properties of language. The book also proposed that the same set of properties would be found to emerge in normal first-language acquisition and must have emerged in the original evolution of language. These proposals, some of which were elaborated in an article in \textit{Behavioral and Brain Sciences} (1984), were immediately controversial and gave rise to a great deal of subsequent research in creoles, much of it aimed at rebutting the theory. The book also served to legitimize and stimulate research in language evolution, a topic regarded as off-limits by linguists for over a century. The present edition contains a foreword by the author bringing the theory up to date; a fuller exposition of many of its aspects can be found in the author’s most recent work, \textit{More than nature needs} (Harvard University Press, 2014).}
\dedication{\parbox[t]{\textwidth}{\begin{flushleft}
	\textit{To the people of Palmares,\\
		\hspace{4em}El Palenque de San Basilio,\\
		\hspace{4em}The Cockpit Country,\\
		\hspace{4em}and the Saramacca River,\\
		who fought for decency, dignity, and freedom\\
		against the Cartesian savagery of Western colonialists\\
		\hspace{2em}and slavemakers;\\
		whose tongues, having survived\\		
		to confound pedagogue and philosopher alike,\\
		now, by an ironic stroke of justice,\\		
		offer us indispensable keys to the knowledge of our species.}\end{flushleft}}
	}
\typesetter{Felix Kopecky, Sebastian Nordhoff}
%\proofreader{Change proofreaders in localmetadata.tex}
\author{Derek Bickerton}
%\newlength{\csspine} 
%\newlength{\bodspine}

%\renewcommand{\lsCoverTitleFont}[1]{\sffamily\addfontfeatures{Scale=MatchUppercase}\fontsize{50pt}{16.75mm}\selectfont #1}

\renewcommand{\lsISBNdigital}{978-3-946234-08-1}
\renewcommand{\lsISBNhardcover}{978-3-946234-09-8}
\renewcommand{\lsISBNsoftcover}{978-3-946234-10-4}

\renewcommand{\lsSeries}{classics} % use lowercase acronym, e.g. sidl, eotms, tgdi
\renewcommand{\lsSeriesNumber}{3} %will be assigned when the book enters the proofreading stage
\renewcommand{\lsURL}{http://langsci-press.org/catalog/book/91} % contact the coordinator for the right number