\documentclass[11pt]{article}
\usepackage{amssymb}
%\usepackage{xunicode}
\usepackage{mathspec}
\usepackage{xltxtra}
\usepackage{tikz}
\usepackage{booktabs}
\usepackage{multirow}
\usepackage[hmargin=1cm]{geometry}
\usetikzlibrary{positioning}
\usetikzlibrary{fit}
\usetikzlibrary{shapes.geometric}
\usetikzlibrary{shapes.multipart}
\usetikzlibrary{calc}
\usetikzlibrary{arrows.meta}
\usetikzlibrary{decorations.markings}
\usetikzlibrary{intersections}

\pgfdeclarelayer{bickertonbg}
\pgfsetlayers{bickertonbg,main}

\setallmainfonts[Mapping=tex-text]{Linux Libertine O}
\begin{document}
	
	\tableofcontents
	\newpage

\begin{center}

\section{Ex 83 Pg 36}

\resizebox{\textwidth}{!}{

	\begin{tikzpicture}[baseline]
	% Draw the nodes
	\node at (0,0) (1S) {S$_0$};
	% Level 2
	\node [below left=\baselineskip and 35mm of 1S, circle, draw] (2NP) {NP};
	\node [below right=\baselineskip and 25mm of 1S] (2VP) {VP};
	\node [below=\baselineskip of 1S] (2Aux) {Aux};
	% Sublevel of 2: "ok"
	\node [below right=3mm of 2NP, circle, draw] (2ok) {o.k.}; 
	% Level 3
	\node [below=2\baselineskip of 2NP] (3S) {S$_1$};
	\node [below left=2\baselineskip and 3mm of 2VP] (3V) {V};
	\node [below right=2\baselineskip and 3mm of 2VP] (3Adv) {Adv};
	% Level 4
	\node [below left=\baselineskip and 8mm of 3S, circle, draw] (4NP) {NP};
	\node [below right=\baselineskip and 8mm of 3S] (4VP) {VP};
	%Sublevel of 4: "*"
	\node [below right=3mm of 4NP, circle, draw] (4*) {*}; 
	% Level 5
	\node [below left=2\baselineskip and 3mm of 4VP] (5V) {V};
	\node [below right=2\baselineskip and 3mm of 4VP] (5Adv) {Adv};
	% Level 6
	% Polygon
	\node [below=1mm of 5Adv, regular polygon, regular polygon sides=3, draw, inner sep=6pt] (6polygon) {};
	\node [on grid, below=1\baselineskip of 6polygon.base, baseline] (6ovadea) {\strut ova dea}; 
	
	\path let \p1 = (5V),  \p2 = (6ovadea) in node at (\x1,\y2) (6goin) {\strut goin};
	\path let \p1 = (4NP),  \p2 = (6ovadea) in node at (\x1,\y2) (6sambadi) {\strut sambadi};
	\path [on grid] let \p1 = (2Aux),  \p2 = (6ovadea) in node at (\x1,\y2) (6gon) {\strut gon};
	\path let \p1 = (3V),  \p2 = (6ovadea) in node at (\x1,\y2) (6hia) {\strut hia};
	\path let \p1 = (3Adv),  \p2 = (6ovadea) in node at (\x1,\y2) (6nau) {\strut nau};
	
	% Draw the paths between nodes
	\draw (node cs:name=1S, anchor=west) -- (node cs:name=2NP, anchor=north east);
	\draw (node cs:name=1S, anchor=east) -- (node cs:name=2VP, anchor=north);
	\draw (node cs:name=1S, anchor=south) -- (node cs:name=2Aux, anchor=north);
	\draw (node cs:name=2NP, anchor=south east) -- (node cs:name=2ok, anchor=north west);
	\draw (node cs:name=2NP, anchor=south) -- (node cs:name=3S);
	\draw (node cs:name=2VP, anchor=south) -- (node cs:name=3V, anchor=north);
	\draw (node cs:name=2VP, anchor=south) -- (node cs:name=3Adv, anchor=north);
	\draw (node cs:name=3S, anchor=south west) -- (node cs:name=4NP, anchor=north east);
	\draw (node cs:name=3S, anchor=south east) -- (node cs:name=4VP, anchor=north);
	\draw (node cs:name=4NP, anchor=south east) -- (node cs:name=4*, anchor=north west); 
	\draw (node cs:name=4VP, anchor=south) -- (node cs:name=5V, anchor=north);
	\draw (node cs:name=4VP, anchor=south) -- (node cs:name=5Adv, anchor=north);
	\draw (node cs:name=4NP, anchor=south) -- (node cs:name=6sambadi);
	\draw (node cs:name=5V, anchor=south) -- (node cs:name=6goin);
	\draw (node cs:name=2Aux, anchor=south) -- (node cs:name=6gon);
	\draw (node cs:name=3V, anchor=south) -- (node cs:name=6hia);
	\draw (node cs:name=3Adv, anchor=south) -- (node cs:name=6nau);
	
	\end{tikzpicture}
}
\newpage

\section{Ex 141 Pg 101}

\resizebox{\textwidth}{!}{
	\begin{tikzpicture}[baseline]
	% Levels 1 & 2
	\node at (0,0) (1S) {S};
	\node [below left=\baselineskip and 18mm of 1S] (2NP) {NP};
	\node [below=\baselineskip of 1S] (2AUX) {AUX};
	\node [below right=\baselineskip and 30mm of 1S] (2VP) {VP};
	
	% Level 3
	\node [rectangle split, rectangle split parts=2, below=\baselineskip of 2NP] (3I) {I\nodepart{two}mi};
	\node [below=\baselineskip of 2AUX] (3Past) {Past};
	\node [below left=\baselineskip and 12mm of 2VP] (3V) {V};
	\node [below right=\baselineskip and 12mm of 2VP] (3S) {S};

	% Level 4
%	\node [below=.1\baselineskip of 3I] (4mi) {mi};
	\node [rectangle split, rectangle split parts=2, below=\baselineskip of 3V] (4hear) {hear\nodepart{two}hia};
	\node [circle, draw, below left=\baselineskip and 12mm of 3S] (4NP) {NP};
	\node [below=\baselineskip of 3S] (4AUX) {AUX};
	\node [below right=\baselineskip and 12mm of 3S] (4VP) {VP};
	
	% Levels 5 and 6
	\node [rectangle split, rectangle split parts=2, below=\baselineskip of 4NP] (5drums) {drums\nodepart{two}drom};
	\node [below=2\baselineskip of 4AUX] (5ASP) {ASP};
	\node [below=2\baselineskip of 4VP] (5V) {V};
	\node [rectangle split, rectangle split parts=2, below=\baselineskip of 5V] (6beat) {beat\nodepart{two}nak};
	
	
	% Connections
	\draw (node cs:name=1S, anchor=south) -- (node cs:name=2NP);
	\draw (node cs:name=1S, anchor=south) -- (node cs:name=2AUX);
	\draw (node cs:name=1S, anchor=south) -- (node cs:name=2VP);	
	\draw (node cs:name=2NP, anchor=south) -- (node cs:name=3I);
	\draw (node cs:name=2AUX, anchor=south) -- (node cs:name=3Past);
	\draw (node cs:name=2VP, anchor=south) -- (node cs:name=3V);
	\draw (node cs:name=2VP, anchor=south) -- (node cs:name=3S);	
	\draw (node cs:name=3V, anchor=south) -- (node cs:name=4hear);
	\draw (node cs:name=3S, anchor=south) -- (node cs:name=4NP);
	\draw (node cs:name=3S, anchor=south) -- (node cs:name=4AUX);
	\draw (node cs:name=3S, anchor=south) -- (node cs:name=4VP);
	\draw (node cs:name=4NP, anchor=south) -- (node cs:name=5drums);
	\draw (node cs:name=4AUX, anchor=south) -- (node cs:name=5ASP);
	\draw (node cs:name=4VP, anchor=south) -- (node cs:name=5V);
	\draw (node cs:name=5V, anchor=south) -- (node cs:name=6beat);
	\end{tikzpicture}
	}
\newpage

\section{Ex 142 Pg 102}

\resizebox{\textwidth}{!}{
	\begin{tikzpicture}[baseline]
	% Levels 1 & 2
	\node at (0,0) (1S) {S};
	\node [below left=\baselineskip and 12mm of 1S] (2NP) {NP};
	\node [below=\baselineskip of 1S] (2AUX) {AUX};
	\node [below right=\baselineskip and 24mm of 1S] (2VP) {VP};
	
	% Level 3
	\node [below=\baselineskip of 2NP] (3I) {I};
	\node [below=\baselineskip of 2AUX] (3Past) {Past};
	\node [below left=\baselineskip and 6mm of 2VP] (3V) {V};
	\node [circle, draw, inner sep=2pt, yshift=.25em, below=\baselineskip of 2VP] (3NP) {NP};
	\node [below right=\baselineskip and 6mm of 2VP] (3VP) {VP};
	
	% Level4
	\node [below=\baselineskip of 3V] (4hear) {\strut hear};
	\node [below=\baselineskip of 3NP, yshift=.3em] (4drums) {\strut drums}; % NOT PERFECT!!
	\node [below=\baselineskip of 3VP] (4beating) {\strut beating};
	
	% Connections
	\draw (node cs:name=1S, anchor=south) -- (node cs:name=2NP);
	\draw (node cs:name=1S, anchor=south) -- (node cs:name=2AUX);
	\draw (node cs:name=1S, anchor=south) -- (node cs:name=2VP);	
	\draw (node cs:name=2NP, anchor=south) -- (node cs:name=3I);
	\draw (node cs:name=2AUX, anchor=south) -- (node cs:name=3Past);
	\draw (node cs:name=2VP, anchor=south) -- (node cs:name=3V);
	\draw (node cs:name=2VP, anchor=south) -- (node cs:name=3NP);
	\draw (node cs:name=2VP, anchor=south) -- (node cs:name=3VP);
	\draw (node cs:name=3V, anchor=south) -- (node cs:name=4hear);
	\draw (node cs:name=3NP, anchor=south) -- (node cs:name=4drums);
	\draw (node cs:name=3VP, anchor=south) -- (node cs:name=4beating);
	
	\end{tikzpicture}
	}


\newpage

\section{Ex 199 Pg 111}

\resizebox{\textwidth}{!}{
\begin{tikzpicture}[baseline]
	% Levels 1 to 3
	\node at (0,0) (1S) {\=S};	
	\node [below left=\baselineskip and 6mm of 1S] (2COMP) {COMP};
	\node [below right=\baselineskip and 24mm of 1S] (2S) {S};
	\node [below left=\baselineskip and 12mm of 2S] (3NP) {NP};
	\node [below=\baselineskip of 2S] (3V) {V};
	\node [below right=\baselineskip and 24mm of 2S] (3S) {S};
	
	% Levels 4 and 5
	\node [below=\baselineskip of 3NP] (4i) {\strut i};
	\node [below=\baselineskip of 3V] (4waan) {\strut waan};
	\node [below left=\baselineskip and 6mm of 3S] (4NP) {\strut NP};
	\node [below=\baselineskip of 3S] (4Aux) {\strut Aux};
	\node [below right=\baselineskip and 6mm of 3S] (4V) {\strut V};
	\node [below right=\baselineskip and 24mm of 3S] (4NP2) {\strut NP};
	\node [below=\baselineskip of 4NP] (5mi) {\strut mi};
	\node [below=\baselineskip of 4Aux] (5fi) {\strut fi};
	\node [below=\baselineskip of 4V] (5go) {\strut go};
	\node [isosceles triangle, isosceles triangle apex angle=100, shape border rotate=90, minimum height=1.5em, draw, below=3mm of 4NP2.center] (5polygon) {};
%	\path [on grid] let \p1 = (4NP2), \p2 = (5go.base) in node at (\x1,\y2) (6WH) {WH-place};
	\node [below=\baselineskip of 4NP2] (6WH) {\strut WH-place};	
	
	% Connections
	\draw (node cs:name=1S) -- (node cs:name=2COMP);
	\draw (node cs:name=1S) -- (node cs:name=2S);
	\draw (node cs:name=2S) -- (node cs:name=3NP);
	\draw (node cs:name=2S) -- (node cs:name=3V);
	\draw (node cs:name=2S) -- (node cs:name=3S);
	\draw (node cs:name=3NP) -- (node cs:name=4i);
	\draw (node cs:name=3V) -- (node cs:name=4waan);
	\draw (node cs:name=3S) -- (node cs:name=4NP);
	\draw (node cs:name=3S) -- (node cs:name=4Aux);
	\draw (node cs:name=3S) -- (node cs:name=4V);
	\draw (node cs:name=3S) -- (node cs:name=4NP2);
	\draw (node cs:name=4NP) -- (node cs:name=5mi);
	\draw (node cs:name=4Aux) -- (node cs:name=5fi);
	\draw (node cs:name=4V) -- (node cs:name=5go);
	
	

\end{tikzpicture}
}

\section{Ex 200 Pg 112}

\resizebox{\textwidth}{!}{
\begin{tikzpicture}[baseline]
	% Levels 1 to 3
	\node at (0,0) (1S) {\=S};	
	\node [below left=\baselineskip and 12mm of 1S.south] (2COMP) {COMP};
	\node [below right=\baselineskip and 6mm of 1S.south] (2S) {S};
	\node [below left=\baselineskip and 12mm of 2S] (3NP) {NP};
	\node [below=\baselineskip of 2S] (3V) {V};
	\node [below right=\baselineskip and 12mm of 2S] (3PP) {PP};
	
	% Levels 4 and 5
	\node [below=\baselineskip of 3NP] (4i) {\strut i};
	\node [below=\baselineskip of 3V] (4waan) {\strut waan};
	\node [below left=4\baselineskip and 3mm of 3PP] (4P) {\strut P};
	\node [below right=4\baselineskip and 24mm of 3PP] (4S) {\strut S};
	\node [below=\baselineskip of 4P] (5fi) {\strut fi};
	\node [below left=4\baselineskip and 18mm of 4S] (5NP) {\strut NP};
	\node [below left=4\baselineskip and 2mm of 4S] (5Aux) {\strut Aux};
	\node [below right=4\baselineskip and 2mm of 4S] (5V) {\strut V};
	\node [below right=4\baselineskip and 18mm of 4S] (5NP2) {\strut NP};

	\node [below=\baselineskip of 5NP] (6mi) {\strut mi};
	\node [below=\baselineskip of 5Aux] (6fi) {\strut fi};
	\node [below=\baselineskip of 5V] (6go) {\strut go};
	\node [isosceles triangle, isosceles triangle apex angle=100, shape border rotate=90, minimum height=1.5em, draw, below=3mm of 5NP2.center] (6polygon) {};
%	\path [on grid] let \p1 = (5NP2), \p2 = (6go.base) in node at (\x1,\y2) (7WH) {WH-place}; 
	\node [below=\baselineskip of 5NP2] (7WH) {\strut WH-place};	
	
	% Connections
	\draw (node cs:name=1S, anchor=south) -- (node cs:name=2COMP, anchor=north);
	\draw (node cs:name=1S, anchor=south) -- (node cs:name=2S, anchor=north);
	\draw (node cs:name=2S, anchor=south) -- (node cs:name=3NP);
	\draw (node cs:name=2S, anchor=south) -- (node cs:name=3V);
	\draw (node cs:name=2S, anchor=south) -- (node cs:name=3PP);
	\draw (node cs:name=3NP) -- (node cs:name=4i);
	\draw (node cs:name=3V) -- (node cs:name=4waan);
	\draw (3PP.south) -- (4P.north);
	\draw (3PP.south) -- (4S.north);
	\draw (4P.south) -- (5fi.north);
	\draw (4S.south) -- (node cs:name=5NP);
	\draw (4S.south) -- (node cs:name=5Aux);
	\draw (4S.south) -- (node cs:name=5V);
	\draw (4S.south) -- (node cs:name=5NP2);
	\draw (node cs:name=5NP) -- (node cs:name=6mi);
	\draw (node cs:name=5Aux) -- (node cs:name=6fi);
	\draw (node cs:name=5V) -- (node cs:name=6go);
	
	% \draw [red] (6mi.base) -- (7WH.base); % This line checks whether basline alignment is correct.
\end{tikzpicture}
}


\section{Ex 201 Pg 113}

\resizebox{\textwidth}{!}{
	\begin{tikzpicture}[baseline]
	% Levels 1 to 3
	\node at (0,0) (1S) {\=S};	
	\node [below left=\baselineskip and 6mm of 1S] (2COMP) {\strut COMP};
	\node [below right=\baselineskip and 24mm of 1S] (2S) {\strut S};
	\node [below left=\baselineskip and 12mm of 2S] (3NP) {\strut NP};
	\node [below=\baselineskip of 2S] (3V) {\strut V};
	\node [below right=\baselineskip and 24mm of 2S] (3S) {\strut \=S};
	\node [below=\baselineskip of 2COMP] (3space) {( )};
	
	% Levels 4 and 5
	\node [below=\baselineskip of 3NP] (4i) {\strut i};
	\node [below=\baselineskip of 3V] (4waan) {\strut waan};
	\node [below left=\baselineskip and 3mm of 3S] (4comp) {\strut COMP};
	\node [below right=\baselineskip and 12mm of 3S] (4S) {\strut S};
	
	\node [below=\baselineskip of 4comp] (5fi) {(fi)};
	\node [below left=\baselineskip and 6mm of 4S] (5NP) {\strut NP};
	\node [below=\baselineskip of 4S] (5Aux) {\strut Aux};
	\node [below right=\baselineskip and 6mm of 4S] (5V) {\strut V};
	\node [below right=\baselineskip and 24mm of 4S] (5NP2) {\strut NP};
	\node [below=\baselineskip of 5NP] (6mi) {\strut mi};
	\node [below=\baselineskip of 5Aux] (6fi) {\strut fi};
	\node [below=\baselineskip of 5V] (6go) {\strut go};
	\node [isosceles triangle, isosceles triangle apex angle=100, shape border rotate=90, minimum height=1.5em, draw, below=3mm of 5NP2.center] (6polygon) {};
%	\path [on grid] let \p1 = (5NP2), \p2 = (6go.base) in node at (\x1,\y2) (6WH) {WH-place};
	\node [below=\baselineskip of 5NP2] (6WH) {\strut WH-place};	
	
	% Connections
	\draw (node cs:name=1S, anchor=south) -- (node cs:name=2COMP);
	\draw (node cs:name=1S, anchor=south) -- (node cs:name=2S);
	\draw (node cs:name=2S, anchor=south) -- (node cs:name=3NP);
	\draw (node cs:name=2S, anchor=south) -- (node cs:name=3V);
	\draw (node cs:name=2S, anchor=south) -- (node cs:name=3S);
	\draw (node cs:name=3NP) -- (node cs:name=4i);
	\draw (node cs:name=3V) -- (node cs:name=4waan);
	\draw (node cs:name=3S, anchor=south) -- (node cs:name=4comp);
	\draw (node cs:name=3S, anchor=south) -- (node cs:name=4S);
	\draw (node cs:name=4S, anchor=south) -- (node cs:name=5NP);
	\draw (node cs:name=4S, anchor=south) -- (node cs:name=5Aux);
	\draw (node cs:name=4S, anchor=south) -- (node cs:name=5V);
	\draw (node cs:name=4S, anchor=south) -- (node cs:name=5NP2);
	\draw (node cs:name=5NP) -- (node cs:name=6mi);
	\draw (node cs:name=5Aux) -- (node cs:name=6fi);
	\draw (node cs:name=5V) -- (node cs:name=6go);
	\draw (node cs:name=2COMP) -- (node cs:name=3space);
	\draw (node cs:name=4comp) -- (node cs:name=5fi);
	
	\draw (6WH.south) edge [ dashed, -{Stealth[]}, bend left=60]  (5fi.south);
	\draw (5fi.south west) edge [ dashed, -{Stealth[]}, bend left=50] (3space.south);
	
	
	
	\end{tikzpicture}

}
\clearpage
\section{Ex 207 Pg 115}

\resizebox{\textwidth}{!}{
	\begin{tikzpicture}[baseline]
	
	\node at (0,0) (1S) {S};
	\node [below right=\baselineskip and 24mm of 1S] (2V) {$V_{adj}$};
	\node [below=\baselineskip of 1S] (2Aux) {Aux};
	\node [below left=\baselineskip and 24mm of 1S] (2NP) {NP};
	
	\node [below=\baselineskip of 2NP] (3S) {\strut\=S};
	\node [below left=\baselineskip and .25mm of 2Aux] (3na) {\strut na};
	\node [below right=\baselineskip  and .25mm of 2Aux] (3bin) {\strut bin};
	\node [below=\baselineskip of 2V] (3iizi) {iizi};
	
	\node [below left=\baselineskip and 12mm of 3S] (4comp) {COMP};
	\node [below right=\baselineskip and 6mm of 3S] (4S) {S};
	
	\node [below=\baselineskip of 4comp] (5fi) {\strut fi};
	\node [below left=\baselineskip and 6mm of 4S] (5NP) {\strut NP};
	\node [below=\baselineskip of 4S] (5V) {\strut V};
	\node [below right=\baselineskip and 6mm of 4S] (5NP2) {\strut NP};
	
	\node [below=\baselineskip of 5NP] (6emptyset) {\strut $\varnothing$}; %from amssymb
	\node [below=\baselineskip of 5V] (6kech) {\strut kech};
	\node [below=\baselineskip of 5NP2] (6taiga) {\strut taiga};
	
	\draw (2NP.north) -- (1S.south) -- (2Aux.north); 
	\draw (1S.south) -- (2V.north); 
	\draw (3na.north) -- (2Aux.south) -- (3bin.north);
	\draw (2NP.south) -- (3S);
	\draw (2V.south) -- (3iizi);
	\draw (4comp.north) -- (3S.south) -- (4S.north);
	\draw (4comp.south) -- (5fi);
	\draw (5NP.north) -- (4S.south) -- (5NP2.north);
	\draw (4S.south) -- (5V.north);
	\draw (5NP.south) -- (6emptyset.north);
	\draw (5NP2.south) -- (6taiga.north);
	\draw (5V.south) -- (6kech.north);

	

	\end{tikzpicture}
}
\clearpage
\section{Ex 222 Pg 118}

\resizebox{\textwidth}{!}{
\begin{tikzpicture}[baseline]

\node at (0,0) (1S) {S};
\node [below right=\baselineskip and 24mm of 1S] (2NP2) {NP};
\node [below left=\baselineskip and 24mm of 1S] (2NP) {NP};
\node [below=\baselineskip of 1S] (2V) {V};
\node [below=\baselineskip of 2NP] (3S) {S};
\node [below=\baselineskip of 2NP2] (3S2) {S};
\node [below=\baselineskip of 2V] (3mek) {mek};
\node [below left=\baselineskip and 6mm of 3S] (4NP) {NP}; \node [below right=\baselineskip and 6mm of 3S] (4V) {V};
\node [below left=\baselineskip and 6mm of 3S2] (4NP2) {NP}; \node [below right=\baselineskip and 6mm of 3S2] (4V2) {V};
\node [below=\baselineskip of 4NP] (5dem) {dem}; \node [below=\baselineskip of 4V] (5gaan) {gaan}; 
\node [below=\baselineskip of 4NP2] (5mi) {mi}; \node [below=\baselineskip of 4V2] (5glad) {glad}; 
\draw (2NP.north) -- (1S.south) -- (2V.north); \draw (1S.south) -- (2NP2.north);
\draw (3S) -- (2NP); \draw (3S2) -- (2NP2); \draw (2V) -- (3mek);
\draw (4NP) -- (3S.south) -- (4V); \draw (4NP2) -- (3S2.south) -- (4V2);
\draw (5dem) -- (4NP); \draw (5gaan) -- (4V); \draw (5mi) -- (4NP2); \draw (5glad) -- (4V2);
\end{tikzpicture}	
}

\section{Ex 246 Pg 125}
\resizebox{\textwidth}{!}{
\begin{tikzpicture}[baseline]

\node at (0,0) (1S) {S};
\node [below left=\baselineskip and 12mm of 1S] (2NP) {NP}; \node [below right=\baselineskip and 12mm of 1S] (2VP) {VP};

\node [below=\baselineskip of 2NP] (3Kofi) {Kofi}; \node [below left=\baselineskip and 12mm of 2VP] (3V) {V};
\node [below=\baselineskip of 2VP] (3NP) {NP}; \node [below right=\baselineskip and 24mm of 2VP] (3S) {S};
\node [below=\baselineskip of 3V] (4teki) {teki}; \node [below=\baselineskip of 3NP] (4nefi) {a nefi};
\node [below left=\baselineskip and 6mm of 3S] (4NP) {NP}; \node [below right=\baselineskip and 6mm of 3S] (4VP) {VP};

\node [below=\baselineskip of 4NP] (5Kofi) {(Kofi)}; \node [below left=\baselineskip and 3mm of 4VP] (5V) {V}; \node [below right=\baselineskip and 3mm of 4VP] (5NP) {NP};
\node [below=\baselineskip of 5V] (6koti) {koti}; \node [below=\baselineskip of 5NP] (6brede) {a brede};

\draw (2NP) -- (1S.south) -- (2VP);
\draw (3Kofi) -- (2NP); \draw (3V) -- (2VP.south) -- (3NP); \draw (2VP.south) -- (3S); \draw (4teki)--(3V); \draw (4nefi)--(3NP); \draw (4NP) -- (3S.south) -- (4VP); 
\draw (5Kofi)--(4NP); \draw (5V) -- (4VP.south) -- (5NP); \draw (6koti) -- (5V); \draw (6brede) -- (5NP);

\end{tikzpicture}
}

\section{Ex 247 Pg 126}

\resizebox{\textwidth}{!}{
\begin{tikzpicture}[baseline]
\node at (0,0) (1S) {S};
\node [below left=\baselineskip and 12mm of 1S] (2NP) {NP}; \node [below right=\baselineskip and 12mm of 1S] (2VP) {VP};
\node [below=\baselineskip of 2NP] (3Kofi) {Kofi}; \node [below left=\baselineskip and 12mm of 2VP] (3V) {V};
\node [below=\baselineskip of 2VP] (3NP) {NP}; \node [below right=\baselineskip and 24mm of 2VP] (3VP) {VP};
\node [below=\baselineskip of 3V] (4teki) {teki}; \node [below=\baselineskip of 3NP] (4nefi) {a nefi};
\node [below left=\baselineskip and 3mm of 3VP] (5V) {V}; \node [below right=\baselineskip and 3mm of 3VP] (5NP) {NP};
\node [below=\baselineskip of 5V] (6koti) {koti}; \node [below=\baselineskip of 5NP] (6brede) {a brede};

\draw (2NP) -- (1S.south) -- (2VP);
\draw (3Kofi) -- (2NP); \draw (3V) -- (2VP.south) -- (3NP); \draw (2VP.south) -- (3S); \draw (4teki)--(3V); \draw (4nefi)--(3NP); \draw (4NP) -- (3S.south) -- (4VP); 
\draw (6koti)--(5V); \draw (6brede)--(5NP);

\end{tikzpicture}		
}

\section{Ex 252 Pg 128}

\resizebox{\textwidth}{!}{
\begin{tikzpicture}[baseline]
\node at (0,0) (1S) {S};
\node [below left=\baselineskip and 12mm of 1S] (2NP) {NP}; \node [below right=\baselineskip and 12mm of 1S] (2VP) {VP};
\node [below=\baselineskip of 2NP] (3a) {\strut a}; \node [below left=\baselineskip and 6mm of 2VP] (3V) {\strut V}; \node [below right=\baselineskip and 6mm of 2VP] (3VP) {\strut VP};
\node [below=\baselineskip of 3V] (4waka) {\strut waka}; \node [below left=\baselineskip and 3mm of 3VP] (4V) {\strut V}; \node [below right=\baselineskip and 3mm of 3VP] (4PP) {\strut PP}; 
\node [below=\baselineskip of 4V] (5go) {go}; \node [below=\baselineskip of 4PP] (5wosu) {a wosu};

\draw (2NP) -- (1S.south) -- (2VP); \draw (3a)--(2NP); \draw (3V)--(2VP.south)--(3VP); \draw (3V)--(4waka); \draw (4V)--(3VP.south)--(4PP); \draw (4V)--(5go); 
\node [isosceles triangle, isosceles triangle apex angle=100, shape border rotate=90, minimum height=1.25em, draw, below=3mm of 4PP.center] (5polygon) {};
%\draw (5wosu.north east) -- (4PP.south) -- (5wosu.north west) -- cycle; % I don't like this, maybe some optimization is possible?

\end{tikzpicture}	
}

\section{Ex 260 Pg 129}

\resizebox{\textwidth}{!}{
\begin{tikzpicture}[baseline]
\node at (0,0) (1VP) {VP}; \node [above left=\baselineskip and 3mm of 1VP] (0inv) {};
\draw [dashed] (1VP.north)--(0inv);
\node [below left=\baselineskip and 6mm of 1VP] (2V) {V}; \node [below=\baselineskip of 1VP] (2NP) {NP}; \node [below right=\baselineskip and 18mm of 1VP] (2VP) {VP};
\draw (2V)--(1VP.south)--(2NP); \draw (2VP)--(1VP.south);
\node [below=\baselineskip of 2V] (3teki) {\strut teki}; \draw (2V)--(3teki);
\node [rectangle split, rectangle split parts=2, below=\baselineskip of 2NP] (3watra) {watra\nodepart{two}buku}; \draw (2NP)--(3watra);
\node [below left=\baselineskip and 6mm of 2VP] (3V) {\strut V}; \node [below right=\baselineskip and 6mm of 2VP] (3NP) {NP}; \draw (3V)--(2VP.south)--(3NP);
\node  [rectangle split, rectangle split parts=2, below=\baselineskip of 3V] (4gi) {gi}; 
\node [rectangle split, rectangle split parts=2, below=\baselineskip of 3NP] (4plantjes) {\strut plantjes\nodepart{two}mi};
\draw (3V) -- (4gi); \draw (3NP) -- (4plantjes);
\end{tikzpicture}	  }
\section{Ex 261 Pg 129}

\resizebox{\textwidth}{!}{
\begin{tikzpicture}[baseline]
\node at (0,0) (1VP) {VP}; \node [above left=\baselineskip and 3mm of 1VP] (0inv) {};
\draw [dashed] (1VP.north)--(0inv);
\node [below left=\baselineskip and 6mm of 1VP] (2V) {V}; \node [below=\baselineskip of 1VP] (2NP) {NP}; \node [below right=\baselineskip and 18mm of 1VP] (2PP) {PP};
\draw (2V)--(1VP.south)--(2NP); \draw (2PP)--(1VP.south);
\node [below=\baselineskip of 2V] (3teki) {\strut teki}; \draw (2V)--(3teki);
\node [rectangle split, rectangle split parts=2, below=\baselineskip of 2NP] (3watra) {watra\nodepart{two}buku}; \draw (2NP)--(3watra);
\node [below left=\baselineskip and 6mm of 2PP] (3P) {\strut P}; \node [below right=\baselineskip and 6mm of 2VP] (3NP) {NP}; \draw (3V)--(2VP.south)--(3NP);
\node  [rectangle split, rectangle split parts=2, below=\baselineskip of 3P] (4gi) {gi}; 
\node [rectangle split, rectangle split parts=2, below=\baselineskip of 3NP] (4plantjes) {\strut plantjes\nodepart{two}mi};
\draw (3P) -- (4gi); \draw (3NP) -- (4plantjes);
\end{tikzpicture}
	
}

\clearpage
% The following table is not very nice :(
\begin{table}
	\begin{center}
	\begin{tabular}{cccccc}
	\multirow{2}{*}{Rank} & & \multicolumn{4}{c}{Age Group}\\
	\cmidrule{3-6}
	& & 3:7 & 4:7 & 5:6 & 6:6 \\
	\midrule
	1 & & P-1 & P-1 & J-1 & J-1\\
	\begin{tikzpicture}[baseline] \node [circle, radius=1.5] {2}; \end{tikzpicture} & & \begin{tikzpicture}[baseline] \node [circle, radius=1.5] {J-2}; \end{tikzpicture} & \begin{tikzpicture}[baseline] \node [circle, radius=1.5] {J-1}; \end{tikzpicture} & \begin{tikzpicture}[baseline] \node [circle, radius=1.5] {J-2}; \end{tikzpicture} & \begin{tikzpicture}[baseline] \node [circle, draw, radius=1.5] {Jx-5}; \end{tikzpicture}\\
	\begin{tikzpicture}[baseline] \node [circle, radius=1.5] {3}; \end{tikzpicture} & & \begin{tikzpicture}[baseline] \node [circle, radius=1.5] {J-1}; \end{tikzpicture} & \begin{tikzpicture}[baseline] \node [circle, draw, radius=1.5] {Jx-5}; \end{tikzpicture} & \begin{tikzpicture}[baseline] \node [circle, radius=1.5, draw] {Jx-5}; \end{tikzpicture} & \begin{tikzpicture}[baseline] \node [circle, radius=1.5] {J-2}; \end{tikzpicture}\\
	\begin{tikzpicture}[baseline] \node [circle, radius=1.5] {4}; \end{tikzpicture} & & \begin{tikzpicture}[baseline] \node [circle, draw, radius=1.5] {Jx-5}; \end{tikzpicture} & \begin{tikzpicture}[baseline] \node [circle, radius=1.5] {J-2}; \end{tikzpicture} & \begin{tikzpicture}[baseline] \node [circle, radius=1.5] {P-1}; \end{tikzpicture} & \begin{tikzpicture}[baseline] \node [circle, radius=1.5, draw] {Jx-10}; \end{tikzpicture}\\
	\begin{tikzpicture}[baseline] \node [circle, radius=1.5] {5}; \end{tikzpicture} & & \begin{tikzpicture}[baseline] \node [circle, draw, radius=1.5] {Jx-10}; \end{tikzpicture} & \begin{tikzpicture}[baseline] \node [draw, circle, radius=1.5] {Jx-10}; \end{tikzpicture} & \begin{tikzpicture}[baseline] \node [circle, radius=1.5, draw] {P-10}; \end{tikzpicture} & \begin{tikzpicture}[baseline] \node [circle, radius=1.5] {P-1}; \end{tikzpicture}\\
	\begin{tikzpicture}[baseline] \node [circle, radius=1.5] {6}; \end{tikzpicture} & & \begin{tikzpicture}[baseline] \node [circle, draw, radius=1.5] {P-10}; \end{tikzpicture} & \begin{tikzpicture}[baseline] \node [draw, circle, radius=1.5] {P-10}; \end{tikzpicture} & \begin{tikzpicture}[baseline] \node [circle, radius=1.5, draw] {Jx-10}; \end{tikzpicture} & \begin{tikzpicture}[baseline] \node [circle, radius=1.5, draw] {P-10}; \end{tikzpicture}\\
	
	
	\end{tabular}
	\end{center}
	\caption{Rank orders for past-marking frequency}
\end{table}


\clearpage
\section{Fig 3.1 Pg 178}
\begin{figure}[h]
	\begin{center}
	\begin{tikzpicture}[baseline]
	\node at (0,0) (16) {1:6}; 
	\node [below=6\baselineskip of 16, text width=4em, right] (english) {English learner};
	\node [right=4em of english] (base) {base};
	\node [above right=2\baselineskip and 6em of base] (ing) {--ing (NP)};
	\node [below right=2\baselineskip and 6em of base] (base2) {base};
	\node [above right=2\baselineskip and 6em of base2] (irr) {irr. past (P)};
	\node [below right=2\baselineskip and 6em of base2] (base3) {base};
	\node [above right=2\baselineskip and 6em of base3] (reg) {reg. past};
	\node [below right=2\baselineskip and 6em of base3] (base4) {base};
	\draw [-Stealth, thick] (base.east) -- (ing.south west); \draw [-Stealth, thick] (base.east) -- (base2.north west);
	\draw [-Stealth, thick] (base2.east) -- (irr.south west);
	\draw [-Stealth, thick] (base2.east) -- (base3.north west);
	\draw [-Stealth, thick] (base3.east) -- (reg.south west);
	\draw [-Stealth, thick] (base3.east) -- (base4.north west);
	\draw [-Stealth, thick] (reg.north west) -- (irr.south east);
	
	
	
	\node [below=15\baselineskip of english, text width=4em] (italian) {Italian learner};
	\node [right=4em of italian] (itbase) {base};
	\node [above right=2\baselineskip and 6em of itbase] (pp) {past part. (P)};
	\node [below right=2\baselineskip and 6em of itbase] (itbase2) {base};
	\path let \p1 = ($(itbase2) + (0,2\baselineskip)$), \p2=(reg.west) in node at (\x2, \y1) [right] (imp) {imperfect (NP)};
	\path let \p1 = ($(itbase2) - (0,2\baselineskip)$), \p2=(imp.west) in node at (\x2, \y1) [right] (itbase3) {base};
	
	\draw [-Stealth, thick] (itbase2.east) -- (imp.west);
	\draw [-Stealth, thick] (itbase2.east) -- (itbase3.west);
	\draw [-Stealth, thick] (itbase.east) -- (pp.south west);
	\draw [-Stealth, thick] (itbase.east) -- (itbase2.north west);

	% use the processed image to create the heading.
	\path let \p1 = (16), \p2 = (imp.east) in node at (\x2,\y1) (26) {2:6};
	\draw [thick] (16) -- (26);
	\node at ($(16)!.5!(26)$) [above=1\baselineskip] {Time line (very approx.)};	

	
	\end{tikzpicture}
	\end{center}
	\caption{Comparative {\sc tma} acquisition (Italian versus English)}
\end{figure}



\clearpage
\section{Fig. 4.1 Pg 228}
\begin{figure}[h]
	\begin{center}
		\begin{tikzpicture}
		\node at (0,0) [circle, fill, inner sep=3pt] (dot) {};
		\node [dashed, color=red, circle, fit=(dot), inner sep=10pt] (guide) {}; % add 'draw' option here to show guiding help lines.
		\foreach \angle in {90,126,...,540} {
			\draw [thick] (guide.\angle) -- (\angle:2cm); 
		}
			\draw [thick] (guide.270) -- (270:6cm);
		\end{tikzpicture}
	\end{center}
	\caption{The minimal ``flower''}
\end{figure}


\clearpage
\section{Fig 4.3 Pg 247}
\begin{figure}[h]
	\begin{center}
	\begin{tikzpicture}[baseline]
	\node at (0,0) [rectangle split, rectangle split parts=2] (def) {+P +S\nodepart{two}``Definite''};
	\node at (4,0) [rectangle split, rectangle split parts=2] (indef) {--P +S\nodepart{two}``Indefinite''};
	\node at (0,-3) [rectangle split, rectangle split parts=2] (gen) {+P --S\nodepart{two}``Generic''};
	\node at (4,-3) [rectangle split, rectangle split parts=2] (oth) {--P --S\nodepart{two}``Other''};
	
	\node (gr1) [draw, fit=(def) (oth), inner xsep=1.25cm, inner ysep=1cm] {};
	\draw (gr1.north) -- (gr1.center); \draw (gr1.east) -- (gr1.west);
	
	\node [left=\baselineskip of gr1.west] (gr1west) {};
	\node [right=\baselineskip of gr1.east] (gr1east) {};
	\node [above=\baselineskip of gr1.north] (gr1north) {};
	\node [below=\baselineskip of gr1.south, fill=white, inner xsep=.25cm] (gr1south) {\huge ⌀};
	
	\path let \p1=(gr1north.base), \p2=(gr1.north east) in node at (\x2, \y1) [fill=white, left] (gr1NE) {\it wan};
	\path let \p1=(gr1north.base), \p2=(gr1.north west) in node at (\x2, \y1) [right, fill=white] (gr1NW) {\it di};
	
	% fill=white in preceeding \paths and the following pgfonlayer is used so that dashed lines to not cross "wan" and "di"
	
	\begin{pgfonlayer}{bickertonbg}
	\draw [decoration={markings, mark=between positions 0 and 1 step 1 with {\arrow[line width=0.1mm, scale=2]{|}}}, postaction={decorate}, dashed] (gr1west.south)  |- (gr1south.west)  -| (gr1east.south) ;
	\draw [decoration={markings, mark=between positions 0 and 1 step 1 with {\arrow[line width=0.1mm, scale=2]{|}}}, postaction={decorate}, dashed] (gr1west.north) |- (gr1north.west);
	\draw [name path=draw1, decoration={markings, mark=between positions 0 and 1 step 1 with {\arrow[line width=0.1mm, scale=2]{|}}}, postaction={decorate}, dashed] (gr1east.north) |- (gr1north.east);
	\end{pgfonlayer}
	
	\end{tikzpicture}
	\end{center}
	\caption{Semantic Space for Guyanese Articles}
\end{figure}

\clearpage
\section{Fig 4.4 Pg 249}
\begin{figure}[h]
	\begin{center}
		\begin{tikzpicture}[baseline]
		
		% This figure needs some more exact measurements
		
		
		\node at (0,0) [rectangle split, rectangle split parts=2] (def) {+P +S\nodepart{two}``Definite''};
		\node at (4,0) [rectangle split, rectangle split parts=2] (indef) {--P +S\nodepart{two}``Indefinite''};
		\node at (0,-3) [rectangle split, rectangle split parts=2] (gen) {+P --S\nodepart{two}``Generic''};
		\node at (4,-3) [rectangle split, rectangle split parts=2] (oth) {--P --S\nodepart{two}``Other''};
		
		\node (gr1) [draw, fit=(def) (oth), inner xsep=1.25cm, inner ysep=1cm] {};
		\draw (gr1.north) -- (gr1.south); \draw (gr1.east) -- (gr1.west);
		
		% These are auxialiary nodes. I think the figure can be produced more elegantly, but I have not figured out the optimal way yet.
		
		\node [left=1em of gr1.west, fill=white, inner xsep=0cm, inner ysep=.25cm] (gr1west) {\it the};
		\node [left=3em of gr1.west] (gr1west2) {};
		\node [right=1em of gr1.east, inner ysep=0cm] (gr1east) {}; %inner ysep here is to prevent dashed lines from making "breaks" between ends of point.
		\node [right=3em of gr1.east] (gr1east2) {};
		\node [above=\baselineskip of gr1.north] (gr1north) {};
		\node [below=\baselineskip of gr1.south] (gr1south) {};
		\node [below=2\baselineskip of gr1.south, inner xsep=0cm] (gr1south2) {};
		\node [left=4.5em of gr1.west] (gr1west3) {};
		\node [below=3\baselineskip of gr1.south] (gr1south3) {\underline{\large ⌀ \hspace{.75em} \textit{s}}};
		
		
		\path let \p1=(gr1east), \p2=(gr1south2.base) in node at (\x1, \y2) [fill=white] (gr1SE) {\it a};
%		\path let \p1=(gr1north.base), \p2=(gr1.north west) in node at (\x2, \y1) [right, fill=white] (gr1NW) {\it di};
		
		% fill=white in preceeding \paths and the following pgfonlayer is used so that dashed lines to not cross "wan" and "di"
		
	\begin{pgfonlayer}{bickertonbg}
		\draw [decoration={markings, mark=between positions 0 and 1 step 1 with {\arrow[line width=0.1mm, scale=2]{|}}}, postaction={decorate}, dashed] (gr1south.west)  -| (gr1west)  |- (gr1north.west) ;
		\draw [decoration={markings, mark=between positions 0 and 1 step 1 with {\arrow[line width=0.1mm, scale=2]{|}}}, postaction={decorate}, dashed] (gr1north.east) -| (gr1east) |- (gr1south2) -| (gr1west2.south);
		\draw [name path=draw1, decoration={markings, mark=between positions 0 and 1 step 1 with {\arrow[line width=0.1mm, scale=2]{|}}}, postaction={decorate}, dashed] (gr1west3.south) |- (gr1south3) -| (gr1east2.south);
	\end{pgfonlayer}
		
		\end{tikzpicture}
	\end{center}
	\caption{Semantic Space for English articles}
\end{figure}


\clearpage
\section{Fig 4.5 Pg 250}
\begin{figure}[h]
	\begin{center}
	\begin{tikzpicture}[baseline]
	\node at (0,0) (ownership) {Ownership \textit{(a)}};
	\node [below=4\baselineskip of ownership] (possession) {Possession \textit{(get)}};
	\node [right=4cm of ownership] (location) {Location \textit{(de)}};
	\node [below=4\baselineskip of location] (existence) {Existence \textit{(get)}};
	\node [draw, thick, fit=(ownership) (existence), inner xsep=2cm, inner ysep=1cm] (group) {};
	\draw [thick] (group.east) -- (group.west); \draw [thick] (group.north) -- (group.center);
	\end{tikzpicture}
	\end{center}
	\caption{Semantic space for location, etc., in GC}
\end{figure}


\clearpage
\section{Fig 4.6 Pg 252}


\begin{figure}[h]
\begin{center}

%\resizebox{\textwidth}{!}{
\begin{tikzpicture}[baseline]
\node at (0,0) (1All) {All entities};
\node [below left=2\baselineskip and 24mm of 1All] (2S) {--S};
\node [below right=2\baselineskip and 24mm of 1All] (2S2) {+S};
\draw (2S2) -- (1All.south) -- (2S);
\node [below left=2\baselineskip and 24mm of 2S2] (3P) {+P \textit{(the)}};
\node [below right=2\baselineskip and 24mm of 2S2] (3P2) {--P \textit{(a)}};
\draw (3P2) -- (2S2.south) -- (3P);
\node [rectangle split, rectangle split parts=2, below left=2\baselineskip and 3mm of 3P] (4R) {+R\nodepart{two}(ownership)};
\node [rectangle split, rectangle split parts=2, below right=2\baselineskip and 3mm of 3P] (4R2) {--R\nodepart{two}(location)};
\node [rectangle split, rectangle split parts=2, below left=2\baselineskip and 3mm of 3P2] (4R3) {+R\nodepart{two}(possession)};
\node [rectangle split, rectangle split parts=2, below right=2\baselineskip and 3mm of 3P2] (4R4) {--R\nodepart{two}(existence)};
\draw (4R) -- (3P.south) -- (4R2);
\draw (4R3) -- (3P2.south) -- (4R4);

\end{tikzpicture} %  }

\caption{Hypothetical tree structure for semantic primes}
\end{center}
\end{figure}

% \begin{figure}
% %fig 4.7. p 253
% \end{figure}
% \newpage

\newpage
\section{Fig. 4.8 Pg 259}
\begin{figure}[h]
\begin{center}
	\begin{tikzpicture}[baseline, every node/.style={rectangle split ignore empty parts=false}]
	\node at (0,0) (habitual) [rectangle split, rectangle split parts=3] {(habitual)\nodepart{two}\nodepart{three}--S};
	\node (habitualfit) [draw, fit=(habitual), inner sep=1cm] {};
	\node (conditional) [above right=\baselineskip of habitualfit.north east, rectangle split, rectangle split parts=3] {(conditional)\nodepart{two}\nodepart{three}+S}; 
	\node (past) [above left=\baselineskip of habitualfit.north west, rectangle split, rectangle split parts=3] {(past)\nodepart{two}\nodepart{three}+S}; 
	\node (cont) [below left=\baselineskip of habitualfit.south west, rectangle split, rectangle split parts=3] {+S\nodepart{two}\nodepart{three}(continuous)}; 
	\node (future) [right=2\baselineskip of habitualfit.340] {(future)};
	\node (group) [draw, fit=(past) (cont) (conditional), inner sep=3\baselineskip] {};
	\draw (group.south) -- (habitualfit.south); \draw (group.west) -- (habitualfit.west); \draw (group.north) -- (habitualfit.north);
	\node [above left = .5\baselineskip of group.north]  {+R}; \node [above right = .5\baselineskip of group.north]  {--R};
	\node [below left = .5\baselineskip of group.south]  {+R}; \node [below right = .5\baselineskip of group.south]  {--R};
	\node [above left = .5\baselineskip of group.west]  {+P}; \node [below left = .5\baselineskip of group.west]  {--P};
	\node [below=3\baselineskip of group.south] {(R = realis, P = punctual, S = specific)};
	\end{tikzpicture}
\end{center}
\caption{Semantic space around habituals}

\end{figure}


\clearpage
\section{Fig 4.9 Pg 260}
\begin{center}
\begin{figure}[h]
\begin{minipage}{7cm}
	\begin{tikzpicture}[baseline]
	\node at (0,0) (PR1) {+P+R} ;
	\node at (4,0) (R1) {--R} ;
	\node at (0,-4) (PR2) {--P+R} ;
	\node at (4,-4) (R2) {--R} ;
	\node (gr1) [fit=(PR1) (R2), draw, inner sep=2em] {};
	\node at (2,-2) (P1) [inner sep=2em] {--P}; \draw (P1.west) |- (P1.north) -| (P1.east) |- (P1.south); %horizontal centering seems to be off for P1
	\draw (gr1.north) |- (P1.north); \draw (gr1.west) -- (P1.west); \draw (P1.south) -| (gr1.south);
	\end{tikzpicture}
	\caption{(a)}
\end{minipage}	
\begin{minipage}{7cm}
\begin{tikzpicture}[baseline]
	\node at (0,0) (PR1) {+P+R} ;
	\node at (4,0) (R1) {--R} ;
	\node at (0,-4) (PR2) {--P+R} ;
	\node at (4,-4) (R2) {--R} ;
	\node (gr1) [fit=(PR1) (R2), draw, inner sep=2em] {};
	\node at (2,-2) (P1) [inner sep=2em] {+P}; \draw (P1.west) |- (P1.south) -| (P1.east) |- (P1.north); %horizontal centering seems to be off for P1
	\draw (gr1.north) |- (P1.north); \draw (gr1.west) -- (P1.west); \draw (P1.south) -| (gr1.south);
	\end{tikzpicture}
	\caption{(b)}
\end{minipage}
\begin{minipage}{7cm}
	\begin{tikzpicture}[baseline]
	\node at (0,0) (PR1) {+P+R} ;
	\node at (4,0) (R1) {--R} ;
	\node at (0,-4) (PR2) {--P+R} ;
	\node at (4,-4) (R2) {--R} ;
	\node (gr1) [fit=(PR1) (R2), draw, inner sep=2em] {};
	\node at (2,-2) (R3) [inner sep=2em] {--R}; \draw (R3.south) -| (R3.west) |- (R3.north); 
	\draw (gr1.north) -| (R3.north); \draw (gr1.west) -- (R3.west); \draw (R3.south) |- (gr1.south);
	\end{tikzpicture}
	\caption{(c)}
\end{minipage}
\end{figure}

\end{center}

\clearpage
\section{Fig 5.1 Pg 298}
\begin{figure}[h]
	\begin{center}
	\hyphenpenalty=100000000
	\begin{tikzpicture}
	
	\node at (0,0) [text width=4cm, align=center] (domain) {Domain of ``natural languages''};
	\node at (6,3) [text width=3cm, align=center] (limit) {Limit imposed by formal universals};
	
	\node at (0,-2.5) [text width=3cm, align=center, draw, thick, inner xsep=1cm, inner ysep=.6666cm] (bioprog) {Bioprogram language};
	\node (gr1) [draw, thick, inner xsep=1.5cm, inner ysep=1cm, fit = (domain) (bioprog)] {};
	
	\draw (limit.south) edge [very thick, draw, -{Stealth[length=15pt,inset=2pt]}, bend left]  (gr1.10);
	\end{tikzpicture}
	\end{center}
	\caption{Relationship of bioprogram to formal universals}
\end{figure}
\end{center}
\end{document}