
% ISBN 0-89720-044-6 (Hardcover)

First hardcover printing, November 18th, 1981; Second hardcover printing, March 31st, 1982; Third hardcover printing, January llth, 1983

First paperback printing, April I st, 1985: ISBN 0-89720-073-X Copyright © 1981 by Karoma Publishers, Inc.

All Rights Reserved

Printed and Published in the United States of America

{\textbackslash}

\textit{To} \textit{the} \textit{people} \textit{of Palmares,}

\textit{El} \textit{Palenque} \textit{d}\textit{e} \textit{San} \textit{Basilio,} \textit{The} \textit{Cockpit} \textit{Country,}

\textit{and} \textit{the} \textit{Saramacca} \textit{River,} \textit{who} \textit{fought} \textit{for} \textit{decency,} \textit{dignity} , \textit{and} \textit{freedom}

\textit{against} \textit{the} \textit{Cartesian} \textit{savagery} \textit{of} \textit{Western} \textit{colonialists} \textit{and} \textit{slavemak} \textit{ers;}

\textit{whose} \textit{tongues,} \textit{having} \textit{survived}

\textit{to} \textit{confound} \textit{pedagogue} \textit{and} \textit{philosopher} \textit{alike,} \textit{now,} \textit{by} \textit{an} \textit{ironic} \textit{stroke} \textit{ofjustice,}

\textit{offer} \textit{us} \textit{indispensable} \textit{keys} \textit{to} \textit{the} \textit{knowledge} \textit{of} \textit{our} \textit{species.}

\bfseries
\hypertarget{TOC250001}{}ACKNOWLEDGMENTS

The research on which the first two chapters of this volume are based would not have been possible without the support of NSF Grants Nos. GS-39748 and S0C75-14481, for which grateful acknowledgment is hereby made. I am also indebted to the University of \isi{Hawaii} for granting me an additional leave of absence which, together with my regular sabbatical leave, gave me two years in which to work out the theory presented here.

The ideas contained in this volume have been discussed, in person and in correspondence, with many colleagues; while it is in a sense in\-vidious to pick out names, Paul Chapin, Talmy Given, Tom Markey, and Dan Slobin have been among the most long-suffering listeners. I am also grateful to Frank Byrne, Chris Corne, Greg Lee, and Dennis Pres\-ton for reading parts of the manuscript. Needless to say, I alone remain responsible for whatever errors and omissions may still be present.

\isi{'}

BC CNCD CR

DJ

Eng. Fr. GC GU HC HCE HPE

roe

JC KR LAC LAD MC

Pg. PIC PIC PK PNPD

pp

PQ RC SA SC

SNSD SPD SR SSC ST

ABBREVIATIONS AND ACRONYMS

\ili{Belize Creole}

Causative-Non\isi{causative} Distinction \ili{Crioulo}

\ili{Djuka} \isi{English} \ili{French}

\isi{Guyanese Creole} \isi{Guyanais}

\ili{Haitian Creole}

\ili{Hawaiian} Creole \isi{English} \ili{Hawaiian} Pidgin \isi{English} Indian Ocean Creole(s) \ili{Jamaican Creole}

\ili{Krio}

\ili{Lesser Antillean Creole} Language Acquisition Device \ili{Mauritian Creole}

\ili{Portuguese}

Propositional Island Constraint \ili{Providence Island Creole}

\ili{Papia Kristang}

Punctual-Nonpunctual Distinction \ili{Papiamentu}

\ili{Palenquero} \ili{Reunion Creole} \textbf{Saramaccan} \ili{Seychelles Creole}

Specific-Nonspecific Distinction State-Process Distinction

\ili{Sranan}

Specified Subject Condition \ili{Sao Tomense}

I

\hypertarget{TOC250000}{}INTRODUCTION

Of all the fields of study to which human beings have devoted themselves, linguistics could lay claim to being the most conservative. Two thousand five hundred years ago, Panini began it by describing an individual human language, and describing individual languages is what the majority of linguists are still doing. Even during the last couple of decades, in which linguists have begun to be interested in some of the larger issues that language involves, the main thrust toward clarifying those issues has involved making more and more detailed and ingenious descriptions of currently existing natural languages. In consequence, little headway has been made toward answering the really important \isi{questions} which language raises, such as: how is language acquired by the individual, and how was it acquired by the species?

The importance of these \isi{questions} is, I think, impossible to exaggerate. Language has made our species what it is, and until we really understand it-that is, understand what is necessary for it to be acquired and transmitted, and how it interacts with the rest of our cognitive apparatus-we cannot hope to understand ourselves. And unless we can understand ourselves, we will continue to watch in helpless frustration while the world we have created slips further and further from our control.

The larger and, in a popular sense, more human issues which language involves lie outside the scope of the present work, and will be dealt with at length in a forthcoming volume, \textit{Language} \textit{and} \textit{Species.} First, there is a good deal of academic spadework to be done. In the chapters that follow, I shall try to develop a unified theory which will·propose at least a partial answer to three \isi{questions}, none of which has as yet been satisfactorily resolved:


\begin{itemize}
\item How did creole languages originate?
\item How do children acquire language?
\item How did human language originate?
\end{itemize}

\isi{'}

\textbf{\textsubscript{xii }}INTRODUCTION \textsuperscript{INTRODUCTION} \textbf{xiii}

Traditionally, these three \isi{questions}, insofar as they have been treated at all, have been treated as wholly unrelated. None of the solutions offered for (1) have had any relevance to (2) or (3); none 'of the solu\-tions offered for (2) have had any relevance to (1) or (3) ; and none of the solutions offered for (3) have had any relevance to (1) or (2). It has even been explicitly denied, although without a shred of support\-ing evidence, that an answer to (1) could possibly be an answer to (3) \citep{Sankoff1979}. Here and there, a few insightful scholars have hinted at possible links between the problems, and such insights will be ac\-knowledged in subsequent pages. However, a single, unified treatment has never even been attempted, and this book, whatever its short\-comings, may therefore claim at least some measure of originality. Doubtless many of its details will need revision or replacement; the explorer is seldom the best cartographer. However, of one thing I am totally convinced: that the three \isi{questions} are really one question, and that an answer to any one of them which does not at the same \isi{time} answer the other two will be, ipso facto, a wrong answer.

I shall begin with the origin of creoles. To some, this may appear the least general and least interesting question of the three. However, as I shall show, creoles constitute the indispensable key to the two larger problems, and this should come as no surprise to those familiar with the history of science, in which, repeatedly, the sideshow of one generation has been the central arena of the next. In Chapter 1, I shall examine the relationship between the variety of Creole \isi{English} spoken in \isi{Hawaii} and the pidgin which immediately preceded it, and I shall show how several elements of that creole could not have been derived from its antecedent pidgin, or from any of the other languages that were in contact at the \isi{time} of creole formation, and that therefore these elements must have been, in some sense, ``invented.\isi{'}\isi{'} In Chap\-ter 2 I shall discuss some (not all-there would not be space for all) of the features which are shared by a wide range of creole languages and show some striking resemblances between the ``inventions\isi{'}\isi{'} of \isi{Hawaii} and ``inventions\isi{'}\isi{'} of other regions which must have emerged

quite independently ; and I shall also try to probe more deeply into

\isi{'}

certain aspects of creole syntax and semantics which may prove signifi\-cant when we come to deal with the other two \isi{questions}. In Chapter 3, which will deal with ``normal\isi{'}\isi{'} language acquisition in noncreole societies, I shall show that some of the things which children seem to acquire effortlessly , as well as some which they get consistently wrong\-both equally puzzling to previous accounts of ``language \isi{learning}{\textquotedbl}{}- follow naturally from the theory which was developed to account for creole origins: that all members of our species are born with a bio\-program for language which can function even in the absence of ade\-quate input. In Chapter 4, I shall try to show where this bioprogram comes from: partly from the species-specific structure of human perception and cognition, and partly from processes inherent in the expansion of a linear language. At the same \isi{time}, we will be able to resolve the continuity paradox ({\textquotedbl}language is too different from animal communication systems to have ever evolved from them{\textquotedbl}; ``language, like any other adaptive mechanism, must have been derived by regular evolutionary processes{\textquotedbl}) which has lain like some huge roadblock across the study of language origins. In the final chapter, I shall briefly summarize and integrate the findings of previous chapters, and suggest answers to some of the criticisms which may be brought against the concept of a genetic program for human language.

\textit{Chapter} \textit{1}

\textbf{PIDGIN} \textbf{INTO} \textbf{CREOLE}

If one wants to account, ultimately, for the \isi{origins of} human language (as I shall try to do in Chapter 4), it seems reasonable for one to begin by trying to find out how individual human languages came into existence. But, in most cases, such a search would be futile. Mod\-ern \isi{Italian}, for example, would be found to fade back into a maze of dialects deriving ultimately from \ili{Latin}, which developed out of Indo\-European, which sprang, presumably, from some antecedent language now wholly inaccessible to us; :md at no point in the continuous transmission of language could we name a date and say, ``Here \ili{Latin} ended,\isi{'}\isi{'} or ``Here \isi{Italian} began.{\textquotedbl}

But there is one class of languages for which we can point, with reasonable accuracy, to the year of birth: we can say that before 1530, there was no \ili{Sao Tomense}; before 1650, no \ili{Sranan}; before 1690, no \ili{Haitian Creole}; and before 1880, no \ili{Hawaiian} Creole. And yet two or three decades after those dates, those languages existed. Of course no one would claim that these languages were quite devoid of ancestry; indeed, their relationships with several sets of putative ancestors have been and continue to be a subject of controversy (for a critical sum\-mary and bibliography of the relevant literature, see Bickerton 1976).

{\textbackslash}

%\originalpage{2} }PIDGIN INTO CREOLE\textsubscript{ }\textsubscript{3}

But even controversy could not exist unless the lines of descent were, at the very least, considerably more obscure than they are for most other languages.

Creole languages arose as a direct result of European colonial

expansion. Between 1500 and 1900, there came into existence, on tropical islands and in isolated sections of tropical littorals, small, autocratic, rigidly stratified societies, mostly engaged in mono\isi{culture} (usually of sugar), which consisted of a ruling minority from some European nation and a large mass of (mainly non-European) laborers, drawn in most cases from many different language groups. The early linguistic history of these enclaves is virtually unknown; it is generally assumed (but see Alleyne 1971, 1979) that speakers of different lan\-guages at first evolved some form of auxiliary contact-language, native to none of them (known as a \textit{pidgin} \textit{)} , and that this language, suitably expanded, eventually became the native (or \textit{creol}\textit{e}\textit{) }language of the community which exists today. These creoles were in most cases different enough from any of the languages of the original contact situation to be considered ``new\isi{'}\isi{'} languages. Superficially, their closest resemblance was to their European parent, but this was mainly because the bulk of the vocabulary items were drawn from that source, and even here, there were extensive phonological and semantic shifts. In the area of syntax, features were much less easily traceable.

In general, the term \textit{creole} is used to refer to any language which was once a pidgin and which subsequently became a native language ; some scholars have extended the term to any language, ex-pidgin or not, that has undergone massive structural change due to language contact (one who shall be nameless confessed to me that he did this solely to obtain access to a conference which, like most creole conferences, was held in an exotic tropical setting!). In fact, I think that even the traditional definition is too wide, since it covers a range of situations which may differ in kind rather than in degree.

Since my aim here is not to account for the \isi{origins of} all lan\-guages known as creoles (which would be an absurd aim anyway since they do not constitute a proper set\{ but rather to search for certain

fundamental properties of human language in general, my interests lie, not in creoles per se, but in situations where the normal continuity of language transmission is most severely disrupted. While it is true that the circumstances under which \isi{pidginization} and creolization take place represent ``a catastrophic break in linguistic tradition that is unparalleled\isi{'}\isi{'} \citep[24]{Sankoff1979}, there are still a number of areas where the severity of that break was mitigated by other factors. Let us consider two quite different cases of such mitigation: \ili{Reunion Creole} and \ili{Tok Pisin} (a.k.a. Neo-Melanesian, \isi{New Guinea} Pidgin, etc.).

One factor that would limit the extent of language disruption would be the presence of any large homogeneous linguistic group in the community-more especially if that group happened to consist of speakers of the dominant language. According to \citet{Chaudenson1974}, in Reunion during the first few decades of settlement nearly half the population consisted of native speakers of \ili{French} ; in conse\-quence, although the resultant language bears the creole label, the distance between that language and \ili{French} is much less than the distance between most creoles and their superstrates, while (more important still from the present viewpoint) the language differs in many respects from creoles formed where access to the superstrate was more restricted.

In \isi{New Guinea}, the percentage of superstrate speakers was low, but the pidgin existed for several generations alongside the indigenous language before it began to acquire native speakers. Thus \ili{Tok Pisin} was able to expand gradually, through normal use, rather than very rapidly, under the communicative pressure of a generation that had, for practical purposes, no other option available as a first language. The bilingual speakers of \ili{Tok Pisin} had ongoing lives in their own languages and, perhaps more importantly still, in their own traditional communities; whereas in the classic creole situation, people had been torn from their traditional communities and forced into wholly novel communities in which the value of traditional languages was low or nil. The two situations are not commensurate, and we would expect

  


 

%\originalpage{4}

to find, as we do, that while \ili{Tok Pisin} differs from \isi{English} much more than \ili{Reunion Creole} does from \ili{French}, it lacks, again, a number of the features found in the classic creole languages, and possesses a number of features which those creoles, in turn, do not share.

Accordingly, in the text that follows, I shall use the word \textit{creole}

to refer to languages which:


\begin{itemize}
\item Arose out of a prior pidgin which had not existed for more than a generation.
\item Arose in a population where not more than 20 percent were
\end{itemize}

native speakers of the dominant language and where the remaining 80 percent was composed of diverse language groups.

The first condition rules out \ili{Tok Pisin} and perhaps other (e.g., Austra\-lian Aboriginal) creoles; the second rules out \ili{Reunion Creole} and perhaps other creoles also (the varieties of \ili{Portuguese} creoles that evolved in Asian trading enclaves such as \isi{Goa} or \isi{Macao} are possible candidates for exclusion under this condition). Given the above, I shall continue to refer to certain languages or groups oflanguages as ``\ili{\isi{English} creoles},{\textquotedbl} ``\ili{French} creoles,\isi{'}\isi{'} etc.; this usage implies no conclusions as to the affiliations of these languages and \textit{is} merely for convenience.

By limiting our research area in this way , it becomes possible to concentrate on those situations in which the human linguistic capacity is stretched to the uttermost. As I have said, we know little or nothing of the early linguistic history of most creoles, but what evidence we do have (e.g., Rens 1953, for \ili{Sranan}) suggests that they emerged from the pidgin stage fairly rapidly, within twenty or thirty years after first settlement of the areas concerned. Such a \isi{time} span gives space for the first locally-born generation to come to maturity, but it hardly gives space for a stable, systematic, and referentially adequate pidgin to be evolved in a community which might initially speak dozens of mutually unintelligible languages; certainly, in the one case of which we have direct knowledge (\isi{Hawaii}), no stale, systematic, or referentially

%\originalpage{5}

adequate pidgin had developed within that \isi{time} frame, and there are no real grounds for supposing that the \ili{Hawaiian} situation was any less favorahle to the development of such a pidgin than were the situations in other creole-forming regions. We can assume, therefore, that in each of these regions, immediately prior to creolization, there existed, just as there existed in \isi{Hawaii}, a highly variable, extremely rudimentary language state such as has been sometimes described as a ``jargon\isi{'}\isi{'} or ``pre-pidgin continuum\isi{'}\isi{'} rather than a developed pidgin language. Since none of the available vernaculars would permit access to more than a tiny proportion of the community, and since the cultures and communities with which those vernaculars were associated were now receding rapidly into the past, the child born of pidgin-speaking parents would seldom have had any other option than to learn that rudi\-mentary language, however inadequate for human purposes it might he.

We should pause here to consider the position of such children, for no one else has done so, even in the vast literature on language acquisition. That position differs crucially from the position of children in more normal communities. The latter have a ready-made, custom· validated, referentially adequate language to learn, and mothers, elder siblings, etc., ready to help them learn it. The former have, instead, something which may be adequate for emergency use, but which is quite unfit to serve as anyone's primary tongue; which, by reason of its variability , does not present even the little it offers in a form that would permit anyone to learn it; and which the parent, with the best will in the world, cannot teach, since that parent knows no more of the language than the child (and will pretty soon know less). Every\-where else in the world it goes without saying that the parent knows more language than the child; here, if the child is to have an adequate language, he must speedily outstrip the knowledge of the parent. Yet every study of first-language acquisition that I know of assumes without question that the more general situation is universal; \textit{every} \textit{existing} \textit{theory} \textit{of} \textit{acquisition }\textit{is} \textit{based on} \textit{the} \textit{presupposition that} \textit{there} \textit{is} \textit{always} \textit{and} \textit{everywhere} \textit{an} \textit{adequate} \textit{language} \textit{to} \textit{be acquired.}

%\originalpage{6} }PIDGIN INTO CREOLE\textsubscript{ }\textsubscript{7}

It is true that the situation I am describing is extremely rare and can indeed occur only once even for a creole language. However, the rarity or frequency of a phenomenon contains no clues as to its scientific importance.

The act of ``expanding\isi{'}\isi{'} the antecedent pidgin, which each

first creole generation has to undertake, involves, among other things, acquiring new rules of syntax. In the conventional wisdom, children are supposed to derive rules by processing input (with or without the help of some specific language-\isi{learning} device) ; in this way, they arrive at a rule system similar to, if not identical with, that of their elders. If this were all children could do, then they would simply learn the pidgin, and there would be no significant gap between the generations. In \isi{Hawaii}, at least, we have empirical proof that this did not happen-that the first creole generation produced rules for which there was no evidence in the previous generation's speech.

How can a child produce a rule for which he has no evidence? No one has answered this question; most people haven't even asked it; and yet, until it is answered, we cannot really claim to know any\-thing about how languages in general are acquired. For it violates both parsimony and common sense to suppose that children use one set of \isi{acquisition strategies} for ``normal\isi{'}\isi{'} acquisition situations, and then switch to another set when they frnd themselves in a pidgin\-speaking community: parsimony because two explanations would be required where one should be adequate, and common sense because there is no way in which a child could tell what kind of community he had been born into, and therefore no way he could decide which

set of strategies to use.

I shall return to the topic of acquisition in Chapter 3; in the present chapter, I shall simply describe what happened in \isi{Hawaii} when pidgin turned into creole.

For a century after the first European contact, the population of \isi{Hawaii} consisted mainly of native Hawaiians, with a small but growing minority of native Englis speakers. A small but growing

minority of Hawaiians spoke \isi{English} with varying degrees of profi\-ciency; the lower end of this spectrum acquired the name \textit{hapa-haole} 'half-white\isi{'}. \textit{Hapa-haole} was not a true pidgin, or even a true pre\-pidgin, in the sense discussed above; rather it was a continuum of ``foreigner's \isi{English},\isi{'}\isi{'} similar to Whinnom\isi{'} s (1971) \isi{description of} \textit{\ili{cocoliche}.} It was, apparently, limited to the towns, still hardly heard in country areas even in the 1870s. There was a small sugar industry, but the labor force was almost entirely \ili{Hawaiian}, and, as far as one can discover, the plantation work-language was \ili{Hawaiian}.

In 1876, a revision of U.S. tariff laws allowing the free importa\-tion of \ili{Hawaiian} sugar caused the industry to increase its productivity by several hundred percent. The native \ili{Hawaiian} population had so declined in numbers that workers had to be imported, first from \isi{China}, and then from \isi{Portugal}, \isi{Japan}, \isi{Korea}, the Philippines, \isi{Puerto Rico}, etc. In a few years there came into existence a multilingual community that greatly outnumbered the former population, \ili{Hawaiian} and \textit{haole} alike.

A pidgin \isi{English} was probably not the first, and certainly not the only, contact language used on the \ili{Hawaiian} plantations after 1876. A pidgin based on \ili{Hawaiian} and known as \textit{olelo} \textit{pa'i'ai }'taro\-language\isi{'} {}-so called because it allegedly originated with the \ili{Chinese}, who largely took over taro growing from the native Hawaiians-was widely used in the last two decades of the 19th century (Bickerton 1977 i307-8) and perhaps for some years afterward, since one speaker, a Filipino, was still alive in the mid-1970s. So far, it has proved impos\-sible to determine whether pidgin \isi{English} grew up alongside pidgin \ili{Hawaiian} or whether the former grew out of the latter by a gradual relexiflcarion process. The two processes may not be mutually exclu\-sive, especially when we consider that population balances and other demographic factors differed widely from island to island. But one certain consequence of the existence of \textit{olelo} \textit{pa'i'ai} is that it delayed the development of any form of pidgin \isi{English}, especially on the outer islands, where the \ili{Hawaiian} language was strongest.

In 1973 and 1974, I and my team of assistants made recordings

%\originalpage{8} PIDGIN} \textsuperscript{INTO} \textsuperscript{CREOLE} 9

of several hundred hours of speech from both immigrant speakers of \ili{Hawaiian} Pidgin \isi{English} (HPE) and locally-born speakers of \ili{Hawaiian} Creole \isi{English} (HCE); this work is described in detail in Bickerton and \citet{Odo1976}, Bickerton and Giv6n (1976), and Bickerton (1977 ). The earliest immigrant arrival in the group we recorded was 1907, a \isi{time} which , if our theory about the delayed development of HPE is correct, may have been only a few years after the beginning of HPE. The latest arrival was 1930. [n other words, a period of from forty\-three to sixty-six years had elapsed between our subjects\isi{'} dates of arrival and our recordings. Can recordings made after such a long period give us an adequate idea of what HPE was like in the period

1907-1930?

Although it is widely held that au individual's speech changes little after maturity has been reached, this may not necessarily be true of second languages or contact languages. Pidginization is a ptocess, not a state, and it is therefore possible that at least some of our subjects may now speak differently from the way they spoke when they first arrived in \isi{Hawaii}, even though the vast majority were already adults at that \isi{time}. However, one thing is certain. If their version of HPE has changed in those intervening years, then it mnst be more complex in structure and less subject to idiosyncratic or ethnic-group variation than it was in the years that immediately followed their arrival. It is unthinkable that after several decades of life in a community that was steadily becoming more integrated their version of HPE should have grown less complex or more idiosyncratic. We must therefore assume that either their HPE now adequately represe11ts the HPE of the early \isi{pidginization} period, or that the latter was even more primitive and more unstable than the versions they use today.

But even if modem HPE represents early HPE quite accurately,

it does not follow that all HPE speakers are equally good guides to the state of HPE as it was when creolization took place. On the basis of evidence discussed at length in Bickerton ( 1977 ), we can place the \isi{time} of creolization somewhere around 1910, and certainly no later than 1920. There are considerable differences between the HPE spoken by

{\textbackslash}

the earliest arrivals among our subjects and that spoken by those who arrived in the 1920s. The former is considerably more rudimentary in its structure; the complications that developed after 1920 could have been due to internal developments in HPE, but were more probably

caused by feedback from the newly-developed HCE, whose earliest speakers would have come to maturity by 1920 if not before (this issue, too, is explicitly dealt with in Bickerton [1977 :Chapter 4]). We shall therefore make the reasonable assumption that at the \isi{time} of creolization HPE was either adequately represented by our recordings of earlier (pre-1920) immigrants, or it was at a still more primitive

level of development.

At first glance, the second possibility might seem hard to credit. The HPE of the older surviving speakers is both highly restricted and highly variable. The main source of instability is first-language influ\-ence. \citet{Labov1971} claimed that these transference-governed versions of HPE were the idiosyncratic inventions of social isolates; our much more widely-based research indicates that instead they represent one of the earliest stages in the \isi{pidginization} process in which the more isolated the speakers, the more likely they are to become fossilized. However, speakers who produced such versions were by no means all socially isolated, and in particular, we noted that speakers of more evolved versions of HPE would sometimes relapse into this mode when they became excited or when they had to deal with complex or un\-familiar topics (the speaker who produced !121 below also produced, in the middle of a long and exciting narrative, \textit{!21).}

Typical of very early HPE as produced by speakers born in \isi{Japan} are the following:

\ea\label{ex:1}
 mista karsan-no \textit{to}\textit{k}\textit{oro} tu eika sel \textit{shite} Mr. Carson-POSS place two acre sell do 'I sold two acres to Mr. Carson's place\isi{'}
\glt
\z

\ea\label{ex:2}
 \textit{sore} \textit{kara} kech \textit{shite} \textit{k} \textit{ara} pul ap and then catch do then pull up
\glt
\z

'When he had caught it, he pulled it up\isi{'}

%\originalpage{10}}

In these examples, the italicized lexical items are \ili{Japanese}, and ana\-phora is maintained by zero forms rather than by pronouns. In \textit{I} 1/ the structure (with both direct and indirect objects preceding the verb and the auxiliary following the main verb) represents direct transference from \ili{Japanese} syntax. Example /2/ can hardly be said to have anything recognizable as structure; lexical items from the \isi{English} and \ili{Japanese} lexicons are simply strung together. But what is striking about these two sentences is that six out of the seven \ili{Japanese} morphemes are grammatical, not lexical, items; it is as if the speakers felt the need for some kind of grammatical glue with which to stick their sentences together, and perforce used the only brand available to them.

·Speakers who immigrated into areas where there was a large native-\ili{Hawaiian} population show a rather different tendency ; here, it is \ili{Hawaiian} rather than native-language vocabulary that is mixed with \isi{English} items:

\ea\label{ex:3}
 ifu laik meiki, mo beta \textit{mak} \textit{e} tairn, mani no kaen \textit{hapai}
\glt
\z

if like make, more better die \isi{time}, money no can carry

'If you wanted to build (a temple), you should do it just before you die-{}-you can't take it with you!\isi{'}

\textit{I} 4\textit{I }\textit{Luna,} Im \textit{hapai?} \textit{Hapai} awl, \textit{hemo} awl Foreman, who carry ? Carry all, cut all

'Who'll carry it, boss? Everyone'll cut it and everyone'll carry it\isi{'}

Example /3/ was uttered by a \ili{Japanese} speaker, /4/ by a Filipino speaker; note that the predominantly OV syntax of the former is replaced by a predominantly VS syntax in the latter, reflecting the

That the macaronic elements in these examples may represent a deliber\-ate str:tegy on the part of speakers is suggested by the following perceptive comment from an old \ili{Hawaiian} woman:

\textit{15/} So we use the \ili{Hawaiian} and \ili{Chinese} together in one sentence, see? And they ask me if that's a \ili{Hawaiian} word, I said no, maybe that's a \ili{Japanese} word we put it in, to make a sentence with a \ili{Hawaiian} word. And the \ili{Chinese} the same way too, in order to make a sentence for them to understand you.

In other w\textsubscript{.}ords, in the original linguistic meltincr pot from which HPE venrnally issued, the more skilled speakers acquired a core vocabulary

m which the commonest items, both lexical and grammatical, might be represented by forms drawn from three or four languages. Small won\-der that a \ili{Japanese} woman, asked if she spoke \isi{English} , answered: ``No, \textit{haf{\textquotedbl}{\textquotedbl}har:a} [Hawaiia;t \isi{'}.half-half\isi{'}] \textit{shite} [\ili{Japanese} 'do\isi{'}] ``{}-i.e., 'I speak a

nuxture {}-and added (m \ili{Japanese}), ``I never know whether I'm speaking one thing or the other.{\textquotedbl}

Even at a subsequent stage of \isi{pidginization}, represented by speaker whose vocabulary is drawn predominantly from \isi{English}, sntacti features characteristic of their native languages will still distmgmsh, for example, \ili{Japanese} from Filipino speakers. The former continue to produce sentences such as /6/ and /7/, with final verbs:

\textsuperscript{/6/ }tumach mani mi tink kechi do plenty money I think catch though

'I think he earns a lot of money, though\isi{'}

speaker's native \ili{Visayan}. Here, however, all the \ili{Hawaiian} items carry lexical meaning, in contrast with the \ili{Japanese} items in /1/ and /2/; this lends support to the claim that even as late as the 1910s, in some

\ea\label{ex:7}

\glt
\z

da pua pip!awl poteito it

'The poor people ate only potatoes\isi{'}

areas, either \textit{olelo} \textit{pa'i'ai }was still dominant, or its relexiflcation by

\isi{English} was still far from complete.

Certainly examples /1/-/4/ suggest an extreme instability in the language model that would confron the first locally-born generation.

Filipinos, however, often produce sentences in which verbs or predicate \isi{adjectives} precede their subjects:

\ea\label{ex:8}
 wok had dis pipl
\glt
\z

'These people work hard\isi{'}

%\originalpage{12} PIDGIN} \textsuperscript{INTO} \textsuperscript{CREOLE} 13

\ea\label{ex:9}
mo plaeni da ilokano en da tagalog
\glt
\z

'Ilocanos were more numerous than Tagalogs\isi{'}

The patterns of /6/-/9/ were probably never categorical for any speaker; all the speakers in onr sample showed some SVO synt\isi{'}.{\textquotedbl}{\textquotedbl}. Variation, however, was fairly unpredictable; \ili{Japanese} speakers vaned

between 30 percent and 60 percent of SOV sentences, although the figures for particular sentence types might range between 1? ercem

and 90 percent (see Bickerton and Giv6n [1976] for full details!{\textgreater} while among Filipinos, percentages of VS structures ranged between 15 percent and 50 percent in sentences where S was a full noun rather than a pronoun. On the other hand, VS sentences om \ili{Japanese} speake:s d OV sentences from Filipino speakers, while extremely rare, did

\isi{articles}, \ili{Japanese} speakers rarely used either definite or indefinite; Filipino speakers, on the other hand, often over-generalized the definite article, as in /9/ above. While both groups relied on zero anaphora more than \isi{English} does, pronouns were far more frequent among Filipino speakers. In short, anyone (in particular a child) trying to learn HPE would have encountered formidable obstacles to even figuring out what the rules of HPE were supposed to be.

But its variability was by no means the only obstacle to child \isi{acquisition of} HPE. The presence of two conflicting models, A and B, still leaves the learner three theoretical choices: learn A, learn B, or learn some mixture of A and B. But when neither of the models, nor the two together, constitutes an adequate variety of human language, the problem is of a different order altogether.

:cur from \isi{time} to \isi{time}. Since the \ili{Japanese} and the Filipinos consti\-

Let us be quite clear as to what the deficiencies of HPE were, for

tuted the two largest immigrant groups, a child in \isi{Hawaii} who sought to learn basic \isi{word order} by inductive processes alone would have ended up in a state of total bewilderment.

Other features besides \isi{word order} distinguish \ili{Japanese} from

Filipino speakers. Among \ili{Japanese}, when-clauses were frequently expressed by compound nominals such as:

\ea\label{ex:10}
 as-bihoa-stei-taim us-before-stay-\isi{time}
\glt
\z

'when we used to live here\isi{'}

Filipinos never used such expressions, except. fr \textit{sma\isi{'}:\isi{'}.} \textit{tim} 'when I was young\isi{'}, which became a universal HPE 1d10m. Fil1pmo peakers inserted pronouns between most full-noun subjects and their verbs,

for example:

\ea\label{ex:11}
josafin brada hi laik \textit{hapai} mi
\glt
\z

'Josephine's brother wants to take me (with him)\isi{'}

\ili{Japanese} speakers seldom if ever ued this structure. With regard to

it has been claimed (e.g., Samarin 1971) that anything at all can be said in a pidgin. There is a sense in which this is probably correct, even of an immature pidgin like HPE, provided we do not count the cost of saying it. Take the following remarkable speech:

\ea\label{ex:12}
 samtaim gud rod get, samtalm, olsem hen get, enguru ['angle\isi{'}] get, no? enikain seim. olsem hyuman laif, olsem. gud rodu get, enguru get, mauntin get-no? awl, enikain, stawmu get, nais dei get-olsem. enibadi, mi olsem, smawl taim.
\glt
\z

'Sometimes there's a good road, sometimes there's, like, bends, corners, right? Everything's like that. Human life'sjust like that. There's good roads, there's sharp corners, there's mountains\-right? All sorts of things, there's storms, nice days-{}-it's like that for everybody, it was for me, too, when I was young\isi{'}

This philosophic statement would be a striking piece of rhetoric in any language. But it is an achievement against the grain of the language, so to speak; the speaker, a retired bus driver (which probably accounts for his choice of imagery), triumphs by sheer force of imagination over

%\originalpage{14}

the minimal vocabulary and narrow range of structural options within which he is obliged to work,

Similarly, HPE does not prevent speakers from finding ingenious ways of replacing lexical items which they lack or are unsure of; here is

% PIDGIN INTO CREOLE\textsubscript{ 15}

\textsuperscript{/15/} aena tu macha ehuren, samawl ehuren, haus mani pei

and too much children, small children, house money pay

'And I HAD too many children, small children, I HAD to pay the rent\isi{'} (Korean speaker)

one subject who cannot recall \textit{library:}

\ea\label{ex:13}
 rai ., . rai . . . \textit{ano} buk eniting boro \textit{dekiru} \textit{tokoro}
\glt
\z

Ii. , . li. ..that book anything borrow can place

'Li . . . Li , ..That place where you can borrow any of the books\isi{'}

\ea\label{ex:16}

\glt
\z

bihoa mil no moa hilipino no nating

before mill no more Filipino no nothing

'Before the mill WAS BUILT, THERE WERE no Filipinos here at all\isi{'} (\ili{Japanese} speaker)

If it were the case that children simply induced rules from input,

It is not absolutely necessary, for communicative purposes, that a language have either an extensive vocabulary or a variety of syntactic structures; but the goals of language, whether social communication or mental computation, seem to be better served if a language has these things. HPE lacks, wholly or partially, many of the building blocks which all native languages possess. Among HPE speakers who arrived prior to 1920, the following features are largely or completely missing: consistent marking of tense, aspect, and modality; relative clauses; \isi{movement rules}, embedded \isi{complements}, in particular infinitival con\-structions; \isi{articles}, especially indefinite. On the rare occasion when such features do appear, they often do so in forms modeled directly on the speaker's native language{}-for example, the relative clauses that precede, rather than follow, their head-nouns that are sometimes produced by \ili{Japanese} speakers:

\ea\label{ex:14}
 aen luk laik pankin kain get
\glt
\z

and look like pumpkin kind get

'And there were some that looked like pumpkins\isi{'}

For the most part, however, sentences would consist of short strings of nouns and verbs paratactically linked. Often even verbs would be omitted, as in the following two examples:

\isi{'}

one might suppose that when children were born to HPE speakers they learned the grammars of their parents. If their parents were Filipinos, they would learn the rules characteristic of Filipino speakers; if their parents were \ili{Japanese}, they would learn the rules characteristic of Jp{\textquotedbl}:'ese speakers, and so on. One might argue that when \ili{Japanese} and Fil1prno children went to school they met one another and ironed out their differences; but if this, or something like it, did rake place, it must ave had mre to do with their being children than with their being rn contact with one another. Fifty years of contact were not enough to

erase language-group differences from the speech of adults, And while similar phenomena have been observed among children of immigrant groups on the U.S. mainland, it must be remembered that .the latter

had a ready-made target, while the first creole generation in \isi{Hawaii} did not.

Whatever processes were involved, the erasure of group differ\-ences in that generation was complete. Even other locally-born persons cannot determine the ethnic background of an HCE speaker by his speech alone, although the same persons can readily identify that of an HPE speaker by listening to him for a few seconds.

Now it is true that we could construct an argument similar to that already constructed for HPE speakers. The reader will recall the claim that while the contemporary speech of old HPE speakers may be the same as, or more developed than, their speech shortly after \isi{time} of

%\originalpage{16}

arrival, it could hardly be less developed. Similarly, one might claim that older HCE speakers do not necessarily speak now as they spoke in their childhood or early maturity; that, again, it would be absurd to suggest that they then spoke a variety more developed than they speak now; and that, therefore, their speech may have changed and become considerably more complex since they reached adulthood. Indeed, on this showing, they might have spoken , as children, varieties as rudimentary and as ethnic-tongue-influenced as their parents did; subsequently, and very gradually, they could have developed the more stable, yet more complex, variety of!anguage that they use today. If this were true, the apparently sudden break between HPE and HCE would be a misleading artifact of the analysis, produced by back\-projection from synchronic data.

Since we lack direct evidence from the period in question, this argument cannot be conclusively disproved. However, it is an im\-plausible one, and for the following reason. If the argument is correct , then the homogeneity of modern HCE must have come about by a gradual leveling process in which group differences were gradually removed through intergroup contacts. What were the critical differ\-ences between the immigrant and first locally-born generations? Not, apparently, bilingualism versus monolingualism, since all the older, locally-born subjects we interviewed spoke at least one other language besides HCE when they were children. The only significant difference between the two generations is that the first encountered HPE as adults, while the second encountered it as small children.

Similar arguments can be mounted with regard to the greater complexity of HCE. Again, we cannot prove empirically that this complexity did not result from gradual increment. If we assume that it did, however, we have to explain why HPE did not also become more complex; and we can only conclude, again, that such an explanation must lie in the difference between language-\isi{learning} by adults and language-\isi{learning} by children.

Finally, as we will see in Chapter 2, the forms and structures arrived at by HCE resemble far beyond the scope of chance the forms

\isi{'}

%\originalpage{17}

and structures arrived at by a variety of other creole languages, often with substrata very different from \isi{Hawaii}'s (and from one another's too). It defies belief that a language formed by the leveling of several substratum-influenced versions of a pidgin should exhibit the degree of identity that will be illustrated with languages so diverse in their origins, all of which must have evolved in a similar manner; the odds against this happening, unless some set of external guiding principles was condi\-tioning the result, must be fantastic. It seems reasonable, therefore, to assume that the gap between HPE and HCE that is reflected in our data is a gen uine phenomenon, accounted for by extremely abrupt changes :'hich took place while the first creole generation was growing

\textit{to} maturity.

I shall now examine some of the substantive differences between HPE and HCE in the following five areas:


\begin{itemize}
\item \isi{movement rules}
\item \isi{articles}
\item verbal auxiliaries
\end{itemize}

d) \textit{for-to }complementization

e) \isi{relativization} and pronoun-copying

I claimed above that HPE had no \isi{movement rules}. In fact, HPE could not have had any \isi{movement rules} if we use the term in a rather restricted sense to cover processes such as those which convert sen· tences like /17/ and /19/ into sentences like /18/ and /20/:

\textit{\{17}\textit{/ }I spoke \textit{to} John.

\ea\label{ex:18}
 It was John that I spoke to.
\glt
\z

\ea\label{ex:19}
 Mary loaned us a book.
\glt
\z

\ea\label{ex:20}
 The one who loaned us a book was Mary.
\glt
\z

Rules of this kind are generally associated with certain functions, e.g.,

%\originalpage{18}

that of focusing one particular constituent of a sentence, and they perform this function in some \isi{English} cases by adding structure but always by changing the basic, unmarked \isi{word order} of the sentence.

We saw that in HPE there were several possible sentence orders: SVO, for all speakers sometimes; SOV, very often for \ili{Japanese} speak\-ers; and VS, quite often for Filipino speakers. However, sfuce \ili{Japanese} speakers hardly ever produced VS sentences, and Filipino speakers hardly ever produced SOV sentences, the use of the non-SVO structures could hardly indicate focns or any similar emphatic device; they served merely as (probably unintentional) signals of ethnicity. Even within a

group-say; the \ili{Japanese}-it could hardly be the case that SVO (or SOV) represented an unmarked order, while SOV (or SVO) represented a marked order. For instance, if SOV were the basic order, and SVO a marked order, those speakers who produced only 30 percent SOV would be using their marked order more than twice as often as their basic order. But if the relationship were reversed, the result would be just as unlikely for those speakers who produced 60 percent SOV. If there were two groups, one with basic SOV and marked SVO, and the other with basic SVO and marked SOV, then it would become impossible for the listener to be sure whether contrastive emphasis was or was not intended, and the whole purpose of \isi{movement rules} would be lost..We can therefore assume that differences in \isi{word order} among HPE speakers are not the result of \isi{movement rules} but are due to a gradual transition from VS or SOV orders, unmarked in the speakers\isi{'} native languages, to the equally . unmarked SVO which characterizes almost all contact languages.

In HCE, the situation is quite different. HCE is homogeneous (except to the extent that it has been increasingly influenced by \isi{English} in recent years) both across and within all groups irrespective of the parents\isi{'} language background. For all speakers, without question, the basic, unmarked \isi{word order} is SVO. All speakers, however, have rules that will move either objects-/21/, /22/-or predicates-/23/, /24/-to the beginning of the sentence:

{\textbackslash}

%\originalpage{19}

\ea\label{ex:21}
 \textit{eni} \textit{k} \textit{ain} \textit{lanwij} ai no kaen spik gud
\glt
\z

'I can't speak any kind of language well\isi{'}

\ea\label{ex:22}
 o, \textit{daet} \textit{wan} ai si 'Oh, I saw that one\isi{'}
\glt
\z

\ea\label{ex:23}
 \textit{es} \textit{wan} \textit{ting} \textit{baed} dakain go futbawl 'That football stuff is a had thing\isi{'}
\glt
\z

\ea\label{ex:24}
 \textit{daes} \textit{leitli} dis pain chri 'These pine trees are recent\isi{'}
\glt
\z

Object-fronting occurs only when the speaker wishes to contrast one NP with another, or to contradict some inference that has been or might be drawn from a previous utterance. This can be shown if we look at some context for /22/, for instance:

\ea\label{ex:25}
 Interviewer: You ever saw any ghost ?
\glt
\z

MJ7 SM: no-ai no si.

Interviewer: What about, you know, dakine fireball? MJ75M: \textit{o,} \textit{daet} \textit{wan ai} \textit{si.}

The interviewer is referring to \textit{\isi{akualele},} supernatural fireballs, allegedly controlled by members of the \textit{kahuna} or \ili{Hawaiian} priestly caste. MJ75M ( the letters and numbers indicate: sex, masculine; ethnicity, \ili{Japanese}; age, 75; and island of residence, Maui, respectively) has just denied knowledge of supernatural entities and uses \isi{object-fronting}

to mark the exception to this denial, as soon as it is brought to his attention.

Predicate-fronting occurs when a predicate that contains new information is introduced in conjunction with a subject which has been explicitly stated or implied in the immediately preceding \isi{discourse}. This can be shown by extended context for /24 /:

\ea\label{ex:26}
 bifoa don haev mach \textit{chriz} hia. in daet hil dea no moa \textit{chriz.}
\glt
\z

daes leitli \textit{dis} \textit{pain} \textit{chri.}

'There weren't many trees here before. There were no trees

  


 

%\originalpage{20}

at all on that hill over there. These pine trees (that you now see there) were planted recently\isi{'}

Here, the speaker realizes that what he said in the first two sentences may seem plainly false in light of what the interviewer can see before him. Since \textit{trees,} although their presence has been denied, have been established as a topic, he can emphasize the recency of the presently visible trees\isi{'} appearance by fronting the predicate,

This congruence between movement rule and \isi{discourse} feature

is, of course, peculiar to HCE; one cannot fmd any similar congruence between \isi{discourse} and variant ordering in HPE. In fact, the result of the HCE rules is a series of orderings which differs markedly from the possible orderings of HPE, both in that it contains orders which HPE does not permit, and in that it does not contain orders which HPE does permit. The situation is shown in \tabref{tab:1}.1:

\textsubscript{Order }HPE \textsuperscript{HCE}

\textsubscript{SVO }yes \textsuperscript{yes}

sov \textsuperscript{yes no}

\textsubscript{vs }yes \textsuperscript{yes}

\textsubscript{VOS }no \textsuperscript{yes}

\textsubscript{osv }no \textsuperscript{yes}

ovs \textsuperscript{no yes}

%%please move \begin{table} just above \begin{tabular
\begin{table}
\caption{1: Word order in HPE and HCE}
\label{tab:1}
\end{table}

The SOV order which is the commonest among older \ili{Japanese} HPE speakers does not exist \isi{in HCE}. While VS may occur in both HPE and HCE, its source is different in each case: in HPE, it stems from the retention of verb-first ordering; \isi{in HCE}, from the operation of a regular rule. That rule, if it applies to transitive sentences, yields VOS order \isi{in HCE} since objects and other constituents of the verb-phrase move

with the verb:

{\textbackslash}

%\originalpage{21}

\ea\label{ex:27}
 no laik plei futbawl, dis gaiz
\glt
\z

'These guys don't want to \isi{play} football\isi{'}

But although VOS is a possible order in Philippine languages, it does not emerge in HPE, possibly because of the absence of either case\-marking or consistent intonation to distinguish the roles of the two NPs.

As for the two remaining orders, OSV and OVS, which are

present \isi{in HCE} but not in HPE, the £rst arises through \isi{object-fronting}, while the second can occur when both object- and \isi{predicate-fronting} apply to the same sentences. The result, though infrequent, is occasion\-ally found and is judged grammatical by native speakers:

\ea\label{ex:28}
 difren bilifs dei get, sam gaiz 'Some guys have different beliefs\isi{'}
\glt
\z

There is no way in which the sentence orders that are produced, or the rules which produce them, could have been acquired by the first creole generation from their pidgin-speaking parents. Moreover, even if we assume extensive bilingualism in that generation, those rules could not have been derived from either the substrate languages, or from \isi{English}. Thtee substrate languages (\ili{Chinese}, \ili{Portuguese}, \ili{Spanish}), as well as \isi{English} , have underlying SVO order, bnt \ili{Chinese} and \isi{English} do not permit verb-first or predicate-fast sentences, except for one or two highly marked structures like \isi{English} left-dislocated pseudo\-clefts \textit{(Told} \textit{the} \textit{landlord} , \textit{that's} \textit{what} \textit{he} \textit{did)}\textit{.} \ili{Portuguese} and \ili{Spanish} are freer in their ordering, tolerating certain \isi{types of} verb-first sen\-tences, but the equivalents of VOS sentences like /27/ and OVS sen\-tences like /28/ would be ungrammatical in these languages. Conversely, the common Iberian VSX, as exemplified by Port uguese, would be ungrammatical \isi{in HCE}:

\ea\label{ex:29}
 Chegaram os generais do exercito anoite no Rio arrived the generals of-the army last-night in-rhe Rio 'The army generals arrived in Rio last night\isi{'}
\glt
\z

%\originalpage{2}

Of course, one might always say something like, ``All the struc\-tures of HCE are found in at least one of the languages in contact (it would be bizarre if they weren't!] , and therefore HCE merely repre\-sents a random mix of the structures available to children of various groups, through either the pidgin or their own ethnic tongues.\isi{'}\isi{'} If anyone seriously believed that a language could be built by random mixture, this answer might be satisfactory. But it would not explain why one of the commonest (SOV) orders should be excluded-still less why the particular mixture illustrated in \tabref{tab:1}.1 should have been chosen, rather than one of the many other possible combinations. However, it can hardly be accidental if that particular distribution turns out to be exactly what is generated if one assumes basic SVO order (which is virtually mandatory when you have no other means of mark\-ing the two major cases) plus a rule which moves either of the two major constituents, NP and VP, to sentence-initial position. \textsuperscript{1} We may 
therefore claim that the rules which move NPs and VPs cannot have been acquired inductively by the original HCE speakers, but must, in some sense of the term, have been ``invented\isi{'}\isi{'} by them ab ovo.

Next, let us look at \isi{articles}. These appear sporadically and unpredictably in HPE; typical of early (mainly \ili{Japanese}) speakers is the 92-year-old 1907 arrival who produced only three indefinite \isi{articles} (out of 32 that would have been required by \isi{English} rules of reference, i.e., 9.4 percent) and seven definite \isi{articles} (out of a total of 40. that \isi{English} rules would have required, i.e., 17.5 percent). Filipino speakers, on the other hand, generalized the definite \isi{articles} to many environ\-ments in which \isi{English} does not require them, for example: with generic NPs, as in \textit{191} above; where there is only one possible refer\-ent, as in /30/ or /31/; where there is a clearly nonspedfic referent, as in /32/; or where noncount nouns are involved, as in /33/:

\ea\label{ex:30}
 hi get \textit{da} hawaian waif 'He has a \ili{Hawaiian} wife\isi{'}
\glt
\z

{\textbackslash}

%\originalpage{23}

\ea\label{ex:31}
 istawri pram \textit{da} gad 'God's story\isi{'}
\glt
\z

\ea\label{ex:32}
 no kaen du nating abaut \textit{da} eniting insai da haus
\glt
\z

'They can't do anything about anything inside the house\isi{'}

\ea\label{ex:33}
 oni tek tu slais \textit{da} bred
\glt
\z

'I only take two slices of bread\isi{'}

HCE speakers, however, follow neither the under-\isi{generalization} of the \ili{Japanese} speaker nor the over-\isi{generalization} of the Filipino speaker. The definite article \textit{da} is used for all and only specific-refer\-ence NPs that can be assumed known to the listener:

\ea\label{ex:34}
 aefta da boi, da wan wen jink daet milk, awl da maut soa 'Afterward, the mouth of the boy who had drunk that milk was
\glt
\z

all sore\isi{'}

The indefinite article \textit{wan} is used for all and only specific-reference NPs that can be assumed unknown to the listener (typically, first-mention use):

\ea\label{ex:35}
 higet wan blaek buk. daet buk no du eni gud
\glt
\z

'He has a black book. That book doesn't do any good\isi{'}

All other NPs have no article and no marker of plurality. This category includes generic NPs, NPs within the scope of \isi{negation}-i.e., clearly nonspecific NPs-and cases where, while a specific referent may exist, the exact identity of that referent is either unknown to the speaker or irrelevant to the point at issue. Examples include:

\ea\label{ex:36}
 \textit{dag} smat
\glt
\z

'The dog is smart\isi{'} (in answer to the question, ``Which is smarter,

\textit{the} \textit{horse} or \textit{the} \textit{dog?{\textquotedbl})}

137I \textit{ya}\textit{n}\textit{g} \textit{fela} dei no du daet 'Young fellows don't do that\isi{'}

%\originalpage{24}

I \textit{38\{ }poho ai neva bai \textit{big} \textit{wan}

'It's a pity l didn't buy a big one\isi{'}

\ea\label{ex:39}
 bat nobadi gon get \textit{jab}
\glt
\z

'But nobody will get a job\isi{'}

\textit{\{} \textit{40/ }hu go daun frs iz \textit{luza}

'The one who goes down first is the loser\isi{'}

\ea\label{ex:41}
 \textit{kaenejan} \textit{waif,} ae, get 'He has a Canadian wife\isi{'}
\glt
\z

\ea\label{ex:42}
 mi ai get \textit{raesh}
\glt
\z

'As for me, I get a rash\isi{'}

\ea\label{ex:43}
 as tu bin get had taim reizing \textit{dag}
\glt
\z

'The two of us used to have a hard \isi{time} raising dogs\isi{'}

These zero-marked forms would be marked three diffe.rent ways in \isi{English}: with \textit{the }in /36/, /40/; with \textit{a} in /38/, /39/, /41/, /42/; with zero, but followed by \isi{plural} \textit{{}-s, }in /37 /, /43/. But note

that the absence of \isi{plural} marking in the last two cases certainly does not stem from any more general absence of pl uralization \isi{in HCE}; specific NPs with \isi{plural} reference are always appropriately marked \isi{in HCE} (although not in HPE), except where numerals or other clear signs of plurality are already present.

The fact that HCE unites in a single category what \isi{English} treats as discrete categories has led to some curious analyses, such as that of \citet[99]{Perlman1973} who writes: ``(/)s that mark generic singular NP should be distinguished from those that mark indefinite singular ones. The distinction may be quite difficult to make. In such cases as \textit{they} \textit{go} \textit{beer} \textit{parlor,} \textit{Ewa} \textit{was} \textit{never} \textit{using} \textit{crane} \textit{that} \textit{date,} \textit{that} \textit{is} \textit{hund} \textit{red-pound} \textit{bag,} \textit{those }\textit{d}\textit{ays} \textit{they} \textit{get} \textit{icebox,} and \textit{olden} \textit{days} \textit{we} \textit{gotta} \textit{ride} \textit{train,} it may actually be neutralized; however, I have dis\-tinguished generic from indefinite where possible and discarded un\-certain cases like the preceding.{\textquotedbl}

In reality, all Perlman's cases and those cited above have in common the fact that no specific reference is intended, or, in most cases, even possible; and the semantc feature \textit{nonspecific} happens to be

%\originalpage{25}

slurred by both generics and what Perlman calls ``indefinite singular.\isi{'}\isi{'} That he uses the word ``singular\isi{'}\isi{'} at all in this context is enough to show that he is looking at his data through \isi{English} spectacles. \isi{English} has an obligatory number distinction; every NP has to be either singular or \isi{plural}. HCE does not have an obligatory number distinction, or rather it has three numbers-singular, \isi{plural}, and nonspecific (number\-less). Thus, while to the \isi{English} speaker \textit{raesh} in /42/ is clearly singular, while \textit{d}\textit{ag} in /43/ is clearly \isi{plural}, HCE speakers treat both cases as unmarked for number, because both are nonspedfic: the rash, because no particular rash is being referred to-simply the usual consequence when the speaker uses a certain brand of soap; the dogs, because it was not one particular dog or group of dogs that gave the speaker trouble, but rather the business of dog-raising in general.

We shall come back to nonspecifidty in each of the next three chapters; for the present, we only need to ask where the specific\-nonspecific, marked-unmarked, distinction that is incorporated in the HCE grammar came from. Based on the evidence already shown, it could not have come from HPE. If HCE speakers had followed the \ili{Japanese} version, they would have marked far fewer NPs than they do, and zero marking would have been assigned to at least some specifics. If they had followed the Filipino model, thev\textsubscript{,}\textsubscript{ }would have had far fewer \zeros, and some definite \isi{articles} would have been assigned to non-specific NPs; compare, for instance, /30/ with /41/. A mixing strategy\- ``Use more \isi{articles} than \ili{Japanese} HPE speakers, fewer than Filiplno speakers{\textquotedbl}-would have achieved roughly the right numerical propor\-tions but, if used alone, would hardly have arrived at the rigorous semantic distinction that all HCE speakers in fact make.

As the influence of the original languages in contact, the glosses for /36/-\{43/ show that \isi{English} can hardly have been a model. Of the substrate languages, many do not have \isi{articles} at all. Of those that do, none show the same distribution of zero forms; \ili{Portuguese} and \ili{Hawaiian} allow, to a varying extent, zero-marked, number-neutral NPs in object position, but demand \isi{articles} for subject generics like that in /36/. Indeed, those who believe in the strength of substrate
%\originalpage{26}
influence might note that the speaker who produced /36/, even after a generic with a definite article had been presented to him by his inter\-viewer, was a 79-year-old pure-blooded \ili{Hawaiian} who had. spoken \ili{Hawaiian} as a child.\textsuperscript{2} We must conclude, as with \isi{word order}, that the zero marking of nonspecifics was an HCE ``invention,\isi{'}\isi{'} and one firmly rooted enough to override counterevid.ence from other languages known to its speakers.

Next, let us examine verbal auxiliaries. HCE has an auxiliary which marks tense, \textit{bin} (which sometimes takes the form \textit{wen,} derived from it by regular phonological rules); an auxiliary which marks modal\-ity, \textit{go }(sometimes \textit{go}\textit{n}\textit{)} \textit{; }and an auxiliary which marks aspect, \textit{stei.} I shall not discuss the first two here as we will return to a fuUer discus\-sion of creole tense-modality-aspect (henceforth TMA) systems in Chapter 2. The aspect marker \textit{stei} will give us a clearer view of how creole creativity works.

\textit{Bin }and \textit{go }(at least as surface forms, though not with their HCE

meanings) occur sporadically and unpredictably in .HPE, but \textit{s}\textit{t}\textit{i} does not-at least not as an auxiliary. It does, however, occur as a mam verb, taking locative \isi{complements}:

\ea\label{ex:44}
 mi iste nalehu tu yia
\glt
\z

'I was in Nalehu for two years\isi{'}

Jn all our recordings of pre{}-1920 immigrants (the only ones who could possibly have provided input to the creolization process) there were only seven sentences in which \textit{stei} preceded another verb. When a feature that occurs so frequently \isi{in HCE} occurs with such vanishing rarity in HPE, there clearly exists the possibility that it was invented by HCE speakers and was only afterward adopted by some HPE speakers\-who, as was suggested earlier, are hardly likely to have lived through a half-century without \textit{any} addition to their grammars. But let us assume the contrary and ask, fu:st, whether the occurrences of \textit{stei} represent
% {\textbackslash}
% PIDGIN INTO CREOLE 27
a true auxiliary, or whether the sentences contain mere sequences of two main verbs; and second, whether uses such as these could have provided evidence for the HCE speaker to develop a true auxiliary with nonpunctual (progressive plus \isi{habitual}) meaning. The seven sentences are:

\ea\label{ex:45}
 haus, haus ai stei go in, jaepan taim.
\glt
\z

\ea\label{ex:46}
 al srei kuk.
\glt
\z

\ea\label{ex:47}
 mi papa stei help.
\glt
\z

\ea\label{ex:48}
 aen istei kam-i kam draib in i ka.
\glt
\z

\ea\label{ex:49}
 mai brad.a hi stei \textit{make} hia.
\glt
\z

\ea\label{ex:50}
 oni tu yia mi ai stei wrk had.
\glt
\z

\ea\label{ex:51}
 samtaim wan dei stei gat twentipai baeg.
\glt
\z

I have not provided glosses since everything turns on what the sentences mean,. and what they mean is far from transparent. If /45/ were an HCE utterance, \textit{stei} \textit{go} \textit{in} would mean something like 'kept entering\isi{'},
which is improbable here; the .most likely meaning, in context, is that the speaker, when she was a girl in \isi{Japan}, seldom used to leave the house. In the latter case, \textit{stei} would be the main verb with the meaning
'stay\isi{'}, while \textit{go} \textit{in} would probably have been learned as an undifferenti\-ated chunk meaning something like 'inside\isi{'}. In /46/, too, the second ``verb\isi{'}\isi{'} may not really be a verb either; \textit{kuk} could represent the noun \textit{cook,} in which case the sentence would mean simply 'I was a cook\isi{'}. Sentence /47/ could represent \isi{verb serialization} as easily as auxiliary\-plus-verb; 'I stayed \textit{0J1d} helped my father\isi{'} or 'l stayed \textit{to} help my father\isi{'} are as plausible as glosses, in context, as 'I was helping my father\isi{'}. Sentences /48/ and /49/ could contain auxiliaries···the reformu\-lation of /48/ makes it impossible to tell what was intended-but in both cases, punctual events are referred to; the most probable glosses are 'He drove up in his car\isi{'} and 'My brother died here\isi{'}, respectively. But /49/ is puzzling since the \ili{Hawaiian} verb \textit{make} alone would mean 'died\isi{'}, while \textit{stei} \textit{make} would mean literally 'is (or was) dead\isi{'}, which in conjunction with \textit{here} makes little sense.

%\originalpage{28}
The only sentences that could have any kind of nonpunctual meaning are / 50/ and /51/. Sentence \textit{/} \textit{50[} is most like an HCE sentence with its suggestion of durative activity. Sentence \textit{/} \textit{51[} is more problema\-

%\originalpage{29}

\ea\label{ex:54}
 Present \isi{habitual}:
\glt
\z

yu no waet dei stei kawl mi, dakain-kawl mi gad

'You know what they call me, that bunch ? They call me God\isi{'}

tic-it could be a past \isi{habitual} with a zero impersonal subject ('Some\-times they \textit{used} \textit{to} \textit{collect} twenty-five bags a day\isi{'}) or some kind of premature attempt at a \isi{passive} with the quasi-copular \textit{stei} ('Sometimes twenty-five bags \textit{were} \textit{collected} in a day\isi{'}).

If the input to the first creole generation was as chaotic as this\-and, as we saw, it could hardly have been \textit{less} chaotic although it may well have been \textit{more} chaotic-··it is impossible to see how children of that generation could have distilled any kind of regular rule out of it,

\ea\label{ex:55}

\glt
\z

Past \isi{habitual}:

da meksikan no tel mi nau, da gai laik daunpeimen{}-i stei tel mi, o, neks wik, hi kamin, kamin

'Now the Mexican didn't tell me that the guy wanted a down\-payment-he kept telling me, oh, next week it's coming \textbf{it;s} \textbf{con1ing\isi{'} }\isi{'}

A further difference between the HPE examples cited above and

still less the particular rule that they did in fact derive. But if, as is highly possible, sentences such as /45/-/51/ would not have been produced by any HPE speaker prior to 1920, and in fact represent a case of partial imperfect \isi{learning} of HCE rules by those speakers (a

possibility that I shall document in another area later in this chapter J,

then the achievement of the first creole generation becomes still more mysterious, since it must have been completely ex nihilo.

For if we look at the \textit{stei} \textit{+} \textit{V} sentences of HCE speakers, we find

no ambiguous cases and no reference to single punct ual events. All their uses of \textit{stei} as auxiliary fall into what an \isi{English} speaker would probably describe as four categories-present continuous, past con\-tinuous, present habit ual, and past habit ual-but these four categories, as I have demonstrated elsewhere (Bickerton 1975:Chapter 2), really constitute a single nonpunctual category, semantically opposed to a punctual category expressing single nondurative actions or events:

\ea\label{ex:52}
 Present continuous:
\glt
\z

ai no kea hu stei hant insai dea, ai gon hunt

'l don't care who's hiinting in there, I'm going to hunt\isi{'}

\ea\label{ex:53}
 Past continuous:
\glt
\z

wail wi stei paedl, jaen stei put wata insai da kanu-hei, da san av a gan haed sink!

'While we were paddling, John was letting water into the canoe\-hey, the son-of-a-gun had sunk it!\isi{'}

he many hundreds of \textit{stei} sentences we recorded from HCE speakers

is that out of the seven examples, two conjoined \textit{stei} with predicates\-\textit{gat} and \textit{make-with} which it is never conjoined \isi{in HCE}. This is because both are perceved as \isi{stative} verbs, and \isi{in HCE}, nonpunctual aspec t cannot be applied to statives: \textit{*shi} \textit{stei} \textit{no} \textit{da} \textit{ansa} is as ungrammatical as its \isi{English} equivalent, \textit{*she} \textit{is} \textit{knowing} \textit{the} \textit{answer.} It is true that on one of our glosses of /51/ \textit{gat} is not \isi{stative}; but then it would not be used to express \textit{collect }by an HCE speaker, for whom \textit{gat }is limited to an alternative form for \textit{get }'have\isi{'}.

How could HCE speakers have invented the \textit{stei} \textit{+} \textit{V} form? \textit{Stei}

is common as a locative in modern HPE, but there is some doubt

:Whether it was so common at the \isi{time} of creole formation. Examples m \citet{Nagara1972}, drawn from data collected from very old speakers decade earlier, contain only \textit{stap} as a locative, and the oldest speaker

m ou: wn sur;ey, who arrived prior to 1910, also has only \textit{stap.} Thus lt 1s conceivable that even the locative use of \textit{stei} was acquired by HPE speakers from HCE speakers. However, I would think it more likely that \textit{stei} was a low-frequency variant around 1910 and that HCE speakers selected it over \textit{stap} because, semantically, \textit{stei} was a more appropriate expression for durativity.

Locativ expressions are a common source of nonpunct ual markers: for mstance, forms such as \textit{I} \textit{am} \textit{worki}\textit{n}\textit{g} derived originally from \textit{I} \textit{am} \textit{AT} \textit{working-main} verb \textit{be} plus locative preposition plus
%\originalpage{30}
gerund-which gave rise first to the form \textit{I'm a-working }(still found in some conservative areas, e.g., West Virginia) and then, via phono\-logical reduction and syntactic reanalysis, to the modern \textit{Aux} + \textit{V} + \textit{i}\textit{n}\textit{g.} These processes in \isi{English} took several centuries to produce a result which HCE must have produced almost instantly.

Again, we will look in vain for any substratum language which unites all the ingredients which make up HCE nonpunctual: preverbal free morpheme, semantic range inclusive of both progressive and habi\-tual, indifference to the past-nonpast distinction. Many substratum languages express aspect via bound morphemes, or by reduplication (e.g., Philippine languages-it is surely surprising, in view of the fre\-quency with which ``reduplication\isi{'}\isi{'} is hailed as a universal creole characteristic, that HCE speakers did not avail themselves of this particular resource!). \ili{Hawaiian} uses free morphemes, but these are placed both before and after the verb; \ili{Chinese} uses preverbal free morphemes, but semantically there is hardly any point of resemblance between the HCE and \ili{Chinese} TMA systems. Perhaps the closest form to \textit{stei,} semantically, is \ili{Japanese} \textit{{}-te} \textit{iru/-te} \textit{ita;} these forms cover (very roughly ) the same semantic range as \textit{stei.} However, it is the discontinuous segment \textit{{}-te }\textit{i- }which carries nonpunctual meaning;

\textit{{}-ru} and \textit{{}-ta} signify nonpast and past, respectively, and nonpunctuality cannot be marked without marking one or the other of them. Thus, the \ili{Japanese} form satisfies neither the first nor the third characteristic of \textit{stei.} To assume \ili{Japanese} influence on \textit{stei} would be to assume that a \isi{TMA system} can be put together like a jigsaw puzzle;\textsuperscript{3} indeed, the implicit supposition that all languages are like erector sets which can be dismantled, cannibalized, and put together again in new combin,.,. tions lies at the heart of all substratum arguments.

Our fourth area involves a particular type of \isi{sentential} comple\-mentation. It has already been mentioned that sentence embedding of any kind is virtually nonexistent in HPE. Certainly there are no examples of anything resembling \isi{English} \textit{for-to }complementat ion,
% {\textbackslash}
whether \textit{for} is obligatorily present /56/, optionally present /57/, or obligatorily deleted /58/:

\ea\label{ex:56}
 Mary bought this for you to read.
\glt
\z

\ea\label{ex:57}
 Mary prefers (for) Bill to go.
\glt
\z

\ea\label{ex:58}
 Mary prefers (*for) to go.
\glt
\z

HCE does have \isi{sentential} \isi{complements}, but these are marked not by \textit{to} but by \textit{fo} (presumably derived from \isi{English} \textit{for} \textit{)} and \textit{go;} the precise distribution of these will be discussed shortly.

First, we must see what, if anything, HCE speakers could have learned from HPE. \textit{Fo} is found in HPE only as a preposition, and it is rare even as a preposition in speakers who arrived prior to the late 1920s. \textit{Go,} however, occurs frequently, and in three con\isi{texts}: as a main verb, as a marker of imperatives, and as a preverbal modifier of extremely indeterminate meaning and wildly fluctuating distribution (some HPE speakers use it before every third or fourth verb; others don't use it at all). The nearest antecedent to a complementizer \textit{go} derives from its \isi{imperative} use. One can have paratactic strings of \isi{imperative} structures, as in /59/:

\ea\label{ex:59}
\textit{I} go tek tu fala go \textit{hapai} dis wan

go take two men go carry this one 'Take two men and take this away\isi{'}
\glt
\z

From their intonation contour, pauses, etc., these are clearly two independen t sentences, but production of such sequences in the more rapid tempo of HCE could conceivably serve as a source of true comple\-mentation. Such a result might be even more likely in the case of reported imperatives, such as /60/:

\ea\label{ex:60}
 ai no tel yu palas, go join pentikosta
\glt
\z

'I'm not telling you guys, ``Join the Pentecostal Church\isi{'}\isi{'} \isi{'}

\section{}
T

%\originalpage{32}

This could, presumably, be reanalyzed as . . . \textit{telling} \textit{you} \textit{guys} \textit{TO} \textit{join} . . .; at least with the benefit of the native \isi{English} speaker's 20/20 hindsight.

Apart from these two constructions- \isi{imperative} strings and

reported imperatives-there is nothing that looks remotely like a \textit{go·} complementizer construction in HPE, and even these two are quite rare. There is thus no precedent far sentences such as the fallowing, which occur with considerable frequency \isi{in HCE}:

\ea\label{ex:61}
 dei wen go ap dea erli in da mawning go plaen
\glt
\z

'They went up there early in the morning TO plant\isi{'}

\ea\label{ex:62}
 so ai go daun kiapu go push
\glt
\z

'So I went down to Kiapu TO push (clear land with a bulldozer)\isi{'}

\ea\label{ex:63}
 ai gata go haia wan kapinta go fiks da fom
\glt
\z

'I had TO hire a carpenter TO fix the form\isi{'}

However, \textit{Jo }often replaces \textit{go} in environments which might appear at frrst giance to be identical:

\ea\label{ex:64}
 aen dei figa, get sambadi fo push dem
\glt
\z

'And they figured there'd be someone TO encourage them\isi{'}

\ea\label{ex:65}
 mo .beta a bin go hanalulu fa bai maiself
\glt
\z

'It would have been better if I'd gone to Honolulu TO buy it

myself\isi{'}

\ea\label{ex:66}
 hau yu ekspek a gai fa mek \textit{pau }hiz haus
\glt
\z

'How do you expect a guy TO finish his house?\isi{'}

In fact, the two sets of environments differ in an interesting way. The actions described in /61/-/63/ all actually occurred, while those de\-scribed in /64/-/66/ were all hypothetical: there wasn't anyone to encourage the basketball team referred to in /64/; the speaker of /65/ hadn't gone to Honolulu; and the hypothetical \textit{guy} in /66/ couldn't complete his hypothetical house because the very real bank manager who was being addressed wouldn't issue a loan for that purpose. In

\isi{'}

%\originalpage{33}

other words, HCE marks grammatically the semantic distinction be\-tween \isi{sentential} \isi{complements} which refer to realized events and those which refer to unrealized events.

The distinction is blurred a little by some sentences which contain both \textit{Jo} and \textit{go:}

\ea\label{ex:67}
 pip!no laik tek om fo go wok 'People don't want TO employ him\isi{'}
\glt
\z

\ea\label{ex:68}
 rumach trabl, ae, fo go fiks om op
\glt
\z

'It's a lot of trouble, you see, TO fix it up\isi{'}

But again, these \isi{complements} express hypothetical or even nonoccur\-ring events; thus, these examples confirm the claim that \textit{Jo} only occurs with unrealized events, and does not affect the claim that \textit{go,} ALONE, occurs only with realized events.

In this area, then, HCE has made two distinct innovations, one semantic, one syntactic. The syntactic innovation consisted of taking \textit{Jo} and \textit{go,} a preposition and an \isi{imperative} marker, respectively, and using them to introduce embedded sentences, which were themselves an innovation. Even with the possible stimulus supplied by HPE sen\-tences such as /59/ and /60/, this represents a massive change. However, the semantic innovation- distinguishing realized from unrealized com\-plements-was completely without precedent in HPE, in \isi{English}, or in any of the substrate languages. We should bear this in mind when we encounter widely separated creoles with identical distinctions in Chapter 2.\textsuperscript{4}

The fifth and final example of HCE innovation which we will examine here is rather more complex than the previous examples, involving, as it does, the interaction of two rules: a rule of \isi{relativization} and a rule of \isi{subject-copying}. Each of these rules itself involves inno\-vation, but I shall say little about these since it is their interaction that shows most dramatically the working of creole creativity.

  


 

%\originalpage{34}}

Insertion of a pronoun between subject and predicate was noted above as a feature of Filipino HPE; this feature is discussed at length in Bickerton and Odo (1976:3.6.1). There is no dear evidence that it is used as anything but a marker of verbal (as distinct from adjectival, nominal, or locative) predicates by any but a small minority of very late (post-1926) arrivals. However, HCE speakers use the same feature for all full-NP subjects on first mention and for all full-NP contrastive subjects; it follows from this that all full-NP subjects ·of indefinite refer\-ence are thus marked (since indefinite reference marks first mention only, and the \textit{some} \textit{guys} who \textit{do X }turn into the \textit{they} or \textit{those} \textit{guys }who \textit{do} \textit{Y}\textit{).} Thus, \isi{in HCE}, sentences such as /69/ and /71/ are common, whereas /70/ and /72/ would be ungrammatical:

\ea\label{ex:69}
 sam gaiz samtaimz dei kam 'Sometimes some guys come\isi{'}
\glt
\z

\ea\label{ex:70}
 *sam gaiz samtaimz kam
\glt
\z

\ea\label{ex:71}
 jaepan gaiz dei no giv a haeng, do
\glt
\z

'Guys from \isi{Japan} don't give a hang, though\isi{'}

\ea\label{ex:72}
 *jaepan gaiz no giv a haeng, do
\glt
\z

The function of pronoun-copying \isi{in HCE} is dearly linked with that of the \isi{movement rules} discussed above. All deal with constituents selected for special focus; \isi{movement rules} move those constituents to the left, but subject NPs are already leftmost constituents and can thus only be ``symbolically\isi{'}\isi{'} moved by inserting something between them and the rest of the sentence.

Relative clauses, among pre-1920 immigrants, are rare, and when they do occur, often they do so in forms influenced by the speaker's native language, cf. /14/ above. Among HCE speakers, relative clauses are common. However, they differ from \isi{English} relative clauses in that they contain no surface marker of \isi{relativization} even where \isi{English} demands one, i.e., in sentences where the noun relativized on is sub\-ject of the clause, and either subject /73/ or object /74/ of the main \textbf{sentence:} 

\ea\label{ex:73}
 da gai gon lei da vainil fo mi bin kwot mi prais
\glt
\z

'The guy WHO is going to lay the vinyl for me had quoted me a price\isi{'}

\textit{17}4\textit{I }yu si di ailan get koknat

'You see the island THAT has coconut palms on it?\isi{'}

We will consider how such sentences may be generated in Chapter 2.

The interaction of those two rules comes about when full NPs of indefinite reference and other NPs which must be copied \textit{occur }as head nouns of relative clauses and subjects of those clauses. In non\-relative sentences, such as /69/ or /71/, the copy either immediately follows the NP, as in /71/, or, if an adverb is present, as in /69/, immediately precedes the verb. In relative-clause sentences, however, the copy must follow the entire relative clause:

\ea\label{ex:75}
 sam filipinoz wok ova hia dei wen kapl yiaz in ftlipin ailaenz 'Some Filipinos WHO worked over here went to the Philippines
\glt
\z

for a couple of years\isi{'}

\ea\label{ex:76}
 *sam filipinoz dei wok ova hia wen kapl yiaz . . .
\glt
\z

\textit{177I }*sam ftlipinoz dei wok ova hia dei wen kapl yiaz . , .

\ea\label{ex:78}
 *sam filipinoz wok ova hia wen kapl yiaz . , ,
\glt
\z

\textit{17}\textit{9}\textit{I }sambadi goin ova dea dei gon hia nau

'Anybody WHO's going over there will hear it now\isi{'}

\ea\label{ex:80}
 *sambadi dei g\isi{'}?in ova dea gon hia it nau
\glt
\z

\ea\label{ex:81}
 *sambadi dei goin ova dea dei gon hia it nau
\glt
\z

\ea\label{ex:82}
 *sambadi goin ova dea gon hia it nau
\glt
\z

It cannot be claimed that in /75/ or /79/ the \textit{c}\textit{o}\textit{py }represents some ``resumptive\isi{'}\isi{'} device whose presence is due to the distance be\-tween subject and main verb; if this were the case, the subject of /73/, which is even further from its verb, would be similarly copied. More\-over, a ``resumptive\isi{'}\isi{'} argument does not explain why /77 / and /81/ are ungrammatical, and cannot account for the presence of copies in

/69/ and /71/.

%\originalpage{36}

The real problem is explaining the different placement of the copy in, e.g., /71/ and /79/. We can see what is happening if we look at what is probably the underlying structure of /79/ (I shall defend the rule that rewrites NP as S, rather than N S, in the next chapter):

\ea\label{ex:83}
 sr-{}-{}-{}-{}-
\glt
\z

%\originalpage{37}

\isi{A-over-A principle}, cannot have done so as a result of experience.

In the first place, no sentences involving both \isi{relativization} and \isi{subject-copying} are found in pre-1920 arrivals, and therefore the varying distribution of copies in relative and nonrelative sentences cannot have been acquired from HPE speakers. The differences from \isi{English} are obvious: sentences like /7 5/ and /79/ would be ungramma\-tical even in those so-called ``substandard\isi{'}\isi{'} dialects of \isi{English} which

NP

o.k.

s,

VP

Aux VP

V Adv

permit \isi{subject-copying} and\{or deletion of relative pronouns in

subject position. No substrate language combines similar modes of \isi{relativization} and focusing; therefore, none of them could have pro\-vided relevant evidence. Moreover, in this case we have much clearer proof than before that the current of innovation ran from HCE back into HPE, rather than vice versa.

Among later immigrants, there was just one (arrival date 1930)

sambadi

V Adv

I 6

goin ova dea

gon

hia

nau

who attempted complex sentences such as /79/. Although he some\-times got them right, he would, with equal frequency, produce sen\-tences with two copies, as in /81/, or no copies, as in \{82/:

\textsuperscript{/84/ }awl diz bigshat pip!dei gat plenti mani dei no kea

If S\textsubscript{1}\textsubscript{ }constituted an independent sentence, then the rule of subject\-copying would place the appropriate pronoun immediately to the right of the NP marked with an asterisk to yield \textit{sambadi} \textit{dei} \textit{goin} \textit{ova} \textit{dea.} However, when S\textsubscript{1}\textsubscript{ }is embedded in S\textsubscript{0}\textsubscript{ }, the higher-circled NP node

\{85/

'All these big shots who have plenty of money don't care\isi{'} sam kam autsaid kam mo was

'Some who come out (of jail) become worse\isi{'}

must have the pronoun adjoined to it in order to yield \textit{/7} \textit{9\{} , rather

than the ungrammatical /80\{.

\citet{Chomsky1964} proposed a universal principle termed the ``\isi{A-over-A principle},\isi{'}\isi{'} which states that if a major category, such as NP, is directly dominated by the same major category, then any rule that would normally apply to the lower category node could apply only to the higher node. Although the principle as there formulated has not been widely accepted (cf. Ross 1967), similar phenomena have been observed in a number of languages, and something resembling such a principle must still be regarded as a likely formal universal. Formal universals must be regarded part of the innate equipment of the species, and HCE speakers, however they may have arrived at the

These sentences are of course ungrammatical \isi{in HCE}, and no locally\-born speaker would have produced them. But they are just the kind of vague approximations that are made by foreign-language learners when they try to apply a new and imperfectly-acquired rule. Indeed, HCE, once established, was just that-a new foreign language-and joined earlier versions of HPE as a part of the input to immigrant speakers who arrived in \isi{Hawaii} after 1920.

We have now surveyed five quite distinct aspects of HCE gram\-mar and found in each of them dear innovations by the earliest HCE

%\originalpage{38}

speakers; developments in the grammar which can have owed little or nothing to HPE, to \isi{English}, or to any of the substrate languages involved. We may briefly review those developments by presenting a more formal summary in terms of the grammatical rules involved, showing first the HPE rules-if HPE can be said to have rules or a grammar of its own; Ithink that HPE would really have to have an analysis like that proposed by \citet{Silverstein1972} for Chinook Jargon, in which the pidgin forms would be produced by extensions and modifications of the HPE speakers\isi{'} original native languages-and then

the HCE rules for each of the £ve areas.

%\originalpage{39}

\ea\label{ex:91}
 NP {}-{\textgreater} \textsubscript{j}\textsubscript{ }\textsubscript{(da) }\textit{\textsubscript{l}}\textsubscript{ }\textsubscript{N}
\glt
\z

/ (wan) {\textbackslash}

There would be no rule that would determine the circumstances under which \textit{da,} \textit{wan,} or \textit{(/)} would be generated. HCE, on the other hand, would have the following rule \{I will ignore determiners other than \isi{articles}): 

\ea\label{ex:92}
 NP \_,. Art N
\glt
\z

\ea\label{ex:93}
 Art \_,. Definite
\glt
\z

With regard to ·basic sentence-structure and \isi{movement rules}, HPE would have the following \isi{phrase-structure} (PS) rules: 

\ea\label{ex:86}
 S {}-+ NP V (NP) {\textbackslash}
\glt
\z

NP (NP) V

\ea\label{ex:94}

\glt
\z

\ea\label{ex:95}

\glt
\z

\ea\label{ex:96}

\glt
\z

Nondefinite

Nonspecific Definite \_,. \textit{da} \textsuperscript{Nondefinite }\textsuperscript{{}-+ }\textit{wan} Nonspecific {}-{\textgreater} \textit{(/)}

V NP

HPE would have no \isi{movement rules}. HCE, on the other hand, would have the following PS rules:

\textsubscript{/87}\textsubscript{ }\textsubscript{/ }s \_,. \textsubscript{NP }Aux VP

\textsubscript{/88/ VP}\textsubscript{ }\textsuperscript{.... }v (NP) \textsuperscript{(PP)}

and in addition the following \isi{movement rules}:

\ea\label{ex:89}
SD: \textsuperscript{NP VP}
\glt
\z

\textsubscript{1 2 }{}-+

SC: \textsuperscript{2 1}

\ea\label{ex:90}
 \textsuperscript{SD: }\textsuperscript{x }\textsuperscript{v }\textsuperscript{NP}
\glt
\z

\textsubscript{1 2 3 }\_,.

\textsuperscript{SC: }3 \textsuperscript{1 }\textsuperscript{2}

For the second area, involving \isi{articles}, HPE, if it had any rule at

all, would have something like /91/ : ·

\isi{'}

For the third area, involving \textit{stei} and other auxiliaries, it is not clear what rules, if any, HPE would have-possibly a rule such as /97/, which would also account for the fact that some auxiliaries, such as \textit{kaen,} may function as main verbs, as in \textit{no} \textit{kaen} \isi{'}(You) can't (do it)\isi{'}/ 'It's impossible\isi{'}:

\ea\label{ex:97}
 V \_,. (VJ V
\glt
\z

HCE, however, would have /87/, plus the following PS rules:

\ea\label{ex:98}
 Aux {}-{\textgreater}- (Tense) (Modal) (Aspect)
\glt
\z

\ea\label{ex:99}
 Tense \_,. Anterior
\glt
\z

\ea\label{ex:100}
 Modal {}-{\textgreater} • • • Irrealis . . .
\glt
\z

\ea\label{ex:101}
 Aspect \_,. Nonpunctual
\glt
\z

\ea\label{ex:102}
 Anterior \_,. \textit{bin}
\glt
\z

\ea\label{ex:103}
 lrrealis {}-+ \textit{go}
\glt
\z

\ea\label{ex:104}
 Nonpunctual {}-+ \textit{stei}
\glt
\z

%\originalpage{40}

For the fourth area, involving \textit{fo,} \textit{go,} and \isi{sentential} comple\-ments, HPE would have no rules. HCE would possibly have something like the following PS rule (but see Chapter 2 on the status of comple\-

mentizers in creoles generally ):

\ea\label{ex:105}
NP {}-+ (COMP) S
\glt
\z

\ea\label{ex:106}
 COMP \_,. {\textbackslash} Realized l
\glt
\z

/ Unrealized {\textbackslash}

\ea\label{ex:107}
 Realized {}-+ \textit{go}
\glt
\z

\ea\label{ex:108}
Unrealized ., \textit{fo}
\glt
\z

In addition, HCE would require something analogous to (but probably not identical with ) the \isi{English} rule of equi-deletion.

Finally, for the nfth area, involving \isi{relativization}, subject\-

copying, and their interaction , we might need a rule for Filipino speak\-ers which would modify /97\textit{I} to something like /109/:

\ea\label{ex:109}
 v
\glt
\z

{}-+ {\textbackslash} (\textsubscript{(}V\textsubscript{i}\textsubscript{ }J\textsubscript{) }V\textsubscript{v}\textsubscript{ }{\textbackslash}

Pred

(All but the later HCE-influenced HPE speakers realize the copy, if indeed for them it is a copy-it is more likely a marker of a particular predicate type-as an invariant \textit{i, }i.e., in contradistinction to the HCE rule, subject features such as \textit{\isi{plural} }or \textit{fem} \textit{inine} are not copied onto it.) A few speakers might have, in addition, a rule for \isi{relativization} that would simply replace NP by S. HCE speakers would have a well\-

established rule:

\ea\label{ex:110}
 NP \_,. S
\glt
\z

In addition, they would have the following transformational rule:

{\textbackslash}

PIDGIN INTO CREOLE 41

\ea\label{ex:111}
 SD: NP VP
\glt
\z

[;\textit{;}\textit{:} \textit{:r} \textit{]}

1 2 ....

SC: 1 + pro 2

[;;:::r]

The A-0ver-A principle, or whatever general constraint governs the \isi{subject-copying} rule in relativized sentences, would not need to be separately stated in the HCE grammar since it would presumably be a universal.

All that remains for us is to ask how these quite substantial innovations could have been produced. There would seem to be only two logically possible alternatives. They could have been produced by some kind of general problem-solving device such as might be applied in any field of human behavior where the required human institutions were lacking-much as survivors of a shipwreck or an atomic holocaust might reconstruct government, laws, and other social institutions. Or they might have been produced by the operation of innate faculties genetically programmed to provide at least the basis for an adequate human language.

If \isi{Hawaii} were the only place where people had been. faced with the problem of reconstructing human language, it would be impossible to decide between these alternatives. However, \isi{Hawaii} is far from unique. There are a number of creole langnages in other. parts of the globe, but produced under very similar circumstances, several of which have been described well enough to make compatison possible. It is true that in these cases we do not have the antecedent pidgin for compatative purposes, \textsuperscript{5} but we shall see that there are still some oblique indications of antecedent structure. In. any case, it is difficult to see, given the rapidity with which creoles arose, how those antecedent pidgins could have developed any further than \isi{Hawaii}'s did.

%\originalpage{42}

Now, of the two alternatives stated above, each would seem to make different predictions about the general nature of creoles. If some general problem-solving device were at work, we would not expect that in every different circumstance it would reach the same set of conclusions. There are any number of possible solutions to the struc\-tural and communicative problems that language poses, as the very diversity of the world's languages shows, and we would expect ;o f\isi{'}.{\textquotedbl}d that, given the differences in geographical region, \isi{culture}, contnbut:{\textquotedbl}g languages, and so on, each group faced with the task of reconstructrng language would arrive at quite different soltions. Indeed: unles I am mistaken, orthodox generativists, even while believing m an innate language faculty, might predict the same result since their thery assigns to that faculty nothing more than those formal and substantive universals which are reflected in all languages. Thus, they could pre\-dict no more of a creole than that it should not violate any universal constraint.

However, if all creoles could be shown to exhibit an identity far beyond the scope of chance, this would constitute strong eidence that some genetic program common to all members of the species was decisively shaping the result.



%\originalpage{137}

\chapter{Acquisition}

In recent work, a number of scholars (e.g., Bruner 1979, Snow 1979) have summarized the development of acquisition studies over the last two decades. In the mid-sixties, the field, which had previously been atheoretical and somewhat underdeveloped, came to be domi\-nated by a type of innatist theory. This theory, derived largely from generative grammar, and in particular from works such .as \citet{Chomsky1962}, held that the child acquired language through simple exposure to linguistic data, much of which was ``degenerate\isi{'}\isi{'} {}-i.e., consisted of st3n tence fragrnant11, mid-111;1ntrncr reformufo tiom, i:ind moi.ny types oJ performance error which would render natural speech a very unreliab mirror to mature native-speaker competence. Somehow the child had to sift the wheat from the chaff, and he could only do this, it wa claimed, if he had some kind of inbuilt Language Acquisition Device (LAD). A LAD would contain a set of linguistic universals, presumed to be innate and genetically transmitted. These universals would not\textsuperscript{1} however, precisely specify a particular potential language, as in the; theory described at the end of the last chapter; rather, they would de\-fine somewhat narrowly the limits on the forms which human language might take, thereby drastically reducing the number of hypotheses
that the child could make about the structure of his future native tongue and rendering it correspondingly easy for him to select the correct hypothesis.

Since it is well known that children, whatever else they may do, do not in fact instantly and unerringly make correct hypotheses about adult structures, but rather approximate to those structures by means of a fairly regular and well-defined series of stages, the innocent ob\-server might have expected the next step to consist of an examination of the initial (and often incorrect) ``hypotheses\isi{'}\isi{'} made by the child, to determine why it was that that particular hypothesis, rather than any other, was originally selected. Further steps might have con\-sisted of determining in what ways the child discovered the falsity of his original hypothesis and how he subsequently modified it (or selected an alternative) in order to approximate more closely to the linguistic models available to him.

Unfortunately, nothing of the kind was done. The founders of \isi{generative theory} remained grandly aloof from the hare they had started, claiming that real-world acquisition processes were still too chaotic and ill understood to constitute a legitimate object of study and taking refuge in the ``\isi{idealization} of instantaneity\isi{'}\isi{'} described in Chomsky and Halle (1968:Chapter 7). Workers in t1'e field were not simply left to their own devices; they were continually harassed by endless revisions of the theory. Doing acquisition work along Chom\-skyan lines became rather like playing a game which few minutes the umpires revise the rules.

Bearing this in mind-and bearing in mind too that workers in the field not only had no training in the analysis of variability and dynamic process generally but also had been given no reason even to think that such training might be necessary- it is not surprising that their results were somewhat unrevealing. In general, as shown, for example, in Brown and \citet{Hanlon1970}, \citet{Brown1973}, Bowerman ( 1973), etc., the predictions that \isi{generative theory} seemed to make about acquisition were simply not borne out: young children did not
show conclusive evidence that they knew S {}-+ NP VP or other basic PS
%\originalpage{138}
rules; syntactic structures were not acquired in the order that was dictated by their relative complexity, and so on.

At the same \isi{time}, and inspired at least in part by the meager results of generative-oriented work, many scholars began to question the assumptions on which this work was based. Was the input really degenerate? Was \isi{learning} as rapid as had been claimed? Did it take place in the cognitive vacuum that at least seemed to be implied, if not actually asserted, in most generative writing? Upon examination, a number of these assumptions appeared to he partly or even wholly incorrect. Thus, there came about in the early seventies a very rapid and extreme swing of the pendulum, leading to an all but universal consensus among those working directly on acquisition which persists, with relatively minor variations, up to the present.

This consensus, while not ruling out entirely the possibility that some kinds of innate mechanisms may be involved \isi{in acquisition}, systematically plays down and degrades the role of such mechanisms, often regarding them as constituting no more than a ``predisposition\isi{'}\isi{'} to acquire language, whatever that might mean (they never do say). The consensus holds, however, that prelinguistic communication and extralinguistic knowledge (acquired, nat urally, through experience) \isi{play} crucially important roles \isi{in acquisition}, hut that perhaps the most critical role of all is that of the interaction, paralinguistic as well as linguistic, which takes place between the child and the mother (or other caregiver). The mother, it is claimed, models language for the child, adapting her outputs to his linguistic level at every stage. Far from being degenerate, the data she provides are highly preadapted, highly contextualized, and patiently repeated. ``Mothers \textit{teach} their children to speak,\isi{'}\isi{'} \citet{Bruner1979} states. When all these factors are taken fully into account, the consensus claims, the need to posit an innate component in language acquisition shrinks to near zero or even disappears altogether.

Unfortunately, the whole position of this consensus is based on a fallacy-a fallacy that should be readily apparent to all readers of the two previous chapters. That fallacy is perhaps most concisely expressed
%\originalpage{39}
by Sow (197 :367) when she remarks that ``Chomsky's position regard.mg the, unimportance of the linguistic input was unproven, since \textit{all} \textit{c}\textit{.}\textit{htldren},\textit{.} m :UWition \textit{to} possessing an innate liuguistic ability, \textit{also} \textit{receive} \textit{a} \textit{smplifie,} \textit{wll-formed} \textit{and} \textit{redundant corpus{\textquotedbl}} (emphasis added). This 1s quite sunply untrue. The input that the first creole g.eneration in \isi{Hawaii} received was over-simplified rather than simpli\-fied, and was as far from beiug well formed as anyone could imagine ;
and we can assume that in other areas where creoles formed the same state of affairs must have existed. Mother could not teach \textit{;}\textit{hese} children to speak, for the simple and inescapable reason that Mother herself di not know the language-the language didn't exist yet. But even so, without Mother, those children learned how to speak.

In adition to:his fallacy of fact, the Bruner-Snow position is base on a sunple log1cal fallacy. If we accept that in the vast majority of ci.rcumstances mothers do teach and children do learn, it by no means. follows that children learn BECAUSE mothers teach. It would be logically quite possible to argue that there is no connection whatso\-ever btween mothers\isi{'} teaching and children's \isi{learning}, any more than there is between ,children\isi{'} walking and uncles\isi{'} dragging them around the room by thetr fmgert1ps. If it could be shown that without well\-formed input from the mother the child could not learn to speak then we might indeed assume a causal connection. fn fact, we hav; sown the reverse: well-formed input from the mother cannot constitute eve a necessary condition for children to acquire language; for, otherwise, creoles could not exist.

But our argument, though logically correct, need not be pushed \textit{t} its logical extreme. I am perfectly willing to accept that if mother did not teach her child \isi{English}. that child might have a much harder tune \isi{learning} it-even that the child might never acquire a perfected form of the language, but might significantly distort it in the direction of ,the. kind d pattern we reviewed in the last chapter. All I want to chum is that 1f we persist in believing that the child must have input m order to learn, we shall continue to misunderstand completely the way in which he does learn a developed, natural language. Just as
%\originalpage{140}
the child does not need mother in order to learn, so he could not learn even with a myriad of mothers if he did not have the genetic program that alone enables him to take advantage of her teaching.

In fact, the evidence we reviewed in the first two chapters of this book has simply never been taken into account in studies of child language acquisition. The vast majority of scholars in the f eld evince no awareness whatsoever of the existence, let alone the possible signi£cance, of pidgins and creoles; an honorable exception is Slobin (especially Slobin 1977). Unfort unately, the data available to Slobin at the \isi{time} were by no means as ample as those given in the present volume; moreover, he makes the common mistake of supposing \ili{Tok Pisin} to be paradigmatic of normal pidgin-creole development. Still, even limited access to pidgin-creole data is better for acquisitionists than none, and in consequence we shall find the work of Slobin and his associates illuminating on a number of points in the pages that
follow.

Meanwhile, in the absence of the insights provided by creolization, the current .paradigm has provided us with much information that we lacked before{}-on the nature of input to the child and of child\-caregiver interaction; on the \isi{acquisition of} turn-taking, conversational routines, and the kind of social appropriateness summed up under

%\originalpage{141}

{\textquotedbl}acquisition strategy{\textquotedbl} ``ha made us aware of some of the ways by which the child may possibly 'get into\isi{'} the linguistic system. It has sl.iown us the importance of perceptual mechanisms for interpreting utterances, and how as adult speakers with full lingllistic competence we nevertheless rely on a number of short cuts to understanding . . . . The concept of language \isi{acquisition strategies} has told us much\-except how the child acquires language.\isi{'}\isi{'} \citet{Bowerman1979} , who cites this passage with approval, further points out that while such strategies may enable children to understand utterances which still lie outside their developing grammars, those strategies do not and indeed cannot, in and of themselves, assign structural descriptions to rhese novel tterances. Yet children must achieve this kind of structural knowledge

If they are subsequently to use such utterances themselves in a produc\-tive and creative way-understanding something is miles awav from manipulating that something freely and voluntarily. fn other ,words, strategies belong in the realm of performance, and the problem is, how do you get from performance to competence? Small wonder that so many supporters of the current consensus seek to downgtade, ignore,
or even abolish the competence-performance distinction. But real problems cannot be defmed away.

. I propose, therefore, to review the literature on acquisition as

Hymes's concept of ``commum•cati•ve competence H; on

;'acqm5• 1•t1• on

1t concerns certain core syntactic and semantic structures, in particular
strategies\isi{'}\isi{'} based on contextualization, semantic and pragmatic clues to the function of novel structures, etc., etc.-and yet, as more and more thoughtful scholars are realizing, the gathering of this information has merely served to conceal the fact that the central question of acquisition, the question with which the early generativists did at least struggle, however unsuccessfully, is simply not being answered:

How can the child acquire syntactic and semantic patterns of great arbitrariness and complexity in such a way that they can be used creatively without making mistakes?

Cromer \{1976:353), for inst:.mce, observes that the concept of
some that we have had occasion to deal with in earlier chapters, to see whether what we know of the acquisition process supports or fails to support the hypotheses adva11ced at the end of the last chapter. To the extent that these hypotheses are supported, the general theory of a human \isi{language bioprogram} will tend to be confirmed. To the
extent that these hypotheses fail to be supported, doubts will be cast upon the theory, although the reader should perhaps be reminded that not even the most thorough refutation, in the arena of child language, would make the initial problem which led to the theory --
the fact that creoles are learned without experience-miraculously ``'O

0

away. At worst, such refutation would merely drive us back to a
reconsideration of that problem.

%\originalpage{142}

But before commencing this review, three words of caution are in order: the first concerning the data; the second concerning the reviewer; the third concerning the theory.

From our point of view, the data suffers from two defects. First,
much of it has been presorted in ways that automatically diminish its utility. There are a variety of reasons for this, but I shall deal with oly one in detail, since it is fairly typical. A.round 1970, when acqu:si\-tionists were still concerned with proving (or disproving) generative predictions about acquisition, it appeared that one way of doing this would be to see whether features of a language were acquired in an order which conformed to some kind of hierarchy of grammatical complexity-simplest first, most complex later on. But In order to do this, it was necessary to determine exactly what one me;uit by ``\isi{acquisition of} a feature.\isi{'}\isi{'} Children are such messy creatues; mste.ad of quietly going to bed one night without a feature, and waking ;'1th it, as the Chomskyan \isi{idealization} of ``instantaneus acqu1S1tion\isi{'}\isi{'} suggests they should, they stubbornly insist on alternatmg p:esence and absence of that feature in appropriate con\isi{texts}, not to mention absence and presence of that feature in inappropriate con\isi{texts}, for periods of weeks, months, and occasionally even years.

Not only that, but the little beasts do not .even p\_roed. as
reason dictates they should, gradually and cumulatively drmmishmg inappropriate usages as they increase appropriate ones; on the contary, a graph of their appropriate productions zigzags up and down like a malaria victim's temperature chart, before finally leveling off at or near the 100 percent mark. The innocent observer might think that the most interesting thing you could do \isi{in acquisition} study would be . to figure out \textit{why} this happens, but as usual, he would e disappointed. Fashion and expediency dictate that order must be imposed on dts\-order: to determine the \isi{order of acquisition}-a ``need\isi{'}\isi{'} dictated merely by current theory{}-\citet{Brown1973} established a .purely :1'bitrary ``.cri\-terion\isi{'}\isi{'} for acquisition, i.e., a 90 percent production rate m appropriate environments, maintained over three consecutive recording sessions. The reign of the criterion mere \{ reinforced what has always been
%\originalpage{143}
a trend \isi{in acquisition} studies, and a deplorable one·\textsubscript{•}\textsubscript{ }to look to the o\textsubscript{0}{}-oal rather than the path, to ask ``What has the child acquired?\isi{'}\isi{'} rather than
``How has he acquired it?\isi{'}\isi{'} In consequence, masses of potentially valuable data, which would be required by any interesting acquisition theory, were simply flushed down the drain.l

In addition to deficiencies of this nature, we have to remember that all the data collected to date were collected for very different purposes than the present one. It is a general law applicable to all research that one tends to find what one is looking for, and not to fmd what one is not looking for. Hence, it would be unrealistic if we expected to find massive quantities of unambiguous evidence pointing toward the truth of our theory, which had yet somehow been missed by previous observers. The most that one can ever hope for from data collected under other assumptions and for other purposes than one's own are oblique hints, gaps that one's own hypotheses might fill, puzzles set aside that might begin to make sense in the context of a different framework. However, if one finds any of these at all, it is a reasonable assumption that a purposeful search of raw data sources would reveal much more-something comparable to the invisible eight-ninths of the iceberg.

With regard to the second word of caution, I can lay claim to no special expertise in the field of child language. In creoles, I have fourteen years\isi{'} experience, most of them spent in direct contact with native creole speakers, so that I can speak in that field with some degree of confidence. In language acquisition, I can claim to be no
 more than an assiduous reader of the literature, and in consequence, both. my knowledge and my understanding may be at fault sometimes. On the credit side, l can only offer complete uninvolvement m any of the controversies that have racked the field (for, as we shall see, my position, although innatist, is really no closer to the orthodox Chomskyan one than it is to the ``motherese\isi{'}\isi{'} school), and the freshness
of perspective that a novel viewpoint may on occasion bring. So be it: the facts will decide.

Finally, a word of caution about the theory. Straw-man versions
%\originalpage{144}
of innatist theories abound, and in particular, those which claim that to stress the function of an innate component \isi{in acquisition} is auto\-matically equivalent to completely writing off all other modes of \isi{learning} and all other aids to \isi{learning}. In the present case, this par\-ticular straw man has even less substance than usual. The \isi{language bioprogram} theory is, as we shall see in Chapter 4, an evolutionary theory, and the bioprogram itself is an adaptive evolutionary device. Now, it is the nature of such devices that they are facilitatory , not pre-emptive; that is to say , their whole adaptive function is lost if they force a species into a position where that species is dependent upon them and upon them alone, by inhibiting the action of other adaptive processes. In addition to whatever we may have in the way of innate language equipment, we also have a wide variety of \isi{learning} strategies and problem-solving routines which are applicable to a range of situations far broader than language. It would be absurd to suppose that in the presence of data classified as ``linguistic,\isi{'}\isi{'} all these routines and strategies should simply switch off.

It would be equally absurd to suppose that they and the innate language component would be always and necessarily at war with one another. Sometimes their respective promptings may combine, some\-times they may point in opposite directions; which way is an empirical issue at anY given point. But their interaction must form the core of any complete \isi{description of} language acquisition. If Ihave ignored
other resources in the present study , and have concentrated solely on the innate component, that is for strategic purposes only; besides, general cognitive processes have had far more than equal \isi{time} in the last decade, and the turn of hardcore syntax and semantics has come around again. But Ibelieve that in order to acquire language-a feat which is, so far as we yet know, without parallel in the entire universe\-we need every ounce of help, particular or general, innate or acquired through experience, that we can get. To pit one kind against another simply demonstrates a failure to understand how complex language really is.\textsuperscript{2}

% {\textbackslash}

%\originalpage{45}

With these preliminaries disposed of, we can begin our review. The eVJdence we shall consider will fall into two quite separate classes. One class will consist of the ``incorrect hypotheses\isi{'}\isi{'} which, in the course of language acquisition, children often make, yet which often seem t have o simpfo explanation either in the structure of the input th child receives or m any general theory of acquisition. The simi\-lanty between such ``hypotheses\isi{'}\isi{'} and the structures which actually em\_ege as part of the grammars of creole languages is often quite smk1:1g, and when I first contemplated writing this chapter, Ifelt certain that examples drawn from this class would constitute by far
the strongest evidence in favor of the bioprogram theory. After writing
the first draft of this chapter, however, I became much less certain, not so much because of the weakness of the original evidence-although there are. some phenomena, as Ishall show, which may allow alternative explanations{}-but because of the growing impression that a much subtler and less obvious class of evidence made on me.

As the ``icorret hypotheses\isi{'}\isi{'} suggested, there were many things m language .which children seemed to find quite difficult to learn, often spendmg years before they acquired full control over the struc. tuts concerned. n the other hand, there were certain other things wluch. seemed to .g e diem no trouble at all, which they learned very early m the acquisition process and/or without any of the ``mistakes\isi{'}\isi{'} which arose so frequently in other areas. On principle, one might suppose that these differences correlated with some kind of scale of relative difficulty, and yet it was extremely difficult to see exactlv what objective factors might constitute such a scale. Indeed, from commonsense linguistic viewpoint, some of the things that were easily and effortlessly acquired looked a lot more difficult to learn than some of the things that gave so much trouble.

But obviously, to talk about things being ``difficult\isi{'}\isi{'} or ``easy\isi{'}\isi{'} from an adult standpoint is totally irrelevant in an acquisition context. What is ``difficult\isi{'}\isi{'} or ``easy\isi{'}\isi{'} for the child is all that is of interest and one might therefore conclude that what seems ``difficult\isi{'}\isi{'} to u; might seem ``easy\isi{'}\isi{'} to the child, and vice versa. However, a moment's
%\originalpage{146}
thought should show that it was not so much the adult viewpoint as the use of the words ``easy\isi{'}\isi{'} and ``difficult\isi{'}\isi{'} themselves that was at fault in our original formulation.

Terms like ``easy\isi{'}\isi{'} and ``difficult\isi{'}\isi{'} imply an act of evaluation
which in t urn depends on the capacity to compare one task with another, which in turn depends on prior experience of tasks with differing levels of difficulty. Thus, when we acquire a second language, we can say that its derivational morphology, for example, is difficult to learn, while its \isi{relativization} processes, say, are relatively easy. Such remarks are meaningful only because we already know a language and can measure features of the second language against those of the first. If we had not previously learned a language, we would have no standard of comparison; moreover, it is at least in part the nature of what we have already learned that determines whether what we are now about to learn will turn out easy or difficult for us.

Now, if we say that something is easy for a two-year-old to
learn, we cannot possibly mean any of this; all we can mean is that the child is somehow preadapted to learn that thing, rather than other things, or that in terms of the present theory , he is programmed to learn it. If, as we shall see is the case, the things that children learn early, effortlessly, and errorlessly turn out repeatedly to be key features of creole languages, which the children of first creole generations acquire in the absence of direct experience, we can then assume that such early, effortless, and errorless \isi{learning} results, not from charac\-teristics of the input, or from the efforts of the mother-since the features involved are often too abstract to be known to any but the professional linguist-but rather from the functioning of the innate bioprogram which we have hypothesized.

I find evidence of this second class to be even more convincing
than that drawn from systematic error, and will accordingly begin by considering some examples of it. The first concerns the \isi{learning} of the specific-nonspecific distinction (henceforth SNSD) by \isi{English}\-speaking children. This distinction, as we saw in Chapters 1 and 2, is explicitly represented in all creole grammars by the opposition
% {\textbackslash}
% ACQUISITION 147
between zero and realized determiners. It is expressed in \isi{English} too, but much more obliquely, as we will see.

The most comprehensive study of the \isi{acquisition of} \isi{English} \isi{articles} is that of Maratsos (1974, 1976), who confirmed by means of ingenious experiments the naturalistic observations of Brown ( 1973), i.e., that the article system is mastered at a very early age. Some of Maratsos\isi{'} findings have been questioned in subsequent work (Warden 1976, Karmiloff-Smith 1979), but such criticisms relate only to the earliness with which the definite-nondefinite distinction is acquired. No one ·has challenged Maratsos\isi{'} finding that the SNSD is handled virtually without error by three-year-olds, well ahead of the earliest date by which the child masters the definite-nondefinite distinction.

At first sight, this is an odd finding since the latter distinction is clearly marked in \isi{English}, while the SNSD is not. In \isi{English}, ``definite\isi{'}\isi{'} really means presumed known to the listener, whether by prior knowledge \textit{(} \textit{{\textquotedbl}the} man you met yesterday{\textquotedbl}), uniqueness in the universe \textit{(} \textit{{\textquotedbl}the} sun is setting{\textquotedbl}) , uniqueness in a given setting \textit{(} \textit{{\textquotedbl}the} battery is dead{\textquotedbl}-cars do not usually have more than one battery), or general knowledge that a named class exists \textit{({\textquotedbl}the} dog is the friend of man{\textquotedbl}). ``Indefinite\isi{'}\isi{'} really means presumed unknown to the listener, whether by absence of prior knowledge ({\textquotedbl}a man you should meet is Mr. Blank{\textquotedbl}), nonexistence of a nameable referent ({\textquotedbl}Bill is looking for \textit{a} wife{\textquotedbl}), or nonexistence of any referent ({\textquotedbl}George couldn't see \textit{an} aardvark anywhere{\textquotedbl}). In other words, the two classes are systematically distinguished by the distribution of \textit{the} and \textit{a/an.}

Specific and nonspecific, however, are not systematically dis\-tinguished. Consider the following:

\ea\label{ex:1}
 If you're sick, you should see \textit{the} \textit{doctor} (NS).
\glt
\z

\ea\label{ex:2}
 Call \textit{the} \textit{doctor} who treated Marge (S).
\glt
\z

\ea\label{ex:3}
 \textit{The} \textit{doctor} may succeed where the priest fails (NS).
\glt
\z

\textit{I}4\textit{I }\textit{Dogs} are mammals (NS).

\ea\label{ex:5}
 \textit{The} \textit{dog} is a mammal (NS).
\glt
\z

\ea\label{ex:6}
 \textit{A} \textit{dog} is a mammal (NS).
\glt
\z

%\originalpage{148}

\ea\label{ex:7}
\textit{A} \textit{dog} just bit me (S).
\glt
\z

\ea\label{ex:8}
 Mary can't stand to have \textit{a} \textit{dog} in the room (NS).
\glt
\z

In fact·, the only way in which \isi{English} distinguishes specifics from nonspecifics is in constructions with at least two \isi{articles}. If a given referent is specific, it will receive \textit{a} on first mention and \textit{the} on second and subsequent mention:

\ea\label{ex:9}
 Bill bought \textit{a} \textit{cat} and \textit{a} \textit{dog,} but the children only like \textit{the} \textit{dog.}
\glt
\z

If a given referent is nonspecific, it will receive \textit{a} on first mention and on second and subsequent mention:

\ea\label{ex:10}
 Bill wanted to buy \textit{a} \textit{cat} and \textit{a} \textit{dog,} but he couldn't find \textit{a} \textit{dog}
\glt
\z

that he really liked.

Maratsos constructed an ingenious set of stories which his child subjects were asked to complete. In some of the stories, reference was made to a specific entity; in others, to a nonspecific entity; in both cases, naturally, the entity was introduced into the story as \textit{a} \textit{N}\textit{P.} However, the completion task required the child to produce \textit{a} \textit{NP} just in case the entity was nonspecific, and \textit{the} \textit{NP} just in case the entity was specific, in accordance with the rule illustrated in /9/ and /10/ above (for full \isi{texts} of the stories and a more complete \isi{description of} the experiments, see Maratsos 1976).

The success rate in this experiment was almost 90 percent for three-vear-olds and over 90 percent for four-year-olds. In order to maintin these high rates, the children had to determine that out of some NPs identically marked, half had specific real-world referents and half had not. The stories were original and contained no contextual clues as to the status of the referents. How did the children succeed so often?

Maratsos himself was surprised and impressed by his subjects\isi{'} capacities, and he discusses the implications of his experiments at
%\originalpage{149}
some length and with great insight. He notes that the high frequency
of \isi{articles} in adult speech is often regarded as an adequate explanation
of the relative earliness and lack of error shown in the \isi{acquisition of} \isi{articles}. He points out, however, that ``although the frequency of [\isi{articles}\isi{'}] use may somehow serve to bring them to the child's atten\-tion and provide data for him, he must still select and attach to the \isi{articles} just those abstract . differences in the circumstances of their use that correspond to the specific-nonspecific distinction. One clear requirement is that he have available some conceptual understanding of such matters as the difference between the notion of any member (or no member) of a class and that of a particular class member. This understanding must be sufficiently well articulated for the child to perceive just this difference in the circumstances of use of the definite and indefinite morphemes and construct the meaning of the terms accordingly\isi{'}\isi{'} \citep[453]{Maratsos1974}.

Let us try to reconstruct the process or processes by which the child might arrive at this perception. We will ignore the problems that arise from the child's original isolation and recognition of \isi{articles}, although these are far from trivial (especially with \textit{a,} so frequently reduced to an unstressed schwa and so closely linked to its following NP that morpheme boundary perception becomes quite difficult), and deal solely with how, having recognized them, he determines their functions. If the conventional accounts are correct, the child can do this in only two ways-through linguistic context or through extralinguistic context.

The nature of the problems involved can be better understood if we compare the \isi{acquisition of} \isi{articles} with the \isi{acquisition of} \isi{plural} marking, which occurs at roughly the same age (a very few weeks later, according to Brown 1973). The \isi{plural} morpheme marks a single, straightforward distinction-one/more than one-and it does so bi\-uniquely, that is to say, in a one-morpheme, one-meaning relationship: when the morpheme is present, one meaning is entailed; when it is absent, the other meaning is entailed. Articles are, from a purely formal viewpoint, much more complex than that. Three \isi{articles}, \textit{the,} \textit{a,} and
%\originalpage{1}
\zero, represent two distinctions-supposed-known-to-listener \textit{I} sup\-posed-unknown-to-listener and specific-referent \textit{I} no-specific-referent\-but without the biuniqueness that relates semantics to surface repre\-sentation in the c'ase of plurals. Instead, with regard to the second distinction only (the SNSD), there are two morphemes with one meaning (both \textit{a} and \textit{the} can have specific reference) and one mor\-pheme with two meanings \textit{(}\textit{a} can be both specific and nonspecific).

Let us suppose that the child can first factor out the distinction between \textit{o,} and \textit{the} (although in fact he cannot even rely on this aid; Warden [1976] and Karmiloff-Smith [1979] show that it will be several years before he is able to overcome this potential distraction). He then has to distinguish specific from nonspecific \textit{a. }One might think he could do this by distinguishing between linguistic environ\-ments. For instance, the scope of \isi{negation} is often crucial in determin\-ing whether a given occurrence of \textit{a} \textit{NP} is specific or nonspecific: the difference between \textit{I} \textit{saw} \textit{a} \textit{dog} (S) and \textit{I} \textit{didn't} \textit{see} \textit{a} \textit{dog} (NS), for instance. So is the scope of desiderative verbs: the difference between \textit{I} \textit{want} \textit{a} \textit{dog} (NS) and \textit{I} \textit{have} \textit{a} \textit{dog} (S). Those who put their trust in extralinguistic context will, however, point out, quite correctly , that things like desiderative scope and negative scope are themselves ex\-tremely abstract relations, unlikely to be capturable by two-year-olds.

But in fact the problem is even tougher than we have suggested; there are many cases in which a mere tense switch marks the SNSD:

\ea\label{ex:11}
 When you see \textit{a} \textit{dog} (NS) , are you frightened?
\glt
\z

\ea\label{ex:12}
 When you saw \textit{a} \textit{dog} (S), were you frightened?
\glt
\z

Since the child's control of tense is, at the appropriate age, highly questionable at best, it is implausible to suppose that he could utilize such clues.\textsuperscript{3} Again, there are cases when desiderative scope alone is insufficient to mark the distinction:

\ea\label{ex:13}
 Your little sister wants \textit{a} \textit{dog-any} kind of dog (NS).
\glt
\z

\ea\label{ex:14}
 Your little sister wants \textit{a} dog-and it's that one (S)!
\glt
\z

% {\textbackslash}
% 
% ACQUISITION 151

In fact, the only reliable indicator of the SNSD is not a single article use, but a series of \isi{articles} uses; an \textit{a-a }sequence, as in /10/ above, as opposed to an \textit{a-the} sequence, as in /9 / above.

However, as \citet[95]{Maratsos1976} again points out, it is at least highly questionable whether the child can take advantage of clues provided by sequences, especially when members of such sequences are not necessarily adjacent-as they are in /9 / and /10/-but may be separated by several sentences: ``It is easy to forget that the child, to the best of our present knowledge, does not have an extensive corpus of data at any one \isi{time} with which to work. He probably cannot record numerous long stretches of conversation and all of the con\-textual information that accompanied them, as can an adult linguist investigating a novel language.{\textquotedbl}

We must therefore conclude that a child would be, at best, highly unlikely to derive the SNSD from analysis of purely linguistic context.

Yet is it any more likely that he could learn it from physical
experience or any other kind of extralinguistic source? As noted above, recent studies have concentrated heavily on the here-and-newness of speech aimed at children, and on the child's prelinguistic experiences in the world of objects. It is hard to see just how either of these could help with the SNSD. As \citet[94]{Maratsos1976} remarks, ``specific and nonspecific reference are connected in no clear way with external physical attributes or relations of perceived objects.\isi{'}\isi{'} For example, nonspecific reference is usually (although not always) made in the absence of any member of the referent class: \textit{we} \textit{don't} \textit{have} \textit{a} \textit{dogg}\textit{y}, \textit{Dad}\textit{d}\textit{y's} \textit{looki}\textit{n}\textit{g} \textit{for} \textit{a} \textit{doggy} \textit{for} \textit{you,} \textit{a} \textit{doggy} \textit{would} \textit{be} \textit{nice} \textit{to} \textit{play} \textit{with,} \textit{wouldn't} \textit{it?} and so on. But specific reference is made just as often in the absence of the referent : \textit{a} \textit{dog} \textit{bit} \textit{Jessie} \textit{yesterd} \textit{ay,} \textit{I} \textit{saw} \textit{a} \textit{dog} \textit{you'd} \textit{really} \textit{have} \textit{liked} \textit{in} \textit{town} \textit{toda}\textit{y}, and so on. How does the child determine that of the two absent sets of referents one is concrete while the other is only hypothetical? If he did not do so, he would score no better than chance on Maratsos\isi{'} tests.

%\originalpage{152}

While it is true that many concepts are formed by the child prior to language \isi{learning}, these are generally concepts which relate to physical objects which the child can see, touch, etc. Moreover, it is reasonably clear that such concepts ARE arrived at by interaction with experience rather than by merely processing language input. If the child only processed linguistic tokens of \textit{dog,} for example, he would pre\-sumably apply the term only to members of the appropriate species; whereas, as is well known, the initial meaning of \textit{dog,} for the child, is likely to be 'any four-legged mammal\isi{'}. Thus, we know that the child reaches out ahead of linguistic experience, so to speak, in order
to derive ways of talking about the world.

But how could the child derive knowledge of purely abstract relationships from direct experience? A comparison with \isi{plural}-marking acquisition is again very much to the point. Plural marking is directly associated with relations that the child is physically able to observe. He can see and feel at any given \isi{time} whether he has one toy or several, whether he is allowed only one cookie or more than one ; the gramma\-tical marking of nouns correlates directly with manifest and obvious differences in his perceptual field. But the distinction between an actual member of a class (which more often than not is not physically present) and an imaginary representative of that same class is in no way one that · can be determined by the organs of perception, or inferred from any kind of direct experience. The SNSD involves comparisons, not between physical entities, but between purely mental representations; one can only marvel that a child, for whom the boundaries between . real and unreal are notoriously vague, should be able to make it at
all, by \textit{any} means.

Indeed, that he should even hypothesize such a distinction-a would, presumably, be claimed by those who believe in a hypothesis• forming, hypothesis-testing LAD-is highly implausible. Even abou possible functions of \textit{a} and \textit{the,} there are many possible hypothese that might be made. Since definites tend to be subjects while indefinite tend to be objects, one might hypothesize that \textit{the} marks agents and
% ACQUISITION 153
\textit{a} marks patients. Since \textit{the} often co-occurs with NPs that are physically present and \textit{a} with NPs that are physically absent one might hy\-pothesize that \textit{the }and \textit{a} mark poles of some kind of proximal-distal distinction. In fact, so far as we know, such hypotheses are never made. In any case, they would affect only \textit{a} and \textit{the; }with regard to \textit{a} alone, why on earth should the child even start by hypothesizing that there are really two kinds of \textit{a? }Moreover, since two-year-olds use few or no \isi{articles}, and the SNSD is acquired by about the age of three, it would have to be just about the first hypothesis the child makes\-there would hardly be \isi{time} to frame and discard any other. To say that the child invariably forms a correct hypothesis about the SNSD as his first hypothesis is simply an issue-dodging way of saying that he is programmed to make the SNSD.

Indeed, we can only conclude that the SNSD would be quite impossible to learn, by means of linguistic data, or of experience, or of any hypothesis-forming process, or of any feasible combination of these. For the child to make the SNSD as early and as successfully as he does, he would have to be somehow preprogrammed to make it.

This proposal is strongly supported by the creole data reviewed in Chapters 1 and 2. We saw there that the SNSD was made by the first creole generation in \isi{Hawaii} (even though none of their HPE-speaking parents made it ) and that it is made consistently, and always by the same means, in all creole languages. If we assume a \isi{language bioprogram} that includes the SNSD in its specifications, the prob]P.m of how the
,child acquires that distinction in \isi{English} becomes a manageable nne. The child knows of the distinction in advance and is therefore looking put (at a purely subconscious level, of course) for surface features
.m the target language that will mark it. If no other feature is pre\-
programmed for NP, which is likely, then the fact that the SNSD
. constitutes the child's first ``hypothesis\isi{'}\isi{'} is no longer bewildering, but
.M automatic consequence of the theory.
The skeptical reader may, however, ask: if creole children follow\-ing the bioprogram universally mark the SNSD by allotting zero marking to nonspecifics, how is it that children \isi{learning} \isi{English}, prior to
%\originalpage{1}
correctly interpreting the two \textit{as,} do not mark, or at least attempt to mark, nonspedfics with zero, as creole children do? The answer is that we do not know that they do not.

Earlier, I referred to deficiencies in the data due to excessive
concentration on the goals rather than the paths of acquisition. Here is a case in point. Even as conscientious and insightful a scholar as Maratsos confesses (1974:450) that ``only full noun phrases of the form article plus noun were counted; answers which included no article, such as \textit{boy, }were not counted in the analysis.\isi{'}\isi{'} Another careful inves\-tigator, \citet{Brown1973}, who allots some sixteen pages to a discussion of \isi{articles} in early child speech, makes no reference to zero forms, and in what he claims is a ``full list of errors in definite and nondefinite reference for Adam, Eve and Sarah from Stages IV and V\isi{'}\isi{'} (1973: \tabref{tab:51}) includes only cases of \textit{a} where \textit{the} is indicated, and cases of \textit{the} where \textit{a} is indicated-no zeros at all. Yet from what Maratsos says, and from mere common sense, one knows there \textit{must} have been zeros; after all, the child has no \isi{articles} at the two-word stage, and obviously does not acquire the surface forms overnight.

The present theory predicts that when a substantial body of early child language is properly examined, there will be found to be a significant skewing in article placement, such that a significantly higher percentage of \isi{articles} will be assigned to specific-reference NP, while zero forms will persist in nonspecific environments longer than elsewhere. Such examination affords a simple and straightforward means of empirically testing the claims made about the innateness of the SNSD in this chapter.

We will now examine another distinction which is made even earlier and without , apparently, even a single reported case of error. This is the distinction between states and processes, including under the latter rubric verbs of experiencing as well as action verbs (hereafter referred to as the state-process distinction, or SPD). The SPD is di\-rectly involved in the \isi{acquisition of} the \isi{English} progressive marker

\textbf{\textit{{}-ing.}}

% {\textbackslash}

%\originalpage{155}

In general, the \isi{acquisition of} novel morphology by the child is attended by cases of over-\isi{generalization}, a number of which are dis\-cussed in \citet{Cazden1968}. Thus, the consistency in the final segments of possessive pronouns leads to production of the aberrant form \textit{*mines,} while plurals such as \textit{*sheeps,} \textit{*foots} (or \textit{*feets}\textit{),} \textit{*mouses,} etc., and past-tense forms such as \textit{*corned,} \textit{*goed} (or \textit{*wented}\textit{),} \textit{*buyed, }etc., occur in the speech of most, if not all, child learners of \isi{English}.

The \textit{{}-i}\textit{n}\textit{g }form is acquired even earlier than the \textit{{}-ed} form (before
any of the other thirteen morphemes studied in Brown [1973], and as early as the second year in at least some cases). Also, just as there are verbs that do not take \textit{{}-ed,} there are verbs that do not take \textit{{}-ing} (with certain qualifications, see Sag 1973), such as \textit{lik} \textit{e,} \textit{want,} \textit{know,} \textit{see,} etc. These verbs are quite common in children's speech, probably as common as many of the irregular verbs to which children incor\-rectly attach \textit{{}-ed.} Yet, apparently, children never ever attach \textit{{}-ing} to \isi{stative} verbs.

\citet{Kuczaj1978} has argued that the two cases are not really commensurate since with the \isi{past tense} there are other ways of mark\-ing than \textit{{}-ed }(just as with plurals there are other ways of marking than

\textit{{}-}\textit{s}\textit{),} whereas in the case of \textit{{}-ing,} \isi{English} has no alternative way of marking progressive aspect. This argument is somewhat disingenuous since zero can be a term in a subsystem, and it is hard to see what the difference would be between, on the one hand, adding \textit{{}-s} to \textit{sheep }to make \textit{sheeps} or \textit{{}-ed} to \textit{put} to make \textit{putted} , and, on the other hand, adding \textit{{}-ing} to \textit{like} to yield \textit{I} \textit{am} \textit{liking} \textit{you. }A more pertinent observa\-tion would be that verbs which do not take \textit{{}-ing,} as opposed to verbs which do not take \textit{{}-ed} or nouns which do not take \textit{{}-s,} constitute a natural semantic class; we shall return to this point in a moment.

In fact, Kuczaj undercuts his own argument by observing that children do indeed over-generalize \textit{{}-ing,} but not to \isi{stative} verbs\-rather, to nonverbal items, as in /15/:

\ea\label{ex:15}
 Why is it weathering?
\glt
\z

(presumably, 'Why is the weather so bad?\isi{'})

%\originalpage{156}

Note that, in fact, \textit{weather }is a plausible candidate for admission to the list of ``climatic\isi{'}\isi{'} verbs that yield expressions such as \textit{it} \textit{is} \textit{raining} \textit{/} \textit{snowing} \textit{/thundering, }etc. But all these verbs have in common the fact that they are nonstatives, as \textit{weather }would be also if it were a verb in the sense of /15/. The fact that children will generalize \textit{{}-ing} even to nouns IF AND ONLY IF SUCH NOUNS HAVE A PLAUSIBLE

NONSTATIVE READING makes their abstemiousness with respect to \isi{stative} verbs even more significant.

Brown (1973:326ff.) rightly regards it as remarkable that chil\-dren should be ``able to learn a concept like involuntary state before they [are] three years old,\isi{'}\isi{'} and explores several hypotheses which might account for such \isi{learning}. In the case of one child, Eve, he was able to show that many nonstatives, as well as statives, were unmarked by \textit{{}-ing,} and that the unmarked nonstatives were precisely those which Eve's mother seldom used with progressive aspect ; on the other hand, the nonstatives which the mother did use frequently with \textit{{}-ing} were precisely those which appeared with \textit{{}-ing }in Eve's speech. However, a similar relationship did not hold for the other children in Brown's study; and as Brown himself pointed out, even if it had held, it would not have provided a solution. For anyone who claimed that children delayed applying \textit{{}-I}\textit{n}\textit{g} to a verb until they learned from experience that it was ``{}-ingable\isi{'}\isi{'} would then be forced to explain why a similar caution and restraint was not applied to other morphemes, like \textit{{}-ed }and \textit{{}-s,} where over-generalizations abounded.

Brown next considered the possibility that the SPD was learned from imperatives and transferred to progressives, since the verbs that will not take \textit{{}-ing} are just those that cannot be used in the \isi{imperative}. Against this possibility, Brown argued that it would depend also on \isi{imperative} usage being errorless; and it was simply impossible to tell whether this was the case since, especially in Stage I, children's impera\-tives are often formally indistinguishable from their declaratives \textit{(} \textit{want} \textit{cookie} looks like an \isi{imperative}, but is probably no more than the child's version of 'I want a cookie\isi{'}).

A stronger argument against t)ie ``\isi{imperative} transfer\isi{'}\isi{'} hypothesis,
%\originalpage{157}
not made by Brown, involves first recognizing that Brown's argument is in error; children could learn imperatives through trial and error and, having learned at last the list of verbs which could not be imperatives, simply apply that knowledge to the \isi{learning} of \textit{{}-ing.} But trial-and-error \isi{learning} of imperatives is more implausible than errorless \isi{learning} of imperatives, and for the following reason: in trial-and-error \isi{learning}, the child must correct himself simply through observing that others produce forms different from his (we know that overt correction of grammar, as opposed to content, is rare among parents). Thus, the child who says \textit{drinked} eventually becomes aware that others say \textit{drank} , and revises his grammar accordingly. If such things did not come to his attention, he would presumably go on saying \textit{d}\textit{rinked} indefinitely.

But negative evidence cannot function in this way. Let us sup\-pose that the child who said \textit{want} \textit{cookie} really was urging someone else to desire a cookie. Would the fact that he did NOT hear others saying

\textit{*want} \textit{some} \textit{chocolate} or \textit{*hate} \textit{naughty} \textit{bunny} deter him? It is hardly likely. I know of no facts which would indicate that a child needs positive reinforcement, as well as an absence of counterexamples, in order to maintain his current grammar. The child may be diverted from that ammar by the existence of contradictory forms to which he is obliged to pay attention; we can hardly expect him to pay atten\-tion to something that is NOT happening.

Moreover, on a purely pra,a,matic basis, trial-and-error \isi{learning} of imperatives is unlikely. A child's early imperatives are all action\-oriented, aimed at getting people to pick him up or put him down, bring nice things to him and take nasty things away. It would be bizarre if he sought instead to influence the thought-processes and emotions of others by commanding them to want, need, know, etc. In fact, the likeliest possibility is that children do not acquire the SPD from imperatives, either errorlessly or by trial and error, because they them\-selves would only ever need non\isi{stative} imperatives for pragmatic reasons. They would not know whether statives could be used as imperatives because the opportunity for such use would simply not
%\originalpage{158}
have occurred-unless of course they were already programmed with the SPD, and thus ``knew\isi{'}\isi{'} that such uses were impossible, without requiring experience to prove it.

But whereas the use of \isi{imperative} statives might seem bizarre, the use of progressive \textit{{}-ing} with statives would surely appear, to a child not programmed with the SPD, to be the most natural thing in the world. For \textit{{}-ing} is applied to verbs with present reference, and when a child wants or sees or likes something, it is right now that he does it.

\textit{*I wanting} \textit{teddy} \textit{(now}\textit{)} or \textit{*I} \textit{seeing} \textit{pussy} \textit{(now}\textit{)} would surely appear, to such a child, every bit as grammatical as \textit{I} \textit{playing} \textit{pee} \textit{k}\textit{aboo} \textit{(now}\textit{)} or \textit{I} \textit{sitting} \textit{potty} \textit{(}\textit{now)}\textit{.} Nobody could claim that the distinction emerged from experience, or from context; on the contrary, both experience and context would point in a contrary direction.

Finally, Brown considered the possibility that the distinction is innately known. However, he rejects this possibility because of what he claims are ``fatal difficulties.\isi{'}\isi{'} Since neither he nor other scholars who have discussed the issue (e.g., Kuczaj 1978, Fletcher 1979) have advanced any serious alternative to the innatist suggestion, we should examine Brown's ``difficulties{\textquotedbl}{}-bearing in mind that they arose out of a theory of innateness quite different from this one-and see whether they are really as ``fatal\isi{'}\isi{'} as he believes.

The first difficulty is that, according to Brown, children do not behave as innatist theory predicts with respect to categories other than state-process. If they came equipped with a full set of syntactic and/or semantic subcategories, ``they ought to attempt to order regular and irregular inflections in terms of one or another of the innate subcate\-gories. They should test the hypothesis that verbs that take \textit{{}-d }[sic] in the past are all transitives and the others intransitives or that those that take \textit{{}-d} are animate actions and the others not, or something of this kind\isi{'}\isi{'} (1973:328). Of course, this does not happen, and because it does not happen with distinctions other than the SPD, Brown con\-cludes that the SPD cannot be innate.

Now this argument makes sense only if you assume, first, that
% {\textbackslash}
% 
% ACQUISITION 159
children are born with all the subcategorization features of an \textit{Aspects} grammar in their heads (a view Brown specifically attributes to McNeil! [1966] ), and second, that children, like junior linguists, acquire gram\-mars by formulating and testing hypotheses. The present theory as\-sumes neither of these things. Thus, the fact that other distinctions were treated differently from the SPD could never constitute an argu\-ment against the innateness of the SPD, unless it could be shown that those other distinctions also formed part of the bioprogram. To do that, it would be necessary to show that those distinctions were for\-mally marked in creole languages. In creoles, animate actions are not formally distinguished from other \isi{types of} action, and transitive verbs are not formally distinguished from intransitive verbs-quite the reverse, indeed, as we saw in the section on Passive Equivalents in Chapter 2, and as we shall see again later in this chapter. Thus, the child hy\-potheses Brown suggests would not make any kind of sense in light of the present theory, even if that theory supposed that children test hypotheses-which it does not.

Brown's second difficulty is that the SPD is ``a poor candidate
for innateness\isi{'}\isi{'} because it is ``very far from being universal in the world's languages.\isi{'}\isi{'} The problems foreign learners have with \isi{English} progressives and a claim by \citet{Joos1964} that \isi{English} is ``unique or almost unique\isi{'}\isi{'} in possessing the SPD are adduced as evidence for this \textbf{contention.}

Again, in the present theory, whether or not a given feature is common to all the world's languages is quite irrelevant. All previous universals theories have been static theories, which assume that lan\-guage is always and everywhere the same; if one accepts this, it follows that only features that occur in all languages can really qualify as can\-didates for innateness. But the present theory is a dynamic, evolu\-tionary theory which assumes that language had a starting point and a sequence of developments, which are recycled, in rather different ways, inhoth creole formation and child acquisition, as well as perhaps in certain \isi{types of} \isi{linguistic change} (consideration of which would take us beyond the scope of the present volume). What is innate is therefore
%\originalpage{1}
what was there at the beginning of the sequence, and thus there is not the slightest reason to suppose that innate features will automati\-cally persist and be found in the structure of all synchronic languages\-indeed, given the nature of dynamic processes, this would be an ex\-tremely unlikely result.

In other words, the SPD is presumed to be innate, not because of its universality, which may well be as low as Brown suggests, but because it plays a crucial role in creole grammars. There, statives are distinguished from nonstatives by the fact that the nonpunctual marker never attaches to the former ; but that is by no means the only signifi\-cant difference between the treatment of the two categories. The SPD causes a characteristic skewing of the creole \isi{TMA system}, not explicitly treated in the present volume, but discussed at some length in Bicker\-ton (197 5 :Chapter 2). Briefly, there is a significant difference between creole and ludo-European systems which takes the following form. In the latter, the same morphological marking applies to both statives and nonstatives in any given tense; this seems so obvious that it is never even remarked on. In creoles, however, present-reference statives and present-reference 11onstatives carmot be marked in the same way, and the same applies to past-reference statives and nonstatives. The pattern for GC, which we may take as typical in this respect, is given in \tabref{tab:3}.1 below:
%\originalpage{161}
in creole grammar, his second objection to the innateness of the SPD is also deprived of its force.

However, Brown's claim that, if a distinction were genuinely

innate, it might be generalized to inappropriate environments (cited above in discussion of his fi'rst objection), is a reasonable one if it is made with respect to distinctions found in the bioprogram (as opposed to distinctions supposedly innate by the standards of other theories). An example which looks, from the data available so far, somewhat like
an inappropriate \isi{generalization} of the SPD is found in data on the
\isi{acquisition of} \ili{Turkish} in Slobin and \citet{Aksu1980}.

\ili{Turkish} has two morphemes used for marking pasHeference verbs: \textit{{}-dI} and \textit{{}-}\textit{m}\textit{i}\textit{s.} These are used in adult speech to mark direct experience (events personally observed by the speaker) and indirect experience (events reported to or inferred by the speaker), respectively. According to Slobin and Aksu, \textit{{}-dI} is usually acquired by age 1:9, and \textit{'mls} about three months later (i.e., about the same age as \textit{{}-i}\textit{n}\textit{g} is ac· quired). But ``at first the \textit{{}-dI} and \textit{{}-}\textit{m}\textit{i}\textit{s} inflections differentiate between dynamic and static events . . . . (C)lear differentiation of the two
forms [according to their adult meanings, D.B.] is not stabilized until.about 4:6.{\textquotedbl}4

The delay \isi{in acquisition} is even more significant since most features of \ili{Turkish} verb morphology are fully acquired at age 3:0. Although ``evidential\isi{'}\isi{'} \isi{tenses} are found elsewhere (for example, in some

Present reference Past reference

Stative (/J bin

%%please move \begin{table} just above \begin{tabular
\begin{table}
\caption{1}
\label{tab:3}
\end{table}

Non\isi{stative} a

(/J

American Indian languages such as \isi{Hopi}), they are completely un\-known in ``all creoles. From these two facts, we may conclude that the ditect/inditect experience distinction does not form part of the bio\-program, and .we may further hypothesize that non-bioprogram dis· tinctions that have emerged in natural languages are particularly vulner·

Stative-non\isi{stative} distinctions \isi{in GC}

Thus, without a clear understanding of the SPD, the creole \isi{TMA system} would be quite unworkable.

Since Brown believed that universality was the criterion for innateness, while the criterion for the bioprogram theory is emergence
% {\textbackslash}
able to reinterpretation in the conrse of the acquisition process. Such reinterpretation would naturally involve the assumption that the

\begin{itemize}
\item surface markers of a non-bioprogram distinction were really marking a distinction . established in the bioprogram which is exactly what seems to be happening in the \ili{Turkish} case. Certainly, any case of unusually delayed acquisition may tum out to be evidence as conclusive
of the workings of the bioprogram as are cases of early and errorless acquisition, once the mechanisms of interaction between bioprogram and target language are adequately understood.
\end{itemize}

%\originalpage{162}


We may therefore place the SPD alongside the SNSD as a second
semantic distinction (with important syntactic consequences) which is innately programmed. But consideration of the SPD naturally prompts the question: since among the most distinctive features of creoles is their distinctive \isi{TMA system}, should it not be the case that this or a similar system emerges at some stage of acquisition, if indeed a universal genetic program generates such a system?

This question will serve to focus more sharply on the nature of bioprogram-target interaction, briefly mentioned two paragraphs above. In the present volume, emphasis is placed on the first member of the pair, for obvious reasons: until students of acquisition are convinced that a bioprogram is really operative, it is premature to talk too much about how such a bioprograrn might interact with other components of the acquisition process. But such emphasis can lead all too easily to a familiar straw man: the innate component which is supposed to roll like some irresistible juggernaut through the years of acquisition, sweeping all other influences aside. After the all too easy demolition of this travesty, the empiricist thinks he has disposed of innatism.

In fact, no innate program could or should behave in this way. From one viewpoint, the child is a biophysical organism evolving along the genetic lines laid down for its species, but from another and equally valid perspective, the child is a sociocultural organism growing up into membership of a particular human community. The pressures from the second side of being human must inevitably mold the impulses of the first-the more so since biophysical characteristics are typically more general and cultural characteristics more highly specified. Thus, from an early age-certainly from age two upward-we would expect that in a ``natural\isi{'}\isi{'} acquisition situation, as distinct from a pidgin\-creole one, the pattern of the bioprogram would be gradually shifted in the direction of the target-language pattern.

Such shifting must inevitably affect the formation of a TMA
%\originalpage{163}
system. Since virtually all the relevant literature deals with \isi{acquisition of} particular pieces of such systems, rather than with such systems as wholes, it is difficult to say at what age the child fully controls the \isi{TMA system} of his mother tongue; but it is highly doubtful whether such control is achieved prior to age four, and likely that it may come considerably later than that. This means that the \isi{acquisition of} TMA must spread over at least two years, two years during which the pres\-sure of the target grammar on the evolving bioprogram is steady and continuous. It would therefore be highly unrealistic to expect any child at any stage of acquisition to exhibit anything like a fully-developed creole \isi{TMA system}. The most that we could expect would be that \isi{acquisition of} the earlier features of TMA systems would be influ\-enced in rather oblique ways. However, if the results of such influence should prove mysterious to other theories of acquisition, yet follow logically from the present one, even such oblique evidence would be significant.

Accordingly, I shall re-examine two of the most influential papers on the \isi{acquisition of} tense: Bronckart and \citet{Sinclair1973} and Antinucci and \citet{Miller1976}; and I shall show that some puzzling features of those studies become immediately clear once we assume that while the subjects of these studies appear to be merely \isi{learning} \ili{French} and \isi{Italian}, respectively, the bioprogram decisively influences the progress of their acquisition.

One point must be made fu:st, however. With few exceptions, students of acquisition assume that when a child uses a past-tense form, he uses it because he fully understands, and deliberately intends \textit{t} mark, pastness of reference.\textsuperscript{5} True, it is often admitted (e.g., by Antinucci and Miller (1976]) that the child's concept of past may be restricted as compared with the adult's, and may extend only to past events. that leave. presently-observable consequences; but it is still assumed, without question, that where past marking appears, some sort of concept of past must be there too.

This by no means necessarily follows. All we can say is that
%\originalpage{164}
%\originalpage{165}

during the period in which \isi{past tense} is being acquired, some past· reference verbs are tense.marked while some are not. There are several possible explanations of why this is so, none of which can be ruled

GC

Past-marking rate

Punctual

38\%

Nonpunctual

12\%

out a priori. The child may have acquired a full past rule but may
apply it unpredictably because of lapses of attention, phonological difficulties, etc. The child may have acquired a partial past rule which applies only to a subset of past.reference verbs. The child may have acquired a rule that has nothing at all to do with pastness or non\-pastness, but which just happens, coincidentally, to mark a certain percentage of past-tense verbs. Which of these explanations is the correct one can only be determined by empirical investigation in each individual case.

Since the study of variable data is much further advanced in \isi{decreolization} than it is \isi{in acquisition}, it should be instructive to look at another situation where variable past-morpheme insertion takes place. In creoles, \isi{past tense} is not a category. But when creoles begin to decreolize, past-tense markers begin to be introduced, occurring sporadically just as they do in child acquisition.

At first, one might interpret such data just as similar child language data have been interpreted the speakers have an established past category but do not always mark it. However, analyses of de\-creolization in both Guyana and \isi{Hawaii} (Bickerton 1975:142-61; 1977:36-51), with a data base of a thousand past-reference verbs in both cases, suggest quite a different picture.

On the assumption that speakers had a past category, one would have to conclude that decreolizing GC speakers randomly inserted past morphemes 27 percent of the \isi{time} while decreolizing HCE speak\-ers did so 30 percent of the \isi{time}. However, when all past-reference verbs were divided into two categories-those that referred to SINGLE, PUNCTUAL EVENTS, and those that referred to \isi{iterative} or \isi{habitual} events-insertion rates were shown to vary widely between the two categories, as shown in \tabref{tab:3}.2:



HCE \textsubscript{53\% }\textsubscript{7\%}

%%please move \begin{table} just above \begin{tabular
\begin{table}
\caption{2}
\label{tab:3}
\end{table}

Past versus punct ual in \isi{decreolization}

In other words, what was being marked in both sets of data was not really pastness, bnt rather punctuality.

The punctual-nonpunctual distinction (henceforth PNPD) is related to, yet distinct from, ·the SPD, and is of equal importance in creole grammar. Since both \isi{decreolization} \isi{and acquisition} involve the introduction of ``past\isi{'}\isi{'} marking where none was before, it should at least be worthwhile examining acquisitional data to see whether punctuality plays the same role in the second as it does in the f:trst. We should certainly do well to bear this possibility in mind while we reconsider previous findings on past-tense acquisition.

We may now turn to the first of the two papers cited above, Bronckart and \citet{Sinclair1973}. The authors\isi{'} starting point was their informal observation that when children were asked to describe past events, their choice of tense often seemed to be influenced by the nature of the event : if the latter was one of some duration, like washing a car, they would tend to use \textit{il} \textit{lave} \textit{la} \textit{voiture} 'He washes/is washing the car\isi{'}, rather thanii\textit{a} \textit{lave} \textit{la} \textit{voiture }'He washed the car\isi{'}; whereas if the
event was a punctual one, like kicking a ball, they would use \textit{ii} \textit{a} \textit{pousse}

\textit{la} \textit{balle }'He kicked the ball\isi{'}, and seldom if ever substitute \textit{il pousse la}

\textit{balle} 'He kicks/is kicking the ball\isi{'}.

Normally, the opposition betweenii\textit{lave} andii\textit{a} \textit{lave }is treated as a simple past-present opposition, but of course this is not the case. The ``present\isi{'}\isi{'} tense in \ili{French} is in fact a \isi{nonpunctual aspect} which does not extend into the past (unlike that of creoles); but, like a creole \isi{nonpunctual aspect}, it embraces both \isi{iterative} and Jurative events.

%\originalpage{1}

Similarly, the ``past\isi{'}\isi{'} is not a simple past in the sense of \isi{English} simple past. \isi{English} simple past is applicable to past-punct ual and past-\isi{iterative} (but not always to past-0.urative) reference; \ili{French} \textit{avoir} + participle is limited to past punct uals, while past iteratives and past duratives are rendered by the so-called ``imperfect\isi{'}\isi{'} form, e.g., \textit{il} \textit{lavait} rather

than \textit{il} \textit{a} \textit{lave}.

In other words, when \ili{French} children use different verb forms for different kinds of past events, they are doing exactly the same as creole speakers, who always mark nonpunctual pasts differently from
punctual pasts.

Bronckart and Sinclair confirmed and quantified their original observation by asking 74 children to describe (after the event) different \isi{types of} actions which the investigators performed with the aid of a series of toys. Ages of the children ranged from under 3 to nearly 9. Eleven actions were performed. Of these, six were actions which had a clear goal or result (e.g., ``a car hits a marble which rolls very rapidly into a pocket{\textquotedbl}), while two were actions which had no perceptible goal
or result (e.g., ``a fish swims in the basin [circular movement J ``). The
three remaining ``actions,\isi{'}\isi{'} which consisted merely of cries of differing types or duration supposedly uttered by various toys, can be disre\-garded for our purposes.
of the six goal-0.irected actions, some were durative and others
were not. While the \ili{French} \textit{passe} \textit{compose} was used more frequently for these six than for the two goalless actions, there was a significant
difference (p {\textless} .01) between durative and nondurative actions, the
former having a much higher probability of being marked with a nonpast (= nonpunctual) verb. The authors concluded that ``the dis· tinction between perfective and imperfective events seems to be of more importance than the temporal relation between action and the moment of enunciation. Imperfective actions are almost never ex\-pressed by past \isi{tenses}, and for perfective actions the use of \textit{presents}
is the more frequent the greater the probability of taking into account the unaccomplished part of the action. This probability is partly determined by duration, frequence, [sic] , and maybe other objective features we have not investigated.{\textquotedbl}

%\originalpage{167}

In terms of the present study, Bronckart and Sinclair have dearly shown that the PNPD overrides the past-nonpast distinction until at least the age of six. However, the situation in \ili{French} acquisi\-tion may be even more creole-like than the authors suggest.

First, they fail to mention the possibility that even when their subjects seem to be marking +past, they are in fuct marking +punctual. \ili{French}-speaking children, like all other children, start out using the bare stem of the verb for every kind of reference, past or nonpast, punctual or nonpunctual, realis or irrealis. Then, at an age when devel\-opmental studies suggest that they have only the vaguest idea of past \isi{time}, they encounter a form (the \textit{passe} \textit{compos}\textit{e}\textit{)} which has exclusively punctual reference. Note that, semantically, the categories of past and punctual overlap. While all pasts need not be punctuals, all punct uals must be pasts-if they were not, they would still be happening, and if they were still happening now, they would be nonpunctual by defini\-tion. Which is likelier-that they would interpret \textit{passe} \textit{compose} in terms of a distinction that they barely yet grasped (past-nonpast), or that they would interpret it in terms of a distinction which would be apparent to them from their own direct observation of actions and events (punctual-nonpunctual) ?

Second, there are a number of facts about the Bronckart and Sinclair data that the authors themselves either skim over or ignore altogether. In order to understand the significance of these facts, we shall have to re-examine their study rather minutely.

Let us begin by describing the six goal-oriented actions used in the study and then coding them in terms of type of action, duration, and iteration (where applicable). The actions (Bronckart and Sinclair's 1{}-6) are.as follows:

\ea\label{ex:16}
 A truck slowly pushes car toward a garage.
\glt
\z

\ea\label{ex:17}
 A car hits a marble which very rapidly rolls into a pocket.
\glt
\z

\ea\label{ex:18}
 The farmer junips over ten fences and reaches the farm.
\glt
\z

\ea\label{ex:19}
 The farmer's wife jumps in one big jump over ten fences and reaches the farm.
\glt
\z

%\originalpage{168} 

% ACQUISITION 169

The authors present (their \figref{fig:1}) a graph which shows the percentages of \textit{passe} \textit{compose} used to describe each of the different actions by members of five age groups, average ages of each group being as follows: 1, 3:7; 2, 4:7; 3, 5:6; 4, 6:6; 5, 7:8. Like all the

e e \textsuperscript{e}

\textsuperscript{4 }J-2 \textsuperscript{P-1 }\textsubscript{e} 

5 \textsubscript{8 }\textsuperscript{P-1} 

% \chapter{8} something went wrong here! FK 15 July

acquisition charts I have ever seen , this one does not show a consistent\textsuperscript{ 6}

and steadv rise from low to high percentages of correct forms; rather it shows he familiar fever-chart zigzags before coming to rest, at age

0

%%please move \begin{table} just above \begin{tabular
\begin{table}
\caption{3}
\label{tab:3}
\end{table}

7 :8, with fairly uniform percentages of past marking across all action types. In Bronckart and Sinclair's presentation, I think quite unin\-tentionally, the amount of zigzagging is reduced by showing two different graphs for durative and nondurative events; thus, certain very interesting crossover phenomena, which are not accounted for in the authors\isi{'}.conclusions, may very easily be overlooked.

In order to dis\isi{play} these phenomena, I shall recast Bronckart

and Sinclair's data into the form given in \tabref{tab:3}.3 on the following page. In this table, the six actions, /16\{-/ 21/, are ranked for each of the first four age groups. Rank order is based on percentage of past\-tense assignment; thus, in each column the event at the head of the column is that which is most frequently assigued past marking, while the event at the foot of the column is that to which past marking is least frequently assigned. The nondurative items are circled in the
table for easier reference:


Rank orders for past-marking frequency

%%please move \begin{table} just above \begin{tabular
\begin{table}
\caption{3 presents a picture rather different from that which appears in Bronckart and Sinclair's tables and analyses. At the earliest age, actions seem to be ranked entirely on the basis of their duration, the shortest be.ing the most likely to be past-marked. The authors\isi{'} \figref{fig:1} shows that the difference between the three highest ranks in the frrst column (i.e., those actions that have a duration of two seconds or less) is less than ten percentage points, while there is a gap of over twenty percentage points between the lowest of the nondurative actions and the highest of the durative actions (Jx-5).}
\label{tab:3}
\end{table}

However, this picture gradually and progressively changes, through the next three age groups: jumping actions irrespective of duration tend to rise in rank, while pushing actions sink to the bottom of the table. The final column shows durative and nondurative actions regularly interspersed, but the four jumping actions are now all placed higher than the two pushing actions. The authors\isi{'} \figref{fig:1}shows that the stratification between jumping and pushing is quite sharp: while the
%\originalpage{170}
four jumping actions in the 6:6 column are grouped in a narrow range around the 90 percent past-insertion mark, the higher of the two push\-ing actions is separated from the lowest jumping action by a span of more than twenty percentage points.

Far from there being an overall rise in past marking of pushing actions, past-marking percentages for these show an absolute decline between ages 4:7 and 6:6, at the same \isi{time} as past-marking percentages for jumping actions are rising fairly steadily. It stretches the imagina\-tion to suppose that children between these ages begin to perceive pushing actions as LESS past and jumping actions as MORE past; yet if we really believe that \isi{past tense} is all that the children are ac\-quiring, we have no alternative but to believe in improbabilities such as this.

Bronckart and Sinclair note some of the fluctuations mentioned here, but attempt to account for only one of them , and that, perhaps, the least significant: the drop in rank for J-2 (their ``event 4{\textquotedbl}) be\-tween ages 3:7 and 4:7. The explanation they offer is that ``this action took objectively more \isi{time} (2 sec.) than the others (1 sec.).\isi{'}\isi{'} By ``the others\isi{'}\isi{'} the authors mean, presumably, J-1 and P-1, but they are a little disingenuous here because they fail to note that the past-marking rate for J-2 ALSO FALLS BELOW THAT FOR Jx-Swhich takes more than twice as much \isi{time} as J-2! Moreover, the rate for J-2 is only a point or two higher than that for Jx-10, which is five times longer! Relative length of \isi{time}, therefore, cannot be the factor involved.

Let us see if we can really determine what underlies the phenomena illustrated in \tabref{tab:3}.3. From the first column, it would appear that children in the lowest age group do indeed discriminate between events on the basis of pure length. However, as they grow older, the criterion of durativity is replaced by another which is also related to the PNPD.

For the punctual-nonpunctual opposition must also be marked in the semantic features of individual verbs. That is to say, some verbs are inherently punctual, while others are inherently nonpunctual. lf you hit something for five minui;es, it must be that you hit it many
%\originalpage{171}
times; similarly, if you jump for five minutes, you must jump many times; both \textit{hit} and \textit{jump} express inherently punctual actions. But on the other hand, if you push something for five minutes you do not ncessaril push it more than once, and if something oils for five mmutes, It des not necessarily roll more than once; both \textit{push} and \textit{roll} express mherently nonpunctual actions (although of course a compound verb like \textit{roll} \textit{over} is inherently punctual).

. .fo other words, although the PNPD is crucial throughout the acqu1smon of past-tense marking, the way in which punctuality and nonpunctuality are interpreted changes as children mature. At f1rst, tey mere! register the relative length of events, and do not ·distinguish either the. mhe\isi{'}.{\textquotedbl}6nt characteristics of different actions or any difference between \isi{iterative} and durative events. Note how, in column 1 \textsubscript{0}f

%%please move \begin{table} just above \begin{tabular
\begin{table}
\caption{3, the wo Jx events, which are sequences of punctual events,}
\label{tab:3}
\end{table}

e grouped with P-10, the only truly durative event. Thus, their JUdgment of what is nonpunctual at age 3:7 accords with the common\-st ceole ju gment of what is nonpunctual: that is, a merger of the 1terat1ve (\isi{habitual}) with the durative (progressive).

However, as \isi{time} goes by, the two Jx events are reinterpreted s squence of events, each one of which, considered individually , 1s bnef and mherently punctual. Thus, iteratives are removed from the nopunctual category (which now contains only duratives) and re\-asigned to the punctual category. This judgment- merging of iteratives with punctuals-corresponds to the minority creole pattern found in

Jamaican.Creol: and perhaps a few others referred to under the heading of Dev1at10n E m Chapter 2 (p. 78J. It is certainly intriguing to speculate that at least some of the relatively few real differences in creoles c?uld result from their having been ``finalized,\isi{'}\isi{'} so to speak, at slightly
different age levels by the inventing generation-but of course we can only speculate at this stage.

What is of more immediate interest is the insight that we can denve, from the process described above, into the wav in which a bio\-program would evolve. Some scenarios for Chomsky innatism seem to suggest that every neonate already has a full \textit{Aspects} grammar curled
%\originalpage{1}
up in Broca's region. This literalistic reading of ``innate\isi{'}\isi{'} has no place in a bioprogram theory. A true bioprogram would grow, develop, and change just as the physical organism that houses it grows, develops, and changes. Increases in the child's cognitive abilities (which o course also form part of the bioprogram in its widest sense) would mteract with the linguistic component and progressively modify it.

For the \ili{French}-speaking child, the shift in the nature of the
nonpunctual category would have the effect of moving ma.re events into the punctual category, thus making more events available for past marking. It would, in other words, help the child in his trai:sfer from predominantly punctual marking to predominantly past markmg\-although whether this result issues from the hand of a beneficent providence, or is merely an accidental bonus, it is far too early to tell.

The suggestion that children may have been marking punctality when they seemed to have been marking pastness may still see.m b12arre to some readers. Let us, therefore, see how well it stands up m hght of another well-known study of tense acquisition-that of Antinucci and
\citet{Miller1976}.

Antinucci and Miller found that the earliest tense form used by
their sample of \isi{Italian}-speaking children was the past participle. At first this always agreed in number and gender with the \isi{sentential} object, suggesting that the children regarded the participles as \isi{adjectives} rather than verbs. Then, around age two, they dropped the agreement rule and began to use the participles in ways suggesting that they now perceived them as true past-tense verbs (the usual \isi{past tense} in \isi{Italian} consists of auxiliary plus participle, \textit{ho venuto }'I came\isi{'}, but the children almost always omitted the auxiliary).

However, the verbs which children used in this way appeared
to be somewhat restricted in number. The authors divided verbs into three classes: activity verbs (where the action has no end result) , \isi{stative} verbs, and \isi{change-of-state verbs} (such as \textit{close,} \textit{fall,} \textit{give,} et\isi{'}:\isi{'}.) which describe actions as a result of which ``an object changes 1ts state.\isi{'}\isi{'} They found that with vry few exceptions, children's past
%\originalpage{173}
forms were found with verbs of the last class only, and they concluded that the child could only assign \isi{past tense} to an action when some\-thing presently in his physical environment-a toy that had been broken, some milk that had been spilled-remained behind as a concrete result of that action.

Now, it happens to be the case that \isi{change-of-state verbs} are all inherently punctual; the rare, apparent exceptions are often due to purely technological developments, as in \textit{the} \textit{abandoned} \textit{astronaut} \textit{fell} \textit{toward} \textit{the} \textit{planet} \textit{for} \textit{several} \textit{hours.} But even sentences like that can be
seen to be underlyingly punctual if we apply another test for inherent punctuality: the question, ``If you stop halfway through \textit{Vi}\textit{n}\textit{g,} have you \textit{Ved?{\textquotedbl}} Thus, if you stop halfway through closing, you have not closed, and if you stop halfway through giving, you have not given; similarly, if the abandoned astronaut stopped halfway through falling, he would not have fallen, although he migh t have lost altitude. But with activity verbs, which are inherently nonpunct ual, the converse applies: if you stop halfway through playing, you have played, if you stop half\-
way throirgh writing, you have written, and so on. If, as the Bronckart
and Sinclair study suggests; the more an action is regarded as punctual,
the more likely it is to be given past-tense marking, then the verbs in
Antinucci and Miller's change-of-state list may be given past marking bei:ause .they are. punctual, rather than for the reason the authors suggest:
 

It is true. that in the Bronckart and Sinclair study the first cri\-terion for the PNPD was raw duration, and that inherent characteristics
of .verbs didn ot become dominant until an age long past that of the

Antinucci and Miller subjects. However, the Bronckart and Sinclair data are ·drawn from experiments, whereas the Antinucci and Miller data ate drawn from naturalistic observation; moreover, Bronckart and inclair do not include any \isi{change-of-state verbs} in their study. The .two studies are therefore not comparable at the fine-grained level 6f,.{\textquotedbl}Just how do children interpret punct uality at age X?{\textquotedbl}; they do, ho{\textbackslash}Ver, seem to be in agreement that some kind of PNPD is involved.

There are other clues in Antinucci and Miller's study which 
%\originalpage{174}
suggest that a punct ual analysis may account for the facts better than a change-of-state one. Early in their third year, \isi{Italian} children gener\-ally acquire a second \isi{Italian} \isi{past tense}, the imperfect. This is used with activity verbs, but is not extended to \isi{change-of-state verbs}, which continue to be past-marked with participial forms (with or without auxiliary):

\ea\label{ex:22}
 (Antinucci and Miller's 82)
\glt
\z

Mamma \textit{e} \textit{andato} (participial) al parco e io \textit{stavo} (imperfect)

\textbf{a} \textbf{casa}

'Mommy \textit{went} to the park and I \textit{stayed} home\isi{'}

\ea\label{ex:23}
 (Antinucci and Miller's 90)
\glt
\z

Li \textit{ha} \textit{messi }(participial) nel saco e dopo gli altri bambini \textit{piange·} \textit{vano }(imperfect)

'He \textit{put} them in a sack and then the other children \textit{cried'}

In other words, imperfects and participials are in complementary distribution, the first being used for punctual verbs, the second for nonpunctual ones. Note that this does not reflect anything in \isi{Italian} grammar; all \isi{Italian} verbs, whether punctual or nonpunctual, activity or \isi{change-of-state verbs}, have both perfective and imperfective past \textbf{\isi{tenses}. ,}

Antinucci and Miller's explanation for this state of affairs is far from satisfactory. They note that the imperfect appears first during story-telling, and suggest that the child uses it to distinguish ``pretend\isi{'}\isi{'} from real events. But if this were really the case, we would expect use of the imperfect to be extended to all \isi{types of} verbs, including change\-of-state verbs; for surely change-of-state (or punct ual) verbs can be used to describe imaginary events as easily as activity (or nonpunctual) verbs. Moreover, in the examples that Antinucci and Miller them\-selves cite, such as /22/ and /23/ above, the events of \textit{staying }and \textit{crying,} rendered by the imperfect, are no more (or less) ``pretend\isi{'}\isi{'} events than the events of \textit{going} and \textit{putting, }which are rendered by the participial form. ,

%\originalpage{175}

It is therefore highly possible that the connection observed by the athors bet{\textquotedbl}\isi{'}.'ee ``pretend\isi{'}\isi{'} evnts and the imperfect is merely a part1 .and comc1den tal one. It 1s hard to tell, since they present no stat1st1cal data that would serve to quantify the distribution of participial and imperfective forms in realis and irrealis con\isi{texts}. And if their hypothesis is falsified even by the few sentences they themselves choose \textit{to} cite, and if both \isi{tenses} are past-reference in adult speech, then there is nothing but the PNPD to prevent children from general\-izing the more regular, imperfect form to \isi{change-of-state verbs} just as \isi{English} children generalize \textit{{}-ed} to irregular pasts. \isi{'}

Indeed, how is it that \isi{English} children generalize in this way when \isi{Italian} children do not? If everyone has the same bioprogram, how com.e everyone doesn't learn the same way? Let us explore this problem 1Il some depth, for by so doing we will not only answer these and other \isi{questions}, but we will also better understand how the same
bioprogram can yield superficially different results when it interacts with two languages that differ in structure.

First, let us dispose of a possible objection. It might be argued tat the two processes- Irian-speaking children \isi{learning} first parti\-ciples, then llnperfects; \isi{English}-speaking children \isi{learning} first irregular, then regular pasts{}-are not really commensurate. So they are not, from an adult point of view. But the child does not have an adult point of view, and for the child they must be completely commen\-surate. The adult knows that \isi{Italian} has two tense forms with different meanings, whereas \isi{English} has only a single form, expressed in diverse ways. But there is no way a child could know this unless he were born with a comparative grammar of Inda-European in his head, as well as \textit{Aspects. }Remember, the children we are talking about are under three. Not only can they not have the slightest idea what the mat ure tense system of their languages will eventually look like, but even on the most favorable accounts, they can have only the vaguest notion of
what past means, and by some accounts, they can have no notion at all.

%\originalpage{176}

What must really happen is something like the following. Around age two, the child who happens to be \isi{learning} \isi{Italian} becomes aware of a set of rather irregular forms, which are past-reference forms in adult grammar (the \isi{Italian} participles), whereas the child who happens to be \isi{learning} \isi{English} also becomes aware of a set of rather irregular forms, which are also past-reference forms in adult speech (the \isi{English} ``strong\isi{'}\isi{'} past \isi{tenses}). Shortly afterward, the child \isi{learning} \isi{Italian} encounters a set of quite regular forms, once again past-reference forms in adult speech (the \isi{Italian} imperfective), while, around the same \isi{time}, the child \isi{learning} \isi{English} also encounters a set of regular forms that are past-reference forms in adult speech (the \isi{English} ``weak\isi{'}\isi{'} past \isi{tenses}).

Up until this point, the experiences of the two children have been , from their point of view, identical. I defy anyone to explain how those experiences could be differently interpreted by the two children-except in a single respect, which we shall deal with shortly. From the child's point of view, in both cases he has begun by finding some irregular fors that mean past (from the traditional perspective) or punctual (from the perspective of this volume), and he has gone on to find some regular forms that also mean past (from the traditional perspective).

But now, the \isi{Italian} learner and the \isi{English} learner part com\-pany. The \isi{Italian} learner keeps the two sets of forms, the regular and the irregular, completely separate, applying one set to one class of verbs and the other to another. The \isi{English} learner, on the contrary, proceeds to generalize the regular set to the irregular set, applying ``weak\isi{'}\isi{'} tense endings to ``strong\isi{'}\isi{'} verbs in defiance of adult grammar rules. Why ? Why doesn't the \isi{Italian} learner make a similar generaliza\-tion? Or, to put it differently, why doesn't the \isi{English} learner make the same kind of distinction as the \isi{Italian} learner, maintaining the irregular forms (like \textit{came,} \textit{went,} \textit{bought,} \textit{sold,} \textit{gave,} \textit{broke-all} good changes-ofstate, note) for punctual verbs, and saving the \textit{{}-ed} affix for activity verbs?

To understand the answer, we have to get used to looking at the
% {\textbackslash}
%\originalpage{177}
acquisition process in a way it has not been looked at hitherto-even though e.verything w.e know about language points to that way as the mst logical and fr1tful. Als, the ``\isi{order of acquisition}\isi{'}\isi{'} gambit set child language studies back fifteen years by concentrating exclusively on the \isi{acquisition of} isolated features. Small wonder if, as we have seen, the sterility of this approach sent acquisitionists gamboling off across the meadows of pragmatics, cognition, ``motherese,\isi{'}\isi{'} etc., which were not much more irrelevant to the central problem of syntax acquisition, but a good ?eal less dull. For what both groups forgot was that lauage IS a tight system composed of even tighter sub\-systems. Children do not learn individual morphemes in isolation from
one another; they build up subsystems and at the same \isi{time} integrate those subsystems into an overall system.

The situation was not helped any by the primitive state of the art in TMA studies, in spite of (I would prefer to say, because of) wo\isi{'}.k in the field from \citet{Reichenbach1947} to \citet{Comrie1976} and Wo1setschlaeger (1977). We shall return to this issue in Chapter 4; for. the .moment, suffice it to say that an approach like Comrie's,
which tries to extract some kind of Platonic core meaning from terms like ``perfective\isi{'}\isi{'} and ``imperfective,\isi{'}\isi{'} totally ignores the fact that the units of grammatical subsystems cannot be defined independently of those systems-that, in consequence, what ``perfective\isi{'}\isi{'} and ``im\-perfec.tive\isi{'}\isi{'} mean, .in any subsystem where such labels are applicable,
IS entirely determmed by how many other units that subsystem has and what the other units mean.

Once this viewpoint is established, we can proceed to look at the .u:quisition of TMA systems AS SYSTEMS, bearing in mind all the while the injunction of the bioprogram; ``Make sure that punctuals and nonpuncts are adequately differentiated.\isi{'}\isi{'} We may then repre\-sent the acqu1Sltlon process for \isi{English} and \isi{Italian} learners as in Figure
3.1 on the following page:
%\originalpage{1}

Time line (very approx.)

\textsubscript{1}\textsubscript{:6 }\textsubscript{{}-}{}-{}-{}-{}-{}-{}-{}-{}-{}-{}-{}-{}-{}-{}-{}-2:6

{}-ing (NP)

% ACQUISITION 179

that the scope of this ``present tense\isi{'}\isi{'} differs in \isi{English} and \isi{Italian}; in \isi{English} it includes \isi{habitual} and \isi{iterative} reference only, whereas in \isi{Italian} it includes progressive and durative reference also. The child,
needless to say, cannot foresee these facts, but they exert a profound influence on the acquisition process, as the following paragraphs will show.

The question of what determines the order in which new forms

\begin{figure}
\isi{English} learner

base{\textless}

base

{\textless}irr. past (P)

base

reg. past

base
\end{figure}

are acquired is too complex to be explored fully here.\textsuperscript{6} However, it seems likely that the difference between \isi{Italian} and \isi{English} present \isi{tenses} determines the first addition to the system. \isi{English} has a dis\-tinct (and frequent) form for present progressives; \isi{Italian} has not. Therefore, the first new term that \isi{English} learners add is a nonpunctual one. But since there is no similar form in \isi{Italian}, the first new form that \isi{Italian} learners add is a past form-the participle-which they interpret as a punctual one.

The second new form acquired by \isi{English} learners is the ir\-%
\begin{figure}
\isi{Italian} learner

base {\textless}P{\textquotedbl}'part. (P)

base

Figure 3.1 .

Comparative TMA acquisition (\isi{Italian} versus Enghsh)

Both \isi{English} and \isi{Italian} learners begin with a single undiffe
\end{figure}%
%
regular past, which they interpret as marking punctuality. They are therefore now able to mark both sides of the PNPD. But shortly after\-ward they become aware of a third form-regular past \textit{{}-ed.} Since they already have markers for punctual and nonpunctual, they cannot ac\-commodate this .new form by assigning to it its own semantic scope; they therefore assume they were wrong in choosing irregular past as a punctual marker, and proceed to extend \textit{{}-ed} to those past punc\-tuals which had previously been allotted irregular forms.

However, the second new form acquired by \isi{Italian} learners
is the imperfect. This, like the past participle, is used for past reference by adults, and if \isi{Italian} learners were really using participles to mark past reference, they would surely generalize the imperfect form to
entiated base form which at first has to cover all intende.d forms \textsubscript{TMA}\textsubscript{ }\textsubscript{r}\textsubscript{e}\textsubscript{£}\textsubscript{erence}\textsubscript{ }\textsubscript{·}·(\textsubscript{m}· f\textsubscript{a}\textsubscript{c}t\textsubscript{'}\textsubscript{ }the \isi{Italian} base form is reall\textsubscript{.}y a s\textsubscript{.}eries of\textsubscript{.}\textsubscript{ }for differentiated for person, but this and similar details will be ignore
here for the sake of clarity of presentation). As ne:V- .forms are. add
verbs of all types, just as \isi{English} learners generalize {}-ed-Jor, as noted
above (p. 176), the \isi{Italian} participial/imperfect opposition and the
\isi{English} irregular/regular past opposition must look formally identi\-
cal to the child learner. The reason why they do not do this can stem only from the unique difference between the situations of the two the

.

semantic scope of this base form contra.cts until it evolves mto t
sets of learners: \isi{English} learners have already marked both sides of
adult, so-called ``present tense\isi{'}\isi{'} in both languages. Note, oweve 
%\originalpage{180}
the PNPD, while \isi{Italian} learners have marked only one side. For them, nonpunctuals are yet to be marked, so instead of generalizing the imperfect, they seize on it as their nonpunctual marker and keep it carefully separate from their marker of punctuality , the participial form.

Note that without the bioprogram the differences in behavior between \isi{Italian} and \isi{English} learners are quite inexplicable. In virtually identical circumstances, the \isi{English} learner over-generalizes, while the \isi{Italian} learner under-generalizes. However, once we see that \isi{English} and \isi{Italian} learners are equipped with an identical program, but still satisfying the requirements of that program in a different order-an order determined by the interaction of the bioprogram with two different languages-such differences are not merely explicable, but follow inevitably from the theory presented here.

Now we can better understand what the child of a first creole generation does. When that child is around 18 to 21 months old, his TMA ``system\isi{'}\isi{'} and the TMA ``system\isi{'}\isi{'} of his parents\isi{'} pidgin exactly coincide; both consist of the ``universal base\isi{'}\isi{'} shown at the left-hand side of \figref{fig:3}.1. The only difference between the child's trying to learn a pidgin and the child's trying to learn \ili{French} or \isi{Italian} is that the latter will be offered a variety of verb forms which he can then interpret according to the specifications of his bioprogram, while the former will not be offered anything new in the way of forms. The creole child therefore decides to mark the nonpunctual side of the opposition.

Two \isi{questions} may be asked here: why does the creole child decide, apparently without exception, to mark nonpunctuals rather than punctuals, and why does he not mark both terms of the opposi\-tion, as I have claimed that both \isi{English} and \isi{Italian} children do?

I think that nonpunctuals rather than punctuals are marked because, from a pragmatic viewpoint, nonpunctuals represent the marked case in a Jakobsonian sense: in the real world, more actions are punctual than nonpunctual; punctual actions constitute the back%
%\originalpage{81}
ground against which nonpunctual actions stand out. Regarding the second question, we should rather ask, why do noncreole children mark both terms? The answer to that clearly is because noncreole children receive, if anything, too great a variety of forms-greater, certainly, than any two-year-old can incorporate into a coherent system. The child feels obliged to assign some kind of significance to terms with which he is constantly bombarded, so he assumes that in the language confronting him both sides of the PNPD are formally marked.

But, of course, both sides of an opposition do not have to be formally marked-it serves to distinguish them if you formally mark one term and zero-mark the other. Considerations of parsimony alone would indicate such a choice, if the opportunity presents itself (and for the creole child, it does). It is quite enough trouble for the creole child
to select, from the pidgin, one content-word (like HPE locative \textit{stei)}
to mark one term of the opposition, without having to search out another to mark the other. In both cases, albeit by differen t means, the demands of the bioprogram are satisfied.

I have little doubt that when acquisitionists begin to study TMA. acquisition from a dynamic, systems-oriented standpoint, and with.the tools of variation analysis already available from \isi{decreolization} studies, many more of the effects of the bioprogram will become visible. For the present, we must leave that area and survey rather more briefly some others in which resemblances between creoles and acqui\-sitional stages are to be found. We shall look at just four areas: comple\-mnt Ss, \isi{questions}, negatives, and \isi{causatives}.

A recent overview of complex sentence acquisition \citep{Bowerman1979} draws heavily on \citet{Brown1973} and \citet{Limber1973}, which

appear to. be the major, if not the only, sources in this area. If true,
surprising, since Brown devotes less than a page (p. 21) to sen\-tential complemen ts, and Umber's only slightly longer (six. {}-page) treatment leaves many crucial \isi{questions} unasked. However, much of
is said by these scholars is highly suggestive.

%\originalpage{182}

Brown cites four examples only of complement Ss produced by children:

\ea\label{ex:24}
 I hope \textit{I} \textit{don't} \textit{hurt} \textit{it.}
\glt
\z

\ea\label{ex:25}
 I think \textit{it's} \textit{the} \textit{wrong} \textit{way.}
\glt
\z

\ea\label{ex:26}
 I mean \textit{that's} \textit{a} \textit{D.}
\glt
\z

\ea\label{ex:27}
 You think \textit{I} \textit{can} \textit{do} \textit{it?}
\glt
\z

Brown comments that ``the embedded sentence appears exactly as it would if it stood alone as an independent simple sentence.{\textquotedbl}He observes that there are other \isi{types of} complement S of which this is not true, such as:

\ea\label{ex:28}
 It annoys the neighbors \textit{for} \textit{John} \textit{to} \textit{\isi{play} the} \textit{bugle.}
\glt
\z

He does not state whether or not the children in his sample produced sentences like /28/, but the implication is that they did not. He does observe, however, that ``there is also a complementizer \textit{that{\textquotedbl}} which can occur in sentences like /24/-/27 /; but ``the children did not use \textbf{it.n}

Limber's data are more problematic in that it is not always clear from his treatment whether a given example is an actual child utterance or one presented for heuristic purposes. Thus, although Limber states that ``marked \isi{infinitive}\isi{'}\isi{'} is acquired early, this is not clearly the case from the example given: \textit{I} \textit{want} \textit{to} \textit{go.} If, as seems probable, this is just an orthographic regularization of the actual utterance, \textit{I} \textit{wanna} \textit{go} (a likelihood increased by the fact that Limber himself includes a similar form, \textit{h}\textit{a}\textit{fta,} in his \tabref{tab:1}), then it is not at all clear \textit{from} \textit{the} \textit{viewpoint} \textit{of} \textit{what} \textit{the} \textit{child }\textit{(as} \textit{opposed} \textit{to} \textit{the} \textit{adult)} \textit{knows} that the child has acquired marked infmitives. Consider the following sentences, which few children can have failed to hear or failed to produce themselves:

\ea\label{ex:29}
 I wanna cookie (unambiguous noun).
\glt
\z

{\textbackslash}

%\originalpage{83}

\begin{itemize}
\item \begin{itemize}
\item \textit{130/} I wanna drink (ambiguous between noun and verb).
\end{itemize}
\end{itemize}

\ea\label{ex:31}
 l wanna go (unambiguous verb).
\glt
\z

Faced with such data, the most reasonable conclusion on the part of the child would be that the canonical form of the verb was \textit{wanna} rather than \textit{want-or} that, at the very least, \textit{hafta,} \textit{liketa,} \textit{wanna} should be. entered in the lexicon as variant (perhaps phonologically condi\-tioned) forms of the verb stems concerned. Such, certainly, is the assumption made by \citet[54]{Brown1973} when establishing rules for the alculation of mean length of utterance: \textit{{\textquotedbl}gonna,} \textit{wanna,} \textit{hafta} \textit{.} . . [were]counted as single morphemes rather than as \textit{going} \textit{to} or \textit{want} \textit{to} because evidence is that they function so for the children.{\textquotedbl}

The following series of examples represents, with one exception, all the \isi{sentential} complement forms cited by Limber which we can assume to be examples of actual child speech:

\ea\label{ex:32}
 lwant \textit{mommy} \textit{do} \textit{it.}
\glt
\z

\ea\label{ex:33}
 I don't want \textit{you} \textit{read} \textit{that} \textit{book.}
\glt
\z


\ea\label{ex:34}
 Watch \textit{me} \textit{draw} \textit{circles.}
\glt
\z

\ea\label{ex:35}
 I see \textit{you} \textit{sit} \textit{down.}
\glt
\z

\ea\label{ex:36}
 Lookit \textit{a} \textit{boy} \textit{play} \textit{ball.}
\glt
\z

If we look at these five examples together with the four cited by Brown (and these, strange to say, seem to be virtually the only coniplement-S constructions cited in the literature), we will note first that not one of them has an overt complementizer, and second,
that with the exception of /34/ the \isi{complements} could stand on their own as independent simple sentences. Moreover, since \textit{me} as subject
has been widely reported for black children, it is by no means certain that for the speaker of /34/, \textit{me} \textit{draw} \textit{circles} would be ungrammatical; and even if it were, the \isi{analogy} with \textit{watch} \textit{me,} \textit{mommy!-an} utterance surely developmentally prior to /34/-may be what is operative ln this case.

It is true that we cannot point to the same kind of evidence
%\originalpage{184}
we used in Chapter 2, when the same question of finite versus non\-finite analysis was at issue; we cannot point to t,he presence of markers of tense or aspect in the embedded sentence. But it would be illegiti\-mate to expect such evidence, since at the ages from which Limber's
%\originalpage{185}
established and that there is no evidence, as there seems not to be, for any VP constituent):
examples are taken (1 :6 to 3:0), the vast majority of children's verb forms consist of unmarked stems anyway.

The only example of Limber's which was 11ot cited above is:

\ea\label{ex:39}
 S {}-+ NP V
\glt
\z

\textsubscript{( }\textsubscript{\{}NSP \})

\ea\label{ex:37}
\textit{I} I all done eating.
\glt
\z

This might at first seem like a clear case of a nonf{\textquotedbl}mite complement S. But Limber himself explicitly observes that in his recordings there is no trace of ``a variety of \textit{{}-ing} \isi{complements}; for example, \textit{I} \textit{like} \textit{eating} \textit{lollipops} in contrast to the very common \textit{I} \textit{like} \textit{to} \textit{eat} \textit{lollipops{\textquotedbl}} (which, as already suggested, is more probably a case of a quasi-modal \textit{liketa).} He further comments that nonf{\textquotedbl}mite \textit{i}\textit{n}\textit{g} forms (as distinct from the ``finite \textit{{}-ing{\textquotedbl}} discussed in a previous section) occurred only in sentences like /37 /, i.e., ``with \textit{finish} or \textit{all} \textit{d}\textit{one.{\textquotedbl}} Although Limber himself does not explicitly draw it, it would seem legitimate to draw the conclusion that \textit{finish} and \textit{all} \textit{done} are interpreted by the child either as quasi\-modals followed by ``fmite \textit{{}-i}\textit{n}\textit{g{\textquotedbl}} or as main verbs followed by NP. Either way, /37/ would not be relevant to the present discussion.

Limber goes on to ``informally summarize\isi{'}\isi{'} the major develop\-ments in complex sentences prior to age three in the following manner: ``An N-V-N sequence is the common simple sentence .. . . [children] expand (or substitute) an N-V-N sequence for certain noun phrases .. .
[butJ do not apply syntactic operations to any subject NPs.\isi{'}\isi{'} Brown (1973:21) also observed that sentences of the type of /38/ below did not occur in child speech:

\ea\label{ex:38}
 \textit{That} \textit{John} \textit{called} \textit{early} annoyed Bill.
\glt
\z

Stated more formally , Limber's study would suggest that children have only the following major PS rule (assuming that Aux is not yet

% {\textbackslash}

Similarities between the foregoing account of complement Ss in child speech by Limber and Brown and the account of complement Ss in creoles given in Chapter 2 are quite striking. They include:


\begin{enumerate}
\item The absence of embedded sentences in subject position.
\item The absence of complementizers.
\item  The identity of form between embedded and nonembedded sentences.
\item The absence of nonfinite and subjectless embeddings.
\end{enumerate}


In addition, we may note the similarity of /39/ to the major creole PS rule /236/, Chapter 2, hypothesized on quite independent grounds for all early-stage creoles, and repeated here for convenience as /40/:

\ea\label{ex:40}
 S {}-+ NP Aux V (NP) (S)
\glt
\z

Rule /40/ is merely a slightly more sophisticated version of /39/, as would befit its more mature users, differing only in that it admits an established Aux: and allows for object NP as well as complement S in the same sentence, instead of only admitting these as alternatives.

Of the similarities listed, the first three are self-explanatory, but perhaps a word should be said about the fourth, which relates to the absence from child speech of sentences like \textit{I} \textit{like} \textit{eating} \textit{lolli\-} \textit{pops} and from creoles of sentences like /41/ or /42/:

\ea\label{ex:41}
 GC.: *mi hia a sing
\glt
\z

'I heard singing\isi{'}

%\originalpage{1}

\ea\label{ex:42}
 GC: *mi laik a sing
\glt
\z

'I like singing\isi{'}

(Sentences whose \isi{complements} have overt subjects, such as \textit{mi} \textit{hia} i\textit{a} \textit{si}\textit{n}\textit{g} 'I heard him singing\isi{'}, are of course in another class entirely.) In fact, more is involved here than there is space to discuss/42/ involves equi-deletion while /41/ involves deletion of an unspecified subject, so the reasons for their ungrammaticality cannot be the samebut I would like to suggest a reason why both creoles and children should reject sentences on the model of J \textit{like} \textit{doing} X.

If children learned language primarily on the basis of \isi{analogy}, the absence of such sentences would be mysterious. The child would observe some specific \isi{questions} and answers:

\ea\label{ex:43}
What are you eating? Cookies.
\glt
\z

\ea\label{ex:44}
 What are you playing with? My ball.
\glt
\z

Answers to such \isi{questions} fit well into the frame, \textit{I} \textit{like} \textit{.} . \textit{.}\textit{:}

\ea\label{ex:45}
 I like cookies.
\glt
\z

\ea\label{ex:46}
I like my ball.
\glt
\z

Once the child had acquired ``finite \textit{{}-ing},\textit{{\textquotedbl}} he would be able to answer slightly less explicit \isi{questions} with \textit{{}-ing} forms:

\ea\label{ex:47}
 What are you doing? Eating cookies.
\glt
\z

\ea\label{ex:48}
 What are you doing? Playing ball.
\glt
\z

By \isi{analogy} with /45/, /46/, these ought to yield:

\ea\label{ex:49}
 I like eating cookies.
\glt
\z

\ea\label{ex:50}
 I like playing ball.
\glt
\z

Of course, children do not;,, learn primarily by \isi{analogy}; analogical
%\originalpage{187}
forms may crop up from \isi{time} to \isi{time}, but not when (as here) they would conflict with important structural aspects of the grammar. For \textit{eating} \textit{cookies} in /47/ is not the same as \textit{eating} \textit{cookies} in /49/. In /47 /, it expresses a particular nonpunctual action in realis \isi{time}; in /49/, it expresses the abstract concept of an action, not necessarily either punctual or nonpunctual, in irrealis \isi{time}. The superficial identity of the forms involved is an illusory one, and the child, for all the little he is supposed to know of language, is not fooled by it. For hoth child and creole, nonpunctual means nonpunctual, nothing more, and be\-cause of the form-meaning biuniq ueness that characterizes both child speech and creoles, the form chosen to mark nonpunctual cannot be assigned any other function.

We mayjustifiably conclude, then, that the mechanisms of child
language and creoles for incorporating sentences within sentences are highly similar, with one exception: children show no evidence of \isi{verb serialization} \{at le'ast in existing accounts; I would not rule out the possibility that it might turn up if people started looking for it). But then,. the reasons why child language doesn't have \isi{verb serialization} are probably the reasons why some creoles don't have it: because prepositions are. available it1 the input, and therefore serialization is not needed to differentiate case roles.

Let us now turn to \isi{questions}. Children acquire \isi{questions} early --
certainly by the two-word stage, and probably earlier even than that. acqtdsition of question forms was first studied intensively by Klima and Bellugi .(1966), and although some subsequent observers have found more ·variation in question development than these authors recognized, their principal findings have not been seriously challenged.

Among \isi{English} learners, \isi{yes-no} \isi{questions} are at frrst distin\-gi;1shed from statements only by a rising intonation contour, and WH\-\isi{questions} onlyhy a . sentence-initial WR-word; in neither type is there 9£ Subject-Aux inversion. This state of affairs changes only slo.wl1r. Sent.en.ces grow longer and more complex, all the question

W'ordls but \isi{yes-no} \isi{questions} retain the form of /51/ and

%\originalpage{88}

\ea\label{ex:51}
 This can't write a flower?
\glt
\z

\ea\label{ex:52}
 You can't fix it?
\glt
\z

At a later stage, when inversion begins to appear in \isi{yes-no} \isi{questions}, · it is still absent from WH-\isi{questions}:

\ea\label{ex:53}
 Why he don't know how to pretend?
\glt
\z

\ea\label{ex:54}
 Where the other Joe will drive?
\glt
\z

This last stage is all the more puzzling because children who remain in it are often at the same \isi{time} producing sentences which indicate·; mastery of rules seemingly more complex than those required for e correctly forming \isi{English} WH-sentences, such as the following:

\ea\label{ex:55}
 You have two things that turn around (\isi{relativization} ).
\glt
\z

\ea\label{ex:56}
 I told you I know how to put the train together (double comple\-ment embedding plus embedded nonfinite WR-clause).
\glt
\z

\ea\label{ex:57}
 Let's go upstairs and take it from him because it's mine (co- · ordination and subordinate-clause \isi{causative} construction ).
\glt
\z

Why do children at this level of development persist in using structures\isi{'} so different from the many well-formed \isi{questions} which they must . have heard?

Clark and \citet[354]{Clark1977} suggest that ``WH-\isi{questions} may be more difficult because they require \textit{two} \textit{rearrangements:} \textit{movement.} \textit{of} \textit{the} \textit{WH-word} \textit{from} \textit{where} \textit{it} \textit{would} \textit{have} \textit{been} to initial position ·

fhe f,'rocess, such as \textit{you} \textit{doing} \textit{what?} or \textit{what} \textit{you} \textit{doi}\textit{n}\textit{g} \textit{what?} in place Of the.{\textgreater}expected \textit{what} \textit{you doing?;}\textit{\textsuperscript{7 }}with a candor as rare as it is comineridable, they observe that such errors have not as yet been repotted anc! that it will constitute counterevidence to their claims if those errors do not in fact occur. If indeed there is no evidence
t() sl!pport the two-rearrangement argument, the only reason for s1.lppoing•that .WE-\isi{questions} are psychologically complex is that they 
•§'taJie. lqnger to acquire. In other words, the whole explanation becomes

{}-{}-\textsuperscript{.},,:,\_\_:: \textsuperscript{{\textquotedbl}}C\textsuperscript{{\textquotedbl}}lt\isi{'}\textsuperscript{·}CU\textsuperscript{1}Jl\textsuperscript{·}\textsuperscript{·} i

{\textbackslash} \textit{,'} \textit{\textsuperscript{1}}\textit{1{\textless}} An alternative explanation is suggested by Ruth Clark, who
claims that uninverted WR-\isi{questions} are modeled on the embedded
clauses produced by mothers and other caregivers \citep{Clark1977}.
For example; the child who asks \textit{where} \textit{Teddy?} may often be answered

Jiy J \textit{c(O.}\textit{h.'t} \textit{krt()UJ} \textit{.} \textit{WHERE} \textit{TEDDY} \textit{IS.} Clark argues that children are

\begin{itemize}
\item su ·fo litp with special attention to the answers to their own ques\-S·eiorsJ.in thi \isi{'}-{\textbackslash}'ay, they acquire the uninverted structures which they
\end{itemize}

, slll)sqtiently use to form \isi{questions} of their own. Clark's explanation is irnplausible on several grounds.

{\textless} c'i i•first; the productive use of \isi{analogy} it entails has little support ih;'aquismon studies generally, and we have just noted one specific c t•ro11f111ite \textit{{}-i}\textit{n}\textit{g{\textquotedbl}} \textit{)} where the predictions it makes are not in fact

;iffllled. \isi{'}.Second, if there is anything children can do with language,

•dt:!{\textgreater}t() tellthe difference between a question and a statement; accord\-

i\{ltt!fJo \citet{Halliday1975}, they learn to do this productively, by applying

;,,fpr()ptUtte intonation contours to some of their earliest one-word

,,,,,,,5,, ,\isi{'},n \isi{'}

\isi{'}· tee,an{\textless}::es, ari:mnd the age of fifteen months. In the two years or so

the sentence and inversion of the subject and auxiliary verb\isi{'}\isi{'} (emphas\isi{'}

\textsuperscript{1}cnl1'Y elapse.between that \isi{time} and their fm\isi{'}

al mastery of \isi{English}

added). This claim assumes (rather uncharacteristically for these au-. thors) that children actually carry out, in the processing of sentences,. the operations which a generative grammar of \isi{English} would apply to derive WH-\isi{questions}. \citet{ErreichEtAl1980}, writing from an orthodox generative standpoint, make the same assumption. However, the latter. note that if children do really derive WR-\isi{questions} in this way, one would expect to find errors reslting from incomplete application of

•• ;i(je);tins; they must receive countless well-formed tokens of the \isi{'} 51\isi{'}.{\textbackslash}Y:•\isi{'}!NChsince the consequences of inattention may in some

. 9a5s . acutely dysfunctional for them-they must listen as acutely

as;·t:. Yf;do j;o answers to their own \isi{questions}. It is well known that

\begin{itemize}
\item \begin{itemize}
\item · ·• · !'l{\textless}try, wherever possible, to maintain ``one form, one function\isi{'}\isi{'} rety iteeth of ``natural\isi{'}\isi{'} languages which insist on having two for•one function and two functions for one form. In the face of
\end{itemize}
\end{itemize}

%\originalpage{190}

all of this, why should children take a form that clearly belongs in answers and use it to make \isi{questions}?

If we assume a \isi{language bioprogram}, however, a much more reasonable explanation emerges. The bioprogram would enjoin just that biuniqueness in form-function and form-meaning relationships which children strive for and which creoles, with a large measure of success, attain. In this, it merely follows the pattern of genetic programs in general, which do not prescribe sets of alternative routines, but leave open the possibility of adapting given routines for other purposes.

One resource in the bioprogram is constituent movement. We will not directly consider \isi{movement rules} in this chapter, although we considered them in the first two, simply because not enough work has been done on the \isi{acquisition of} \isi{movement rules} for any valid comparisons to be drawn. But it would appear from creoles that inove\-ment has, as the overall model would suggest, only one function\-expressing shifts from the expected pattern of focus and presupposi\-tion. Certainly no creole rule that I know of moves any constituent for any other reason than this.

\isi{English}, therefore, goes contrary to the bioprogram when it uses a movement rule-subject-aux inversion-to distinguish between ques\-tions and statements. The child, therefore, either fails to hear correctly or simply ignores the sentences that depart so radically from his expec\-tations. Eventually (perhaps as a result of misunderstandings; it would be interesting to have some ``caregiver interaction\isi{'}\isi{'} data on this) the child observes that subject-aux inversion is required in \isi{yes-no} \isi{questions}. Two factors could reasonably be expected to delay the \isi{generalization} of this rule to WH-\isi{questions}. First, WH-\isi{questions} are unambiguously marked by the initial WH-word, so that misunderstanding is corre\-spondingly less likely to occur. Second, the fact that WH-\isi{questions} are already formally distinguishable from statements could well deter the child from applying what, to him, would be a quite redundant rule-why mark a question as a question twice over?

Finally, of course, he has to capitulate; the child \isi{learning} a creole does not. The \isi{yes-no} \isi{questions} of children in Klima and Bellugi's 
% ACQUISITION 191
second stage and the WH-\isi{questions} of children in their third stage are identical with the \isi{yes-no} and WH-\isi{questions} cited in Chapter 2.

Let us turn to \isi{negation} where, again, the findings· of Klima and Bellugi ( 1966) have hardly been superseded (except, again, that they may not have paid enough attention to individual differences). At the earliest stage of \isi{negation}, a negative morpheme- occasionally \textit{not,} most often no-is placed at the beginning or end of the utterance. These forms persist into the second stage, but here, one or two spe\-cialized negative forms, such as \textit{don't} or \textit{can't,} are also acquired. Since \textit{cannot} and \textit{do} \textit{not} never appear at this stage, we can conclude that for the child, \textit{can't} and \textit{don't} constitute monomorphemic utterances. Also these forms seem to be more restricted in distribution than they are in adult language; \textit{don't} seems to be confined (in Klima and Bellugi's examples, at least ) to \isi{stative} verbs and imperatives.

\textit{Can't} and \textit{don't} are, presumably, superimposed on the bio\-program by sheer force of parental repetition; I know of no statistics on the subject , but casual observation alone suggests that these must be among the most frequent words addressed to small children-perhaps to creole speakers too; for I cannot resist interrupting this account to describe two striking similarities between acquisition and, this \isi{time}, \isi{decreolization}.

It may seem illogical at first to compare acquisition with de\-creolization in a study whose main thrust is the comparison of acquisi\-tion and creolization; but regarding later stages of acquisition, such comparisons are apt and pertinent. The position of the bioprogram\-activating child vis-a-vis the target-language-enforcing adult is highly comparable to the position of the creole speaker vis-a-vis the super\-strate speaker. Both adult and superstrate speaker believe that both child and creole speaker are speaking merely a ``broken\isi{'}\isi{'} form of their own ``proper\isi{'}\isi{'} language. Both child and creole speaker are eventually forced to modify their natural behavior by the bombardment from above.\textsuperscript{8} With regard to changes in negative forms, the results seem to be identical. Both basilectal GC and basilectal HCE order Neg
%\originalpage{192}
before Aux in surface structure; in the \isi{decreolization} of both languages, the thin end of the wedge of \isi{English} negative placement (i.e., in surface structure as the second member of Aux) consists of adoption of the negative form of \textit{can} (GC \textit{kyaan,} HCE \textit{k} \textit{aenat} \textit{)} to replace, respectively, GC \textit{na} \textit{kyan,} HCE \textit{no} \textit{k} \textit{aen.} \textit{Kyaan} and \textit{kaenat} are both perceived and treated as monomorphemic units. In GC, \textit{kyaan} is first acquired, and \textit{doon} (the equivalent of \textit{don't} \textit{)} is acquired second and some \isi{time} later in the \isi{decreolization} process; \isi{in HCE}, \textit{kaenat} and \textit{don} are acquired around the same \isi{time}. In GC, exactly as in child language, \textit{doon} is initially applied to statives, imperatives, and little else (see Bickerton [197S:\tabref{tab:3}.9] , where these two types account for 84 percent of the output of early \textit{doon} users); comparable figures are, unfortunately, unavailable for HCE.

We return \textit{now} to the normal \isi{evolution} of child negatives. Around the \isi{time} that \textit{don't} and \textit{can't} make their first appearance, there also appear sentences such as the following:

\ea\label{ex:58}
 That no fish school.
\glt
\z

\ea\label{ex:59}
 He no bite you.
\glt
\z

\ea\label{ex:60}
 I no want envelope.
\glt
\z

These sentences find exact parallels not in \isi{decreolization}, but in the classic form of creole negative sentences. As in creoles, the negative morpheme is identical with the morpheme of denial. As in creoles, the negative morpheme is inserted directly after the subject, before any verbal or auxiliary element, rather than sentence-externally (as in the first phase of child \isi{negation}) or after a first auxiliary or verbal constituent (as in \isi{English}). This second similarity is maintained even after \textit{no} begins to be replaced by \textit{not:}

\ea\label{ex:61}
 He not taking the walls down.
\glt
\z

\ea\label{ex:62}
 Ask me if I not made mistake.
\glt
\z

% {\textbackslash}

%\originalpage{193}

In part, at least, these developments are natural, perhaps inevit\-able. At the two-word stage there is nowhere the child could put a negative except sentence-externally. Moreover, since \textit{no} is heard as an isolated unit with heavy emphasis, while \textit{not} often occurs in contracted forms which may be unrecognizable to the child, it is hardly surprising that \textit{no} rather than \textit{not} is selected for sentence \isi{negation}.

It is less clear why negative placement in longer sentences takes the position it does. To judge from the examples cited above, that placement involves post-subject, rather than preverbal, placement\-inapplicable in /58/ which has no verb-{}-or second position in sentence\-ruled out by /62/ where a subordinate clause is negated. Yet there is no support for a ``post-subject\isi{'}\isi{'} hypothesis in \isi{English}. One might claim that the child knows roughly where the negative should go but doesn't yet have any auxiliary to place in front of it, so he ru:rives at post\-subject placement by default, so to speak. This may sound plausible at first. But note that post-subject placement is arrived at BEFORE the child acquires \textit{not-not} merely usurps the place staked out by \textit{no.} Just how would the child know ``roughly where the negative goes\isi{'}\isi{'} in \isi{English} if that child has not yet succeeded in even IDENTIFYING \textit{not?} Recognition of the ``\isi{English} position\isi{'}\isi{'} for Neg-placement depends crucially on the ability to realize that \textit{not} (all that ever occurs in that position) is the marker of \isi{negation}. Those who would argue for the ``commonsense\isi{'}\isi{'} explanation of how children acquire negative place\-ment will have to explain how you can learn what a form means and where it is placed WITHOUT ACTUALLY LEARNING THE FORM ITSELF.

There is empirical evidence, too, to confirm that people can't and don't learn in this way. I have already suggested some ways in which child acquisition is somewhat like \isi{decreolization}: or, to put it more precisely , the child's actuation of the bioprogram is like creoliza\-tion, and the child's modification of bioprogram specifications is like \isi{decreolization}. Now, as I abundantly demonstrated in \citet{Bickerton1975}, \isi{decreolization} proceeds by acquiring new forms fust and new functions later.\textsuperscript{9 }Newly acquired morphemes are at first assigned
%\originalpage{1} \textsubscript{ACQUISITION }\textsubscript{1}\textsubscript{9}\textsubscript{5}
meanings and functions that already exist in the speaker's grammar; \_

.\_, {}-,,. ,,. 1.'c{\textgreater}nlci tie argued that sentences like /63/ are nothing more

in other words, these morphemes have to be stripped of the meanings -

,J ·J'l 2 jo•£:

the order in which \textit{somebody,} \textit{nobody,} and \textit{anybody}
and functions which they had in the superstrate before they can be
incorporated into the existing creole grammar. Only later, as that grammar itself changes, do they reacquire all or part of their original • superstrate meanings and functions. I know of no counterexamples· to this empirical finding, nor has it been challenged in the literature; {}-\-We should therefore be highly skeptical of any claims about child\-acquisiticn which involve the assumption that meanings· and func, tions can be acquired in the absence of the formal units which act\isi{'} as bearers of those meanings and functions.

Another puzzle concerns the slow spread of \textit{don't.} If \textit{don't} acquired at the same \isi{time} as post-subject \textit{no,} how is it that the child does not straight away adopt the hypothesis that \textit{don't }is the ``real\isi{'} negative marker, and spread its use to all environments? In fact, \textit{don't\_} MUST be perceived simply as an alternative to \textit{no,} rather than \textit{do} + negative, since the child has no independent \textit{do} at this stage or for some \isi{time} to come. If a child applied this hypothesis just to the exam\-ples /58/-/62/, he would score two almost correct sentences out o the five-he \textit{don't} \textit{bite} \textit{you,} \textit{I} \textit{don't} \textit{want} \textit{envelope-as} opposed to on! three incorrect ones: \textit{*that} \textit{don't} \textit{fish school, *he} \textit{don't taki}\textit{n}\textit{g} \textit{th} \textit{walls} \textit{down,} \textit{*ask} \textit{me} \textit{if} \textit{I} \textit{don't} \textit{made} \textit{mistake.} This is better than fiv out of five incorrect, which is what he now has.\textsuperscript{10}

Resemblances to creole structures are not exhausted even wheti the child has fully mastered the negative placement rule. McNeil! (1966) reports the case of a child who uttered /63/ on eight consecutiv\isi{'} occasions, despite overt parental correction:

\ea\label{ex:63}
 Nobody don't like me.
\glt
\z

Such sentences, though not reported from all children so far studie are by no means uncommon at age four or thereabouts. We saw f Chapter 2 that the use of negative subjects with negated verbs is co mon to a number of creole langu,ages, although it would appear to {}- uncommon in languages generally.

,:.; is hardly surprising that \textit{somebo}\textit{d}\textit{y,} the only one that

, . ; coi;, 1·ete referen t, is learned first. In consequence, children

:.,. than those who produce sentences like /63/ often say

\textit{I} \textit{don't} \textit{see} \textit{somebo}\textit{d}\textit{y} rather than \textit{I} \textit{don't} \textit{see} \textit{anybody.} \textit{fl} \textit{{\textless}tbc•dy} is je trn{\textless}od before \textit{anybody,} it tends to replace \textit{somebo}\textit{d}\textit{y} n£n,,es;Jil{\textless}e this, giving \textit{I} \textit{don't} \textit{see nobody.} At the same \isi{time}, used in subject position, it would be unrealistic

child to realize immediately that no further formal bfi;\isi{negation} is required; unreasonable, too, to expect him to

\begin{itemize}
\item a \textit{once,} in sentences like /63/, the system of verbal \isi{negation}
\end{itemize}

·w(! have just seen, cost him so much difficulty to acquire.

\_ jSt:\isi{'}.sfi:owing, sentences like /63/ would issue, not from some \isi{'}\isi{'} d:.of•the bioprogram to produce multiple negatives, but rather

·• ·1;\$:il):herent in the process of \isi{learning} \isi{English}.

{}- {}- gumel1t stands up much better than most others which

!!tin away creole-like structures in child language. However,
:means immune to question, It depends crucially on inde\-

·{}-{}- 8tivation for the fact that \textit{nobody }is acquired before \textit{any\-}

\textit{;.} \textit{ft} \textit{:}\textit{·}\textit{may} be as common or more common than \textit{nobo}\textit{d}\textit{y} in

{}- {}- {}- \_ Irleaves mysterious both the frequency of negative sub\-

·verb in creoles and the greater frequency of double predi\-

,in languages generally. There must be some way in which

\_tion \textit{is} more natural than single \isi{negation}, despite the

•and logicians.

\# a case where fuller and more carefully collected data may

·.lve the issue. In creoles, negative subject/negative verb is

..s restricted to generic indefinites like \textit{nobody,} \textit{no} \textit{one,}

ajso involves \textit{Neg }\textit{+} \textit{NP }as in the following GC sentence:

.na bait non kyat not bite no cat

%\originalpage{196}

I have not seen any reports of sentences like /64/, but that in itself is no indication that they never occur in child language. If they do not occur, then the ``commonsense\isi{'}\isi{'} argument given above could well be the answer. If they do occur, then an argument based on the \isi{order of acquisition} of negative indefinites cannot account for all the data, and in light of the creole evidence, the workings of the bioprogram

must again be suspected.

For our fourth and final area of creole acquisition comparison, we will look at the \isi{acquisition of} \isi{causative} constructions.

First (for we shall be drawing evidence from the \isi{acquisition of}
more than one language in this area), we must bear in mind that there are many different ways of making the \isi{causative}-non\isi{causative} dis\-tinction (henceforth the CNCD). This distinction may be marked on the subject (as in \isi{ergative languages}) or on the verb. In either case, there may be several different \isi{types of} marking, especially where verb. marking is the option chosen. \isi{English} excludes subject marking, but marks the CNCD on the verb in several ways.

The simplest way of marking the CNCD is by using the same
verb for \isi{causative} and non\isi{causative} versions of the same event-Le., for cases where the subject must be the \isi{causative} agent but also for cases where the subject is the patient, experiencer, or whatever. These cases are differentiated only by transitivity versus intransitivity: causa· tive-agent cases will have both subject and object NP, non\isi{causative}
cases will have only subject NP.

\ea\label{ex:65}
 The door opened.
\glt
\z

\ea\label{ex:66}
 Bill opened the door.
\glt
\z

There are other cases in which the same verb is used for both \isi{causative} and non\isi{causative} versions, but where the non\isi{causative} version must be marked by use of the \isi{passive}:

\ea\label{ex:67}
 *The tree planted.
\glt
\z 

% ACQUISITION 197

\ea\label{ex:68}
 The forester planted the tree.
\glt
\z

\ea\label{ex:69}
The tree was planted (by the forester).
\glt
\z


In yet other cases, a different lexical verb is required for causa\-tive and non\isi{causative} versions:

\ea\label{ex:70}
 The sheep \textit{ate} (non\isi{causative}).
\glt
\z

\ea\label{ex:71}
 John ate the sheep ('f John caused the sheep to eat).
\glt
\z

\ea\label{ex:72}
 John \textit{fed }the sheep (\isi{causative}).
\glt
\z

In a fourth set of cases, no appropriate leltical alternation exists, and for \isi{causative} versions a periphrastic structure must be used:

\ea\label{ex:73}
Mary \textit{suffered} (non\isi{causative}).
\glt
\z

\ea\label{ex:74}
John suffered Mary ('f John caused Mary to suffer).
\glt
\z


\ea\label{ex:75}
 John \textit{made} Mary \textit{suffer} (\isi{causative}).
\glt
\z

Yet another verb-marking method, not used by \isi{English} but found, for example, in \ili{Turkish}, is to employ the same lexical verb in both cases but differentiate them by means of a verbal affix. Ergative languages, too, generally use the same lexical verb, but mark \isi{causative} subjects only with the ergative case-marker; subjects of non\isi{causatives} are marked, like objects of \isi{causatives}, with the accusative case-marker. The particular strategy or selection of strategies chosen by any lan\-guage to make the CNCD will, of course, reflect the typology of that language. But the function of all of these varying devices is identical.

Some methods of expressing the CNCD would seem to be more easily acquired than others. \citet{Slobin1978} reports a cross-linguistic experiment on the interpretation of \isi{causative} constructions in which the subjects were child learners of \isi{English}, \isi{Italian}, Serbo-Croat, and \ili{Turkish}. Subjects were required to act out with toy animals sentences such as \textit{the} \textit{horse} \textit{made} \textit{the} \textit{camel} \textit{run.} In \isi{English}, \isi{Italian}, and Serbo\-Croat, such sentences have rather similar structures, involving two dis· tinct verbs, one of them a lexical \isi{causative} like \textit{make} \{Slobin did not
%\originalpage{1}
include examples of the three other \isi{English} ways of marking the CNCD). \ili{Turkish}, however, uses single verb + afftx:

\ea\label{ex:76}
 At deveyi kotttrsun
\glt
\z

horse-nom camel-ace run-\isi{causative}-0ptative-3rd pers. (lit., the horse ran the camel)

'The horse made the camel run\isi{'}

The task was performed with almost 100 percent accuracy by \ili{Turkish}\-speaking children before the age of three. Serbo-Croat speakers, how\-ever, did not reach this level until they were four or over, while even at age four the \isi{English} and \isi{Italian} speakers averaged between only 60 and 80 percent.

This fmding is hardly surprising in light of the fact that the \ili{Turkish} \isi{causative} suffix is learned and used productively and correctly by the age of two{}-another of those cases of ``errorless \isi{learning}\isi{'}\isi{'} we discussed earlier in this chapter. Equally early and errorless marking of the CNCD is reported by \citet{Schiefflin1979} for \ili{Kaluli}, an ergative language of Papua-\isi{New Guinea}. Here, the suffix which is applied to \isi{causative} agents is fully acquired and appropriately used by age 2:2, withont ever being generalized to nonagentive subjects.

The fact that CNCD strategies that involve marking of \isi{causatives} by bound morphemes and single-clause structures (the case in both \ili{Turkish} and \ili{Kaluli}) are acquired earlier and more easily than struc\-tures involving two clauses and a \isi{causative} verb casts strong doubts on those generative-semanticist analyses that would assume something like \textit{Bill} \textit{caused the} \textit{door to} \textit{become} \textit{open} as the underlying structure of sentences like /66/. We shall return to this point shortly when we discuss the treatment by \citet{Bowerman1974} of the \isi{acquisition of} \isi{English} \isi{causatives}.

First, however, we should ask how the cases of \ili{Turkish} and

\ili{Kaluli} relate to the creole case. We saw in Chapter 2 that out of the six potential strategies for expressing the CNCD described above (case- . marking\textsubscript{,}verbal affixation, causal{}-v\textsubscript{'}erb periphrasis, passivization, lexical
%\originalpage{199}
alternation, and simple transitive-intransitive alternation ), creoles use only the last named. The examples given/86/-/91/, Chapter 2- were identical in structure with the \isi{English} examples in the present chapter, i.e., /65/-/66/. Notoriously, creoles avoid bound-morphology solutions. ls it not then counterevidence to the language bioprogratn that the bound-morphology solutions of \ili{Turkish} and \ili{Kaluli} are so quickly acquired?

The answer is: not in the slightest. To provide counterevidence of any value, one would have to show that \ili{Turkish}-type or \ili{Kaluli}\-type solutions were acquired BEFORE the simple transitive-intransitive alternations of the kind that creoles make. This is a \textit{most} unlikely finding because, in fact, the \ili{Turkish} and \ili{Kaluli} solutions ARE AL\-READY transitive-intransitive alternations which are simply under\-lined, as it were, by the addition of a further marker. Moreover, \isi{English} \isi{causatives} of the \textit{door} \textit{opened} \textit{/Bill} \textit{opened} \textit{door} type are certainly acquired at an equally early age; it is the three other \isi{types of} \isi{causative} that create problems, as we shall see.

Far from being counterevidence, the \ili{Turkish} and \ili{Kaluli} cases are confirmatory. If there is a \isi{language bioprogram}, then children are programmed with a set of basic distinctions which they expect that their native tongue will implement somehow. It is less clear whether, or ..to what extent, they are specifically programmed with the means t realize these distinctions should their native tongue fail to meet their expectations (as is the case, most drastically, if they are born into a pidgin-speaking community). I suspect that the bioprogram may turn out as follows: both distinctions and means for implementing them ar{\textless};l programmed, but are not necessarily conjoint in the program. We have already claimed that the bioprogram is not present at birth, but unfolds progressively during the course of the first four years or
so; .of Hfe. The distinctions would then be programmed to emerge
prior to .two, possibly around eighteen months or earlier, while the m7ans of implementation would not necessarily emerge until the
third or. fourth year. Thus, children would start early searching for cme;UJs t.o express the distinction, and only if they failed to fmd any
%\originalpage{200}
would they need the implementation part of the program.

% ACQUISITION 201

Put like this, without any supporting evidence, the structure of the bioprogram may look too much like some bizarre kind of provi\-dentiality, as if a well-meaning deity had foreseen the consequeces of European imperialism and specially equipped his creatures to circum\-vent them. However, the picture will change considerably in the next
\citet{Bowerman1974} observed that from around 2:3 on, but more
articularly arund the age of three, children would employ intransi\-t1ve (non\isi{causative}) verbs in \isi{causative} sentences:

\ea\label{ex:77}
Mommy, can you stay this open ?
\glt
\z

(sc., make this stay open, keep this open)

chapter, when Ishall discuss the ways in which the bioprogram may

have come into existence. Creole languages will then appear not as a case of divine foresight and beneficence, but rather as the quite acci\-dental consequence of a much vaster design.

As for those who claim that the \isi{causative}-non\isi{causative} distinc\-tion is one that is salient to the child and important in his interaction with his environment (and therefore easily learnable from experience),

\ea\label{ex:78}

\glt
\z

\ea\label{ex:79}

\glt
\z

\ea\label{ex:80}

\glt
\z

I'm gonna fall this on her.

(sc., make this fall on her, drop this on her) She came it over there.

(sc., made it come over there) How would you flat it?

(sc., make it flat, flatten it)

it does not follow, even if the claim is correct, that he can learn from this alone that the CNCD is marked in the language he is \isi{learning}. There are innumerable facts about the real world that a child has learned by age two, and many of them are extremely important to him, but extremely few of them are explicitly coded in language. How, without prior knowledge, can he know which is to be coded and which is not ? And this is without even considering other kinds of problems involved in correct \isi{learning} of the various CNCD expressions, some of which we will review after discussing \isi{English} acquisitions.

One of the things that facilitates \isi{acquisition of} \ili{Turkish} and K 11l nli i q th:i t th{\textquotedbl}y are uniform· there is hut one way to form catJsatives, and the morpheme involved is unique and undergoes only phonologically-conditioned forms of variation. The picture in \isi{English}, with its four ways of expressing the CNCD, is at an op.posite extreme. Since conflicting evidence is not much better than no evidence at all, the theory would predict that \isi{English} learners would treat En\-glish, in this respect, just as creole children treat a pidgin; that is,.having failed to extract from their input a consistent way of expressmg the CNCD, they would generalize the simplest transitive-versus-intransitive solution, already available to them from \textit{open-}\textit{t}\textit{y} \textit{pe }verbs, to other classes of verbs. And this is, in fact, exactly what they do.

Note that this creative process extends to \isi{adjectives} as well as verbs

\textit{(/} \textit{Of}\textit{)}, and that the line between \isi{adjectives} and verbs may therefore, at this stage, be as thin as it is in creoles.

. This process does not limit itself to intransitives. Transitive verbs hke \textit{eat} which are restricted to non\isi{causative} meanings (see /70/, /71/ abve) and hence, except where cannibalism is practised, to nonhuman objects are also treated as if they were potential \isi{causatives}:

\ea\label{ex:81}
 Child (pretending to feed doll) : See, she can't eat!
\glt
\z

Mother: Just pretend, honey. Child: But Ican't eat h er!

(sc., make her eat, feed her )

. .Clark and \citet[511]{Clark1977}, in discussing these developments, explicitly compare them with the child's over-\isi{generalization} of regular \isi{plural} forms. Ideed, what is .significant about these cases is precisely that they constitute a \isi{generalization} to \isi{English} of the regular creole strategy. But a good deal more is involved than that.

\isi{'}\isi{'} Let s ppose that children learn language by adopting a series of strategies ; whether learned or innate is immaterial here. Such

%\originalpage{202}

strategies would clearly include \isi{generalization}, one of the best-attested concomitants of acquisition. The strategy of \isi{generalization} might be informally defined as follows:

Step 1: Look for any regular form with a consistent core of meaning.

Step 2: Apply that form in all possible environments.

Step 3: Compare output with input, and note cases (if any) where these do not match.

Step 4: Remove the exceptions (if any) which appear when Step 3 is applied.

This strategy would be applied in a wide variety of cases: in \isi{English} pluralization, \isi{past tense}, and, again, in \isi{causatives}. The child would note the existence of a number of pairs like X \textit{o}\textit{pened} \textit{/Y} \textit{opened} X (Step 1); he would generalize this, yielding pairs like X \textit{ate}\textit{/} \textit{Y} \textit{ate} X (Step 2); he would note counterevidence such as \textit{Y} \textit{fed }X (Step 3) ; he would then gradually substitute ``irregular\isi{'}\isi{'} forms like \textit{Y fed }X for false ``regular\isi{'}\isi{'} forms like \textit{Y} \textit{ate} X (Step 4 ).

Let us suppose that the \ili{Kaluli} learner applied a similar strategy. He would fast observe that a number of nouns in subject position had an ergative affix (Step 1); he would then generalize the affix to all NPs in subject position (Step 2); he would then note that in fact a number of subjects had a different kind of affix (Step 3); he would then work toward a correct distribution of the ergative and accusative affo{\textless}:es (Step 4).

Unfortw1ately, while the \isi{generalization} strategy provides an exact \isi{description of} what \isi{English} learners do about \isi{causative} marking, it provides a completely inaccurate \isi{description of} what \ili{Kaluli} learners do about their \isi{causative} marking. If \ili{Kaluli} learners applied the same strategy, then we should find large numbers of ergative case-markers applied to experiencer or patient subjects which, according to Schief\-flin, we do not do. Why is the \isi{generalization} strategy chosen in one case, but not in the other? \textsubscript{{\textbackslash}}

%\originalpage{3}

A simplistic answer might be: because the two cases are not really comparable. In \ili{Kaluli}, there is a semantic and pragmatic distinc\-tion between subjects that cause things to happen and subjects that do not. In \isi{English}, no such distinction is involved. The sets of verbs that take simple transitive-intransitive alternation, as opposed to those that take lexical alternation, passivization, or causal-verb periphrasis, is not a natural semantic class; nothing but experience could tell one that \textit{the} \textit{jock.}\textit{ey} \textit{wal.ked} \textit{th}\textit{:} \textit{horse} and \textit{the} \textit{jockey} \textit{galloped} \textit{the} \textit{horse} are gram\-matical, but \textit{*the;ockey} \textit{ran} \textit{the} \textit{horse} is not.

It is true that the two cases are not comparable from the stand\-point of an adult who knows something of the grammar of both lan\-guages-but from the CHILD's viewpoint? How is the child supposed to recognize that semantic sets are involved in one case, but not in the other, unless he already knows what the relevant semantic sets are? He cannot construct semantic sets from experience alone until he has at leat eperienced the full\_range of \isi{semantic classes} that the language contruns (1f then!). Each lexical item has so many parameters of mean\-ing, could fit into so many partially overlapping classes, that one could never say for certain, given any body of partial data, whether semantic lasses did or did not coincide with the formal differences perceptible m those data. But production does not stand still until the child has master:d possible \isi{semantic classes} in the language confronting him. The child 1s under pressure to talk, whether he is ready or not.

If the child formed hypotheses, as so many suppose, then there woul be many different hypotheses that the \ili{Kaluli} child might make. He .might assume that ergative and accusative case-markers are merely sub3ect markers that happened to be in free variation or that the ergative :{\textquotedbl}.arker m:irked subjects that happened also to 'be topics, or that sta.t1v1ty was mvolved somehow (since many \isi{causatives} are non\-statives, while.many statives are non\isi{causatives}, this hypothesis might be a very.attractive one). But wrong choices of hypothesis would inevit\-ab\_ly yield misplaced case-markers, and this does not seem to happen. Miraculu:ly, somhow:he first ``hypothesis\isi{'}\isi{'} is the right ``hypothesis.\isi{'}\isi{'} Similar considerat1.0ns apply to the \isi{acquisition of} \ili{Turkish}, except
%\originalpage{204}
that here the child's task is made more complex by the fact that the \isi{causative} marker is only one of a string of verbal suffixes which fre\-quently co-occur: suffixes which indicate reciprocity, \isi{negation}, person, number, tense, and the direct/indirect knowledge distinction which, as we saw above, is the only one that seems to cause problems. These strings of suffixes present two quite distinct kinds of problems. The first is a problem of segmentation, which the child presumably solves by some kind of substitution-in-frame process. The second-figuring out what each of the suffixes means, once they have been segmented\-is less often considered, perhaps because it looks easy to the adult, who can ``look in the back of the book,\isi{'}\isi{'} so to speak. In fact, it is much more difficult than the first, and the fact that speech to children is strongly oriented toward the here-and-now, often urged as a reason why children do not need an innate component, in reality makes the task harder rather than easier; every situational context is composed of innumerable factors, any of which, for all the child is supposed to know, could be directly reflected in linguistic structure, and sets of context ual features are seldom constant from one situation to another. The child who tried to figure out which semantic factors were marked grammatically- assuming that a two-year-old mind would be remotely capable of this, even at an unconscious level-would be in the position of someone who tries to solve a maze problem; he would have to take the most promising-looking path, pursue it until it was blocked, then retrace his steps to the beginning again and repeat the process. But when we consider that the same semantic factors are marked gram\-matically over and over again across the range of human languages, that in effect languages select out of a very short list of \isi{semantic primes} the ones that they are going to mark, much as they select their phonological inventory from the set of distinctive features, it becomes more reasonable to assume that the child has advance knowledge of the
contents of the category ``grammatically-markable semantic feature.\isi{'}\isi{'} Thus, both a ``strategies\isi{'}\isi{'} approach and a ``hypothesis-forming\isi{'}\isi{'}
approach fail to account for the \isi{learning} of the CNCD in \isi{English},
\ili{Turkish}, and \ili{Kaluli}. A ``strategic\isi{'}{\textbackslash}\isi{'} approach fails to explain why the
%\originalpage{205}
child over-generalizes in the case of \isi{English} \isi{causatives} but not in the case of \ili{Kaluli} \isi{causatives}{}-unless it introduces some ``hyperstrategic\isi{'}\isi{'} device which would tell the child which strategy to use inwhich case.11 A ``hypothesis-forming\isi{'}\isi{'} approach fails because it cannot show how, out of a wide range of hypotheses that the child could form about the nature of \ili{Turkish} and K.aluli morphemes, that child invariably picks the correct one the first \isi{time} around. A language-bioprogram approach is able to deal with both problems. It has no strategies, so the first problem is a ghost problem. It specifies the set of distinctions to be marked, so the second problem does not arise.

However, before leaving \isi{causatives} we should consider an obser\-vation made in \citet{Bowerman1974} that while ``correct\isi{'}\isi{'} \isi{causatives} like \textit{Mornmy open door }are acquired before periphrastic \isi{causatives} like \textit{Billy }\textit{mtike} \textit{me} \textit{cry,} ``incorrect\isi{'}\isi{'} \isi{causatives} do not appear until AFTER the emergence of correct \textit{mtik} \textit{e} sentences. From these facts, Bowerman argued (and the argument sounded a lot better in the days when genera\-tive semantics was still alive) that although the child at an early stage might PRODUCE sentences like \textit{Mommy open door, }he would not yet be {\textless}ible to ``break down\isi{'}\isi{'} such sentences into ``a cause proposition and an effect proposition.\isi{'}\isi{'} However, once he had acquired \textit{make} sentences, which do formally divide the sentence into these proposi\-tions, he could then analyze sentences with \textit{open, }etc., in just the sam.e way; and, once this reanalysis was complete, it could be generalized. to both transitive and intransitive \isi{causatives}, as we saw in examples
/77/\isi{'}/81/.

There are several problems with this argument. It is far from certainthat two distinct propositions do underlie \textit{X-open-Y }sentences; mere existence of \textit{make-X-do-Y }sentences is not itself evidence one

\textit{ot.} the other. Certainly, the results of Slobin's experiments, dis.

.c..,-·-{}- above•. suggest that the latter sentences are perceptually more

comf•.leic than the former, therefore intrinsically unlikely candidate tr mud.erlyii1g forms.

a more serious objection stems from Slobin's (1978) work

%\originalpage{206}

Slobin found that even at age four, \isi{English} learners often could not act out \textit{make-X} \textit{{}-do-Y} sentences correctly, which suggests that even at that age, they understood them only imperfectly. If this is the case, then it is hardly likely that children a little over two could understand them structurally in the way that Bowerman claims. Of course, Bower\-man could not be expected to foresee Slobin's results, but she assumes that children understand \textit{make} sentences on the basis of no evidence whatsoever.

Let us suppose that children could analyze sentences as she . suggests. In that case, why do they not generalize \textit{make-X-do-Y }to newly acquired non\isi{causatives}, instead of going back to \textit{X-open-Y} and generalizing that? If they took this surely very plausible step, they would produce perfectly grammatical sentences like \textit{can you} \textit{mak} \textit{e} \textit{this} \textit{stay} \textit{open?,} \textit{I'm} \textit{gonna} \textit{make} \textit{this} \textit{fall on} \textit{you,} etc., in place of the ungrammatical /77/-/81/. The fact that they do not do this, viewed in light of Slobin's results, suggests that the earliest periphrastic \textit{make }\isi{causatives} are acquired as idiomatic chunks which are not yet analyzed and therefore not yet generalizable. If they are not analyzed, their analysis cannot be what triggers the spread of incorrect \textit{or{\textgreater}et1-} type \isi{causatives}. Bowetman's argument is simply the logical
\textit{post} \textit{hoc,} \textit{ergo} \textit{propter} \textit{hoc}. 
As for the alleged delay in the appearance of incorrect \textit{open\-} type \isi{causatives}, this could be due to nothing more complex than the interaction of communicative need with available vocabulary. As long as the child can handle his needs with a relatively small vocabulary, the need to ``invent\isi{'}\isi{'} new \isi{causatives} simply will not arise. But when the number of things he wants to (and potentially can) say is expanding more rapidly than his vocabulary. which is the case as he gets deeper into his third year, he will need to express concepts like those expressed by \textit{drop,} \textit{flatten,} etc., before he has had the opportunity to acquire the appropriate lexical items. And it is from this period, say 2: 6 to 3:3, that most of Bowerman's examples are drawn.

We have now reviewed a ``\{ide range of evidence, dealing with the
%\originalpage{7}
\isi{acquisition of} a number of widely different features in several different languages, which cannot easily, if at all, be accounted for by existing theories of language acquisition, but which follow naturally if we assume the existence of an innate bioprogram for language. Moreover, the view of acquisition which this assumption provides is more satisfactory on a commonsense level. Hitherto, we have had to assume that small creatures who could barely control their own bowel movements were capable of \isi{learning} things-whether you choose to call them ``rules\isi{'}\isi{'} or ``behavior\isi{'}\isi{'} is quite irrelevant at this level{}-of such abstractness and complexity that when brought to the level of consciousness, mature scholars often misanalyze them. This paradox was not very often alluded to, but of course it was always there whether it was alluded to or not. Now we can see that children can only learn language because, in effect, they already know a language.

Interestingly enough, a similar view was arrived at by \citet{Fodor1975}, arguing in a completely different way from a completely different starting point. According to Fodor, it is not just common sense improbable, it is logically impossible for anyone to learn a language unless he already knows a language. ``Learning a language (includ\-ing, of course, a fast language) involves \isi{learning} what the predicates of the language mean. Learning what the predicates of a language mean involves \isi{learning} a determination of the extension of these predicates. Learning a determination of the extensions of the predicates involves \isi{learning} that they fall under certain rules (i.e., truth rules). But one cannot learn that (P)redicate falls under (R)ule unless one has a lan\-guage in which P and R can be represented. So one cannot learn a language unless one has a language\isi{'}\isi{'} \citep[63-64]{Fodor1975}.

Thus, to give a concrete example from the first case we looked at in this chapter, a child cannot know which members of the class
\textit{a.} \textit{NP} are specific and which are nonspecific unless he knows what specific and nonspecific mean, and he cannot know what they mean unless he has, in some sense, a language in which that meaning is some\-how represented. As to how it might be represented, that must be
reserved for the next chapter.

%\originalpage{208}

\citet{Marshall1979} notes that ``no-one has yet brought forth a convincing counter-argument\isi{'}\isi{'} to Fodor's claim, although most people agree ``that this conclusion is untenable.\isi{'}\isi{'} I find it bizarre that a strictly logical conclusion should be regarded as untenable, especially when neither Marshall nor anyone else has been able to suggest any cogent or coherent reason why it should be untenable. I find it doubly bizarre now that Fodor's claim can be supported by the large body of empirical evidence surveyed in the preceding chapters-evidence arrived at by methods totally different fi:om Fodor's and, at the \isi{time} of gathering, in total ignorance of his claims. When two such dissimilar approaches
agree so completely in their results, neither coincidence nor \textit{Jolie} \textit{a}
\textit{deux} provides a convincing explanation.

However, there is tremendous emotional resistance to the idea that language is innate, some of the reasons for which I would like to glance at briefly in Chapter 5. In part, this emotional resistance is rationalized by some curious ideas about what is entailed in the making of innatist claims. Typical are the following:

It is not very helpful, however, to stop with the conclu\-sion that linguistic universals spring from innate predispositions (Clark and Clark 1977 :517). ·

. .. to assume that deep structures are ``innate\isi{'}\isi{'} makes a postulate out of a problem and this in itself means that all further study can lead us nowhere \citep[383]{Luria1975}.

Similarly, I am quite certain that many students of acquisition who have read this far will at the moment of reaching this very para\-graph be thinking something along the following lines: ``Sure, he says that the problems of accounting for acquisition are much simpler if you assume an innate bioprogram. Of course they are; you can simply avoid them by making a completely untestable claim. Everybody knows that children lern; the real job is finding out how they do it, and he's
just shirking that.\isi{'}\isi{'}

%\originalpage{9}

There are so many replies to this, one hardly knows where to start. Let me begin by saying that students of acquisition have shirked
two tasks, not just one: the task of accounting for how creoles were learned, and the task of accounting fof how the ftrst human language, whatever that was, was learned. If they think that these two tasks are somehow different in kind from, or irrelevant to, the processes of

{\textquotedbl}normal\isi{'}\isi{'} language acquisition, the onus is now. squarely on them to prove this.

Next, nobody is denying that children learn. Children \isi{learning} \isi{English} learn the difference between \isi{English} and the bioprogram language, and I am sure that they use a whole battery of \isi{learning} strategies, inductive processes, etc., in the course of doing this. Students of \isi{learning} in the traditional sense need have no fears that the rug will be pulled out from under them; their field is still ample, and, if nar· rower, at least better defined than before.

All that is threatened is the assumption underlying their attitude: that language cannot be innate. This is in fact an a priori assumption for which the only evidence ever advanced is the ostrich-like pooh\-poohing typified by Luria's comment. What is more, it is an inherently improbable assumption in view of the fact that the vast majority of hehavior by animate creatures, especially behavior as crucial to a species as language is to ours, is biologically programmed. To suppose that language is not is against the balance of the evidence and a mere
piece of species arrogance, as I am sure any Martian arbiter (if only there \textit{were }Martians!) would quickly agree.

If indeed language is innate, then to continue looking for ways in. which it could be learned from experience makes about as much sense as dropping your keys on the left-hand side of the road and then looking for them on the right-hand side because there aren't any streetlamps. on the left·hand side. Further, the claim that the theory is untestable, like the claim that innateness represents a necessary terminus for research, is simply untrue.


%\originalpage{210}

In the course of the present chapter Ihave mentioned specifically a number of predictions which the theory makes about acquisition processes; these should be easily testable by reference to primary data. Moreover, acquisition has yet to be studied in the vast majority of human languages. All of them should show reflexes of the bioprogram features claimed in this chapter, although clearly those reflexes will differ from language to language since we cannot study the activity of the bioprogram directly, but only its interaction with particular target languages. Thus, the evidence available will not always be clear, and its interpretation will be more often than not a matter of legiti\-mate controversy; but nobody can claim that such evidence is either scarce or hard to obtain. The fact that I have been able to derive so much evidence from works whose authors were not even looking for phenomena crucial to the present theory lends further support to the claim that evidence will be plentiful; if even the crude plow of the pioneer throws up nuggets, there can be little doubt that the trained prospector following on his heels will fmd many more.

Moreover, there are other ways in. which the theory can be tested. One is by a study of the present-day \isi{acquisition of} creole languages, a study which has yet to be carried out. Although creoles are nowadays acquired in just the same way as other languages, the nature of their origins ought to mean that they are acquired with far fewer mistakes on the part of the children, and in a far shorter period

,,f Comp1.ri:o:ons betwet{\textgreater}n ricquisition in creole and noncreole

i:-· sf\isi{'}{\textless}; \textit{cm} f'mpiricalJy test this hypothesis, and if differences in \isi{time} span and/or quantity of error do indeed exist, they can give them a reasonably accurate statistical measurement. Of course, the results

will be more meaningful the more that creoles relatively free of super . strate influence, which have remained relatively unchanged since their origin, are made the subject of study. Little value would be obtained from a study of HCE acquisition, for example, given the rising tide of \isi{English} that is presently eroding it , and the fact that in its purest forn1, it is spoken only by a minority of the population, few if any of wholllf are now under forty-five.

%\originalpage{211}

Eventully, of curse, empirical testing of the theory will depend on advances m the field of neurology, since whatever is innate must have n bjective physical foundation in the structure and/or mode of funct10ng of the human brain. Indeed, linguists are all too often oefull ignort of this field. For example, Alleyne ( 1979) writes: There is nothmg \textit{readily} \textit{apparent} in the neurological and cognitive sysems of humans that makes it natural or inevitable\isi{'}\isi{'} that the cate\-gories I ave proposed for TMA systems should be the appropriate ones
(emphasis added). The expectation that the appropriateness of semantic c.ategones should be ``readily apparent\isi{'}\isi{'} in our neurological and cogni\-tive systems would appear to presuppose a human brain charted labeled, and numbered like the old-\isi{time} phrenologist's diagrams. We ar: ower'e nea: that stage yet; if we get there, it will be due in part to the lmu:st s tlhng the neurologist some things to look for, and the neurologists telling the linguist whether what he has found confirms or disconfirms what the linguist predicted.

Remarks like those of Luria or Clark and Clark cited above seem o nvision the linguist as some kind of bucolic sheriff, shaking his fist impotence because the perpetrator has just fled across the county hne. So what if w hae o go learn neurology ? So what if neurologists hve to g learn ln;i.gmstics? We are boring the same mountain from different sides, and the idea that innateness spells scholarly impotence
reflects only the lack of imagination of those who entertain it.

We have not even yet exhausted the remedies available to us t1¥11t; h.ere and now. There is a diachronic aspect to the whole issue w111.ch has not yet been appreciated. The bioprogram itself must have a history and an origin, and that history and origin cannot lie beyond all tracing. It is true that the attempt to trace them will kad us into what s proved a veritable Sargasso Sea of theories: glottogenesis, the ong1 of human language. But we will have at least one advantage over earher voyagers. We will be equipped with a much more explicit theor, and oe moreo.ver that can draw on the many advances in
evolutionary science which have taken place since the last \isi{time} glotto\-genetic speculation was fashionable.
%\originalpage{212}}

In the next chapter, accordingly, we will attempt to reconstruct the prehistory and early history of human language, in order to deter\-mine, if at all possible, what might be the origin s of the language bi\textsubscript{0}, program whose consequences the first three chapters have explored, In particular, we will try to suggest specific bases for a least some of those semantic distinctions which, as the present chapter has suggested; . must constitute an important, although far from the only, part of . the structure of the bioprogram. For the convenience of the reader, \textbf{I·} repeat the four major distinctions dealt with in this chapter, together • with evidence for each: ·
%\originalpage{213}
and quite unavoidably, we must take off for a more speculative realm. And yet in that realm, we must never lose sight of the fact that there is at least one thing there that is as certain as death or taxes. Even if it could be shown that natural languages are learned, even if it could
be shown that creoles are learned, it cannot be shown that the original human language was learned{}-for it could not have been learned. Even jf one believes that our ancestors were taught by spacemen, then the spacemen weren't taught, or whoever taught the spacemen wasn't taught. There is no escape in regress. Somewhere, some\isi{time}, somehow, human language begar1, and it could not have begun through \isi{acquisition strategies}, or inductive processes, or \isi{hypothesis formation}, or mother's

1\textsubscript{,}1\isi{'}\textsubscript{.}1\isi{'}

1) \textit{S}\textit{pec}\textit{i}\textit{fic-nonspeci} \textit{fic. }Evidence: universality of creole home-cooked language lessons. It must have been ``invented.\isi{'}\isi{'} And if

versus indefinite article; errorless \isi{English acquisition} of a\textsubscript{1 }there were already processes by which language could be invented, versus \textit{a}\textit{\textsubscript{2}}\textsubscript{ }• it goes against parsimony to suppose that the human species then had

2) \textit{State}\textit{·}\textit{process.} Evidence: ``skewing\isi{'}\isi{'} of creole verbal systems;· to acquire a whole lot of new processes in order to learn what it could distribution of nonpunctuals in creoles; errorless acquisition already invent and therefore, presumably, reinvent, whenever occasion of \isi{English} \textit{·}\textit{ing }distribution; errorful \isi{acquisition of} \ili{Turkish} l{\textquotedbl}ight arise. As we shall see in Chapter 4, it is much more plausible to

\textit{{}-dI/.mis} distinction. suppose that each step slowly and painfully made in the direction of

\textit{3}\textit{)} \textit{Punctual-nonpunctual. }Evidence: universality of nonpunctual language was then-harl to be-incorporated into the genotype so that marking in creoles; mode of \isi{acquisition of} past \isi{tenses} in it could serve as the take-off point for the next step.

\ili{French} and \isi{Italian}. Imagine a man ascending the face of a glacier. Painfully and \textit{4)Causative-non\isi{causative}. }Evidence: N;V/NVNi alternation in laboriously he hacks out each step. Each step has to be hacked out to creoles and \isi{English acquisition}; errorless \isi{acquisition of} causa- give him space so that he can stand and hack out the next step. But tive marking in \ili{Turkish} and \ili{Kaluli}; problems of \isi{English}, the steps remain behind him when he has moved on, and once they

\isi{Italian}, and Serbo-Croat learners with ``ii;erneracti{\textbackslash}re-;;enaa11ti•CS· l ·..·· are complete, the merest novice can attain the summit with ease. type\isi{'}\isi{'} \isi{causatives}.

This list is by no means intended to be exhaustive. Its members are merely those distinctions best attested so far in both creoles \isi{and acquisition}. In addition, we will look out for factors which might have influenced the more purely syntactic features we have surveyed, such as the order of auxiliaries, \isi{sentential} \isi{complementation}, \isi{verb serialization}, etc.

So far, our flight has h d the ground of empirical fact. Now,

%\originalpage{215}

\chapter{Origins}

Ever since it has been anything you could even remotely call a science, linguistics has been set in a mold of static formalism. I do not make this remark in a spirit of reproach, as do so many who offer in exchange some form of quantitative, communication-oriented, or functionalist approach , equally static but a whole lot less rigorous. There are many points in the history of a science when developments·. that may not seem desirable from an ideal viewpoint may be strategi\-cally .necessary if the discipline is to advance; such I believe were the
longer be ignored, they are verbalized away, rather than grappled with; I am thinking, for instance, of Chomsky's response to a perfectly legitimate question by Harnad (Hamad et al. 1976:57). \textsuperscript{1} Of course, if you believe that human language is always and everywhere tbe same, that all languages are equal in expressive power, and that how human language developed can have no conceivable relevance to what language is like today, if indeed it developed at all, if indeed it did not spring irt its entirety from Jove's brow by some beneficent and unprecedented mutation-if you believe all of this, then you are ill-adapted to under\-stand the dynamic processes which, as every man of sense since Hera\-ditus has realized, govern all that takes place in our universe. Those in whom the malady is less advanced are hereby requested to retune their receivers to a processual wavelength, if they wish to ge.t the most out of the present chapter.

The fact that static formalism has prevented linguists from grappling with the \isi{origins of} language has not, of course, prevented persons from other disciplines-with, unfortunately but inevitably, rather less understanding of all that language entails-from trying iheir hands at it. Their efforts-and those of earlier generations of linguists-have yielded a host of purely speculative theories which I shall not attempt to review here.\textsuperscript{2} Suffice it to say that all of them suffer from the same defect: they concentrate, exclusively or almost so, on the moment when recognizable speech f'irst emerged, when Ug first
idealizations initiated by de Saussure and refmed by Chomsky. I would

\textsuperscript{sa}\textsuperscript{i}. \textsuperscript{d} to

\textsuperscript{0}g,

\textsuperscript{{\textquotedbl}}. . . . .\textsuperscript{{\textquotedbl}} \textsuperscript{(}\textsuperscript{{\textquotedbl}}. . .. .\textsuperscript{{\textquotedbl}} \textsuperscript{b}em\textsuperscript{.} g some

\textsuperscript{k}m\textsuperscript{'} \textsuperscript{d} o\textsuperscript{f} meam\textsuperscript{.}ngr\textsuperscript{,}\textsuperscript{.}w\textsuperscript{\_}\textsuperscript{,}, even

not even go so far as to say that such idealizations had outlived their

usefulness. In particular, work by Chomsky and his associates over the last decade-which I understand is aimed principally at establishing, as it were, the outer limits of language-is I think extremely important and complementary, rather than opposed, to the present approach, as I shall try to show in the fmal chapter.

The only problem from my point of view with the generativist approach is that it tends to create a mind-set rather difficult to
to the kinds of problems we have to address in the present study. acute cases, the mind·set may be so rigid that when problems can no


if only monolexical, proposition, delivered in the vocal mode). This, which one can only characterize as the \isi{Flintstones approach} to lan\-guage origins, totally ignores the vast amount of preadaptation that was necessary before you could even get to that point, and equally ignores the vast amount of postadaptation that was necessary in order to get from that point to fully developed human language.

That the \isi{Flintstones approach} lives is shown by the current hottest number in origins studies-the ``gestural-0rigin\isi{'}\isi{'} theory (Hewes 1973; 1976, etc.). Whether or not the theory that a gestural language preceded spoken language is a violation of parsimony (as suggested by

%\originalpage{216}

Hill and Most [1978] ) is really beside the point. Either this gestural language was of a structure as complex and noniconic as modern sign-in which case the real question would be, how did this gestural language develop?-or it was some much simpler and more iconic system-in which case the real question would be, how did it get to be more abstract and complex? In fact, the ``gestural-origins\isi{'}\isi{'} theory is just as much focused on the supposed ``critical point\isi{'}\isi{'} of language development, and just as indifferent to any of the substantive \isi{questions} about language origins as any of the other ``Flintstone\isi{'}\isi{'} theories.

The trouble with almost all previous attempts to look at origins is that they do not go back far enough. If we were to understand thoroughly all that language involved, we would probably have to go back to the birth of the lowliest animate creat ures, for language depends crucially on a matrix of volition and primitive consciousness which must have begun to be laid down hundreds of millions of years ago. Such an approach lies far beyond the scope of the present volume, and will be addressed in a subsequent work, \textit{Language} \textit{and} \textit{Species.} Here, I shall go back no further than, say, \textit{\isi{Dryopithecus},} although a brief glance at the frog will not hurt us.

Again , as in previous chapters, a certain amount of ground\-clearing work must be done if we are to avoid irrelevant distractions. At the very least, we will have to dispose of what I shall call the Para\-dox of Continuity.

The Paradox of Continuity is, at the present moment, perhaps the greatest obstacle to a proper understanding of language origins, as well as a powerful factor in keeping linguistics isolated from other human studies. It may be expressed as follows. On the one hand, all the species-specific adaptive developments that we know of have come about through regular evolutionary processes, and language, remarkable though it may be, is only one such development; therefore, language must have evolved out of prior \isi{mammalian communication systems}. On the other hand, if one has anything like a complete understanding of what language is and does, one realizes that there is not simply a quantitative, but a qualitative and indeed unbridgeable, gulf between
% ORIGINS 217
the abstractions and complexities of language and the most abstract and complex of known mammalian systems (which, indeed, seem pretty direct and simple); therefore, language cannot have evolved out of prior \isi{mammalian communication systems}. Thus, there must have been evolutionary continuity in the development of language, yet there cannot have been evolutionary continuity in the development of language. This is the Paradox of Continuity, and debate on it has followed the approved political model of both sides hurling slogans at one another from their different and, as we shall see, mutually irrele\-vant positions.

All paradoxes that are resolvable are resolved by showing that one or more of the presuppositions on which they are based is incorrect or at best misleadingly stated. In the present case, both sides of the paradox have weak legs. The weak leg of the discontinuity side is a belief in the unitary nature of language; the weak leg of the continuity side is the belief that if language evolved out of anything, it must have evolved out of another communication system. Let us examine each of these in turn.

The belief that language is one and indissoluble has also taken its toll on the primate-experiment debate. In both areas, the central point of debate has been ``When can X be said to have language?{\textquotedbl}\- ``language\isi{'}\isi{'} being defined, by the discontinuity side, as something virtually indistinguishable from Modern \isi{English} or Ancient (ancient!) Greek. Although the continuity side may have protested the definition, few have protested the gambit; instead of pointing out that the ques\-tion ``Has X got language or hasn't he?\isi{'}\isi{'} is an intrinsically stupid, irrelevant, and actively misleading question, they have mostly con\-tented themselves with trying to get linguists to lower the target (for a generally bracing and insightful, if overly soft on the continuity side, account of the ape debate, see Linden 1974 ).

In fact, we will get nowhere until we appreciate that anything as complex as language cannot possibly be an internally undifferenti\-ated object, but rather must consist of a number of interacting systems,
%\originalpage{218}
some of which may originally have developed for other purposes and many, perhaps all, of which must have developed at different times and under different circumstances. Once we accept this, we can per\-ceive the development of language as a succession of stages and there\-fore amenable to reconstruction and study, rather than as a quantum leap, which then imposes on us, whether we will or no, some kirid of catastrophe theory as the only possible origins story. It then becomes possible to replace ``Has X got language or hasn't he?\isi{'}\isi{'} with the more interesting, and more answerable question, ``How far has X come along the road to language-specifically, which of the necessary pre\-requisites does he have, and which does he still lack?{\textquotedbl}

In the opposite camp, the belief that language must have evolved from some prior communicative system if it evolved at all is clearly connected with the belief that language is only, or originally, or pri\-marily, a communicative system. Any doubt cast on this is enough to send the continuity side into a flurry of pooh-poohing. Typical is the attitude of \citet[175]{Young1978}, who finds it ``a further \textit{problem\isi{'}\isi{'} }that ``a major use of language in each of us is internal-for thinking, that is \textit{for} \textit{speaking} \textit{to} \textit{ourselves,{\textquotedbl}} but nevertheless concludes that it is \textit{{\textquotedbl}rather} \textit{perverse }not to consider human spoken or written language as primarily a functional system evolved for communication\isi{'}\isi{'} (emphasis added). Perhaps a biologist may be forgiven for not realizing that language is not just ``for communication\isi{'}\isi{'} but is also that which is com\-municated. But in fact the belief is widespread that all language in\-volved was giving labels to things and stringing the labels together. It is assumed as self-evident that when we were ready to talk, all the things in the universe stood there waiting-rock and river, dodo and elephant, storm and sunrise, thirst and evil, love and dishonor-all waiting patiently for their labels. That the world had to be recreated in the image of language before anyone could communicate about any\-thing at all is an idea that seems simply not to have occurred on the continuity side. How that recreation was carried out will form an essential part of the analysis that follows.

Crucial to extant continuity models, even the most recent, is
% ORIGINS 219
the belief, whether implicit or explicit, that you could get from a call system to modern language (with or without an intermediate stop at a gestural system) by a series of imperceptible stages. Thus, \citet{Stephenson1979} proposes a ``dialectical\isi{'}\isi{'} evolutionary process by which, when our ancestors were more preyed upon than preying, they learned to control involuntary vocalizations and replace them with manual signs (how a four-foot hominid in five-foot grass informs his cohorts of the imminent approach of a sabertooth by gesture is left unclear); then when they got to be better predators and were less concerned with unobtrusiveness, they were able to return to voice, which was now under cortical control. ``The dialectic consists in \textit{an} \textit{increase} \textit{in} \textit{the} \textit{level} \textit{o}\textit{f} \textit{complexity} \textit{of} \textit{messages} coincident with a decrease in the limbic content of messages as one proceeds through calls, through gesture, to spoken language and into written language\isi{'}\isi{'} (emphasis added). Steklis and \citet{Raleigh1979} dismiss the gestural phase on principles of parsimony, but maintain call-language continuity by accepting the claim by Hockett and \citet{Ascher1964} that progressive blending and differentiation of primate calls could have mediated the transition. \textsuperscript{3}

All these views share the assumption that the only significant difference between \isi{call systems} and language lies in ``an increase in the level of complexity of messages.\isi{'}\isi{'} In fact, complexity is not the issue. A given alarm call could well receive the reading, ``Look out, you guys, a large predator is already near and rapidly approaching, so get up the nearest tree as quick as you can,\isi{'}\isi{'} which is surely at least as complex as ``the boy ran\isi{'}\isi{'} or ``John kicked Bill.\isi{'}\isi{'} Language depends crucially not on complexification but on the power to abstract, \textit{as} \textit{units,} classes of objects, classes of actions, classes of events, and classes
of yet more abstract kinds (think, for example, for a moment of all the different kinds of relationships that can be conveyed by so simple a predication as X \textit{has} \textit{Y).} It is these classes, not the particular objects, actions, etc., of which they are composed, that constitute the units that language must represent; but in order to represent them it must first abstract them from the constant sensory bombardment to which all creatures are subject (we will see how in a moment). An alarm call
%\originalpage{220}
abstracts nothing from that bombardment, but merely selects from it a set of stimuli (smells, colors, physical movement, etc.) to which some kind of immediate reaction is the only appropriate response. A call and a sentence may both constitute communication, but in the ways in which they work they are more at odds than chalk and cheese; for some chalks and some cheeses at least have the same color and texture, whereas language and \isi{call systems} do not even have this superficial resemblance.

However, once we are prepared to consider the possibility that language could have developed in a regular evolutionary fashion with\-out having sprung from some primitive repertoire of grunts, groans, and grimaces, all the objections to a continuity approach melt like snow in August. Once we have gotten over the ``communicative\isi{'}\isi{'} hang-up, we can see that where we must look for the distinctiveness of human language is not in what it shares with \isi{call systems}-both communicate\-but in how it differs from \isi{call systems}-language communicates con\-cepts, \isi{call systems} communicate stimuli. If we don't understand con\-ceptualization, we don't understand language, period.

However, if we are to write an evolutionary history of con\-ceptualization, there is one more ghost to be exorcised-the ghost of Descartes. This particular specter is still haunting the behavioral sci\-ences . even though the naturalistic observations on which its man\-animal dichotomy was based are now over three hundred years out of date. If you believe, as Descartes believed, that animal behavior can be explained by principles as simple as (and similar to) those hydraulic forces which activated the ``living statuary\isi{'}\isi{'} of 17th-century \ili{French} gardens, then it does not seem so unreasonable to suppose that animals are automata but that we (with souls stashed in our pineal glands) are not. In light of all that has been learned about both the structure of the nervous system and the behavior of species since Descartes\isi{'} day, it is merely absurd-possible to salvage only with the logical, if counter\-factual, strategy of the hardcore behaviorist who would claim that animals are automata but that so too are we.

% {\textbackslash}
%\originalpage{221}

There are four possible answers to the question, ``Who has con\-sciousness, volition, etc.?\isi{'}\isi{'} (all the so-called ``nonphysical\isi{'}\isi{'} attributes summed up under the illegitimate label \textit{mind} \textit{):} animals do, but we don't; animals don't, but we do; animals don't, and we don't ; animals do, and we do. In all the history of human folly, I know of no one who has seriously asserted the first. The second is Descartes\isi{'} answer. The third is the hardcore behaviorist's answer. The fourth, curiously enough, has seldom been made and has been scorned almost as often as it has been made, although in light of what we know now it would seem the most logical. Since it now appears that \isi{evolution} has ad\-vanced not by leaps and bounds but by infinitesimal gradations, we either have to claim that with respect to a particular set of attributes (volition, consciousness, thought, language), \isi{evolution} behaved in quite a different-and, incidentally, completely mysterious-way from that in which it behaved with regard to all other attributes, or we have to accept that at least some of, or some ingredients critical to, these attributes were and presumably still are shared by species other than our own. I know of no logical argument against the second move although the emotional arguments against it seem as numerous as they are strong. Let us therefore see how \isi{conceptualization}- without which language would have been impossible- could have evolved.

Conceptualization is intimately linked to perception, if only in the sense that if there were no perception, \isi{conceptualization} could not take place. But there is, I think, a great deal of difference between a concept and a percept, which tends to be obscured by loose ways of talking and thinking. We use ``concept\isi{'}\isi{'} for any kind of mental image. In fact, there are mental images of \isi{percepts} and of concepts. We might say, loosely, that I have a concept of the glass that is presently standing on my table, meaning, I can close my eyes and present myself with a mental image of my glass. That is a mental image of a percept, i.e., my glass as I perceived it now-empty, but for a small slice of lemon. Of course, I could imagine it completely empty-which is a percept of how it was at another \isi{time}; or full-which is the same. However, I can also have a mental image of the category \textit{glass,} which embraces
%\originalpage{222}
my glass and all other glasses, and which is not a percept, but a true concept.

In some species, percept and concept may not be so far apart.
Consider the frog. The frog can discriminate ``fly\isi{'}\isi{'} and ``not fly,\isi{'}\isi{'} at least as long as the fly is moving. It is unlikely that a frog can tell one fly from another or would preserve \isi{memory} images of individual flies, even if it had a \isi{memory} to preserve them in. In a sense, percep\-tion and \isi{conceptualization} in the frog are one. Only in a sense, of course; for true \isi{conceptualization}, you have to have volitional con\-trol of concepts, and the frog is as far from that as from flying. But in the sense that perception in the frog is generalized, it is like con\-ceptualization.

Now, the fibers which connect a frog's retina with its brain are capable of passing only about four kinds of information, of which only two are relevant to the perception of flies. The first of these two kinds is supplied by a set of neurons specialized to detect small moving objects with curving edges; the second is another set specialized to detect sharp boundaries of light and shade \citep{Burton1970}. There is then a rather tenuous and metaphorical sense in which we could say that the froggy concept of ``fly\isi{'}\isi{'} IS the firing of these two sets of neurons. Of course, we are several scores of millions of years away from true \isi{conceptualization}; but the journey has certainly begun.

In the course of those years, it might have seemed that per\-ception and \isi{conceptualization} were moving apart, as perception became not only wider in range (most of the environment seems quite un\-differentiated to the frog) but more particularized, with so many parameters recoverable that maybe even individuals could be recogniz\-able to one sense or another, or a combination of several (when our dog recognizes us, smell is presumably dominant).\textsuperscript{4} And yet the basic mechanics by means of which this enhanced perception was carried out were in fact no different from those of the frog. There might be many more sets of neurons programmed to respond to many more varied \isi{types of} stimuli, but a percept would be still the particular
firing pattern of the particular set of neurons activated by that set of
% {\textbackslash}
stimuli which constituted the object perceived.
% ORIGINS 223

The problem of how a percept becomes a \isi{memory} is still far from solved. Part of the problem may be that most studies. of \isi{memory} have really been studies of \isi{learning}-that is, of forced situations in which given factors caused changes of behavior. Thus, if an octopus were trained to attack horizontal but not vertical rectangles, two sets of feature detectors, each of which could formerly initiate more than one program of action-advancement, withdrawal, indifference\-can now only initiate one each, with corresponding changes in the neural connections involved (Bradley and Young 1975). Unfortunately, such findings do not seem to generalize to mammals, where ``the search for the engram\isi{'}\isi{'} remains as fruitless as it was thirty years ago \citep{Lashley1950}. Furthermore, it would be unreasonable to expect them to generalize to the quite qualitatively different kinds of \isi{memory} traces which concern us here. For the kinds of \isi{memory} traces that modify behavior- those traditionally studied by psychologists-may be (although of course they are not necessarily ) laid down in ways quite different from those of memories which may only modify beha\-vior in the most indirect of ways, if indeed at all (e.g., my stored \isi{memory} image of my neighbor's new car, accurate enough to enable me to distinguish it from others, but unlikely to prompt me to steal it, polish it, avoid it, etc., and hardly to be described as having been \textit{learned} by me except under the vaguest and most vacuous reading of \textit{learning} \textit{).}

Therefore, I shall assume that long-term storage is achieved (precisely how need not concern us) by storing features of images rather than images themselves. Let us assume I can reliably identify several hundred human faces. Now, the mental representations of these faces that I need for matching purposes-I can think of no other way in which recognition could be carried out-are not stored separately in some analogue of a box in my head, not even in the form of macro\-molecules. Rather, each of the horizontal, vertical, slanted, curving, etc., lines that go to make up faces-as well as lots of other things, of course-is represented by a particular set of neurons. The superset
%\originalpage{224}
composed by these sets is simply an analogue of the superset of feature\-detecting neuron sets. The data recorded by the straight-vertical-line perceiving set of neurons (or however much of them are transferable) are simply transferred to the long-term storage set for straight vertical lines, and so on. The fact that a particular batch of data went into a particular batch of storage sets must also somehow be recorded, in terms of sensitized synaptic pathways or whatever, or l could never recover Aunt Emma's face from its component bits. But in some such general manner-and Iapologize to neurologists for my rather Rube Goldberg picture-the processes of perception, storing, coding, and accessing must be carried out.

It follows that individual images would not be individually stored-members of the same class of images would not necessarily be stored together, while the storage of unlike objects might be strikingly similar. Let us consider the possible storage of the percept images of three objects, any one of which would have to be separately and individually recoverable: \textit{Aunt} \textit{Emma's} \textit{latest} \textit{hat,} \textit{the} \textit{S} \textit{ugarloaf} \textit{at} \textit{Rio} \textit{de} \textit{]} \textit{aneiro, the} \textit{distribution} \textit{curve} \textit{for} \textit{IQ} \textit{in} \textit{an} \textit{average} \textit{population.} These objects belong, respectively , to three quite distinct classes; the class of hats, the class of mountains, and the class of distribution curves. However, in their general shape they share some obvious simi\-larities. Let A through G represent sets of storage neurons, each set representing storage of a particular parameter. Then \textit{Aunt} \textit{Emma's latest} \textit{hat} migh t be represented by sets ABCDE, \textit{the} \textit{Sugarloaf} \textit{at} \textit{Rio} \textit{de} \textit{Janeiro} by sets BCDEFG, and \textit{the} \textit{distribution} \textit{curve} \textit{for} \textit{IQ} \textit{in} \textit{an} \textit{average} \textit{population} by sets CDEF. If I wish to visualize an image of any one of the three, l activate just these sets,

(And what constitutes the ``I\isi{'}\isi{'} that activates? Analogy from observed conspecifics, use of mirrors and other reflecting substances, plus the higher-order ``traffic-control\isi{'}\isi{'} neurons which must exist to establish priorities in brain activity if the whole thing isn't to degen\-erate into electrochemical chaos.)

Some kind of \isi{memory} storage of particular experiences must go pretty far down the mammal{\textbackslash}(m phylum. So too, I suggest, must
%\originalpage{225}
the power of playback-voluntary recall of images or sequences of images. At the very least, involuntary playback (another name for dreaming) does. Reptiles don't dreani, mammals do. Moreover, dream\-ing (human dreaming, for sure; mammalian dreaming, very likely) consists not of just straight playback but of the recombination of stored imagery, something that would be difficult or impossible if memories were individually stored. Once playback, straight or crooked, came under cortical control, our ancestors were well on their way to the world map that is a prerequisite for language-without which there is hardly anything worth communicating to communicate.

The question evolutionists will ask at this point is: why ? Why should mammals develop these capacities? Prehistory was not a dress rehearsal for \textit{homo} \textit{sapiens.} What selective advantage did the species gain ? Some psychologists have attempted answers in very vague and general terms. For instance, \citet{Harlow1958}, discussing the fact that some apes and monkeys can solve in captivity problems far more complex than they would ever meet in nature, pins his faith on receptor system development, since more finely calibrated receptor systems entail an increase in the central nervous system: ``As long as increas\-ingly complex receptor systems provide the organism with slight survival advantages, one can be assured that increasingly complex nervous systems will develop; and as long as increasingly complex
nervous systems develop, the organism will be endowed with greater
potentialities which lead inevitably to \isi{learning}.\isi{'}\isi{'} Similarly, \citet{Passingham1979}, who finds it ``puzzling\isi{'}\isi{'} that chimpanzees should have language capacities which ``do not appear to be used in the wild,\isi{'}\isi{'} surmises either that ``chimpanzees do in fact use their language capacities in the wild{\textquotedbl}\-in ways which two decades of patient and trained observation have somehow still failed to reveal!{}-or that ``their abilities . .. must be general ones, allowing them to do other things of importance to them in the wild.{\textquotedbl}

The vagueness and timidity of these suggestions are, I am sure, due to the Cartesian hangover, although the class ``Cartesian \isi{evolution}\-ist\isi{'}·\isi{'}· ought to constitute a logical contradiction. It should be pretty
%\originalpage{226}
obvious that the power to review the past would provide its possessor with another power of the highest value in natural selection: the power to predict. Psychics aside, \isi{prediction} is based on analysis of past events, and becomes of greater importance as creatures evolve and become more complex. Relatively simple creatures lead relatively simple lives; it is possible to program them, up to around the frog level, so that they will respond automatically to all or almost all the contingencies they are likely to encounter. With more complex crea\-tures, in particular with predators who have to keep (literally!) one jump ahead of their prey, not only does the list of conceivable contin\-gencies get too long to program, it would probably be dysfunctional for such creatures to be programmed down to the wire, so to speak. Such programming would leave them unable to respond appropriately, to vary the moment of attack in accordance with the wind, the light, the prey's motions, and countless other environmental factors, to determine which member of a herd to attack, and so on. The power to review past sequences of events, whether at a conscious or an un\-conscious level, and to abstract those factors which made for success or failure in particular cases, would confer a massive advantage on its possessor- or one that would have been massive had the prey not de\-veloped along similar lines. As shown in the excellent survey by Jerison ( 1973), \isi{brain size} for both prey and predator has gradually but con\-tinuously increased throughout the mammalian era, with the predators always slightly ahead of the prey.

We do not of course know, and have as yet no way of determin\-ing, how far down the evolutionary scale such capacities might extend. But such capacities and more might have been needed to ensure the survival of the primates, creatures who were predators to some species and prey to others; and it is a reasonable assumption that \textit{\isi{Dryopithecus},} the presumed common ancestor of ourselves and the chimpanzees, who lived between five and fifteen million years ago, had them and probably had more.

We have so far surveyed the capacity to form and store \isi{percepts}
and to review \isi{percepts} and sequences of \isi{percepts} under voluntary
%\originalpage{227}
control. But we have not yet considered how \isi{percepts} can become concepts. Until a percept- the image of a particular entity on a par\-ticular occasion-can be replaced at will by a concept-the image of a class of entities, divorced from all particular instantiations of that class-then the power to predict is limited. A creature concerned with \isi{prediction} does not want to have to say, ``The boar I wounded three years ago did such and such, and the boar I wounded a year ago did the same, but this wounded boar is a different boar so I suppose I'll just have to wait and see.\isi{'}\isi{'} It wants to be able to say, ``Wounded boars do such and such, so I can anticipate what will happen and be ready to act appropriately.\isi{'}\isi{'} The \isi{prediction} may be quite wrong, of course-with fatal results. But if it is right just that little bit more often than chance, the survival chances of the species are perceptibly enhanced.

Indeed, instantaneity would have rated above accuracy. The creature that could achieve 60 percent accuracy in one second would surely have outlasted the creature that could achieve 100 percent accuracy in ten seconds-because in those ten seconds, too many of the latter would have gotten themselves killed.

``To generalize is to be an idiot\isi{'}\isi{'} \citep{Blake1808},s but for better or worse, the road to humanity was paved with generalizations. We assume our ancestors to have been primates not adapted for predation but driven by climatic change to adopt, in part, the habits of preda\-tors-or so most anthropologists have held for a good many years. Lacking the tiger's fang, the leopard's speed, the disciplined pack strategies of the canids, they had to compete with these and more in a dry epoch when pickings were scarce. Under such circumstances they would have selected very fast and very naturally for some kind of instant recognition-and-reaction device-one that would respond not merely to the briefest of glimpses of possible prey or rival predator, but to the most minimal clues in the environment: movement of a branch at a particular altitude, say, coupled with the appearance of a patch of brown slightly different in shade and texture from the fall leaves that surround it.

%\originalpage{22}

That modern hunters still have such a device, even though changing times have made it more dysfunctional, was illustrated a few years ago when a national magazine carried out an inquiry into shooting accidents during the \isi{deer} season. Hunters were shooting one another instead of the \isi{deer}. The vast majority of these incidents oc\-curred in the half-light of dawn or dusk. The shooters, when inter- · viewed, almost invariably said that they had seen a \isi{deer}-not ``thought{\textquotedbl}{\textbackslash} they had seen one, but actually seen it-and only seconds after they. had pulled the trigger did this image resolve itself into a wounded
dying fellow-hunter. From a few half-perceived dues of color, shape/ . and texture, they had created phantom \isi{deer} and reacted to their ...
own creation before additional sensory input could replace the pro\textsubscript{0}\textsubscript{ };;
jected image wirh a real one. E·
The reader may easily demonstrate a similar if less lethal effect. \isi{'}.\isi{'}
Simply draw, on a piece of paper or a blackboard, the structure por\-trayed in \figref{fig:4}.1below :

  


 
\begin{figure}
Figure \textbf{4.1}
\end{figure}

The \textbf{minimal} ``flower{\textquotedbl}

%\originalpage{229}

you then ask what this is, people will reply, nine times out of ten, flower.n Ihave tried this many times on students; you can often see their jaws literally drop when you tell them, ``No, it's a dot, nine short

•. ··••·· Jnres, and. one long one.{\textquotedbl}

capacity to construct predictive images obviously entails the

Pfexistence of class concepts. The misguided hunters did not project

an· ag of some specific \isi{deer}, but rather that of any member of the

ignus \isi{deer}. To us, with the elaborate and labeled cognitive map which

{\textbackslash}•;li!llguage provides for us, the genus \isi{deer} seems self-evident. But try to

:ffi:illlllgin1;:, if you can, the task of constructing the category \textit{deer} from

:· ni:tch, by inductive reasoning, without benefit of labels or of map.

{\textless};J) t {\textless}:ome in a number of shapes, sizes, and species. Some are dark,

illt{\textgreater}i'rie• are light. Some have horns, some don't, some sometimes do,

.•,fud. sol11e sometimes don't. Where do \isi{deer} stop and other genera

•.'liegiri? The problems are endless.

,\textit{.}\textit{:}\textit{?}\textit{·•}•\textit{· }.Yet as work by \citet{Berlin1972} and his associates has shown, gneric rt:imes are the most richly represented in natural languages\-

.{\textless}.iliore; .numerous than both the higher-order categories ({\textquotedbl}unique begin-

·\isi{'};p. r'i,\isi{'}{\textgreater} e.g., \textit{plant,} \textit{animal,} or ``life forms,\isi{'}\isi{'} e.g., \textit{tree, bus}\textit{h}\textit{)} or the lower\-4rde;r.ones ({\textquotedbl}specHlc name,\isi{'}\isi{'} e.g., \textit{Ponderosa} \textit{pine,} \textit{jack pine, }or ``vari\-r \isi{'}.'n:ime/\isi{'} e.g., \textit{northern} \textit{Ponderosa} \textit{pine, }\textit{UJ}\textit{estem} \textit{Ponderosa pine} \textit{);}

.;. v(!tiably monomorphemic (contrasted with specific or varietal names);

\begin{itemize}
\item je;ctiyely perceived as primary; the first to be learned by children. U.tr{\textquotedbl}a/gopd bet they were among the first words of human language.
\end{itemize}

.. {\textless} Why the genus and not the species? If your eyes are as sharp as

t:slii;e\isi{'}.out ancestors\isi{'} were, differences between species must often

. . e{\textless} been as salient as, or more salient than, differences between

·ghta{\textbackslash} But behavioral differences would have had greater significance

;for{\textless}..our ancestors than visual differences. All \isi{deer}, whether large or c;s)11all.{\textgreater} plainor spotted, with or without horns, had a number of things jii•.fo)l.1!11611: they were fleet of foot; nervous enough to make stalking

\begin{itemize}
\item .. •t3,ifftcUlt but not impossible; often camouflaged by the light-and\-
\end{itemize}

\}effects of foliage; excellent eating if you could get them, and so

{\textbackslash}

%\originalpage{230}

But the fact that classification of the genera would have been selectively advantageous for our ancestors does not in and of itself make such classification possible. A number of preadaptations would also have had to occur, perhaps the most obvious of which is cross\-modal association-the importance of which for language has been stressed in a number of papers by \citet{Geschwind1974}. For the concepts of genera could not have been built on sight alone; each of the senses must have contributed in varying degrees. But the real key to the crucial developments must lie in the nature of the recognition device.

I have said that members of our species, and even dogs, if their behavior is anything to go by, must be able to distinguish individuals, and there is no reason to suppose that our capacity to summon up at will the visual images of individuals necessarily indicates any very recent evolutionary development. I suggested also that these images might be stored in a fractured manner, so that similar sets of bits, when put together in various ways, could constitute very distinct images. But why in that case do we not project genera that would include in the same category, say, \textit{Aunt} \textit{Emma's} \textit{latest} \textit{hat,} \textit{the} \textit{Sugar\-} \textit{loaf} \textit{at} \textit{Rio} \textit{de} \textit{Janeiro,} and \textit{the} \textit{distribution} \textit{curve} \textit{for} \textit{IQ} \textit{in} \textit{an} \textit{average} \textit{population?}

The utility, or lack of it, that such a category might have is
beside the point. If we built category concepts on the basis of in\-dividual \isi{percepts}, such would be the kinds of categories we would most likely wind up with. One of the main reasons that we do not is that concepts, as distinct from \isi{percepts}, do not exist in isolation. We do not delimit \isi{percepts} in terms of \isi{percepts}. I do not distinguish Aunt Emma's face because it is bounded on one side by Aunt Mary's face, on another by Cousin Emily's face, and so on; nor (just to show that this fact has nothing to do with any heightened perception of
conspecificsJ do I distinguish my toothbrush because it is bounded
on one side by my wife's toothbrush, on another by my son's tooth\-brush, etc. But I do distinguish \isi{deer} because they are bounded by horses on one side, cattle on another side, and so on; I do distinguish toothbrushes in general because ey are bounded by hairbrushes,
%\originalpage{31}
nailbrushes, bootbrushes, etc. Deer stop where horses begin. Tooth\-brushes stop where nailbrushes begin. But Aunt Emma's face does not stop where Cousin Emily's begins, my toothbrush does hot stop where my wife's begins. That is the difference between percept and concept, class member and class. Concepts are delimited in terms of one another, \isi{percepts} only in terms of themselves.

Concepts are like the counties on a state map, in several ways. Where one stops, another starts. Although each is composed of so many acres and contains so many individuals, none is merely the sum of the acres that compose it or the people who inhabit those acres. Each of them has its place with respect to the others. The same with con\-cepts. Percepts inhabit concepts, ,but concepts are not the sum of their \isi{percepts}. \textit{Aunt Emma's latest hat,} a percept, belongs to the concept \textit{hat} and not the concept \textit{mountain;} \textit{the} \textit{Sugarloaf,} a percept, belongs to the concept \textit{mountain }and not the concept \textit{distribution} \textit{curve.} It is not because of its individual characteristics that Ido not expect to fmd the Sugarloaf on Aunt Bmma's head ; it is because I know that the Sugarloaf is a mountain and mountains do not belong on people's heads. When Isee the Sugarloaf for the first \isi{time},I'do not have to work out from scratch all the properties it has, including that of not being on Aunt Emma's head; all those properties follow auto\-matically once I have determined that it is a mountain. Mountains have their place on the map, and so do hats; and those places are different.

But how, on first seeing its picture, did I recognize that it was a mountain (albeit a rather small one) at a distance, rather than a large if rather eccentric hat, up dose? Not by computing its peculiar proper\-ties. If I had, I could have gone wrong, because in outline it is less like the archetypical mountain than it is like some hats. I did so by putting two things together: the fact that it fulfilled at least some of the qualifications for being a mountain together with its relations to other objects: a harbor, clouds, a cable-car line to its summit. Of course, it could still have been a hat in a Lilliputian model city; just as the \isi{deer} the hunter shot at could have been (and in fact was) a
%\originalpage{2}
fellow-hunter. The fact that our maps may occasionally let us down does not mean we could get along without them. We would be no\-where without them.

If \isi{percepts} could be filed fractured, so to speak, concepts must
be filed as gestalts. I do not have the slightest idea how this is done, nor to the best of my knowledge has anyone else, although neurologists will quite likely find out in the next century or two. However, since speculation is usef ul if only to provide candidates for elimination, let us speculate. Having been stored one way as \isi{percepts}, in a manner based on their immediate sensory images, phenomena would be copied and stored another way in accordance with their observed behavioral properties. Did they lie still or move? If they moved, did they soar, lope, or slither? These heterogeneous bundles of information would again, presumably, be stored in sets of neurons synaptically linked with, and as specialized in function as, those sets of perceptual neurons that distinguish movement from nonmovement, loping from slithering, and the smells characteristic of one set of phenomena from the smells characteristic of another set. However, instead of the same set of neurons representing similar aspects of the images of quite different things, as we supposed was the case in the storage of \isi{percepts}, separate
sets of sets, involving heavy duplication of function, would be required
%\originalpage{233}
a patch of dappled light, motionless, at a certain height above ground in a forest at dawn{}-the superset from which that subset is drawn would immediately be activat.ed in such a way as to yield the full concept\-\textit{\isi{deer},} in this case. The subject has then ``seen\isi{'}\isi{'} a \isi{deer}, or whatever, and reacts accordingly.

This account is, again, necessarily crude, necessarily vague\-what constitutes a ``sufficient subset of features\isi{'}\isi{'} for concept trig\-gering, for instance ?-and quite possibly wrong in most or all of its particulars. It is crude, vague, and possibly wrong because, as Blake\-more (1977) observed, studies of \isi{memory} (and allied human capacities) have been ``concerned with the \textit{machinery} . . . not the \textit{code-the }sym\-bolic form in which the events are registered\isi{'}\isi{'} (original emphasis); by ``machinery,\isi{'}\isi{'} Blakemore means ``the manner in which events can cause changes in physical structures,\isi{'}\isi{'} but the kinds of changes likely to be caused by the mental processes we are considering, consisting as they would consist of no more than the forging of additional links between specialized neurons, would hardly be amenable to observation in the state of today's technology (ten billion neurons with sixty thousand connections each-where do you start to look?). It is crude, vague, and possibly wrong because neurologists have considered the more ``meta\-physical\isi{'}\isi{'} implications of their task as somebody else's business, be-
for each. network of neurons that represented a concept.

cause
\textsuperscript{{\textquotedbl}}contm\textsuperscript{.} m\textsuperscript{.}sts\textsuperscript{,,} o\textsuperscript{f}

t\textsuperscript{h}e grunt-groan-or-gesture sc\textsuperscript{h}oo\textsuperscript{l} \textsuperscript{h}ave thought

If this were so, it would explain why, while \isi{percepts} may be stored in literally infinite quantity, the list of concepts, or at least the list of \isi{primitive concepts}, is certainly finite and probably quite short (of course, an infinite number of secondary \textit{concept} \textit{s-timber} \textit{wolf,} \textit{prairie} \textit{dog,} etc.-can be constructed by combining two or more \isi{primitive concepts}). It would also explain why the human recognition device works as it does. Each superset of concept-representing neurons would include representations of all features of a concept. If indeed the major genera constituted the first concepts (and there would seem to be no likelier candidates), this would mean in effect all features of a genus-sensory , behavioral, distributional, whatever. Then whenever any sufficient subset of features is dtected by the perception neurons-
that the nature of reality is self-evident and therefore didn't need to be constructed, and because philosophers, who alone could be expected to perceive the problems inherent in perceiving anything at all, have resolutely refused to tie their ballooning speculations down to the nuts\-and-bolts of what we already know about what we have in our heads and what we might be expected to be able to do with it. So the whole area slipped between the cracks of the disciplines. But that area still exists, and is crucial in the explanation of human capacities, so a bad map is better than no map at all.

These last few pages may seem to have taken us a long way from language, but I do not think that is the case. Unless we have some
%\originalpage{234}
notion of all that must have been involved in moving from moment-by\-moment perceptions to class concepts-and it is class concepts that are named, not perceptions, \isi{percepts}, or the extramental stimuli for these\-then we simply do not know what it entailed for a species to get language. Moreover, until we appreciate just how difficult it must have been to name the major genera-the flora and fauna, successful inter\-action with which was our ancestors\isi{'} very lifeline-we shall not even begin to conceive the role which is played by the perceiving mechan\-ism, rather than the perceived data, as progressively more abstract phenomena are involved. We will come to that in a moment.

But before we do, we should note that if the foregoing account
is correct even in its broadest outlines, we have already suggested an infrastructural motivation for one of the major semantic distinctions observed in the preceding chapters (the SNSD). We saw that both creole speakers and children were able to distinguish with great ease between specific and nonspecific (generic) reference. Now, if \isi{percepts}-images of particular entities on particular occasions, therefore specific-and concepts-images of classes of entities, therefore nonspecific- are stored in different places and in different ways, this distinction would be built into the neural system. In consequence, something which seems highly
%\originalpage{235}
impressive than wh.at they were trained to do. The fact that they applied .names to different-looking objects of the same class, as well as to pictures of such objects, and their frequent generalizations of names to broader classes, shows that to them, names were class names\-on\_cpt label\_s-and not mechanical responses linked to particular, md1viual objects after the fashion of proper names. The fact that they mvented names for classes of objects whose names they had not been taught-for refrigerators \textit{(open-eat-d} \textit{rink),} for ducks \textit{(water-bird} \textit{)} for \isi{Brazil} nuts \textit{(roc}\textit{k}\textit{{}-berr}\textit{y}\textit{), }for oranges \textit{(orange-apple):} Sarah kne only ``apple\isi{'}\isi{'} as a fruit descriptor but had orange separately as a co\-lor)-shows tht they \_had far more concepts floating around in their heads than their caregivers had the \isi{time} or patience to name for them and also showed power of creativity on a lexical (nonsyntactic) level\isi{'} a fact we will return to in due course.\textsuperscript{6 }\isi{'}

With reg?t:he . e?endence of this power from anything you could call trammg, 1t 1s worth citing a passage from Mounin ( 976):. ``Sarah, all alone in her cage (outside any experimental situa\-t10n) picked up objects or signs and composed utterances on the

\textsuperscript{mod}\textsuperscript{e}.\textsuperscript{l}\textsuperscript{s}. \textsuperscript{of} \textsuperscript{the} \textsuperscript{str}\textsuperscript{u}.\textsuperscript{ctures} \textsuperscript{she} \textsuperscript{had} \textsuperscript{J}\textsuperscript{'ust} \textsuperscript{learned} . . . . \textsuperscript{c}an one \textsuperscript{d}1\textsuperscript{'} scern t\textsuperscript{h}e

tranltlo of the main and primary function of her code, social com\-mumcat10n, to a secondary use of it, the possibility of developing

abstract and far beyond the powers of two-year-olds, if we suppose it

to be acquired in the traditional fashion, would apply automatically

\textit{provided that no alternative but incompatible distinctions }(such as

fo\textsubscript{h}\textsubscript{{\textquotedbl}}r oneself \textsubscript{·}the expression of one's own view of the world?\textsubscript{. }t is expression \textit{only} represent \isi{play}?\isi{'}\isi{'} (emphasis added).

This passage r.om P{\textless}{\textgreater} vPry cl0e tn blin din R

\textsuperscript{o}\textit{r} \textsuperscript{d}oes

those i1wolved ln \ili{Japanese} cas.:: and topk marking, for cXa.i!.iplb) wet

simultaneously ben1g iu1po»eJ 011 the .::hil-1 by he lans:iagc:. Indeed, I shall later suggest that it was just those category distinctions based on sharply differing modes of cerebral coding and storage which were the first to be grammaticized, and which were thus to serve as

manages\_ to get things the wrong way around . Mounin is right; 0£ coure, n that the Premacks. taught Sarah her code for strictly com\-mumcatlve purposes, so that 1f she turned it to private, computational

?urposes, that use would be secondary in a rather narrow sense. But

in a much broader, evolutionary sense, things were the other way

a kind of scaffolding by which language was able to rise from an initial

a\textsubscript{£}ro\textsubscript{1}\textsubscript{1}und. Possession of an elaborated world\textsubscript{{}-}view mu\textsubscript{s}t

prece\textsuperscript{d}e, not

low plateau of short and relatively structureless utterances.

Now, from recent experimentation, we know that the power to abstract concepts from nature is something that we share with the great apes. As \citet{Mounin1976}, among others, has pointed out, the evidence of what chimps did voluntarily, after training, is much more

\begin{itemize}
\item o ow, communication on even the lowest of linguistic levels. Sarah and Sarah's species must already have had an interior world of con\-cepts, not of \isi{percepts}, or they would have been unable to transfer names from one object to another, still less from one class to another.
\end{itemize}

%\originalpage{236}

Moreover, a name like Washoe's \textit{rock-berry} (for \isi{Brazil} nut) is a meta\-phor at a level appropriate for barroom joking, if not poetry ; it shows awareness of a superclass of which both berries and nuts are members, and the sharing of an abstract quality-hardness, not normally asso\-ciated with that superclass-by nuts and a member of another, non\-vegetable superclass. The coiner of such expressions has a cognitive map of no mean quality.

A further telling, if oblique, bit of evidence for this claim comes
from Gill and \citet{Rumbaugh1974}, who report that it took Lana 1,600 trials to learn the names for \textit{banana} and \textit{M\& M} , but that the next five items were acquired in less than five trials each-two of them in two only. This stunning and instantaneous increment is inexplicable in terms of Lana's having ``learned how to learn\isi{'}\isi{'} in the course of those 1,600 trials; \isi{learning} curves just don't jump like that. It is much more plausible to suppose that for a long \isi{time} Lana simply couldn't figure out what her trainers were trying to do, and then suddenly it clicked: ``My God, they're feeding me concept names-why couldn't they have \textit{told} me, the dummies?\isi{'}\isi{'} The concepts had been there all the while, and only the link between them and these mysterious new things that people were doing to her needed to be forged.

Finally, Mounin's use of \textit{or} and \textit{only} is a striking example of anthropocentric, or perhaps I should say Puritan business ethic, modes of thinking. On the one hand, you have ``developing for oneself the expression of one's own view of the world,\isi{'}\isi{'} an activity automatically assumed to be solemn, to be heavy, to be \textit{work} , in fact; therefore, on the other hand, something that is \textit{play} couldn't possibly be ``de\-veloping for oneself . . . ,\isi{'}\isi{'} etc. \citet{Piaget1962} came nearer the bone when he claimed that the symbolic function arises first in \isi{play}, and anyone who has watched a young mammal exploring the environment for the first \isi{time} knows that ``\isi{play}\isi{'}\isi{'} and ``building a cognitive map\isi{'}\isi{'} are isomorphic activities (young lizards don't \isi{play} because they don't have the spare brain cells). It may look like ``mere \isi{play}\isi{'}\isi{'} to the super\-cilious human observer, and indeed it is \isi{play}-the animal wouldn't do it if it weren't fun-but it is also +he means by which lower creatures
%\originalpage{237}
as well as human children set about constructing the mental representa\-tion of the world which gives them varying degrees of predictability
and thus enables them to control to a greater extent their own chances of survival.

Now, if chimps can have concepts and label them just as we have concepts and label them, and if we know (or are reasonably sure) that we have a common ancestor in \textit{\isi{Dryopithecus},} then we can begin to get some kind of evolutionary perspective on the development oflinguistic infrastructure. When we find behavioral homologies in closely related species, we can reasonably assume that these homologies represent a common inheritance from a common ancestor (Campbell and Hodos 1970; Hodos 1976;Dingwall 1979:\figref{fig:1}.4). This would mean that the power to conceptualize and the potential for naming go back at least as far as \textit{Dryopithecus} and maybe further back than that. In a moment I will try to answer \textit{the} fascinating question that everyone must want to ask at this point: ``Why , if language is so spectacularly

ap;ie, and if the basic infrastructure has been around for so long, dtdn t It develop millions of years sooner?\isi{'}\isi{'} But first, there is more to be said about the problems of \isi{conceptualization} and naming.

We began by tackling the infrastructure of language at just that point where the gap between language and the external world was most easily bridged ; where the classes to be named were at least classes of

discrete ·entities. Let us now turn from entities to their attributes and in particular to color. \isi{'}

Color is not something that really exists in the external universe. We have all seen the landscape ``change color\isi{'}\isi{'} as the sun declines with\isi{'}\isi{'}

out tinkingit .at ::U odd r stopping to remember that the purpling of

noor S green hills IS due sunp!y to the shortening of \textit{the} Wavelengths oflight reflected from them. Coler is simply created bv the interaction between those wavelengths and sets of specialized pe;ceptor neurons, longer wavelengths appearing as red, slightly shorter ones as orange, and so on .across the spectrum. But color vision adds yet another set
 of parameters to those that are already sorting percpts into their 
%\originalpage{238}

appropriate . classes. The boundary between two colors, for instance, is often a boundary and sometimes perhaps the only boundary between another entity and its background.

Useful though color vision is, it presents serious problems for language, problems of a kind quite different from those inherent in the naming of the species or genera. Creatures are discrete; the spec\-trum is one and continuous. We can perceive light at wavelengths of between roughly 380 and 800 millimicrons, and we can perceive it equally well at any wavelength within those limits. It is true that we can say of some colors, ``that's a real green,\isi{'}\isi{'} or ``a true yellow,\isi{'}\isi{'} but there are points in between where we cannot say whether green or yellow is involved.

If Og and Ug had sat down, as in some of the more simplistic Flintstone scenarios, with a bunch of different-colored pebbles to help them, maybe, and started out to ``name the colors,\isi{'}\isi{'} they would have been stymied from the word go. Even more sophisticated accounts which would still assume some degree of arbitrariness and voluntary control in naming run up against the insuperable obstacle that words demand concepts, and concepts demand boundaries; hut colors have no boundaries, so theoretically you should be free to cut up the spec\-trum into as many chunks as you fancy and draw the lines between them just where you feel like drawing them. In fact, as Berlin and \citet{Kay1969} demonstrated in their pioneering study, nobody is free to do any u rh thing. Basic mlor terms (terms neither borrowecl from names of ore-existinP- obi{\textless}'cts, \textit{0mn}\textit{{\textless}Je} , nor comoounded, e.g.. \textit{d}\textit{ark} \textit{qreen,} \textit{yellowish} \textit{br}\textit{c}\textit{;}\textit{wn,} etc.-that is t say, \isi{primitive concepts}) are highly predictable across languages, and the semantic range of each term is determined by the number of terms in any given language system and by the ranges of pre-existing terms (if this sounds familiar, remember it was exactly the way I said TMA systems were structured, back in Chapter 3, and we will look at these, too, later in the present chapter). This is to say that if a given language has only two basic \isi{color terms}, those terms must be ``dark\isi{'}\isi{'} and ``light{\textquotedbl}; if it has three, they can be only ``dark,{\textquotedbl} ``light,\isi{'}\isi{'} and ``red{\textquotedbl}; and so on.
%\originalpage{9}
The neurological substrate of this structuring of color has been explamed (McDaniel 1974; Kay and McDaniel 1978) in terms of ering's ``oponent\isi{'}\isi{'} theory of \isi{color discrimination} (Hering 1920; smce epenmentally confirmed for certain species of primates, cf. de Valois and Jacobs 1968). Primate brains, and those of some other orders, have various sets of perceptor neurons each adapted to different hands of the spectrum and activated only by stimuli that fall within
those wavelengths. One pair of sets monitors the ranges corresponding to red d green. One member of the pair hits its maximal firing rate when sttmulated by central red and its minimal firing rate when stimu\-lated by central green. The other member of the pair hits its maximal firing rate \isi{'}:hen stimulated by central green and its minimal firing

ra e when stimulated by central red. Similar pairs deal in a similar way with yellow-blue and the light/bright versus dark/dull distinction.

The far-reaching implications of the Berlin and Kay discovery have yet to be absorbed by the scientific community. The conclusions eache in a summary by Clark and Clark ( 1977: 527) are fairly typical m their unrevealig, indeed inaccurate, nature: ``The very physiology of the human visual system makes some colors more salient than others. Children find these colors eye-catching and easy to remem\-ber ·. · . There is more occasion to talk about salient colors and listeners assume that speakers are more likely to be talking about hem.

Color terminology is universal because the human visual system is universal.{\textquotedbl}

Quite apart from it ch:itty, wasn't P{\textquotedbl}\textsuperscript{0}rything-si: rle:::frer-all tone, and its evident confusion of perception with \isi{lexicalization}, this passage makes a grave factual error. The whole point of the Berlin and Kay thesis is that color terminology is NOT universal. If the color systems of languages reflected universalities of the human visual system
then color terminology would always be the same and always me the same. But it is not and does not.

What happens in these languages that have only ``dark\isi{'}\isi{'} and ``light{\textquotedbl}? Presumably speakers of these languages have the same visual system as everyone else, and presumably children \isi{learning} these lan%
%\originalpage{240}
guages find red, yellow, blue, etc., as ``eye-catching\isi{'}\isi{'} as any other children. And if all the primary colors are equally salient, how is it that no language starts by distinguishing only green and blue, and then works its way back across to red in the opposite direction?

It is worth going into the structuring of \isi{color terms} in some depth here as \isi{'} believe this structuring is paradigmatic of a number of other semantic areas, some of them much more important than that of color.

First, and contra Clark and Clark, there is no simple one-to-one relationship between neurological equipment and semantic structure. Rather, the nature of neurological equipment enables semantic struc\-ture to be divided up in a number of possible ways. At the same \isi{time}, it prohibits semantic structure from being divided in an infinitely greater number of ways, any of which might seem a priori no less logical or possible, and imposes rigid constraints on the sequence in which any given analysis of the semantic structure can be rendered more complex.

Let us look at some prohibited \isi{color terms}. No language has the term \textit{*reen} meaning 'red and/or green, but nothing in between\isi{'}, or the term \textit{*yellue} meaning 'yellow and/or blue, but nothing in between\isi{'}. There would seem to be no a priori reason for the absence of these terms, for it is easy to construct not only meanings but also possible neuroloical substrates for them. Thus, \textit{*reen} would be the represen\-tation of activity in the red-green receptors (and no others), while
\textit{*yellue} would be the representation of activity in the blue-yellow receptors (and no others). However, we know that \isi{lexicalization} does not simply represent outputs of particular neuronal sets, for two reasons. First, many languages have a term equivalent to \textit{grue} 'green and/or blue\isi{'}, which represents partial outputs of two opponent sets, rather than the full output of one opponent set. Second, the factor that allows \textit{grue} to exist, while blocking \textit{*reen,} seems to be perceived spatial contiguity, which of course corresponds to wavelength con\isi{'}\isi{'} tiguity. Green and blue are contiguous on the spectrum; red and green,{\textless} or yellow and blue, are not.

%\originalpage{241}

It has often been noted that spatiotemporal contiguity is a condition on naming; no language has a word such as \textit{*larm} meaning 'leg and/or arm\isi{'}, or \textit{*shee} meaning 'shoulder and/or knee\isi{'}; in no lan\-guage can I say, \textit{*I} \textit{teach} \textit{on} \textit{mwidays} meaning 'I teach on Mondays, Wednesdays, and Fridays\isi{'}. However, a further look at \isi{color terms} will show that spatiotemporal contiguity, although a necessary condition on naming, is not a sufficient condition. If it were, some languages would have a word \textit{*yeen} meaning 'yellow and/or green\isi{'} instead of \textit{grue.} Yellow and green are just as much contiguous as green and blue. Their conjunction would mean conjoining the outputs of two oppo\-nent sets, but the same is true of green and blue.

There would seem to be two possibilities. \textit{Grue }conjoins the two short-wavelength outputs of two opponent sets: \textit{*yeen} would conjoin the long-wavelength output of one opponent set (yellow) with the short-wavelength output of the other (green). Perhaps one kind of conjunction can be lexicalized and the other cannot; we simply do not know enough to say. But it is also possible that what can be lexicalized at any given stage of development may be constrained by the order in which \isi{lexicalization} takes place.\textsuperscript{7}

This brings us inevitably to the much deeper question: why were the basic \isi{color terms} added to human language in just the order that Berlin and Kay showed them to be? The answer may lie in a suggestion of potentially immense explanatory power first made by \citet{Stephenson1973} but not, to the hest of mv knowledge, s11hfl!\isi{'}!J11ently deeloped: that the Berlin and Kay sequence of dark/light-red-green/yellow-blue may reflect the order in which color perception became established phylogenetic ally.

The argument, although hard to support from empirical studies\-species representing the appropriate evolutionary stages may all be extinct-is nevertheless a highly plausible one. Stephenson points out that mammals were originally nocturnal and could probably only make
lighter-darker distinctions; as they began to shift to diurnal habits, after the extinction of major reptilian predators, the perce ption of light-wavelength distinctions became selectively advantageous (it 
%\originalpage{242}
would permit a much sharper and finer differentiation of the environ\-ment ). Stephenson argues that such perception would have begun with the longer wavelengths.

The transfer from the phylogeny of perception to the phylogeny
of language would then have come about in the following manner. In any line of development, neurological structure is always incremental; no species sloughs off its neural inheritance in the act of adding new layers; the new layers are simply superimposed on the old ones.\textsuperscript{8} It follows that older layers have a longer \isi{time} in which to establish themselves, to multiply numbers cells and cell connections. This process is likely to be halted or reversed only if the related capacity becomes dysfunctional to a species-which color perception is unlikely to do unless our species is forced back to a nocturnal pattern. Thus, other things being equal, the older of any two capacities should be the stronger. The greater neural strength of the oldest-the light-dark distinction-would then lead to its being first lexicalized; the neural strength of the next oldest-long-wavelength (red) perception- would lead to its being second lexicalized , and so on.

I shall therefore propose the following hypothesis: \textit{those} \textit{seman\-}

\textit{tic} \textit{distinctions} \textit{whose} \textit{neural} \textit{i}\textit{n}\textit{frastructure was} \textit{laid} \textit{down first} \textit{in} \textit{the} \textit{course} \textit{o}\textit{f} \textit{mammalian} \textit{development} \textit{will} \textit{be} \textit{the} \textit{first to} \textit{be} \textit{lexicalized} \textit{and} \textit{/or} \textit{grammaticized in} \textit{the} \textit{course} \textit{o}\textit{f} \textit{human} \textit{language} \textit{development.} In the present state of our knowledge, such a hypothesis can have only a tentative status; yet we will see, when we consider the possible lution of TMA systems, that it can still have considerable

power.
After red, languages can lexicalize either yellow (next wavelength
down from red, also the ``high\isi{'}\isi{'} member of the next color-opponent set) or green (the ``low\isi{'}\isi{'} member of the set already activated). It ma be that herein lies the reason for the absence of \textit{*yeen.} If one or the other member of \textit{*yeen} must be individually lexicalized, then a large category consisting of just those members cannot subsequently be constructed: \isi{lexicalization} proceeds unidirectionally toward an eve finer dissection of the color area, so that while existing categories
% ORIGINS 243
may be split, they can never be added to or collapsed. But again, research into the color vision capacities of more species of primates may clarify the situation by demonstrating a phylogenetic \isi{order of acquisition} for the shorter wavelengths too.

We should also look at how the meaning of individual terms is affected by sequential development of semantic subsystems since there is good reason here also to suppose that similar phenomena will be found elsewhere. In their original (1969) treatment, Berlin and Kay referred to ``dark\isi{'}\isi{'} and ``light\isi{'}\isi{'} as ``black\isi{'}\isi{'} and ``white.\isi{'}\isi{'} Indeed, ``black\isi{'}\isi{'} and ``white\isi{'}\isi{'} is what these terms shrink to in an eleven-term system like that of \isi{English} where other terms have spread over most of the seman\-tic ground. Yet it should surely be obvious that as \isi{lexicalization} pro\-gressively dissects semantic areas, the meanings of the earliest lexical items must change: ``black,\isi{'}\isi{'} which originally embraces half the spec\-trum, must gradually reduce in scope until it eventually occupies only a narrow band of it. Similarly, ``red\isi{'}\isi{'} in a three-term system must include much-orange, maybe the darker yellows-which it cannot possibly include in the eleven-term \isi{English} system, which contains orange, yellow, and pink as units. Thus, the semantic range of terms in subsystems is determined by the number of terms in such subsystems and by the semantic ranges of the other terms.

Constraints such as these will loom ever larger as we continue to traverse \isi{semantic space} away from representations of concrete entities and toward representations of ever more abstract relationshi ps. So far. the semantic infrastructure we have dealt with is in all probability shared by \textit{Homo} \textit{sapiens,} \textit{Pan} \textit{troglodytes,} and \textit{\isi{Dryopithecus}.} I·doubt whether similar sharing extends to much or even any of the areas we are about to enter. Indeed, if we were reconstructing to a strict chrono\-logical timetable, we should probably drop semantics here and start talking about syntax, since from here on out, syntactic and semantic developments were almost certainly intercalated and their interaction served to drive language up along a beneficial spiral. However, in the interests of clarity of presentation, and to counteract the obsession with ``communication\isi{'}\isi{'} that has so far vitiated any understanding
%\originalpage{244}
of how language must have evolved, I shall continue to deal with semantic infrastructure (or rather with such small patches of it as there is space to deal with) in order to show just how much conceptual preadaptation was necessary before a ``communicative system\isi{'}\isi{'} as
simple as the simplest of early creoles could be made to function corn- . municatively. Later on, we will retrace our steps to the present point and deal with the early development of syntax, relating the latter, wherever possible, to concomitant developments in semantics already touched on.

The first of the semantic areas I shall touch on concerns predica\-tions which may be felt to be central in any structured language sys, tern: \textit{There }\textit{is} \textit{an} \textit{X} , \textit{X }\textit{is }\textit{at }\textit{Y,} \textit{Z }\textit{has} X, \textit{X }\textit{is} \textit{Z's. }I shall refer to the relationships expressed by these predications as Existence, Lc{\textgreater}ca.tl{\textless}m;{\textgreater}J\isi{'}

Possession, and Ownership. I should emphasize that these labels
chosen only for convenience of reference and are not meant to nave {\textless}l!\isi{'}\isi{'}. any particular semantic significance: ``\isi{possession},\isi{'}\isi{'} for instance, is grossly inadequate for the semantics of \textit{has,} which might better, though still inadequately, be defined as ``stands in a close and superordinate relationship to.{\textquotedbl}

In an original and insightful study, Eve \citet{Clark1970} reviewed the ways in which these four relationships are represented across a sample of fifty-odd languages. She found a high degree of similarity in the syntactic structures involved, but a good deal less similarity in \isi{lexicalization}. Some languages (indeed, almost half the sample) used only a single morpheme to lexicalize the entire area; others used four different items, i.e., lexicalized each of the four relationships differently. Between these extremes there were several different patterns, with two or three of the relationships being jointly lexicalized, \textit{i} but seldom the same two or the same three from one language to the next. Not surprisingly, Clark concluded that, in this area, the lexicon was without internal structure.

At first sight, an area such as this might seem to be affected by
constraints far different from those which would affect the area of ..
%\originalpage{245}
body parts or even the more abstract area of \isi{color terms}. One would not expect to fmd, for example, \isi{contiguity constraints} of the type that bar items like \textit{*yeen} and \textit{*shee,} since the relationships we are now talking about do not seem to have any discernible concrete correlatives of which contiguity or noncontiguity could reasonably be predicated. And yet, \isi{contiguity constraints} exist here too.

Consider \figref{fig:4}.2 below:

\begin{tabular}{ll}
\lsptoprule

\multicolumn{1}{l}{Ownership} & Location\\
% \hhline{~-}
Possession & Existence

 {}-\\
\lspbottomrule
\end{tabular}
\begin{figure}
\caption{2 Semantic space for four relationships}
\label{fig:4}
\end{figure}



If the four relationships are arranged spatially as in \figref{fig:4}.2, and if we consider only the primary (shortest, simplest, most frequently used) morphemes in each language-e.g., not allowing \textit{exist }to sub-

. s*ute for \textit{there IS, }or \textit{possess }for \textit{have-the }following constraint 'on exicalization seems to apply: no language can use the same mor\-pheme to express any two noncontiguous relationships (i.e., \isi{location} and \isi{possession}, or existence and \isi{ownership}) unless that same mor· p\}ieme is also used to express one of the intervening relationships

(l-e;, existence or \isi{ownership} in the first case, \isi{location} or \isi{possession}

i1; thesecond). In other words, the \isi{semantic space} mapped in Figure

·2 is as structured as real space, and, as with real space, only con- tiguous sectors can be jointly lexicalized. This constraint operates on all the languages i11 Clark's sample, on all creoles for which adequate
%\originalpage{246}
data are available, and for at least thirty other languages checked so far, or at least one hundred languages in total; Ihave not yet met with any counterexamples.

The reasons for the existence of such a constraint in this par. ticular case are far from obvious. The categories involved are not
highly abstract, but seem to be mutually inclusive. Species and co1or .'I terms are mutually exclusive: if something is a cat, it is not a dog; if something is red, it is not yellow or blue, and so on. If the semantic . space associated with species, color, and certain other areas is sh;arnhr. I divided, then such divisions can be regarded as no more than analogues
of divisions which exist in the material universe. But it is hard to see what real{}-world divisions would be correlates of the constraint go·vern• .I ing . the \isi{location}-existence-\isi{possession}-\isi{ownership} area, since extst{\textless}mce \isi{'}·l and \isi{possession} (in the relational sense given aboveJ can be predicated
of all entities whether abstract or concrete, while \isi{location} and owner- : ship can be predicated of all concrete (and perhaps some abstract) ·

%\originalpage{247}

less apical environments {}-voi, +car), or in voiced velar or bilabial environments +voi, {}-{\textless}:or), or both. In the same way, semantic change could not spread from an environment which had minus values for two \isi{semantic primes} to an environment which had pius values for those same primes without first affecting at least one environment

;{\textbackslash}'hich had a minus value for one prime and a plus value for the other
prime.

An example can be found if we compare the article system of Modern \isi{English} with the article system of \isi{Guyanese Creole}, which is probably not much different from the article system of Middle \isi{English} (the theory predicts that when an article system arises, it will be governed by similar constraints irrespective of whether it arises in a creole system or elsewhere). The Guyanese system is shown in \figref{fig:4}.3 below:
entities. So why should it not be possible to conjointly express locat1{\textless}m and \isi{possession}, or existence and \isi{ownership}, given that any other µairs. : any triples, or all four together may be conjointly expressed?

I can think of two possible explanations, not necessarily

ally exclusive. Both explanations involve principles of broad, mueeu universal, application. As yet, Iknow of no way in which these natives could be tested.

The first explanation involves \isi{semantic primes}. The term \textit{se1nai1·}

r \textit{di} {}-{}-{}-{}- {}-{}- {}-{}-{}-l

I I I I

I

I

.L

r {}-{}- {}-{}-{}- {}- \textit{wan} \textit{{}-} \textit{\textsubscript{1}}

I 

I

I I I I I

\section{}
.l..

\textit{tic} \textit{prime} is normally used in reference to unanalyzable concepts; hfl\isi{'}{\textquotedbl}\isi{'}\isi{'} {\textless}o Iuse it in a rather different sense, to refer to a very limited set binary oppositions; any concept can then be defmed in terms of

and minus (and perhaps null) values for these oppositions, in
the same way as phonological units can be defined in terms of mu{\textgreater},.

and minus values for Jakobsonian distinctive features.\textsuperscript{9}

Semantic change would then proceed in a manner ana10:guu• to phonological change. A phonological change cannot spread voiceless velar or bilabial environments ( {}-voi, {}-cor) to vo1ce1J , apical environments +voi, +car) without frrst occurring in

\isi{'}

T T

I

I

I I

l I

t I

L-{}-{}-{}-{}-{}- {}-{}-{}-{}-{}- {}-{}-{}-{}-{}-{}-{}-{}-{}-{}-{}-

\begin{figure}
\caption{3}
\label{fig:4}
\end{figure}

Semantic space for Guyanese \isi{articles}

%\originalpage{248}

%\originalpage{249}

In \figref{fig:4}.3, ``definite\isi{'}\isi{'} and ``indefinite\isi{'}\isi{'} have their traditional mean· ings; ``generic\isi{'}\isi{'} refers to the subject NP in \textit{The} \textit{dog}\textit{/} \textit{A} \textit{dog/Dogs} \textit{is/are}

\textit{(a)} \textit{mammal(s), }and ``other\isi{'}\isi{'} includes NP in the scope of \isi{negation}, ``a book or books,\isi{'}\isi{'} and similar cases (see Chapters 1 and 2 for a more; detailed analysis of the creole system ). It was claimed earlier in this chapter that the specific-nonspecific distinction had as its cerebral foundation the differential storage of \isi{percepts} and concepts; if this is so, then the SNSD must represent one of the oldest (phylogenetically
speaking) of \isi{semantic primes}. If it is old, it must (by the infrastructural

r i

I

I

I

I

I I I

I

\section{}
the

T T I

{}-{}-{}-{}- {}-{}-{}-{}-{}-{}-{}-{}-,

I 

I I I

I

I

I ,..

I I

I I I \textsubscript{I }\textsuperscript{I} 

hypothesis proposed in our discussion of \isi{color terms}) be strong, and . this superior strength may accoun t for the config1m1tion of Fi!]iur

4.3. Although the SNSD divides the entire semantic area, with ``zero\isi{'}\isi{'} on one side and ``some marker or other\isi{'}\isi{'} on the other, the presupposed-; nonpresupposed distinction \textit{(presupposed} in this context refers to ``information presumed shared by speaker and listener{\textquotedbl}) divides only the +specific area. Now, there is no a priori reason why the latter distinction should not divide the entire area; generics are +P because everyone can be assumed to know class names, while ``other\isi{'}\isi{'} is {}-P because no one can be expected to know which was the dog that X DIDN'T see or which was or were the book or books that Y might have bought. But, as we saw with colors, semantic infrastructure tells you where lines MAY, but not where lines MUST, be drawn between lexicalizable areas of meaning.

One feature of \figref{fig:4}.3 is that there is no overlapping of the territory of different lexicalizations; there is no such thing \isi{in GC} as a sentence in which you could change the article of any NP without simultaneously changing the meaning. This \isi{generalization} does not, of course, apply to \isi{English}. In \textit{The} \textit{dog} \textit{is} \textit{a} \textit{mammal} you may change the article to anything you like without materially affecting meaning, and sentences such as \textit{there} \textit{are} \textit{no} \textit{cows} \textit{here} and \textit{there} \textit{isn't a} \textit{cow} \textit{here} are synonymous. In fact, we may compare the GC situation shown in \figref{fig:4}.3 with the \isi{English} situation shown in \figref{fig:4}.4 on the following page:

I I \textsubscript{I }\textsubscript{I}

\textsuperscript{I }\textsuperscript{I }I I 

\textsubscript{I }I I I

\textsuperscript{I }\textsuperscript{1 }I {\textbackslash}

\textsuperscript{I }\textsuperscript{1 }I I 

I

\textsuperscript{I }L- {}-{}- {}- {}- {}-{}-{}- {}-{}-{}-1 \textsuperscript{I }\textsuperscript{I} 

I I 1 

I \isi{'} - {}-{}- - 

L-{}-{}-{}-{}-{}-{}-{}-{}-{}-{}-{}-

\begin{figure}
\caption{4}
\label{fig:4}
\end{figure}

Semantic space for \isi{English} \isi{articles}

Here, \textit{the} has spread from +P +S to +P -S, on the basis of both ``defi\-nite\isi{'}\isi{'} and ``generic\isi{'}\isi{'} being +P, while has spread from {}-P +S +P {}-S, Lut only by virtue of having first spread to {}-P {}-S, on the basis of both ``indefinite\isi{'}\isi{'} and ``other\isi{'}\isi{'} being {}-P; of course, once \textit{a} has r.eached ``other `` it can then spread to ``generic\isi{'}\isi{'} on the basis of their both being . In other words, a contiguity constraint similar to that goven\-mg \figref{fig:4}.2 obtains, preventing ``definite{\textquotedbl}. an.d ``other,\isi{'}\isi{'}. or ``m\-

,lcfinite\isi{'}\isi{'} and ``generic,\isi{'}\isi{'} from being jointly lex1cal1zed unless mterme-

,liate categories are also jointly lexicalized. . . .

A similar spreading process of individual lex1cal1zat10.ns cross \isi{semantic space} could account for the variable ranges of lexical items 
%\originalpage{250}
in the \isi{location}-existence-\isi{possession}-\isi{ownership} area. Let us make the same assumption for that area as we made for \isi{articles}: that the con\-figuration that emerges in creoles is the primary configuration whenever \isi{articles} appear (including, of course, in the original development of human language as well as in the development of every existing natural language). Then the primary configuration for the \isi{location}, etc., area will be as shown in \figref{fig:4}.5 below:

\begin{tabular}{ll}
\lsptoprule

\multicolumn{1}{l}{Ownership \textit{(a)}} & Location \textit{(}\textit{de)}\\
\multicolumn{2}{l}{Possession \textit{(get) }Existence \textit{(get)}}\\
\lspbottomrule
\end{tabular}
\begin{figure}
\caption{5}
\label{fig:4}
\end{figure}

Semantic space for \isi{location}, etc., \isi{in GC}

The fl'semhlance to th P rnnfigurati nn of \figref{fig:4} 1 nhvious. Again, two semantic areas are jointly lexicalized, while the remaining two are separately lexicalized. Again, as with \figref{fig:4}.3, we know that the pattern illustrated is one that is followed by most, perhaps all, creole languages, and one that cannot be explained by appeal to the structures of the languages that were in contact at the \isi{time} the creoles came into existence. If this represents the primordial pattern, then those languages which separately lexicalize all four relationships would have reached that state by dividing and separately lexicalizing the tw lower quadrants, while those that jointly lexicalize three or even four quadrants would have reached that state by procedures similar to thos
which spread \textit{the} and \textit{a} to the second and third quadrants, respectively, but without at any stage of the process jointly lexicalizing noncon · tiguous quadrants.
However, it remains to identify the \isi{semantic primes} by virtue of which the contiguity constraint is maintained in the domain of Figures 4.2 and 4.5. Clearly, the specificity prime, dominant in article systems, cannot be involved, for any entity must be marked as +specific before it can have existence, \isi{location}, \isi{possession}, or \isi{ownership} pre\-dicated of it. However, there is evidence that presupposedness, the next prime down, so to speak, may be crucially involved. Entities of which \isi{location} . and \isi{ownership} can be predicated must be assumed known to the listener: compare \textit{the} \textit{desk is} \textit{in} \textit{a} \textit{comer} with \textit{*a} \textit{desk} \textit{is} \textit{in} \textit{a} \textit{comer,} or compare \textit{the} \textit{briefcase} \textit{is} \textit{mine} with \textit{*a} \textit{briefcase} \textit{is} \textit{mine.} On the other hand, entities of which existence and \isi{possession} can be predicated must be assumed unknown to the listener; thus, we have \textit{there} \textit{is} \textit{an} \textit{answer} versus \textit{*there} \textit{is} \textit{the} \textit{answer},\textsuperscript{1}\textsuperscript{0 }and \textit{I} \textit{have} \textit{a} \textit{cold} versus \textit{*I} \textit{have} \textit{the} \textit{cold} (the fact that the latter sentence is grammatical under contrastive stress, as in \textit{(It's)} \textit{I }\textit{(that)} \textit{have the} \textit{cold} , \textit{not} \textit{Mary} , is, of course, completely beside the point). 

Perhaps less clear here is exactly what the second prime in\-volved is. I shall suggest, very tentatively and provisionally, some- 1thing Ishall call ``relatedness. `` A claim that something exists entails
no claim that that something is significantly related to anything else;

Ultmilarly, a daim that something is \}optPrt snmewherl\isi{'} l'nt nils no daim that there is any significant connection between that c:.ome\-thing and its \isi{location}. However, claims of \isi{possession} and \isi{ownership} ttivolve a substantive link of some kind, whether genetic \textit{(Bill} \textit{has} \textit{children,} \textit{those} \textit{children} \textit{are} \textit{Bill's),} creative \textit{(Mary} \textit{had} \textit{an} \textit{idea,} \textit{that} \textit{idea} \textit{was} \textit{M} \textit{ary's),} or of some other nature. We could then illustrate temantic primes and their interrelationship as in \figref{fig:4}.6 on the
owing page:

%\originalpage{2}

%\originalpage{253}

All entities 

IS INTERESTING

IS THOUGHT ABOUT

\textsuperscript{{}-}{}-{}- 

\begin{itemize}
\item {}-{}-.love
\end{itemize}

fear

{}-S {}-{}-{}-{}-{}-+{}-S{}-{}-{}-{}-{}-{}-{}-

IS N EA RBY IS ABOUT X IS AT THE CORNER IS TRUE 

+P \textit{(th}\textit{e}\textit{) }{}-P \textit{(}\textit{a}\textit{)} 

\isi{'}....., -

......

story

IS AN HOU R LONG 

idea

(R = related)

\begin{figure}
\caption{6}
\label{fig:4}
\end{figure}

IS RED HAPPENED YESTERDA Y IS HEAVY

\textit{I },\textit{.},\textit{,,}

IS ON l'URPOSE

IS X'S FAULT

\isi{'},,

sunrise

Hypothetical tree structure for \isi{semantic primes}

This would enable us to define \isi{ownership} as +P +R, \isi{location} as

+P {}-R, \isi{possession} as {}-P +R, and existence as {}-P {}-R. If this were the case, no lexkalization could spread directly from \isi{ownership} to exis\-tence (or vice versa) or from \isi{location} to \isi{possession} (or vice versa), since in either case the process would involve simultaneously reversing the

polarity of two \isi{semantic primes}. The observed facts for this area

IS TALL

IS SKINNY 

IS DEA D IS SICK

OF BOXES

IS FIXED

IS BROKEN

\textsuperscript{'}\isi{'},

\isi{'}

\isi{'},milk

water

....\_ ......, car

\textsuperscript{'}\textsuperscript{'}..... ....... fight

.... kiss

would thus be fully accounted for. 

However, doubts about the status of ``relatedness\isi{'}\isi{'} {}-which does not appear to figure crucially in any other semantic area, unlike pre\-

supposedness- may make it worthwhile to consider an alternative

IS W ILTED BLOOM S 

... refrigerator

{}-.. {}-....., ... flower

explanation for these facts.

A slightly different kind of contiguity constraint has recently been claimed by Keil (1979, 1981). The constraint envisaged by Keil derives from a structure which he terms a ``Predicability Tree.\isi{'}\isi{'} A \isi{predicability tree} defmes the range of different predication types

over various \isi{semantic classes} of NP (see \figref{fig:4}.7 on p. 253). Each

IS ASLEEP IS HU NGRY

IS\textsuperscript{I}HONEST 

tree

\begin{figure}
\caption{7}
\label{fig:4}
\end{figure}

predication type ranges only over those classes of NP which it domi\-nates in the structure. Thus, predications such as \textit{X is interesting} and \textit{X is} \textit{thought} \textit{about} can be made of any class of NP, while at the

\isi{'}

IS SO\textsubscript{.}R\textsubscript{.,}\textsubscript{\_}RY

\isi{'}, man 

...... girl

The \isi{predicability tree}

\textit{(from} \textit{Keil} \textit{1979:Figure} \textit{1)}

%\originalpage{254}

furthest extreme, predications such as \textit{X is} \textit{honest} and \textit{X} \textit{is} \textit{sorry} can be made only of the class of NP that is +animate, +human.

Keil's \isi{predicability tree} is based on work by Sommers (1959, 1963, etc.), which first pointed out the existence of what Keil calls ``the \isi{M constraint}.\isi{'}\isi{'} The \isi{M constraint} prevents any pair of predicates (A, B) from intersecting, i.e., A and B ``can never span terms in com\-mon and also have terms that just A spans and terms that just B spans\isi{'}\isi{'} \citep[16]{Keil1979}. It also follows from this that no predicate can span two noncontiguous . sets of terms unless it also spans all intervening sets of terms.

Experiments carried out by Keil with children as young as kindergarten age suggest that the \isi{M constraint} is unlikely to be learned by experience. Even the youngest children had somewhat truncated versions of the \isi{predicability tree}, and such violations of the M con\-straint as were found tended to support Keil's hypothesis rather than disconfrrm it. For example, children who claimed that \isi{dreams} were tall (thereby apparently violating the hierarchy of predicability ) re\-vealed under further questioning that they believed \isi{dreams} to be physical objects: ``They're made out of rock,{\textquotedbl} ``They just got grass on them,\isi{'}\isi{'} ``They turn white and go up in the sky\isi{'}\isi{'} were among their answers \citep[110]{Keil1979}. Thus, the violations arose through assignment of ``\isi{dreams}\isi{'}\isi{'} to an inappropriate category, rather than a violation of the hierarchy per se. Since children's output, as we have seen, is any\-thing but isomorphic with their input, it cannot be claimed that the absence of M-constraint violations in their speech is merely a reflex of a similar absence in adult speech. There would appear to be no way in which they could negatively defme the scope of predications as a result of inductive processes; thus, Keil's results further support the argument of Fodor ( 1975), referred to at the end of Chapter 3, that one could not learn the extensions of the predicates of a natural lan\-guage unless one already knew the extensions of those predicates.

Although the \isi{M constraint} is strikingly similar to the types

of \isi{contiguity constraints} that we observed in connection with \isi{color terms} and the area of \isi{location}, etc., it relates to predication (the
% {\textbackslash}
%\originalpage{255}
establishment of a relationship between two lexicalizations) rather than delimiting the scope of \isi{lexicalization} itself. Still, since \textit{have,.} \textit{be,} etc., and their cross-linguistic equivalents are indeed predications, it may be that the predicability hierarchy affects permissible lexicalizations within that semantic area. We noted above that existence and posses\-sion could be predicated of all things, and thus would include the entire tree in their scope; \isi{location}, however, could only be predicated of classes dominated by the second node down \textit{(is} \textit{nearby} \textit{I}\textit{at} \textit{the} \textit{cor\-} \textit{ner\},} while \isi{ownership} could only be predicated of classes dominated by the third node down \textit{(is} \textit{red} \textit{/} \textit{heavy).} Joint scope of existence and \isi{possession} could therefore have favored their joint \isi{lexicalization}, while the disjoint scopes of \isi{location} and \isi{ownership} would have favored disjoint \isi{lexicalization}.

In the present state of our knowledge, there is no principled way
to choose between the two explanations. Those explanations, however, have served to show us other ways in which \isi{semantic space} is struc\-tured, and suggest that the contiguity-constraint approach may yield a rich store of insights into the massive conceptual infrastructure that underlies, and that alone could make possible, the simplest ``communi\-cative\isi{'}\isi{'} uses of language.

Before turning back to survey the growth of the syntactic struc\-tures that were based upon that infrastructure, we should look at one last area of \isi{semantic space} where \isi{contiguity constraints} arising from \isi{semantic primes}, rather than from the predicability hierarchy, appear to be operative. At the same \isi{time}, problems that were deferred to the present chapter when we encountered them in Chapter 2 (in connection with creole variability in the treatment of iteratives) and again in Chapter 3 (in connection with variable treatment of iteratives by children) may now be dealt with.

This area can best be understood if we start from those problems. Readers will recall that although a majority of creoles (including \isi{Guyanese Creole}, which here as elsewhere will be taken as representa\-tive) merge iteratives with duratives in a single nonpunctual category,

%\originalpage{256}

\ili{Jamaican Creole} (and perhaps one or two others) treats iteratives th{\textquotedbl}{\textquotedbl}. o(Thursdays or for any combination of Thursdays (provided that such same way as past punctuals, while \ili{Sao Tomense} (and perhaps one or l\isi{'} a combination did not equal the sum of all Thursdays) and still be
two others) treats iteratives the same way as futures (and perha s ·f.i
true. Let us suppose that John drives to work every Monday, Tuesday,
other members of the irrealis category-{}-existing descriptions are t!o · . •••• Wednesday, and Friday, and also on a majority of Thursdays, but on inadequate for one to tell). ·• the remaining Thursdays, he walks to work. Then it is true that \textit{John}

. Ch As was sehen in. the discussion of Bronckart and Sinclair (19\textsubscript{7}\textsubscript{3}\textsubscript{1}\textsubscript{ }f....··.··• \textit{walks} \textit{to} \textit{w}\textit{ork on} \textit{Thursd}\textit{a}\textit{h}\textit{ys }(buht nhot on

Tuesdays,\isi{'} Fdridays, etc.).

m apter 3, t ere is more than one way of looking at iteratives,c\_ 1.Jvloreover, \textit{\textsuperscript{1}}et us suppose t at Jo n as on\textsuperscript{1}y ever waike to work on From one viewpoint, an \isi{iterative} predication such as \textit{John} \textit{wal}\textit{k}\textit{s} \textit{t}\textit{a}\textit{'}\textit{.}\textit{\isi{'}!.: }'otle Thursday. In that case, the sentence \textit{John} \textit{does} \textit{not} \textit{walk} \textit{to} \textit{work}

\textit{work }r;;.nges over an ill-defined series of instances in which Johti;f . \textit{'}\textit{on} \textit{Tlmrsdays} is false and can be shown to be false by instancing the

alread has walked (and may be expected to continue to walk) to work;.. / fulitary occasion on which he did walk to work on a Thursday. If

Smee it does not represent a single event perceived as a unit (such an\isi{'} · · that sentence is false, its converse, \textit{John} \textit{walks} \textit{to} \textit{work} \textit{on} \textit{Thursdays,} event. as might be represented by \textit{John} \textit{walked to} \textit{work} \textit{last} \textit{Thursday,;} \textit{; }:111ust be true, no matter how uninformative \textit{or} misleading it might sa ), tt can be reg:{\textquotedbl}ded as falling into the nonpunctual category along .; i{\textgreater}i ,appea:rto be.

with events perceived as extended and uncompleted processes \textit{(John{\textless}} \textit{'}\textit{;}\textit{.} \textit{\} }· It should b'e obvious then that predications of the \isi{iterative} class

\textit{is/was} \textit{walki}\textit{n}\textit{g} \textit{to} \textit{wor}\textit{k}\textit{]. }··.••··. . ``:;·{\textless}lc:L not refer to events in the same kind of way that other \isi{types of} But, from another viewpoint, each of the series of actions over ;\isi{'}..\isi{'}{\textbackslash} i:edication refer to them. \textit{John} \textit{walked} \textit{to} \textit{work} \textit{last} \textit{Thursday} is true

which \textit{John wal.k} \textit{s to work }ranges is itself an isolated event seen as a \textit{:}\textit{c}\textit{;} Jfand only if John walked to work last Thursday, and \textit{John} \textit{is} \textit{wal.king}

unit. If one regards the nature of the units in the series as primary, \textit{to} \textit{work.} \textit{today} is true if and only if John is walking to work today. rather than the fact that those units constitute a series, then iteratives \textit{!} \textit{\}n} fact, we could claim that \textit{John} \textit{walks} \textit{to} \textit{work} \textit{on Thurs}\textit{d}\textit{ays} does can be perceived as falling into the punctual category. , \isi{'};\isi{'}.not refer at all to any specific events, but rather to a generalized con- From yet a third viewpoint, one can poin t to the fact that while ;\isi{'}\isi{'} c;ept which may be based on one or more such events. Since the realis sentences like \textit{John worked yesterday }or \textit{John} \textit{is} \textit{working today} refetc\isi{'} \isi{'}:\isi{'} category embraces real events in real \isi{time}, it could be concluded that

t specific occasions on which John worked or is working, sentences{\textbackslash}; :/ jt tative ``really\isi{'}\isi{'} belongs in the irrealis category.

hke \textit{J}\textit{o}\textit{;}\textit{n} \textit{works} do not. A sentence such as \textit{John} \textit{wor}\textit{k}\textit{s} may be true ; (.{\textless} ;:\_lat· Th foegoing paragraphs constitute an informal acconnt of the even i John is not working now and even if John works consider- \textit{¥}\textit{i}\textit{:} ;re ions ip etween \isi{iterative} and the nonpunctual, punctual, and

ably less than the average person. The key to the difference here lies :. \textit{P} \textit{·} \isi{'} irtalis categories. The question is now whether, in terms of well\-

mL· what it tkakes to establish the truth value of an \isi{iterative} predication.: \textit{Z}\textit{!i•} iw!Qooutilvdamt\_etd.ersemt tic prm,ies, wefcahn show formally hfow thos.e categories

et us

1\textsuperscript{oo a}

\textsubscript{1}1tt\textsubscript{1}e more c\textsubscript{1}ose\textsubscript{1}y at the problem of how we would {\textbackslash} .. . .. .. . ac m an anaiogue o t e re\textsuperscript{1}evant area o \isi{semantic space}.

assign truth value to the sentence \textit{John} \textit{wal}\textit{k}\textit{s} \textit{to} \textit{work} \textit{on} \textit{Thursdays. }:...• \isi{'}\isi{'}·\isi{'}\isi{'}.\isi{'}. .. The status of punctual-nonpunctual and realis-irrealis as semantic

We cannot falsify this sentence by pointing to a particular .;'pximes will be dealt with a little later on in this chapter, when we try Thursday on which John did not walk to work. But we would be wrong .to see whether the ordering of TMA markers can be accounted for in if we assumed that the sentence means ``John walks to work on most ,· evo!Utionary terms. For the present analysis we need only these and Thursdays.\isi{'}\isi{'} Not only does it not mean this, but it is also the case that \isi{'} :;..Olli\isi{'} o.ther pimary distinction which has already been independently the sentence could be falsified fr any individual member of any set{\textless} {\textbackslash}.. stahhshed, i.e., specific-nonspecific. Predications like \textit{John works }or

  


 

%\originalpage{258}

\textit{John} \textit{walks} \textit{to} \textit{work }may be regarded as having the same relationship to predications like \textit{John} \textit{work} \textit{ed} \textit{yesterd} \textit{ay} or \textit{John} \textit{is} \textit{walking} \textit{today} as generic NPs have to particular-reference NPs, or as concepts do to \isi{percepts}. In other words, habituals are {}-specific, while nonhabituals are +specific.

The SNSD thus crosscuts the area of \isi{semantic space} which includes the punctual-nonpunctual and realis-irreafis distinctions. In order to adequately represent this situation, we would require a three\-dimensional model, but for convenience we will represent the SNSD as a square boundary within a larger square, as shown in \figref{fig:4}.8 on
p. 259. Since, as has been stated already, semantic infrastructure
determines where conceptual boundaries MAY, but not where they \textsubscript{+P }MUST, be drawn, the configuration of \figref{fig:4}.8 leaves the three
analyses of Figures 4.9(a), (b), and (c) (p. 260) as further possibilities. \textsubscript{{}-}\textsubscript{P}
Analysis (a) of \figref{fig:4}.9 corresponds to the Guyanese (majority creole) analysis; analysis (b), to the \ili{Jamaican Creole} analysis; and analysis (c), to the \ili{Sao Tomense} analysis. It leaves open, of course, a fourth analysis: that of \isi{English}, \ili{Yoruba}, and a number of other languages which would correspond to \figref{fig:4}.8. In this analysis, habituals are separately grammaticized \textit{\{John} \textit{work} \textit{s)} from continua\-tives \textit{(John is} \textit{workin}\textit{g}\textit{), }punctuals \textit{\{John} \textit{worked} \textit{), }and various kinds of irrealis \textit{(John} \textit{will} \textit{work} , \textit{John} \textit{would} \textit{wor}\textit{k}\textit{).} It should be noted, however, that the Romance languages in general follow the analysis of \figref{fig:4}.9(a), the majority r:reolP: an:ilyi:is, ffH're1y   

upon it the past-non past distinction : \ili{Spanish} \textit{yo }\textit{trabtt\}o }me:.UlR \isi{'}! acm

ORIGINS

+R -R 

+R -R

259

working\isi{'} or 'I work\isi{'}, while \textit{yo} \textit{trabajaba} means 'I was working\isi{'} or 'I worked (habitually)\isi{'}, and in consequence, \textit{yo trabaje} is limited to 'I worked (punctually, on a particular occasion) \isi{'}.

We now have a vague inkling · (probably little more than that, as it may turn out) of the coplexities of \isi{semantic space}: a space that had to come into existence before language as we know it could be born. Some of that space was required for the very first, earliest, and simplest stages of language. Other parts, although they were not

(R = realis, P = punctual, S = specific)

\begin{figure}
\caption{8}
\label{fig:4}
\end{figure}

Semantic space around habituals

%\originalpage{2}

\textsubscript{I} I

\textsuperscript{I }I I 

\section{}
1-'4

it

I

%\originalpage{261}

immediately required, probably came into existence prior to the emergence of language, as we shall see, but were only incorporated into language as language grew. Yet other parts may only have come into existence subsequent to the initial stages oflanguage development. However, Isuspect that such parts, if they exist, will prove to be minor, and that the common notion, more often implicit than explicit (if made explicit, it is hard to defend), that language bootstrapped its way upward, creating the conceptual categories it needed as it grew, pro\-ducing thought , consciousness, and volition as mere epiphenomena, is simply false.

As I try to develop the scenario of how a language based on the conceptual categories we have surveyed could have developed, Ishall incur a heavy debt to a seminal work in glottogenesis, Lamen\-della (1976). This paper represents the first systematic attempt to use the development of language in the child as a possible model for the development of language in the species. Lamendella claims that ``on\-togeny manifests a repetition of several phylogenetic stages in the neurofunctional system that allows human infants to learn languages.{\textquotedbl}

In his
case,
as in mine, ``it strains credulity to pretend that language
as we know it \textit{suddenly} \textit{sprang} \textit{up} \textit{intact} \textit{as} \textit{a} \textit{cultural} \textit{invention} in the absence of \textit{extensive} \textit{cognitive} \textit{and} \textit{communicative} \textit{preadaptations{\textquotedbl}} (emphasis added); he envisages, accordingly, a series of hominid species, each developing a particular element or stage of language and then transmitting that development to the next species via the genetic code. 

Lamendella defends the foregoing model against accusations of Larnarckism by pointing out that in all species, individual members show differences in their capacities. Thus, at any given stage of the development toward full language, relatively slight differences in the associated capacities would have conferred a selective advantage on their possessors, so that there would have developed ``a concentration
of genotypes producing [these capacities J in the gene pool of the species.\isi{'}\isi{'} Thus, the average capacity of the hominid line at stage n+l would have equaled the maximum capacity at stage \textit{n,} and the bio%
%\originalpage{262}
logical foundations of language would have been laid down, not in a single cataclysmic event, but in an ordered series of steps.

This series would then necessarily repeat itself in the course of child acquisition since, as Lamendella points out, ``more recently en\-coded genetic information generally tends to unfold later in ontogeny so as to preserve the temporal sequence in which the new components of the genetic information code were laid down.\isi{'}\isi{'} Lamendella is careful to show that his claims do not fall under the two main criticisms to which early recapitulationist theories in biology were subject. First, he points out that ``embryonic\isi{'}\isi{'} stages of language may reproduce not the developmental stages of \textit{adult} language but the language of children at corresponding stages. At any stage, adults using general\-purpose strategies might have developed language beyond the range of contemporary children, although without being able to transmit such developments via the genotype. Second, he is aware that embryological features do not always or necessarily occur in the same order as their corresponding evolutionary features, so that the developmental stages of child language do not necessarily occur in the same order as corre\-sponding stages in the original development of language.

With regard to this latter point I think that Lamendella is top cautious. Recapitulationist theory in general biology had to cover a very wide range of phenomena, many of which were only very re\-motely connected. Consider any pair of such phenomena, say, dentition and the structure of the foot in a given !\isi{'}:pecies. C'leli!rly, two things are not wholly unconnected since we do not normally fwd herbivores with claws or carnivores with hooves. However, within both herbivorous and carnivorous species there is quite a wide range of tooth and foot structures, detailed development of each of which must have proceeded with a good deal of independence from the other. It should therefore be unsurprising that on occasion the precise sequence of developments should have been shuffled somewhat between phylogeny and ontogeny{}-that, for example, in a given phylum, a certain type of tooth might have developed earlier than a certain type of foot, but that, in the embryonic forms of some contemporary species, that type
% ORIGINS 263
of foot might develop earlier than that type of tooth. However, when we are dealing with the development of language, we are dealing with a very tight subsystem of neural structures rather than with a wide range of quite dissimilar physical features; and within such a sub\-system, a high degree of mutual interdependence might be expected to obtain. We would expect, therefore, that reversals of phylogenetic
ordering in the ontogeny oflanguage would be quite rare, ifindeed any . exist at all.
I shall not examine in detail the stages that Lamendella proposes, which differ in some respects from those to be suggested here; his work was carried but from a slightly different perspective and his conclusions are worthy of study in their own right. Ishall return to the last of our speechless ancestors, whose cognitive equipment need not have differed in any material respect from that of the contemporary great apes.

Earlier in this chapter reference was made to the question why, if there was such massive preadaptation for language, it did not arise earlier. Attempts to account for this fact often take the form of simply pointing to the parlous stat of \isi{hominids} expelled from an arboreal Eden and forced to compete with fitter predators; the compensatory advantage offered by language then seems self-evident. But \isi{evolution} does not behave like the U.S. Cavalry ; if it did, it would surely have ridden to the 11:t{\textgreater}cue of orilla, now thrcate.n.ed tw.m: seriously by our species our was ever threatened hy Need
not create function unless that function is already within a species\isi{'} grasp.

This view might seem to directly contradict the view expressed
earlier that intense interspecific competition may have rapidly ex\-panded the cognitive capacities of our species. In fact, there is no contradiction. Cognitive growth-the increase in the capacity of crea\-tures to analyze the environment and predict outcomes-has always been the major thrust of \isi{evolution}, and to claim this is in no sense to be guilty of teleology, since the more cognitively developed any species
%\originalpage{264}
becomes, the greater would be its chances of survival. Thus, the homi\-nid line may have been capable of a relatively rapid growth of cognitive capacity . precisely because there was already a broad evolutionary base to build on. But there could be no basis to build on with regard to language, since the kind of cognitive capacity \isi{hominids} were only now building- the conceptual capstone, so to speak, on the vast arch of perception that had been building ever since the first microorganism responded to ligh t or to the touch of another-was the necessary prerequisite for the most rudimentary form of language.

Yet the question remains. If apes have adequate prerequisites for at least a fraction of what we have in the way oflanguage, then the probability is that \textit{Dryopithecus} had similar prerequisites, and that gives a period of at least five million years in the pongid line, and \textit{x} million years in the hominid line, in which the capacity for language existed, and the need for language existed, but there was no language.

Here we must consider the channel problem. However refined the conceptual schemata, however detailed and accurate the cognitive map that a species can construct, it will profit that species little (except in terms of individual survival) unless there is also a mode of expression. The only two modes of expression that seem to have even a chance of being viable for primates are the vocal and the manual. Without full cortical (consequently voluntary) control over one or another of these channels, language \isi{as communication} would not have been possible.

However, the channel problem has quite another dimension, a dimension seldom referred to but equally critical. This dimension is, in fact, twofold. We will take the second half and then the fu:st half. The second half is: when A, the first hominid ever to use either a sound sequence or a gesture referentially, made such sequence or gesture to B, another hominid, how did B know that A was communicating referenti\-ally , and not merely coughing, clearing his throat, scratching himself, or brushing a fly away? The first half is: given the same situation, how did A, totally ah ovo, conceive the idea of represen ting some object or event in the environment in terms of a sound sequence or gesture-an act unprecedented since the Big Bang?

%\originalpage{5}

These problems cannot be dismissed by hand-waving. Either language began as a consciously intended performance, in which case we have to show both how the intent could have been formed and how a conspecific could have grasped both the fact that there was an intent \isi{'}.{\textquotedbl}\isi{'}\isi{'} the r ference. that was intended, or it began accidentally. Although it 1S no aim of this chapter to add to the already overlon\isi{'}\isi{'} list of Plint\-stone scenaris, one of the (possibly numerous) ways in which language could have ansen accidentally is the following: Mrs. Og, breast-feeding a lusty one-year-old with one hand, is trying to feed herself with the other. Young Og, ready for a change of diet, makes a grab for the meat. Mrs. Og pushes him away. He tries harder, babbling in his frustra\-tion: \textit{gaga.} His stubbornness amuses Mrs. Ug, sitting nearby, and she imitates \textit{gaga} and maybe makes a playful grab for the meat. For a while after that, the favorite joke in the tribe is to creep up on somebody, hout \textit{gaga, }and try and grab his or her meat. Perhaps it dies out, as jokes do. Perhaps words were found and lost and found again a score f times before they took root, or perhaps the slightly older kids picked
it up and began to use it seriously when they got hungry or when they though t the grown-ups were dividing the food unfairly.

.or a slight variant on this: Ig, young Og's uncle, is pretending to be a tiger, an avuncular activity still widespread today and presumably of no very recent evolutionary history. Young Og withdraws in real or simulated fear, shouting \textit{wawa!} Uncle lg imitates him, and again very.one laughs, but tigers are not evetyday occurrences, so the thing is qmckly forgot ten. But a couple of days later Ig sees a real tiger about to pounce on Og. By a sheer fluke he yells out \textit{wawa!} instead of the regular alarm call, and Og saves himself in the nick of \isi{time}. Maybe they and the rest of the band are able to kill the tiger, and dance and em\-brace. around its carcass like European soccer players after a goal, shouting \textit{wawa! }And the word, perhaps soon followed by others of a
similar nature, gets incorporated into the earliest of human rituals; for, as \citet{Marshack1976} insightfully observed, ``Language, in fact, may have been as useful, or more useful, in this cultural realm than in the
comparatively self-evident strategies utilized in hunting, butchering and gathering.{\textquotedbl}

%\originalpage{266}

Many such scenarios could be elaborated, all equally probable (or improbable). How the Rubicon was crossed is of minor concern; what matters is how it was reached and what happened after it was crossed. But origin stories like these which feature Judie and jocular compo\-nents do have some advantages. First, they do not require intent on anybody's part, and since they do not require intent, they do not require understanding, at least in the everyday sense of. that term. Thus, t.liey neatly avoid both halves of the understanding-intentionality problem referred to a few paragraphs earlier. Secondly, they are based on behaviors- imitative and joking behaviors-which are independently attested for other members of the primate family and which therefore must have been common to all our immediate ancestors. Thirdly, they provide an element which may be essential in the \isi{acquisition of} lan\-guage anywhere in the universe: external modeling.

It is not, I think, accidental that chimps did not acquire language until we taught them. It cannot be the case that they lacked the intelli\-gence to invent it, since they can use it creatively (within, admittedly, quite narrow limits) once their pump has been primed, so to speak. It could be that the conceptual leap is too great to be made in a single stride by any species-that some kind of external model is needed, whether that model is intentional (as was the case with human teaching of apes) or unintentional (by young human children, as in the stories above); otherwise, the whole idea of referential communication would have been just too radical to work out (in either sense of \textit{work} \textit{out}\textit{).} But if this is so, there is a channel block for the pongid line that did not exist for the hominid line.

In other primates, vocal outputs have not come under sufficient cortical control for the \isi{vocal channel} to be viable for linguistic use; the great apes cannot suppress spontaneous vocalizations, have very little if any capacity for voluntary vocalization, and ``show little or no ability to imitate sounds\isi{'}\isi{'} \citep{Dingwall1979}. But if \isi{hominids} could have imitated and assigned meaning to the spontaneous vocalizations of children, why could not chimps or other primates have done the same with their own infants\isi{'} spontaneous,gestures?

%\originalpage{267}

If we re\isi{play} the two scenarios given above with an ape cast, the reason will become obvious. Instead of \textit{gaga} for 'meat\isi{'} or more prob\-ably some more general 'food\isi{'}, you would have had some kind of grabbing motion. Instead of \textit{wawa} for 'tiger\isi{'} you would have had some kind of fear behavior.- Paradoxically, infant gestures could not have served as proto-words \textit{because} \textit{they} \textit{were} \textit{not} \textit{arbitrary} \textit{enough.} For the first signs to have had a narrow enough range to fit individual concepts, they must have had NO RANGE, have been quite empty, communi\-catively speaking, so that they could be filled by the particular refer\-ence of the immediate context, hy ``food\isi{'}\isi{'} or ``tiger,\isi{'}\isi{'} as the case might be. You could not use a grabbing motion as a symbol for food because there were so many other things you might grab for. You could not use a fear gesture as a symbol for a tiger because there were so many other things you might be afraid of. But something that had no clear meaning for the parent, such as a child's pre-speech utterance, could be hooked to any of the hominid's preexisting concepts precisely because it lacked any such general associations.

There would be little point in spending so much \isi{time} on the actual emergence point of language if the suggestions just given were not a logical outgrowth of all that we have already discussed. The major point of this chapter has been that language grew out of the cognitive system used for individual orientation, \isi{prediction}, etc., rather than out of prior communicative systems. It would follow from this that the most likely means of expression, when this cognitive infra\-structure finally emerged as a communicative system in its own right, would have been one which was quite separate from, and unlikely to be confused with, the prior system. True, both hominid calls and hominid proto-words would have been in the \isi{vocal channel}, but the acoustic ranges of the modern ``call system{\textquotedbl}-shrieks, laughter, etc.\-
and those of speech sounds do not overlap and very likely have never overlapped.

Once the Rubicon was crossed, progress may well have been rapid, a matter of a few generations, since the necessary infrastructure for a fairly rudimentary level of language would have already been in
I
%\originalpage{2}
place. Chimps have progressed (with training, granted) from one-word to several-word utterances in a matter of months, so Isuspect that the one-word, two-word, etc., stages of early child development do not necessarily reflect stages in adult language development, but rather rehearse cognitive growth stages in the hominid line that PRECEDED
the emergence oflanguage. Not a lot turns on this, either way, and even how we would decide between the two alternatives is at present very far from being clear. But somehow the idea of our ancestors communi\-cating via one-word utterances for several millennia while awaiting the growth of the requisite neurological infrastructure (whatever that might
have been!) that would permit them to add word two to word one falls short of being wholly persuasive. In the absence of any evidence to the contrary (but bearing in mind the possibility that such evidence might appear at any \isi{time}) we will conclude that in the first flush of language, our ancestors were able to get about as far as chimps have; that is, they could:


\begin{itemize}
\item Lexicalize concepts corresponding to classes of sensorily perceptible entities and sensorily perceptible attributes of entities (things like \textit{color} and \textit{size} as opposed to things like \textit{courage} and \textit{justice} \textit{).}
\item Lexicalize secondary concepts by conjuncts of primary con- ·· cept names.
\item Organize brief (up to 3-5 word) utterances on a predominantly topic-comment basis (i.e., proceeding from the proximal to the distal, the old to the new, more or less irrespective of case\-role relations).
\item Despite (c), distinguish in a pinch between X-Vs-Y and Y-Vs-X sequences (e.g., form appropriately, and react appropriately to, the difference between \textit{Roger} \textit{tick} \textit{le} \textit{Lucy} and \textit{Lucy} \textit{tick} \textit{le} \textit{Roger,} in at least a majority of cases).
\end{itemize}

On the other hand, it is likely that they, in common with modern primates, were not able to: ,

%\originalpage{9}

%\setcounter{itemize}{4}
\begin{itemize}
\item Produce utterances of more than one clause.
\item Grammatidze even the most basic semantic distinctions, such
\end{itemize}

as those of tense, plurality, \isi{possession}, etc.

These two capacities are, as Ishall try to show, phylogenetically lmked. Together they constitute minimal requirements for anything even approaching the kind of language we have today, and the reluc\-tance of many linguists and psychologists to accept that, lacking them, modern apes could be capable of language, is very understandable. owever, such linguists and psychologists feel under no \isi{obligation} to give an account of how language was initially acquired, which makes things easier for them, but does not do anything toward helping us to understand ourselves. What apes have, what our ancestors had, you may or my not all language, but it seems to me simply bizarre to suppose
that 1t wasn t smethmg that you had to have in order eventually to
have languages hke those of today. Those who disagree have no righ t to
d.o so. unless they can provide a more plausible route to our present s1tuat10n.
. . S.eidenerg and \citet{Petitto1979}, in reviewing (pessimistically) the h1stic achievements of apes, make the point that no convincing

evide?ce h{\textquotedbl}{\textquotedbl}. ye.t beu prvided that an ape can use a sign for any object

that 1s not m 1ts immediate environment. This is equally true of chil\-dren's language in the first few months of acquisition, of course· and indeed, with only (a)-(d) as one's resources, it is hard to see 'how reference could escape from the prison of the here-and-now. But being able to talk, if only about the here-and-now, is an immense advantage

\begin{itemize}
\item \begin{itemize}
\item ;•c for children, and was presumably at least an equal advantage for our forefathers, over not befog able to talk about anything at all. You culd convey. highly specific warnings, bring about cooperative beha\-v10r,. settle disputes, even construct primitive rituals. In rituals, dis\-placement begins; the head of the \isi{cave-bear} on a pole stands for all
\end{itemize}
\end{itemize}

. ;··· the cave-bears who control the warm caves you will need in order to \isi{'}\isi{'}\isi{'} gt through the next Ice Age. But it is one thing to be able to think

·• displacement, and quite another to be able to talk it.

% 270
 

Consider the following situation. You are Og. Your band has just

%\originalpage{271}

its accuracy, degree of detail, and universality or lack of it. A language

severely wounded a \isi{cave-bear}. The \isi{cave-bear} has withdrawn into its cave. Ug wants to go in after it. ``Look blood. Bear plenty blood. Bear weak. Ug go in. Ug kill bear. Ug plenty strong.\isi{'}\isi{'} You want to be able to say something along the lines of \textit{the} \textit{bear} \textit{we} \textit{tried} \textit{to} \textit{kill} \textit{last} \textit{winter} \textit{had} \textit{bled} \textit{at} \textit{least} \textit{as} \textit{much} \textit{as} \textit{this one,} \textit{but} \textit{when} \textit{l}\textit{g} \textit{went} \textit{in} \textit{after} \textit{it} \textit{to} \textit{finish} \textit{it,} \textit{it} \textit{killed} \textit{him} \textit{instead} \textit{so} \textit{don't} \textit{be such} \textit{an idiot.} Since in order to think this all you had to be able to do was to re\isi{play} the \isi{memory} of events you yourself had witnessed, I can see no reason to believe that you could not have thought it because you didn't have the words to think it in. But saying it is another story. Let's suppose you try. Since you have nothing approaching embedding, there is no way you can use a relative clause to let the others know which bear you are thinking about. Since you have no \isi{articles} or any comparable device, there is no way you can let the others know that y ou are talking about a bear that they know about too. Since you have no way of marking relative \isi{time} by automatic tense assignment or even adverbs, there is no way you can let the others know that the bear you want to talk about is one that is not here anymore. Since you have no verbs of psychological action (we'll see why in a moment), there is no way you can use the verb form itself to inform the others that you are speaking of a past \isi{time} \textit{(remind} , \textit{recall,} \textit{remember,} etc.). You can try ``Og see other bear.\isi{'}\isi{'} Everybody panics. ``Where? Bear where?{\textquotedbl} ``Bear not here.\isi{'}\isi{'} Some laugh, some get angry ; Og's up to his practical joking again. ``Bear kill Ig,\isi{'}\isi{'} you try. Now even the ones who are laughing are sneering. ``lg! lg dead! Og crazy!\isi{'}\isi{'} If you have any sense, you
shut up, or someone will get the idea to push you into the cave instead
of Ug.

It was mentioned earlier in the chapter that the power to predict the course of future events was what gave a selective advantage to those species which developed their cognitive capacities, and that this power depended crucially on the power to categorize and analyze past events. Both powers in turn depend upon the quality of the cognitive map-

that could advance beyond the initial plateau of the here-and-now could potentially do two things which would lead to an exponential increment in the survival power of the species possessing it.

First, it could code the cognitive map in such a way that process\-ing \isi{time} would be drastically reduced. One can think nonverbally, by processing images, or one can think verbally, using lexical items instead of images; in order . to utter, or comprehend, or merely mentally con\-struct the sentence \textit{John} \textit{drove} \textit{the} \textit{tan} \textit{Oldsmobile} \textit{from} \textit{Arkansas} \textit{to Texas,} it is not necessary to frame mental images of John, or driving, or Oldsmobiles, or Arkansas, or Texas. A number of psychological experi\-ments (several referenced in Hamilton [1974] ) have shown that where labels for objects are available, h uman subjects perform more effec\-tively and much more rapidly; for instance, Glucksberg and \citet{Weisberg1966} showed that solution times for label-aided as against label-free versions of a problem differed by a factor of fifteen to one. The mere fact that processing \isi{time} is reduced automatically makes possible many analyses that could not previously have been attempted. For instance, where previously there might have been only \isi{time} to model a single
hypothetical solution to a practical problem (such as that of dealing with an angry \isi{cave-bear} in its cave without getting killed in the process), there is now \isi{time} not only to model several hypothetical solutions but also to compare them and make a choice on the basis of that comparison.

Second, it could make solutions available to other members of the species. Cognitive development without the power to communi\-cate the results achieved by it may serve the survival of the individual but cannot serve the survival of the group. You could remember about the bear that killed lg, but if there is no way in which you can convey your thinking to Ug, then the odds are that although you won't get killed, Ug will, and so will lots of Ug's children and grandchildren.

True, your smart genes will multiply, while their dumb ones won't, but all that will do is bring a little bit.nearer the \isi{time} when the species will break through the here-and-now barrier and achieve, not just
%\originalpage{2}
predictability, but the dissemination of predictability- the unique capacity that launched a single primate species on its unprecedented
career
Let us consider some fairly minimal requirements that language

I
%\originalpage{273} 
with shared, old information fu:st (or zeroed) and nonshared informa\-tion second. There is now a potential conflict. Let X be old information and Y new information, and let it be the case that Y killed X. Topic\-comment order would then call for \textit{X} \textit{keke} \textit{Y,} equivalent to 'X was killed \textit{by} Y\isi{'}.However, \textit{X} \textit{keke} \textit{Y} would also correspond to the structure

must
have satisfied before it could emancipate itself from the here-

NiVNii, in which V has its \isi{causative} sense and Nj is agentive-yielding

and-now.

First, the structure of one-clause sentences must have been
stabilized. It must have been stabilized because freely variable word\-order minus case-marking devices equals growing ambiguity as two\-and three-clause sentences develop. In fact, word-order cannot have stabilized \textit{so} \textit{that} longer sentences should be unambiguous; there must have been some motivation at the single-clause level. Let us consider what such motivation might have been.

To begin with, we need to look at something apparently quite
unconnected with sentence-order- that is, word-formation. Although there is an extensive and controversial literature on the range of distinc\-tive speech sounds that early species (in particular, Neanderthal man) could have produced- see \citet{Spuhler1977} for references-there can be little doubt that that range was considerably smaller than the range of our own species. Let us assume a capacity for five consonants and three vowels (not much less than the range of modern \ili{Hawaiian}, with eight consonants and five vowels) together with CV syllable s:ructure; this would give a maximum capacity of only 15 monosyllabic words and 225 disyllabic words. Moreover, all languages we know of under\-utilize their inventories, leaving numernus lexical gaps, so that the practically attainable total would be lower still. Factors such as these would encourage use of the same lexical item in \isi{causative} and non\-\isi{causative} senses. In that case, as we saw in Chapters 2 and 3, only the frames NuV and NiVNu (where Nii is nonagentive and Ni, agentive) would distinguish \isi{causative} from non\isi{causative} senses of V.

Let us now consider a hypothetical word, \textit{keke,} which means
'die\isi{'} in the context N \textit{keke,} but 'kill\isi{'} in rhe context N \textit{keke} N. We have assumed that the first word-order in early language was topic-comment,
an alternative reading, 'X killed Y\isi{'}. In theory the conflict might be
resolved by adopting either strict topic-comment or strict SVO order, but since the latter holds less chance of ambiguity than the former, and is therefore fractionally more economical in processing \isi{time}, we can assume either that it was universally adopted or that those languages that failed to adopt it died without issue. In fact , languages that did
fail to adopt SVO must surely have died out when the strict-order
languages achieved embedding and complex structure; it is tempting, although quite futile, to speculate that what caused the large size of the Neanclerthal cranium was the apparatus needed to process (and store in short-term \isi{memory}) multiply-ambiguous parsings of multiclause sentences in which the constituents were not systematically ordered.

If we accept Lamendella's hypothesis, there is support for the foregoing picture from acquisition processes. Bever ( 1970) has demon\-strated the existence of what he terms ``Strategy C{\textquotedbl}{}-{\textquotedbl}Constituents are functionally related internally according to semantic constraints{\textquotedbl}- and ``Strategy D{\textquotedbl}{}-{\textquotedbl}Any Noun-Verb-Noun (NVN) sequence within a potential internal unit in the surface structure corresponds to actor\-action-object.\isi{'}\isi{'} Experiments carried out by Bever and his associates indicate that children between two and three rely on Strategy C to comprehend sentences but that a little later they switch to Strategy D. This serves to explain the otherwise quite baffling fact that children's performance with regard to sentence types which involve nonagentive initial NPs (passives, clefts) actually deteriorates rather than improves between three and four.

The acquisitional sequence Strategy C-Strategy D would then replicate stages in the early development of language. Strategy C would have had to be developed in order to interpret case roles in the stage
%\originalpage{274}
in which topic-comment ordering was dominant. Strategy D would have succeeded it as soon as sentence-order stabilized and became the pri\-mary marker of case relations. Note that, originally, adoption of Stra\-tegy D would have had none of the dysfunctional side effects that it does nowadays with children acquiring \isi{English} since, at that \isi{time}, there were no passives and no clefts; Strategy D would have given the right answer every \isi{time}.

We can assume that neural modifications accompanied the change. What these may have been is still beyond anyone's power to determine; what they would have had to be able to accomplish is slightly less opaque. There is no evidence that SVO ever got hardwired into the system, but assignment of case roles must have become auto\-matic, and underlying this must have been a hierarchy of cases with the rank-order, \textit{agent-experiencer-patient-su}\textit{b}\textit{jecthood }in any given sen\-tence being assigned to the highest-ranking case in that sentence. Also, the ape experiments are anything to go by, speech would have had
to be speeded up considerably (Rumbaugh and Gill [197 6] report a human-chimp conversation of only twenty-one sentences which took nine minutes) : more rapid processing would presumably have required qualitative as well as quantitative increases in neurons and neuron connections.

So far we have assumed a limit of two case roles per sentence. However, this does not mean that there were only two case roles in the h ominid repertoire Insofar as :ipt{\textgreater}s can use tools and give thinp;s to one another, one must assume that cases such as instrumental and dative are potentially within their grasp. But problems arise once a third case role is added.

Any two case roles can be ordered around V so that the higher of the two precedes V and the lower follows. But presence of a third means that two NPs must be conjoined. This creates \isi{parsing problems}. If you want to say something like \textit{Ug} \textit{gave} \textit{Jg's} \textit{meat} \textit{to} \textit{Og,} there are three possible ways in which you can overcome these problems.

You can indicate the oblique cases with prepositions, or post\-positions, or some other purely grammatical case-marking device.

% ORIGINS 275

ill \textit{Ug} \textit{gave} \textit{Jg's} \textit{meat} \textit{to} \textit{Og,} \textit{'s} marks the genitive and \textit{to,} the dative case. I t 1s perhaps possible, but highly unlikely, that our predecessors could have invented grammatical case marking ah ovo; even in many syn\-chronic languages, case markers can be traced back to original content which have been bleached of their original semantics and down\-

·•cr••.{\textless}u·:u from their original syntactic roles.

In the absence of grammatical marking, you can simply string case roles together and hope that Strategy C-which must simply be wverridden, not erased, by Strategy D-will suffice to parse the result. In most cases it may, but in many it will not. The sentence introduced above, for example, would come out as \textit{Ug} \textit{give} \textit{Ig} \textit{meat} \textit{Og,} which might be parsed as 'Ug gave Ig's meat to Og\isi{'} buf could also be parsed as 'Ug gave lg the meat of Og\isi{'}. Note that parsing mistakes in \isi{discourse} must be cumulative ; the listener who thought lg got Og's meat and the listener who thought Og got Ig's meat would put quite different con\-structions on the sentences that folfowed.

The third alternative would be to preserve the two-case-roles-per\-scntence restriction and conjoin sentences: \textit{Ig} \textit{have} \textit{meat,} \textit{Ug} \textit{take} \textit{meat,} \textit{Ug} \textit{give} \textit{Og.} This is cumbersome but unambiguous. But note that if the second occurrence of \textit{Ug} is omitted, you get \textit{Ug} \textit{take} \textit{meat} \textit{give} \textit{Og\-} the same serial structuring of dative-incorporating sentences that we found as a frequent feature of creoles in Chapter 2.

In fact, \isi{verb serialization}, probably arising out of paratactic con\-junction plus equi-deletion, represents the only plausible means by which early language could have broken out of smgle-clause structure. It is difficult for us now to appreciate the magnitude of the advance that was involved. The single-clause apelike proto-language was, as we have said, almost certainly limited to dealing with physical activities in the here and now. Sentences representing mental activities almost always demand more than one clause. If we say that something hap\-pened \textit{when} something else happened, or will happen \textit{if} something else happens, or happened \textit{because} something else happened, we are repre\-senting not something that we have perceived directly through the senses, but the result of some kind of mental computation performed
%\originalpage{276}
on sensory inputs (or, to be more precise, on things that originated as sensory inputs but that have already had a lot of processing done to them, along lines suggested earlier in this chapter). Still more clearly, if we \textit{remember,} or \textit{believe,} or \textit{think} , or \textit{hope,} or \textit{expect} that something happened or will happen, or if we \textit{want} or \textit{hope} or \textit{decide} to do some\-thing, we are again directly representing a mental operation on the product of past inputs or the projected product of possible future ones, and in either case, one that cannot be expressed in a single clause. The gap between monopropositional sentences and \isi{multipropositional sentences} is the gap between talking only about observables in the external world and communicating the contents of one's mind. And the bridging of that gap must have constituted the greatest single step in what anthropologists mean by the ugly terms ``hominization\isi{'}\isi{'} or ``sapienization{\textquotedbl}{}-the process of becoming the kind of species that \textbf{we} \textbf{now} \textbf{are.}

Verb serialization helped to bridge this gap, and the way in which it accomplished this is worth looking at a little more closely. At first glance, the results of \isi{verb serialization} may look a little like those underlying structures that were once proposed by generative semanti\-cists in which verbs were distintegrated into what were supposedly \isi{primitive concepts}; perhaps the most widely discussed of these was the
proposed derivation of \textit{kill} from \textit{cause} \textit{to} \textit{become} \textit{not} \textit{alive.} Similarly,
if one found a language that expressed the meaning of \textit{bring} \textit{the} \textit{book} \textit{to} \textit{me} as the equivalent of \textit{carry} \textit{the} \textit{book} \textit{come} \textit{give} \textit{me,} it might seem that sentences of the latter type reflected the absence of a means for generating derived lexical forms. Such a view might lead to the conclu\-sion that there were two \isi{types of} action, a type that was ``semantically complex\isi{'}\isi{'} (capable of being broken down into primes) like \textit{kill }or \textit{bring,} and a type that was ``semantically simple\isi{'}\isi{'} (its members being themselves primes) like \textit{cause} or \textit{carry.}

Such a view is, Ithink, incorrect. There is probably no action verb which is either intrinsically simple or intrinsically complex in the ways suggested above. How actions came to be lexicahzed is something which, like so many other things of equal or greater importance, we
%\originalpage{277}
have had to skim over or ignore altogether in an account as compressed as this one. However, we need to note that verbs are abstractions from sensory input in a way that nouns are not. At first glance, one might think that the referent of a verb like \textit{hit} was as unambiguously unitary as the referent of a noun like \textit{dog,} although in fact \textit{John} \textit{hit} \textit{Bill} could be rendered more accurately (if more circumlocutiously ) as \textit{John} \textit{clenched} \textit{fist} \textit{John} \textit{drew-back} \textit{a}\textit{r}\textit{m} \textit{John} \textit{thrust}\textit{{}-}\textit{forward} \textit{arm} \textit{fist} \textit{met} \textit{Bil[} ln fact, there are perhaps no ``semantically simple\isi{'}\isi{'} verbs that could not be represented in a ``semantically complex\isi{'}\isi{'} way, and vice versa. What determines whether a particular referent action is repre\-sented by one verb or more than one is nothing to do with semantic complexity, but has a lot to do with the number of case roles the action involves. It is precisely those actions which involve a number of case roles that are singly lexicalized in prepositional languages, and multiply lexicalized in verb-serializing languages.

The aid supplied by \isi{verb serialization} in bridging the gap between monopropositional and \isi{multipropositional sentences} had nothing to do with semantics. Verb serialization simply made available structures more complex than had existed hithertostructures that added the possibility of NVNVN to the previous NV and NVN structures.

Now we need to consider how the representation of mental activities could have commenced. We have a syntactic bridge, but we also need a semantic bridge. Ishall propose that the semantic bridge was provided by two classes of verbs: verbs of reporting and verbs of perception. Both represent actions that are in some sense ``more phy\-sical\isi{'}\isi{'} than the true \isi{psychological verbs} of thinking, hoping, remem\-bering, etc. Both entail dual-propositional sentences. Both are likely to be of high utility in hunting-and-gathering communities where members frequently split up in their search for food and need to con\-vey to the others information about their degree of success in that search. Ishall not attempt to determine priority as between these two classes.

Verb-class membership would now become critical in parsing.

%\originalpage{27}

The developing grammar would generate NVNVN sequences, but these would be ambiguous between two interpretations, e.g., NV[NVN] or NVN[ (N)VN] (where the constituent in parentheses had been equi\-deleted). However, if the first V was one of perception or reporting, the second N would be subject of the second V; if the first V belonged to some other class, the second N would be the object of the frrst V, and the sentence would be parsed as a serialization. When the ``true\isi{'}\isi{'} \isi{psychological verbs} came to be added, they too would follow the first of these patterns.

The development of reporting verbs would have begun at the

same \isi{time} as the development of displacement. If I report what another person said and that other person is not present, then obviously the saying must have occurred on a previous occasion. If I tell you \textit{U}\textit{g} \textit{say} \textit{honey} \textit{here,} it requires no Socratic intellect to figure out that the honey may be here now (although of course it need not be) but that the saying of \textit{honey} \textit{here} by Ug must have occurred at a previous \isi{time} (and perhaps in another place, although my capacity to translate Ug's actual utterance of \textit{honey} \textit{here} is something else that cannot simply be assumed). However, as sentences become more complex, the need to distinguish observation from computation, earlier from later, and general from specific statements must increase. Failure to make such distinctions, preferably in some quite rigorous and automatic way, leads to \isi{parsing problems} which could compound even more rapidly than \isi{parsing problems} arising from case assignment. Some scaffolding is required that will accurately fix the place of sentences in the world of \isi{time} and reality; TMA systems supply this scaffolding.

\citet[170]{Quine1960} expressed frustration and puzzlement with the fact that all sentences of all human languages must obligatorily express tense, but then, from Reichenbach on, philosophers have glaringly failed to make any kind of sense out of TMA systems. Their recipe has always been, ``Take the distinctions that are said by tradi\-tional grammarians to be made in modern \isi{English} and reduce them to some kind of a formal schema.\isi{'}\isi{'} Since the advantages, if any, of this
approach are totally opaque to me, I shall discuss it no further. From
% {\textbackslash}
% ORIGINS 279
an evolutionary viewpoint, it appears plausible that the only distinc\-tions the first \isi{TMA system} could grammaticize must have been dis\-tinctions which were somehow already implicit in the ways in which the brain processed and stored information. If certain \isi{types of} infor\-mation were already stored in different places or in different ways, then attaching some kind of grammatical index to the products of different stores would not have presented too much difficulty. On the other hand, the only possible alternative- that the species invented cate\-gories for which there was no such preexisting infrastructure, and then either built a redundant set of infrastructures to reprocess it, or some\-how assigned marking with 100 percent efficiency in the absence of such infrastructure- is at best an improbable one.

People find this hard to comprehend because categories such as \textbf{{\textquotedbl}past,{\textquotedbl}} \textbf{{\textquotedbl}present,{\textquotedbl}} \textbf{and} \textbf{{\textquotedbl}future{\textquotedbl}} \textbf{seem} \textbf{quite} \textbf{natural} \textbf{and} \textbf{transparent.} In fact, the so-called ``moment of speech\isi{'}\isi{'} which marks the elusive ``point present\isi{'}\isi{'} which is the linchpin of Reichenbachian analysis is an abstraction which can never be experienced but can only be inferred by beings who already have produced some kind of \isi{time}-marking device. People talk about ``present\isi{'}\isi{'} as if it were somehow on a par with ``past\isi{'}\isi{'} and ``future{\textquotedbl}; but while any single point action can be in \textbf{the} \textbf{{\textquotedbl}past{\textquotedbl}} \textbf{or} \textbf{the} \textbf{{\textquotedbl}future,{\textquotedbl}} \textbf{no} \textbf{such} \textbf{action} \textbf{can} \textbf{be} \textbf{in} \textbf{the} \textbf{{\textquotedbl}present,{\textquotedbl}} simply because it must already be in the ``past\isi{'}\isi{'} before you can get \isi{time} to open your mouth to talk about it. As for ``future\isi{'}\isi{'} and ``past,\isi{'}\isi{'} the former is of dubious status in that events assigned to it, however probable or plausible, are artifacts of the imagination ontologically indistinguishable from wants and wishes, however unlikely or even counterfactual, while the latter, although the most tangible of the three, suffers in utility from being internally undifferentiated.

In fact, \isi{time} presents itself in experience as an unchanging state\-it always was, is, and will be ``now,\isi{'}\isi{'} as far as the experiencing indi\-vidual is concerned- and, with an assist from \isi{memory} and \isi{prediction}, as a constant and unbroken flow pouring against us. No remotely plausible mechanisms of perception or neural processing would seem to yield the neat bisection of \isi{time} into two equal portions divided by a
%\originalpage{280}
constantly moving point which constitutes the ``commonsense\isi{'}\isi{'} analysis of \isi{time}-imprisoned Western man. \textsuperscript{11} Far different is the case for the distinctions to be argued here.

I shall propose that if the distinctions of ±anterior, ±irrealis, and
±nonpunctual are the TMA distinctions consistently made in creole languages, and if these distinctions struggle to emerge, as they seem to, in the course of natural language acquisition, then they represent the primary TMA distinctions made in the earliest human language(s), and appear in all three places because of their \isi{naturalness}. Since the word ``\isi{naturalness}\isi{'}\isi{'} has been subjected to so many abuses, I had better make very plain what I mean by it in this context. The \isi{naturalness} of a disc tinction is assumed to be an all-or-nothing characteristic and not a matter of degree. A distinction is natural just in case it corresponds to a difference in the mode of perceiving, processing, storing, or access\-ing data in the brain, such difference in turn depending on specific features of the brain's physical structure. It is assumed that only dis\-tinctions of this kind can be candidates for primary grammaticization.

Quite obviously, in the present state of our knowledge, any claims about brain structure can have no more than hypothetical status. This fact is no excuse for imitating the ostrich, as does Muysken when he claims (1981a) that an earlier and much sketchier account of this area\textsuperscript{12} {\textquotedbl}will remain arbitrary until we know a lot more about the functioning of the brain.\isi{'}\isi{'} We will not know a lot more about the functioning of the brain until we have made and compared and evalu\-ated a lot more models of the brain along the lines of the one I have tried to construct in this chapter. The idea that scientists ``increase knowledge\isi{'}\isi{'} simply by ``finding out facts\isi{'}\isi{'} in the absence of any kind of theoretical \isi{model-building} is an illusion which, remarkably enough, flourishes only among those who themselves deal with mainly theore\-tical issues, while it simply does not exist among workers in the physi\isi{'}\isi{'} cal sciences, who so mnch take for granted the interaction of specula\-tive models and empirical findings that they never even see the need to defend their methods. In fact, it is quite conceivable that we could track every dendrite to its synapse and still not have the faintest idea
% {\textbackslash}
%\originalpage{1}
how the brain worked, just because we had no adequate model of what all its electrical and molecular activity might be designed to accomplish. Therefore, to try to rebut the claims made here with the mere cry of ``You can't prove it!\isi{'}\isi{'} is as irrelevant as it is redundant. Those who disagree with the model presented here, which may indeed be wrong
in detail or even in totality, have no recourse \_other than to construct better ones.

Muysken was right, however, in poumng out that my earlier account had failed to explain the syntactic ordering of TMA markers, and accordingly, the present account will remedy that def!Ciency. Let tis review some relevant evidence. We know that the ordering of mark\-ers in creoles is always TMA, anterior-irrealis-nonpunctual. We know
that in VO languages, as we are assuming the primordial language(s) to have been, free verbal elements which modify the meaning of the main verb usually precede the main verb. We know that the commonest source for TMA markers in creoles (and in other languages) is that of former full lexical verbs. \textsuperscript{13} We will assume that whatever distinctions the original markers may have made, the markers themselves were
derived from full lexical verbs. We will further assume that the markers
must have been added in some order, that is to say, they were not ai!ded all at the same \isi{time}. These two assumptions seem to me to be unexceptionable.

Almost as unexceptionable is the assumption that when the first marker was added , it became an immediate constituent of the verb, a.rid. hence any markers added subsequently would have to be posi{\textquotedbl}

J ti6ned externally to the unit formed by the verb and the first marker.

{\textless} Certainly, it is hard to think of any motivation there could have been for inserting a new marker BETWEEN the original marker and the verb. Granted, we cannot prove that this did not happen. But there are ways in'Yhich the assumption can be tested.

Earlier in this chapter it was claimed that between any pair of

··•·:it:...•·•disldil{\textless}:ti{\textless}ms, the distinction whose \isi{neural infrastructure} had been laid

\isi{'}.fi({\textbackslash} dlow'n earliest in phylogeny would be the first to be realized in language.

%\originalpage{282}

Also, by Lamendella's hypothesis, the distinction first realized in language should be the first to be realized in the \isi{acquisition of} lan. guage. Thus, in principle, we have two different ways of testing the assumption made above about adding order, two ways that are coin. pletely independent of one another: if both yield the same answer, then this constitutes mutual support for the two hypotheses, and if the answer yielded by both is the answer yielded by ou.r (indepen\-dently motivated) assumption, then further support is provided for the assumption.

According to that assumption, the nonpunctual marker, being always closest to the verb, represents the first of the three distinctions to be grammaticized. Therefore, punctual-nonpunctual should be the first of the three distinctions to acquire the appropriate neural infra\-structure, and it should also be the first to be acquired by children.

In Chapter 3, we surveyed a considerable body of evidence which suggested that whatever children might be thought to be \isi{learning} (and sometimes they might be thought to be \isi{learning} past-nonpast before other distinctions), what they really learned first was punctual\-nonpunctual. Our second criterion is thus satisfied. With respect to the frrst, let us consider possible neural infrastructures for punctual\-nonpunctual. One of the earliest neural structures known to us is that which underlies the phenomenon known as \textit{\isi{habituation}. }In \textit{\isi{Aplysia},} a sluglike marine mollusk whose nervous system contains only a hand\-ful of ganglia with a few hundred neurons each, the sensitive organs are extruded from a mantled cavity and consist of a gill for breathing, a siphon for eating, and a purple gland. The last two serve as primitive organs of perception. If anything touches the siphon or the gland, the gill retracts into the cavity. However, if you touch either gland or siphon at regular, brief intervals, the withdrawal response will diminish in both speed and intensity until eventually it is extinguished. \textit{Aplysia} has done its equivalent of deciding that your actions are nonthreatening and thus it is wasteful to respond to them.

\textit{\isi{Aplysia}'s} actions are of course entirely automatic and represent the workings of a mechanism whih has evolved in the vast majority of
%\originalpage{3}
animate creatures to prevent them from being wholly at the mercy of every external stimulus. It enables them to disregard irrelevant stimuli and reserve their energies to react to imminent danger or to seize feeding opportunities. If mechanisms such as this go back as far as mollusks, they antedate by some hundreds of millions of years any mechanisms that might underlie the other two TMA distinctions.

Clearly, \isi{habituation} mechanisms have grown considerably more sophisticated since \textit{Aplysia} emerged. Each of our senses has mechanisms that filter abrupt and sudden outside stimuli, which require immediate action on our part, from ongoing or persistently repeated stimuli, which do not. Driving down a crowded street, we are not at all perturbed by the constant movement of countless pedestrians, but let a ball bounce off the edge of the pavement, and (hopefully ) we instantly slam on the brakes in anticipation of the child that may follow it. Working alone in an empty house, we pay no attention to sporadic noises in the street outside or the background murmur from the freeway a block or two off; we probably do not even consciously hear these things if we are engrossed in what we are doing; but let the slightest sound come from the rooms around us, and we stop whatever we are doing and become instantly alert. After the event we may tell the story as if we DECIDED to brake or DECIDED to attend to the strange sound, but these are post hoe rationalizations of our neuronal watchdogs\isi{'} purely autono\-mous activities. In other words, the sorting of punctual from non\-punctual actions is done for us automatically, and it seems reasonable to suppose (although it is still far from provable) that \isi{percepts} in the \isi{memory} store are somehow coded with reference to whether they otiginated from sets of neurons specialized for perception of punctual events or from sets specialized for nonpunctual ones.

This capacity to make the punctual-nonpunctual distinction in real life, so to speak, must have been crucial to our ancestors, inter\-mediate as they were between those who might prey on them and those on whom they might prey. It is hard to think of any distinction
which could have been more important for them to make as they began to build up the store of communal experience that would become
%\originalpage{284}}
the traditional wisdom of human groups. The interrelation of punctual of the external world or whether they lacked any such origin.
and nonpunctual, foreground and background, provided basic ways of At whatever point our ancestors achieved the power to con- analyzing and classifying the diverse experiences with natural forces and sciously manipulate the \isi{memory} store in order to generate hypothetical with other species which could now be handed on from generation to future events or ``might-have-beens,\isi{'}\isi{'} a new dimension would have generation, growing as it spread through \isi{time}, yielding knowledge of been added to irrealis, but the essential coding difference would not a wide range of phenomena and the actions appropriate in the presence have been affected; whatever was internally generated would be coded of those phenomena. differently from whatever was externally generated. Nowadays, of Alone of the three TMA distinctions, punctual-nonpunctual course, most of us can consciously and quite explicitly distinguish, if correlated {\textbackslash}vith observable phenomena. The reaJig,-irrealis distinction required to do so, between events that really happened and events contrasts observed events with events that are unobservable, at least that are the product of \isi{dreams}, desires, wishes, and expectations; if at the \isi{time} of speech ; the anterior-nonanterior contrasts, not events we cannot, we are separated from the remainder of the species and at all, but the relative timing of events, a high!y abstract feature. maintained in institutions specializing in this and similar conditions Punctua!-nonpunctual, however, can be directly observed whenever a until such \isi{time} as we recover the capacity. To be able to tell realis single action interrupts a more protracted or a repeated one. It could from irrealis is a crucial part of being fully human. But the foundation therefore have been grammaticized at a stage when only physical of this capacity and of our capacity to mark verbs in a way appro- objects or events were capable of being grammaticized or lexicalized. priate to the status of their referents must be the same: some kind of Whether or not grammaticization took place this early, all the evi- neural coding of \isi{memory} items that reflects internal as opposed to

dence suggests that punctual-nonpunctual was the first TMA distinction external source.

to be grammaticized, and accordingly, the form that marked the Unfortunately, in the case of realis-irrealis and anterior-non distinction would have been juxtaposed to the main verb. anterior, we cannot draw the evidence that we drew from child acqui{}-

To find the evolutionary ancestry of the second distinction, sition in the case of punctual-nonpunctual. While it is highly possible realis-irrealis, we need to know the earliest source for items in the that children make irrealis distinctions before they make anterior \isi{memory} store that did not originate in the organs of perception. In all distinctions, the point is not easy to prove since the lack of correspon- probability that source was dreaming. Dreaming began with mammals- dence between the bioprogram \isi{TMA system} and the systems of most reptiles, as far as we are aware, do not dream-and its origins and evo- target languages is such that it is by no means easy to tell when children lutionary function remain mysterious. One explanation treats \isi{dreams} have acquired whatever forms may correspond to irrealis and whatever as ``providing for better building of the \isi{memory} model by continued fom!s may correspond to anterior. It is true, for instance, that children operation of the mechanism for memorizing during the night, even \isi{learning} \isi{English} acquire futures long before they acquire pluperfects, when no further information from external sources is available\isi{'}\isi{'} (Young but since future does not correspond one-to-one with irrealis and plu- 1978:209). Certainly , \isi{dreams} appear to permute actual experiences in perfect does not correspond one-to-one with anterior, it would be
ways that produce novel constructs. However, we have hypothesized unfair and unrealistic to base any claims on this fact alone. Hopefully, that items in the \isi{memory} store are coded in ways that reveal the source studies of \isi{acquisition of} those creole languages which preserve the
of each item. If this is so, all items in the store must have (at least) i; 6rigil1al distinctions fairly intact, as well as more sophisticated studies
coding which will indicate whether they originated from perceptions the \isi{acquisition of} other \isi{types of} language, may be able to provide \isi{'} needed evidence.

%\originalpage{2}

However, even in the absence of such evidence, it seems likely that anterior was the last of the TMA distinctions to be added. In order to make the distinction, the order of past events has to be accessible. Perhaps all creatures that have memories have mechanisms, as we must, by which the order in which memories are laid down corresponds with the order in which the relevant experiences occurred. However, recover\-ability of that order is another matter, for any kind of recoverability entails volitional manipulation of the \isi{memory} store, and there is no reason to suppose that volitional manipulation preceded nonvolitional manipulation (i.e., dreaming). Thus, the antecedents of anterior are almost certainly more recent than the antecedents of irrealis.

Furthermore, the utility of anterior as a category would be unlikely to arise until \isi{discourse} had become fairly complex. Anterior marking is primarily a device which alerts the listener to backward shifts of \isi{time} in a narrative or a conversation, thus enabling him to preserve the correct sequence of reported events-a must if features such as causality are to be extracted from it-even when the reporting diverges from that sequence. Thus, not only are the mechanisms under\-lying anterior probably more recent than the mechanisms underlying irrealis, but the functional need for anterior is almost certainly more recent than the functional need for irrealis.

If this is the case, we can claim that according to four sets of criteria-age of infrastructure, age of functional utility, \isi{time} of child acquisitions, and sequence within Aux-the three basic TMA distinc\-tions are ranked in the order: nonpunctual first, irrealis second, and anterior third. \textsuperscript{1}\textsuperscript{4} Note that the surface order of tense-modality-aspect which this yields is not only the order of creoles but also what has been assumed from \citet{Chomsky1957} on to be the underlying order for \isi{English}, and perhaps other languages too.

Although considerations of space have prevented us from survey\-ing a number of other features-such as the development of pronouns, pluralization, \isi{movement rules}, etc.-which must have accompanied or closely followed the developments actually described, we have carried our account of the early history of language to a point at which, in its
% {\textbackslash}
%\originalpage{287}
degree of complexity, it can have fallen but little short of an early creole language. In other words, we have brought language close to a point at which, for all practical purposes, the biological development of language ceased, and the cultural development of language began. I have not even attempted to provide a \isi{time} scale for these develop\-ments, either absolute or relative; they may have been spread out over two or three million years or they may have come in a series of bursts or even (though this is intrinsically less likely ) in a single explosion of creativity.

Although at present we can do little more than guess, the sugges\-tion by \citet{Hockett1973} that there might be some connection berween the emergence of fully-developed language and the sudden and ex\-tremely rapid series of cultural changes that were initiated some ten thousand years ago is quite an appealing one. There is something inherently implausible in the idea that an evolutionary line which had existed for countless centuries within the hunting-and-gathering frame\-work in which some members of our species still exist should suddenly begin to grow crops, herd animals, build permanent settlements, con\-struct complex belief systems, and evince countless other behaviors typical of our species, and highly atypical of all others, merely because certain small areas had exceeded their carrying capacity (if indeed they had). With all species, areas exceed their carrying capacity from \isi{time} to \isi{time}, and the result is always the same-the species moves, if there is anywhere to move to, and if it is not an unduly territorial species; otherwise, individual members of the species die off until the balance of nature is restored. Similar experiences must have happened to our ancestors countless times and in countless places during the \isi{Pleistocene}, with its sudden and extreme changes of climate, but the responses must always have been the same: migration or population loss.

It seems likely that agri\isi{culture} commenced not as a reaction to climatic change, population imbalance, or any other external cause, but rather as a result of vast changes in the computational and com\-municative power of the species. Dearth was feared rather than experi%
%\originalpage{88}
enced, and plans were made to prevent it. The shift from taking what nature provided, an attitude characterizing all previous species, to attempting to control nature was a vast one involving the power to construct an imaginary future and then communicate that construct to others so that concerted efforts could be made to realize it. Such attempts could hardly have been carried out without the aid of a language developed at least to the extent that we have envisaged here; but if such a language long antedated the birth of agri\isi{culture}, how was it that that and all the other arts and sciences were not born far earlier than in fact they were?

At present, no adequate answer can be attempted. In any case, the precise dating of the events detailed in this account is really irrele\-vant. Some series of events such as have been described must have happened at some \isi{time} during the last couple of million years or so for our ancestors to have passed from a state of no language to a state in which there existed languages recognizably similar to those of today. When those events occurred is a matter of legitimate interest, but not one that can affect either positively or negatively the validity of the foregoing account.

No one can be more acutely aware than I that the account given here is provisional, hypothetical, and can at best serve as no more than a rickety bridge between our present condition of almost total ignor\-ance and some future state in which we may have at least a handful of relative certainties to build upon. However, the purpose of this chapter never was to write a definite prehistory and early history of language, but rather to show that, first, a series of capacities that might be plausibly held to have been latent in our last speechless ancestors, plus some capacities that could plausibly have evolved in the course of constructing a linear vocal language, could have yielded something recognizably similar to an early creole language, and second, that on the basis of what we at present know about our own species, such an outcome-a creole-like language-would have been intrinsically likelier than any other kind of possible language. The test of such an account lies not in whether this detail or hat detail of it may be proven true
%\originalpage{289}
or false, but in whether or not it proves possible to build better (more plausible, more detailed, more explanatory ) models.

If the present model is in essence correct, and if a creole-like language was the end product of a long period of biological \isi{evolution}, then the overall capacity to produce languages of this type (itself a composite of neural capacities that preexisted any kind of language and neural capacities that were added as language evolved) \textsuperscript{1}\textsuperscript{5} must at that point (and for the rest of the life of the species, it should go without saying) have formed a part of the genetic inheritance of every individual member of the species. It would then unfold, as we have claimed, as part of the normal growth development of every child\-in most cases, being quickly overlaid by the local cultural language, but in a few, emerging in something not too different from its original form. It would merely require triggering by SOME form of linguistic activity from others-how much, and of what kind, remains one of the most interesting \isi{questions} we can ask about language-which is why wolf children, who share our biological inheritance, cannot speak, and why the interesting experiments of Psammetichus, James IV, Frederick II, and Akbar the Great all failed.

It is not without some interest that the account given here resembles, in some respects, the Biblical account of language. The Bible claims that language is a divine gift. This account can offer no objection to such a belief, assuming that God has chosen to work through evolutionary process; certainly, both accounts firmly reject the suggestion that language was in any sense a conscious or deliberate human invention. The Bible claims that our species originally spoke a single language. This account claims the same, with a slight qualifi\-cation: the issue of whether language frrst arose in one group or in several independently is entirely irrelevant since, assuming the latter, all groups would have had the same neurological equipment, and thus their languages, although perhaps differing in lexical choices (as modern creoles do, for that matter) would have been structurally identical or almost so. The Bible claims that human language diversified coinci\-dentally with a sudden surge of technological capacity, symbolized
%\originalpage{290}
by the erection of the Tower of Babel (a tower aimed at reaching heaven, i.e., usurping powers over nature which were properly part of the divine prerogative). This account would also claim (and I will develop the claim a little further in the next paragraph) that human language diversified as a direct result of rapid cultural and technological diversification, aiming, consciously or not (and in our \isi{time} it has become a conscious goal) at ``The Conquest of Nature:\isi{'}\isi{'} I would be the last person to adduce Scriptural authority in support of a scientific theory, bu!: the resemblances are intriguing, to say the least.

The question most frequently asked about the theory presented in this volume is: ``If our biological inheritance provides for us a ready\-made language, so to speak, how is it that we ever abandoned that language in favor of the diverse and far more complex languages of today ?\isi{'}\isi{'} The answer is that, in a sense, the biological language self\-destructed. It had made possible the construction of cognitive maps more detailed and complete than those available to any previous species, maps which enabled their users to enter what was in effect a wholly new cognitive domain, a domain, in which events could be predicted and forestalled and even altered rather than passively endured as all previous species had endured them. It had conferred on our species the power to LIVE DIFFERENTLY {}-differently from the past, and differently from one another.

So, differently was how they lived. Previously, as in all other species, our ancestors had all lived roughly the same kind of life; if they happened to live near a mud flat, they would include shellfish in their diet; if they didn't, they wouldn't; and that was about the extent of the difference. Now, some went on hunting and gathering and some became pastoralists and some became cultivators and some founded cities and lived by farming other people. New needs arose. New cate\-gories were established to take care of those needs. Some groups found it convenient to code verbs in such a way that the evidential status of any remark was immediately apparent. Some groups found it con\-venient to code nouns in such a way that the major \isi{semantic classes}
%\originalpage{291}
to which they belonged were immediately apparent. These new cate\-gories were superimposed on the old ones, but a language is a system or it is nothing, so that this superimposition shifted and distorted the older, more ``natural\isi{'}\isi{'} categories and in some cases, perhaps, overlaid them completely. This, too, was natural, in its way. No biological language could have been designed to su'it the needs of all humans under all the different circumstances in which humans could live; indeed, if any such language could have been designed, it would either have been itself subject to change (since cultural \isi{evolution} is not a closed process) or if not so subject would have been positively dys\-functional, since it could not have adapted to our changing needs and priorities. Thus, one hundred centuries of cultural change and develop\-ment have produced the world of diverse, yet underlyingly similar, languages which we know today.

But not only cultural factors served to change the bioprogram language. Factors concerned with language processing are also opera\-tive. I will illustrate just two different \isi{types of} such factors here.

The first involves relative clauses. As we saw in Chapter 1, HCE has no surface marker of \isi{relativization}, even where \isi{English} obligatorily requires one, provided that there is a head noun. If there is not a head noun, then an \isi{English} relative pronoun is supplied. Thus, we get headed relatives like \textit{da} \textit{gai} \textit{gon} \textit{lei} \textit{da} \textit{vainil} \textit{fo} \textit{mi} \textit{bin} \textit{kwot} \textit{mi} \textit{prais} 'The guy WHO was going to lay the vinyl for me had quoted me a price\isi{'}, with no marking, but headless relatives like \textit{hu} \textit{go} \textit{daun} \textit{frs} \textit{iz} \textit{luza} \isi{'}(THE ONE) who goes down first is the loser\isi{'}, with an \isi{English} relative pro\-noun. Obviously, the difficulty of incorporating \isi{English} relatives per se cannot be what is responsible for sentences of the first type. Rather, the cause must be, first, that all HCE sentences require some kind of overt subject (except imperatives, of course), and, second, that as we hypothesized in earlier chapters, HCE lacks-{\textquotedbl}used to lack\isi{'}\isi{'} might be more accurate-a rule that would rewrite NP as N S, but possesses a rule that would rewrite NP simply as S, thus yielding the structure NP(NP V X] VP for both the above sentences.

%\originalpage{2}
However, as was shown in Bever and \citet{Langendoen1971}, zero relative pronouns in sentences where the head noun is subject of the relative clause can cause serious ambiguities in a minority of sentences. Practically all creoles now have some kind of relative marking, pre\-sumably as a consequence of such processing problems. Thus, change away from the bioprogram pattern can set in very quickly even whee it is not triggered by language contact, if the functional pressure is sufficient.

The second factor involves \isi{word order}. It has been claimed here that the original language order was SVO with serialization but that this order was not necessarily hard-wired. This directly contradicts a claim by Giv6n (1979:Chapter 7) that the original language order was SOV. Giv6n's evidence is that a majority of the world's language families are either synchronically SOV or reconstruct back to SOV, while change in the reverse direction is rare. But in fact, serial SVO often forms an intermediate stage between SVO and SOV in Austro\-nesian languages which are changing under the influence of \ili{Papuan} languages \citep{Bradshaw1979}. In our original language, a similar change could have come about in the following manner: first, there occur a number of NVNVN sequences in which the final N is realized as a pronoun; second, object pronouns become diticized; third, the first ``. is reanalyzed as a preposition. In this way, a structure that was .or\-ginally analyzed as Subject-Verb-Object-Verb-Object changes unt il it can be reanalyzed as Subject-Preposition-Oblique Case-Verb- SXV, in fact. This is then interpreted as the canonical order, and any foll\-NP objects left behind the verb are moved in front of it in order to remove what now appears to be an anomaly. It is, of course, not necessarily implied that all early languages followed this course; but if a number of them did, then the data which Giv6n took as proof of original SOV could easily be accounted for.

There is not space here to discuss in detail how the bioprogram theory would affect the theory of \isi{linguistic change}. It shuld ?e apparent that an entire volume could easily be written on this topic.


ORIGdINS 293

The study of \isi{linguistic change} has been effectively paralyzed for many decades by the empirically groundless belief that all the world's current languages are at a similar level of development. Even the study of Greenbergian universals led only to suggestions of a kind of ceaseless seesawing between OV and VO orders. Iwould predict that, tight or wrong, the present theory would at least give something tangible for diachronic linguistics to chew on.

However, it should at least be pointed out that the present theory does not claim a steady progression away from the biopro\-grarnmed base. Quite apart from the drastic recyclings which pidginiza\-tion precipitates, there are likely to be partial reemergences of bio\-program features in a number of linguistic situations, prominent among these being, first, the constant surfacing of so-called ``subscandard\isi{'}\isi{'} varieties in classes where prescriptive monitoring is minimal, and second, contacts between typologically different languages (such as the \ili{Austronesian}-\ili{Papuan} clash mentioned above) which set in motion extreme change processes in one party or the other. Thus, despite a very rightly skeptical survey by \citet{Polome1980} which concludes that creolization may hardly ever or never have been responsible for histo\-rical changes, . there may still be some truth in the persistent claims that \ili{Germanic}, or \ili{Egyptian}, or Old \ili{Japanese} may owe some of their features to ``creolization.\isi{'}\isi{'} In fact, Polome would still be correct in claiming that true creolization had not taken place; the creole-like features would be derived from the same bioprogram that is responsible for creoles and for many acquisitional features, but surfacing under rather different and somewhat less radical circumstances than those which give rise to creoles.

We have now completed our survey of creoles, acquisition, and origins, showing the wide range of similarities that unite the first two and that could derive in both cases from the reenactment of the third. In the fifth and final chapter I shall briefly summarize the theory which these findings support, and place it in the context of existing linguistic theories, and I shall also glance at a few of the more obvious arguments that may be brought against it.

% CONCLUSIONS 295

\textit{Chapter} \textit{5}
\chapter{Conclusions}

The foregoing chapters have surveyed the three major areas of language development: development in the individual, developn:ent of new languages, and original development of language. Parsimony alo1 e would suggest that these developmental processes might have much m common with one another, and the common pattern that emerges has an independent support that no other linguistic theory that I know of could claim: it is in accord with all we have so far learned about evo\-lutionary. processes and it is in accord with all we have.so far le:irned about how processes in the brain determine the hehavior of animate creatures. During the sixties and seventies, we heard a good deal about something called ``psychological reality,\isi{'}\isi{'} although what it was was never well defined; I would suggest that whatever the fate of the theory argued here, any future linguistic theory will have to be able to claim

``\isi{biological reality}\isi{'}\isi{'} if it is to be taken seriously. • The theory argued here has claimed that many of the prerequisites for human language were laid down in the course of mammalian \isi{evolution}, and that the most critical of those prerequisites- for even things like vocal tract development were necessary, but in no sense
sufficient requirements \textsuperscript{1}{}-was the capacity to construct quite elaborate
mental representations of the external world in terms of concepts rather than \isi{percepts}. In other words, something recognizable as thought (though clearly far more primitive than developed. human thought\} necessarily preceded the earliest forms of anything recognizable as language.

Circumstances still obscure enabled our (fairly remote) ancestors

{}-perhaps \textit{Homo} \textit{erectus,} perhaps some other species{}-to lexicalize concepts and construct a primitive form of language probably not too dissimilar to that achievable, under training, by modern apes. Language even at so primitive a level conferred a sharp, selective advantage to its users. Over a long period, language developed biologically in the follow\-ing manner. In any group of any species, there is a certain amount of random variation which allows for variation in individual skill. Those individuals who had higher skills in the manipulation of language had those skills as a direct result of the fact that such random variation had produced, in their brains, mechanisms better adapted for converting preexisting mental representations into linguistic fonn by lexicalizing and grammaticizing the categories into which those representations were already sorted by neurological processes. Since language-skilled individuals possessed a higher potential for survival, they would pro\-duce more offspring than other individuals, and the capacities that had arisen in them by random variation would he preserved and trans\-mitted intact to their descendants.

Note that there is nothing particularly novel about all of this; most people nowadays would agree without any hesitation that the giraffe's neck, the hummingbird\isi{'} s bill, and all other adaptive develop\-ments ON A PHYSICAL LEVEL have originated in precisely this manner. It is merely the superstitious persistence of Cartesian \isi{dualism} that makes people reluctant to admit that since mental characteristics just as firm a physical foundation in neurological structures,
the same processes of biological \isi{evolution} must apply to them also.

If language arose in the way I have indicated, then what was passed on from generation to generation was not some vague, abstract ``general \isi{learning} capacity,\isi{'}\isi{'} or even some highly-specified ``language
%\originalpage{296}
\isi{learning} capacity.\isi{'}\isi{'} Biological \isi{evolution} does not trade in nebulous concepts like these ; it hands out concrete features, concrete capacities for specific operations. What was passed on was precisely the capacity to produce a particular, highly-specified language, given only some (perhaps quite minimal) triggering in the form of communal language use. This capacity had attained the level of contemporary creoles when the computational power it bestowed on its owners triggered the cultural explosion of the last ten millennia;\textsuperscript{2} and since cultural evolu\-tion works far faster than biological \isi{evolution}, and since it operates at a far more abstract level, the effects of cultural \isi{evolution} on language could not be transferred to the gene pool. Therefore, biological lan\-guage remained right where it was, while cultural language rode off in all directions. However, it was always there, under the surface, waiting to emerge whenever cultural language hit a bad patch, so to speak ; and the worst patch that cultural language ever hit was the unprece\-dented, \isi{culture}-shattering act of the European colonialists who set up the slave trade. But it is true that out of evil, good may come, and if they had not done this, we might never have found the one crucial clue to the history of our species.

However, even without such setbacks, cultural language could not expand away from the biological base indefinitely. Just as biology produced a floor below which human language could not fall, so it produced a ceiling above which human language could not rise. The realm of variability of any species has upper limits consisting of capa\-cities from which it is barred genetically from ever having. There can be little doubt that what we genetically have determines how far (and in what directions) we can go culturally in ways which, hopefully, will be major focal points of linguistics, philosophy , psychology, and anthropology in the decades to come. Thus, though languages may diversify and complexify, they can never become unlearnable- or if they do, children will soon pull them back to earth again.

The child does not, initially, ``learn language.\isi{'}\isi{'} As he develops,
the genetic program for language which is his hominid inheritance unrolls exactly as does the genetic program that determines his increase
% {\textbackslash}
%\originalpage{297}
in size, muscular control, etc. ``Learning\isi{'}\isi{'} consists of adapting this program, rev1smg it, adjusting it to fit the realities of the cultural language he happens to encounter. Without such a program, the sim\-plest of cultural languages would presumably be quite unlearnable. But the \isi{learning} process is not without its tensions-the child tends to hang on to his innate grammar for as long as possible- so that the ``\isi{learning} trajectory\isi{'}\isi{'} of any human child will show traces of the bioprogram, and bioprogram rules and structures may make their way into adult speech whenever the model of the cultural language is weakened.

This, then, in outline is the unified theory of language acquisi\-tion, creole language origins, and general language origins for which the present volume has amassed numerous and diverse \isi{types of} evi\-dence. The question must now arise: how does this theory relate to existing linguistic theories, and what modifications in such theories does it appear to demand?

Generative theory has now survived for more than two decades as the leading theory in modern linguistics, despite attacks from diverse quarters. Although in the course of this book I have said some harsh words about some current generative stances, it should have been apparent, first, that the theory expressed here would probably have been impossible to frame if generative grammar had never existed, and second, that there is no hostility between the two theories on major issues. The present theory \isi{complements} and amplifies \isi{generative theory}. The latter has, in fact, ceded most of the former's territory. The leading figures in generative grammar have simply ignored creoles and shown a positive antipathy to the mere idea of language origins; as for acquisition, while they have theorized about it, they have not deigned to get their hands dirty by actually examining it.

In fact, bioprogram theory and Chomskyan formal universals fit rather well together, as illustrated in \figref{fig:5}.1 on the following page. The bioprogram language would constitute a core structure for human language. Natural languages would be free to vary within the space between the outer limit of the bioprogram and the overall limit
%\originalpage{29}
imposed by formal universals, which represent neural limits-species\-specific limits-on human capacity to process language.

Limit imposed by formal universals

\begin{figure}
\caption{1}
\label{fig:5}
\end{figure}

Relationship of bioprogram to formal universals

Note, however, that the bioprogram does not correspond directly to superficially similar concepts such as ``substantive universals\isi{'}\isi{'} or (in one of its several senses) ``universal grammar.\isi{'}\isi{'} That is, it does not constitute a body of categories, rules, and structures that are necessarily shared by all languages. Indeed, above the trivial level on which all languages have nouns, verbs, oral vowels, etc., would argue that such a body could not exist. Language systems are wholes, and earlier parts necessarily get mutated to accommodate later parts. Such a statement would be wholly uncontroversial save for the hostility to process that is shown, quite gratuitously, by genertive grammar.

%\originalpage{9}

In fact, what linguistics will have to change is not \isi{generative theory}, in its essential rather than accidental aspects, but a set of much more widely held beliefs, central to which is the belief that all existing languages are at the same level of development. Beliefs that have no empirical foundation generally stem from some kind of politi\-cal commitment, and Iam sure that this one, often expressed as ``there are no \isi{primitive languages},\isi{'}\isi{'} arose as a natural and indeed laudable reaction to the claim that thick lips and subhuman minds underlie the characteristics of both creole and tribal languages. According to 19th-{}-century racists, languages and people alike were ranged along a scale of being from the primitive Bushman with his clicks, grunts, and shortage of artifacts, to the modern Western European with his high pale brow and plethora of gadgets. That was when everyone, racist or anti-racist, did believe that Western Man was superior; the only argument was about how nasty this superiority permitted him to be toward ``lesser\isi{'}\isi{'} breeds. Now that we are rapidly disabusing our\-selves of this kind of mental garbage, it becomes possible to uncouple language from ``level of cultural attainment\isi{'}\isi{'} and look at it develop\-mentally without any pejorative implications.

That there is indeed no simple connection between language development and cultural development should be obvious from just two facts. First, many peoples with hunting-and-gathering cultures have languages of horrendous complexity which seem to be a lot further from the bioprogram than ``rich cultural\isi{'}\isi{'} languages like \isi{English} or \ili{Chinese}.\textsuperscript{3} Second, creoie languages originated in the most advanced cultures of their day. I do not mean that the strains of Mozart nightly pervaded the barracoons; I mean that it was in the slave colonies that the Western powers developed the industrial technology and systems of disciplined mass labor which later, with the aid of the
capital amassed by so doing, they generously bestowed upon their own citizens. While creole speakers were working in organized bodies of hundreds or even thousands and operating complex mechanical processes, the leading technocrats of Western Europe were sitting in their own kitchens with their handlooms. So much for simplistic ``\isi{culture}-and-language\isi{'}\isi{'} equations.

%\originalpage{300}

However, old beliefs die hard, and assuredly, no matter what l say, racists will pounce on the phrase ``developmental differences\isi{'}\isi{'} and use it to suggest that in some never-to-be-precisely-specified fashion my work ``proves\isi{'}\isi{'} that creoles, or their speakers, or both, are inferior to those who \textit{s} their third person singulars and cross their as, \textit{thes,} and zeros when they come to generics. Assuredly, too, progressives, rallying indiscriminately to the struggle, will feel obliged to include this theory in their denunciations, and to accuse me of having called creoles ``primi\-tive languages\isi{'}\isi{'} and of having revived the despised ``baby-talk theory\isi{'}\isi{'} of creole origins. There is no prophylactic against ignorance. But to anyone who has read this book with even a minimum of care, it should
be apparent that the theory presented here is at an opposite pole to those which sought to derive creoles from the babyish imitations of Europeans\isi{'} condescending simplifications, and that creoles, far from being ``primitive\isi{'}\isi{'} in anything but the sense of ``primary,\isi{'}\isi{'} give us access to the essential bedrock on which our humanity is founded; their re-creation, in the face of what the \ili{French} sociologist Roger Bastide aptly termed the ``Cartesian savagery\isi{'}\isi{'} of \isi{colonialism}, repre\-sents a triumph of the human spirit, and if it were necessary to justify them in such a fashion, l could show a dozen ways in which they are more lucid, more elegant, more logical, and less easy to lie in than \isi{English} or other European languages. But I will let the dedication of this volume speak for itself.

The idea of language development is not, I would suspect, the only aspect of the present theory that is likely to arouse ideological rather than logical opposition. A great deal of human self{\textquotedbl}esteem is vested in the belief that there is a qualitative difference between ours and other species, and there is much in this volume that might be thought to weaken such a belief. Weakening such a belief, it is often claimed, may destroy ``the Dignity of Man\isi{'}\isi{'} and lead members of our species to treat other members as if they were no more than beasts.

One could ask a Tasmanian what he thought of this claim, if the advanced techniques of tral{\textbackslash}splanted \isi{English} foxhunters had
%\originalpage{301}
left any \isi{Tasmanians} to \textit{be} asked. Anyone who casts a candid eye down the perspective of human history must find it hard to explain how the idea that ``people are no more than animals\isi{'}\isi{'} could get people any worse treatment than they have gotten already. Moreover, as the discussion of Cartesian \isi{dualism} at the beginning of Chapter 4 made clear, the position of this theory is not ``Animals don't have souls, and we don't either{\textquotedbl}; rather, it is ``We have souls, and animals do too.{\textquotedbl}

The result, I should have thought, would have been to upgrade animals rather than downgrade ourselves.

Further hostility may arise from fears that the theory threatens \isi{free will} and human perfectibility. If we speak what we are biologically programmed \textit{to} speak, and if what we are biologically programmed to speak directly reflects the structure of our central nervous system, then the thoughts we think must be biologically programmed too.

If other reactions to the theory can be dismissed as knee-jerk alarm1sm, this one cannot. It is, I think, pretty likely that our think\-ing is species-specific, and therefore, almost by definition, incapable of providing adequate solutions \textit{to }the problems we see ourselves facing or of answering the \isi{questions} about the nature of the universe which we find so easy to ask. If this is so, it is so. If it is even possible that it could be so, then the appropriate reaction is not to hide behind a smokescreen of rhetoric, but to determine whether or not it is so. rf it is not so, we have a green light to go ahead with human perfectibility, despite the unpromising auguries of our previous efforts in that direc\-tion. If it is so, then we have to learn either to live within our limits
or to change those limits, if we can. For one thing is certain: if they exist, they cannot \textit{be} talked away.

Although I am convinced that future research will show the scope of human freedom \textit{to} be narrower than we had believed, and although there is no value that I personally rate above human freedom, I do not fmd myself in the least depressed by the prospect. Evolution has maintained a steady increase in the autonomy of its creatures with\-out, so far as I am aware, a single retrogressive step. We as a species may lack the infinite capacities which some members of it have thought,
%\originalpage{302}
and continue to think, that we possess, but the range of options open to us is still infinitely greater than that available to any other species, . and the peculiar powers we have inherited allow the possibility that we may one day transcend the limits of species. But we will not do this by laying claim to capacities that we do not possess. We will do it only by determining what the limits of our species are, and then decid\-ing what we want to do about that knowledge.

We may decide that less is more, small is beautiful, and that we must live within our cognitive means, even if so living entails perpetu\-ating the cycle of injustice, revolt, and more injustice which constitutes the major part of our history to date. But somehow I do not think that this will be our choice.

One recalls the TV game show in which the quizmaster asks, ``Will you take the money or open the box?{\textquotedbl} ``Open the box, open the box!\isi{'}\isi{'} the studio audience roars. I think we would try to open the box of species that encloses us, even if we knew that it was an inside-out Pandora's Box, and that once we had broken :\& ee of it, all the terrors of the universe would rain down upon our heads.


\addchap{Notes}


\textbf{CHAPTER} \textbf{1}


\begin{enumerate}
\item I have not found any evidence for rule-ordering in either 

]EfmN'aiian Creole \isi{English} or \isi{Guyanese Creole}. It would seem that rules apply wherever their structural description is met. It may be that below level of linguistic complexity, rule-ordering is not required.

topic merits further study.
 
\item The appropriate response in \ili{Hawaiian} would have been \textit{ka} \textit{ilio} 

'THE dog\isi{'}; \textit{ilio} alone is quite ungrammatical.
 
\item In fact, it is very difficult to answer substratomaniac argu\-ments because of the profound vagueness in which they are invariably couched. For instance, \citet{Alleyne1979} states: ``In dealing with the [s{\textbackslash}lbstratal] input source, we have \textit{to} \textit{make allowances} \textit{for plausible} \textit{of change }analogous to what in anthropology are called reiint{\textless},rpcret;aticms . . . . It is the \textit{failure to} \textit{make} \textit{such} \textit{allowances} that
 

\isi{'}:reduc•s the merit \textit{of} those statements that seek to refute the derivation 'substratomaniacs\isi{'} of Atlantic creole verbal systems from \textit{general\-} \textit{West} \textit{African verbal} \textit{systems,} because the two do not match up

.ex:ictlv point by point\isi{'}\isi{'} (emphasis added). Since nowhere are we told kind of allowances to make or what is or is not plausible, this simply amounts to a plea to swallow anything that fits the substrato%
%\originalpage{304}
maniac case{}-even such an absurdity as the existence of ``generalized West African verbal systems\isi{'}\isi{'} (if you want to flavor the condescension implicit in that concept, substitute ``generalized European verbal systems{\textquotedbl}) , or the greater absurdity that real-world speakers could derive anything from such a chin1era. In the case under discussion, if we took the semantic range of the \ili{Japanese} form, half the syntax of the \isi{English} form, and the HPE indifference to tense, we MIGHT wind up with something approximating HCE \textit{stei} V{}-but would anyone seriously propose that you can construct a language in this way ? Moreover, if anyone did, the burden would be on that person to show why that particular mix of features from those particular languages, rather than dozens of other possible mixes from the dozen or so lan\-guages in contact, happened to get chosen. Until substratomaniacs are prepared to deal with problems of this nature, there is really nothing
to argue against.
 
\item I am aware, of course, of the research that shows that \isi{English} 
does make realized-unrealized distinctions, although in a much more oblique and clumsy fashion: e.g., the contrast between \textit{I} \textit{believed} \textit{that} \textit{John} \textit{was} \textit{guilty} , \textit{but} \textit{he} \textit{wasn't} and \textit{*I} \textit{realized} \textit{that} \textit{John} \textit{was} \textit{guilty} , \textit{but} \textit{he} \textit{wasn't.} But (a) this distinction is made in that-complement sentences rather than \textit{for-to} complement sentences; and (b) it is not surface-marked in the form of complementizer differences, but rather has to be inferred from the semantics of individual verbs. Again, it is true that \textit{{}-i}\textit{n}\textit{g} \isi{complementation} is in general ``more £active\isi{'}\isi{'} than \textit{for-to} \isi{complementation}, but many cases go the opposite way, e.g., \textit{Bill} \textit{managed} \textit{to} \textit{see} \textit{Mary} (entails \textit{Bill} \textit{saw} \textit{M} \textit{ary} \textit{)} versus Bill \textit{dreaded} \textit{seeing} \textit{Mary} (does not entail \textit{Bill} \textit{saw} \textit{M} \textit{ary).} For more relevant exam\-ples, see Chapter 2, examples /31/-/34/.
 
\item \citet{Alleyne1979} uses this fact to argue that there never were 

antecedent pidginsif there had been, he claims, they should have left traces in contemporary creoles, but he denies the existence of such traces. This argument will be dealt with further in Chapter 2. Meanwhile, the reader may well wonder how much pidgin structure one could legitimately expect to be left in creoles, given the relation\-ship between the rules of HPE and HCE illustrated in /86/-/111/ above.

\end{enumerate}
% \textsubscript{{\textbackslash} }\textsuperscript{NOTES} 305

\textbf{CHAPTER} \textbf{2}


\begin{enumerate} 
\item Very few writers on creoles seem to have much background or experience in variation study, and on all the numerous occasions on which writers have used historical citations to make claims about earlier stages of creoles, I cannot recall a single one where the possi\-bility of codeswitching was even mentioned. It may well be that the average fieldhand was monolectal, but the slaves whose speech was most likely to be cited by Europeans were precisely the domestics and artisans who had the most access to superstrate models and who would therefore be the likeliest to be able and willing to adapt their speech in a superstrate direction when interacting with superstrate speakers. Historical citations should therefore be handled with great care, especially when they suggest earlier stages of a creole which would show a heavier \isi{superstrate influence} than is found in the con\-temporary basilect of that creole.
 
\item It is at least highly questionable whether even an absolute majority of speakers of a single substrate language can influence the formation of a creole. Just after the tum of the century, when creoli\isi{'}\isi{'} zation must have been actively in progress, the \ili{Japanese} constituted 50 percent of the population of \isi{Hawaii}, yet there is virtually no trace of \ili{Japanese} influence on HCE. It would be interesting to hear the substratomaniac explanation for this fact, but dealing with counter\-evidence has never been the strong point of that particular approach.
\item ``It is clear that R(eunion) C(reole) is, to quite a large degree,


. a different aninlal from M(auritian) C(reole), Ro(drigues) C(reole), and S(eychelles\} C(reole) . . . . There can be no doubt that RC shares many features in common with MC, RoC and SC . . . . The usual explanation

. . . is that RC is a 'decreollzed\isi{'} version of proto-l(ndian) 0(cean) C(reole) . . . . Another, and perhaps more plausible explanation, is that RC is, on the contrary , a modified version of a variety of \textit{French} (original emphasis) . . . . The modification of this \textit{lete} \textit{ki} \ili{French} may be seen in terms of convergence . . .\isi{'}\isi{'} Come is led to conclude that Bourbonnais (the conventional term for proto-IOC\} did not originate on the Ile de Bourbon (the old name for Reunion), but he is unable
%\originalpage{306}
to say where it did originate, or to commit himself as to whether there was or was not a true proto-IOC. In fact, only au analysis along the lines of Bickerton (197 5) can hope to make sense of RC history; but so far, no such analysis has been attempted.

%\setcounter{itemize}{1} 
\item Note that \textit{fak} \textit{ter} 'postman\isi{'} also lacks au article, although the 
definite article is required in \isi{English}. But in fact, the NP here is as nonspecific as let. 'THE postman\isi{'}, 'THE doctor\isi{'}, 'THE cashier\isi{'}, etc., are really role titles. Postmen often change routes and schedules, and there is no indication in the sentence that one particular postman might have brought the letter, that either the speaker or the listener could have answered the question ``WHICH postman?\isi{'}\isi{'} or that the identity of the postman had the slightest relevance to the topic.
 
\item The anterior-nonauterior distinction is not an easy one for the naive speaker (i.e., anyone who does not speak a creole) to under\-stand, as I have found in trying to teach it to several classes of graduate students. The reader who wishes to understand this is strongly recom\-mended to read the account in Bickerton (1975:Chapter 2).
\item Jansen et al. have a different (and much more complex) explanation involving logical form, propositional islands, \isi{truth values}, etc. Although they cite \citet{Roberts1975} in another context, they appear to be unaware of the JC examples in that paper, cited above as /27 / and 
/28/, as.well as of the other parallels cited here.
 
\item There is the possibility that an African source may also be involved. \ili{Yoruba}, for instance, has both \textit{fi} and \textit{fan }(final nasals in 
\ili{Yoruba} orthography mean that the preceding vowel is nasalized, and do not indicate the presence of a nasal consonant). Both verbs have a number of functions, but perhaps the most relevant for creoles are those found in sentences like \textit{6} \textit{fi} \textit{ow6 naa} \textit{fun} \textit{mi,} lit. 'He take money the give me\isi{'}, or 'He gave me the money\isi{'}. The similarity to creole
instrumentals is obvious, but if \ili{Yoruba} \textit{fi} is the source for JC \textit{ft,} the

shift in meaning is baffling. \textit{Fun} is puzzling in a slightly different way. \citet{Rowlands1969} notes that ``Bilingual Yorubas tend to use \textit{fen} rather indiscriminately to translate 'for\isi{'},\isi{'}\isi{'} making a joint source for GC \textit{fu,} SR \textit{foe }(phonetically /fu/) sound ver;r plausible. Also many creoles use
%\originalpage{307}
verbs meaning 'give\isi{'} to introduce dative and/or benefactive cases (e.g., HC \textit{bay,} ST \textit{da,} etc.). But if SR \textit{foe} is derived from \ili{Yoruba} \textit{[Un,} why did SR select \textit{gi} (from Eng. \textit{give}\textit{)} to mark oblique cases and use \textit{foe} as a complementizer? Moreover, HCE uses \textit{fo} as a complementizer without the benefit of any \ili{Yoruba} model, and \ili{French} and \ili{Portuguese} creoles turn Fr. \textit{pour} 'for\isi{'} and Pg. \textit{para} 'for\isi{'} into complementizers even though no one, to my knowledge, has suggested any verb with the form \textit{pu} or \textit{pa} in \ili{Yoruba} or any West African language that could have served as a model. The question is by no means closed, however ; it merely underlines the fact that we need to know a lot more both about different West African grammars and about what African lan\-guages were spoken in which creole areas. 
\item Both \citet{Christie1976} for LAC and \citet{Corne1981} for SC propose a tripartite division of verbs into Action, State, and Process. As far as I can tell (neither treatment is particularly rigorous), this proposal arises from a confusion of syntactic rules with semantic interpretation. For instance, it is not syntactic rules that (normally) 
bar co-occurrence between \isi{stative} verbs and nonpunctual markers, as is shown in the discussion of the sentence I \textit{bina} \textit{waan} \textit{ju} \textit{no} in \citet[38]{Bickerton1975}, which shows that pragmatic factors can also be involved.
\item A problem not faced by those who call for the examination of non··European creoles is that it is far from clear that there are any.

The only languages without a European superstrate which might qualify under the conditions specified in Chapter 1, above, are \ili{Ki-Nubi} and \ili{Juba Arabic}. Although the data that have emerged on these lan\-guages so far are scanty and unclear (and for this reason I have refrained from citing them in the present volume), most of what is available suggests that they follow the creole pattern described here. But even these languages do not have a third condition which may be necessary to qualify for true creolehood: their populations were not, in general, displaced from their native homelands. It is a historical fact that it was only Europeans who uprooted people from their cultures and carried them across thousands of miles of ocean in order to exploit them;
%\originalpage{308}
therefore, it is only in European colonies that one would expect to fmd the massive disruption of normal language continuity which would permit the emergence of innate faculties.

\item However, anyone wishing to use Quow as a historical source should be warned that the above remarks apply only to his rendering of basilectal speakers. Like many whites, he did not feel threatened by illiterate blacks, and could therefore treat them objectively; but he did feel threatened by literate blacks, and in consequence, his ren\-derings of \textit{their} speech are spoiled by facetiousness and condescension.
\item There have been some nonserious nonchallenges, of course. \citet{Christie1976} produced an analysis of LAC which showed it to be not far short of identity with GC but insisted on preserving traditional terms, obvious though it was that these did not fit (getting the distri\-bution of anterior correct and then calling it past is, to me at least, a quite incomprehensible maneuver). \citet{Seuren1980} endorsed the analysis of Voorhoeve (1957 ), shown in \citet{Bickerton1975} to be intern\-ally incoherent, and neatly avoided having to consider the latter analy\-sis by calling it ``sociolinguistic\isi{'}\isi{'} \textit{[} \textit{sic!} \textit{]} \textit{.} But no one has systematically attempted to criticize my analyses of GC, SR, HC, and HCE, for the obvious reasons.
\item It is perhaps worth observing that no account of \ili{Papiamentu} that I know of translates \textit{I} \textit{had} \textit{worked} , so that the PP \isi{TMA system} may not, in fact, differ as much from the classic system as those ac\-counts might suggest. In general, not only are most analyses of TMA systems incorrect, nine out of ten of them are simply incomplete, lacking the critical information which would make it possible to deter\-mine how they work. Yet, since these defective analyses buttress Euro\-centric prejudices, they are hardly ever questioned, let alone criticized.
\item When I wrote this paragraph, I was quite unaware that Baker had produced an extremely interesting account of the historical de\-velopment of MC, based in part on an analysis of all currently known historical citations (Baker 197 6), which provides a striking piece of independent support for this analysis. While \textit{fini} is recorded as a pre\-verbal marker in 1780, \textit{ti }is not recorded until 1818; but the \textit{ti} \textit{va}
% {\textbackslash}
%\originalpage{309}
combination is recorded in 1828, while the \textit{ti} \textit{fin} combination is not recorded until 1867! Granted that these dates are probably al!late\-nonstandard speech phenomena tend to have a long and lively life before they tickle the bourgeoisie, cf. \textit{olelo} \textit{pa'i'ai} (see Chapter 1) which blushed unseen in \isi{Hawaii} for nearly a century-there is no need to doubt that their order and spacing are substantially correct. Baker seems not to realize, however, that the 17 80 source derives, on both intemal and external evidence, from a pidgin and not a creole speaker.
\item \citet{Corne1981} observes that ``with state 'Verbals\isi{'} \textit{fin} does not occur, since a state has by definition already been attained.\isi{'}\isi{'} Thus, the failure of \textit{fin} to take over anterior marking in statives is a principled one, and not some inexplicable accident.
\item Here Corne falls victim to the \isi{First Law (of Creole Studies)}, since he himself stated five pages earlier (1977:103) that \textit{ti} is omitted from subordinate clauses. But I suspect that he was mostly right on this occasion and that h.e had not made allowances for the nonhomogeneiry of SC. l would be prepared to bet that /110/ came from a higher-class, more decreolized consultant.
\item lf you believe in raising. If you don't, substitute ``whatever rule marks the second NP as object of the first V.{\textquotedbl}
\item As mentioned earlier in this chapter, it seems likely that in reality GC does not have VP as a constituent at the basilectal level. The contrary is assumed here merely in order to simplify the com\-parison between the \isi{English} and GC \textit{processes,} and is not meant to imply any substantive claim about GC structure.
\item It is interesting to note that while ft-clauses in complement position can refer to one-\isi{time} actions (as in /210/), and in consequence the higher verb can take punctual marking, preposed ft-clauses can refer only to \isi{habitual} actions, and in consequence the higher verb must take nonpunctual marking. At the moment I have no idea why this is so.
\item Washabaugh's analysis of \textit{fi} differs radically from that made in the present chapter, although there is no reason to suppose that the facts of PIC differ significantly from those of GC. However, since I
%\originalpage{310}
have dealt with that analysis in \citet{Bickerton1980}, I will not repeat my criticisms of it here.
\item It would seem highly likely, indeed, that the inadequacies
of existing creole descriptions, often referred to in this volume, have served to diminish, rather than exaggerate, the degree of creole simi\-larity. To give just one very recent instance, it was long held that the verb-focusing rule discussed earlier in this chapter was not found in the grammars of any of the \ili{Indian Ocean creoles}. Substratomaniacs could point to the nature of the substratum-Eastern Bantu, Malagasy, and Indian languages- as an explanation of this. Now Corne (p.c.) reports the finding of verb-focusing structures with a copied verb identical to those discussed in this chapter. Substratomaniacs will now doubtless seize on the claim by \citet{Baker1976} that in 1735, 60 percent ·of the nonwhite population of \isi{Mauritius} was from West Africa. However, this finding is strongly challenged by \citet{Chaudenson1979} on .the basis of historical documents which he claims Baker did not examine; according to Chaudenson, the percentage of West Africans never rose much above 33.

In fact, the outcome of the disagreement is rather irrelevant to
the real issue. Baker's ``60 percent\isi{'}\isi{'} contained 66 percent of speakers from Guinea, and \ili{Guinean languages} differ markedly in structure from the \ili{Kwa languages} which are usually claimed as the source of creole structures. On Baker's own figures, the Kwa speakers in \isi{Mauritius} in 1735 must have amounted to about 130! Within a few years, the population of \isi{Mauritius} topped the 10,000 mark, swelled by recruits from India and Madagascar (Baker admits that hardly any Kwa speakers arrived after 1735). The question that substratomaniacs have to answer is: how did 130 people manage to impose their grammar (assuming they had a common one, which is a big assumption) upon a population in which they were outnumbered 100 to 1?


\item I am only too well aware that Piaget draws conclusions from
his studies quite contrary to those drawn here. That he does so, how\-ever, has always seemed to me baffling in light of the fact that the developmental stages he posits bear a nativistic explanation much more
%\originalpage{311}
easily than they do an experiential one. But there is not space here to attempt a reinterpretation of Piagetian findings, desirable though such an activity might seem. We. will see in the next chapter, however, that some linguistic findings of Piaget's disciples can very easily (and very fruitfully ) be reinterpreted in a nativistic manner (see especially the discussion of Bronckart and Sinclair [1973] ) 

\end{enumerate}
\textbf{CHAPTER} \textbf{3}


\begin{enumerate}
\item Even today, I know of no study of child language acquisition in any language which follows the simple and obvious procedure of noting the very first emergence of a given form or structure in a child's speech, then following the development of that feature until Brown's ``criterion\isi{'}\isi{'} is reached-meanwhile noting what that form or structure alternated with in those con\isi{texts} where it was inappropriate, as well as those where it was appropriate, with the aim of figuring out why variation occurred and what the form or structure might mean to the child. Normally, second-language acquisition trots along obediently in the footsteps of first-language acquisition, but here roles are reversed, as my student, Tom Huebner, is about to complete a dissertation which applies the above approach to the \isi{acquisition of} \isi{English} by an immi\-grant Hmong speaker (see also Huebner 1979). The field is wide open for similar first-language studies, which should help to revolutionize our understanding of acquisition.
\item In fact, rather than such a conflict, the present theory entails a division of labor. The innate component is necessary in order to get the child into a position where he can learn any human language, for as \citet{Fodor1975} argues (see below), it is impossible to learn a language unless you already know a language. Some other kind of component is necessary to get the child from the innate creole-like grammar to the idiosyncratic grammars of \isi{Italian}, \ili{Yoruba}, Akawaio, Walbiri, or what\-ever language that particular child is going to have to learn as part of his socialization. Because I have not discussed this second component in the present volume, the reader should not conclude that I deny its 
%\originalpage{312}
importance. My failure to say anything about it is, as I said, strategic; until we know where the innate component stops, we cannot know where any other devices start.
%\setcounter{itemize}{2}
\item Or at least it is implausible to suppose that he could utilize them if he did not have some overall conceptual framework in which \isi{past tense} (punctual, in our treatment) was associated with unique events and present tense (nonpunctual, in our treatment) was associ\-ated with generic events. How such an arbitrary framework could be derived from experience is totally opaque to me. But it might be derivable from species-specific or even genus-specific neural wiring, along the lines suggested in Chapter 4.
\item Students of the \isi{acquisition of} \ili{Turkish} please note: it would be most revealing to analyze 43 hours of a single child's speech (one hour at three-week intervals from 2:0 to 4:6) in order to determine exactly how he moves from a state-process to a direct-indirect analysis, along the lines indicated in Note 1above.
\item One of these exceptions is \citet{Miller1978}. In a brilliant flash of insight, Miller suggests that ``perhaps the difference between \textit{go} and \textit{went} is used to mark something else, like momentary happen\-ings as opposed to persisting states{\textquotedbl}; and, in discussing forms like \textit{wented} , adds that ``if they did not understand \textit{went} as incorporating a concept of pastness, then adding pastness with \textit{{}-ed} would not seem redundant.\isi{'}\isi{'} However, a stiff dose of Reichenbach and formal logic enables him to climb back into the sheepfold of the conventional wisdom. It should be noted, however, that one of his presuppositions\-that forms like \textit{wented} are quite uncommon in child speech-fails to take into account forms like \textit{did} \textit{he} \textit{went?,} \textit{he} \textit{didn't} \textit{went,} etc., which are semantically identical and much more common. These forms are discussed in Hurford (197 5), \citet{Kuczaj1976}, \citet{Fay1978}, Maratsos and \citet{Kuczaj1978}, and \citet{ErreichEtAl1980}; but unfortunately, it seems not to have occurred to any of these writers to look at the sentences with ``double pasts\isi{'}\isi{'} and the sentences with ``single pasts\isi{'}\isi{'} in their appropriate con\isi{texts} and see whether, semantically or pragmatically, there are any differences betwn them. This is the first thing rhat
%\originalpage{313}
an investigator should do, as a matter of simple routine, whenever he is confronted by variable data of this kind.
\item The question is the more interesting in that the form auxiliary
+ past participle- the first to be acquired by \ili{French} and \isi{Italian} learners
{}-is among the last to be acquired by \isi{English} learners. \citet{Maratsos1979} observes of the latter that ``its late acquisition, coming after children hear it used around them for years, probably stems from its subtle meaning,\isi{'}\isi{'} and indeed it is surely the case that the meaning of the ``composite past\isi{'}\isi{'} in \ili{French} or \isi{Italian} (a punctual meaning) is easier for the child to grasp than the meaning of the \isi{English} perfect (a com\-pletive meaning). But this only opens up a host of other issues. For instance, if the meaning of \isi{English} perfect is ``relevance to present state,\isi{'}\isi{'} and if, as Antinucci and Miller suggest, the child assigns his early past marking on the basis of ``relevance to present state,\isi{'}\isi{'} why should the meaning of perfect be so ``subtle\isi{'}\isi{'} in the child's view, and why should it not be the first, rather than the last, verb form to be acquired? Further, is it a matter of mere coincidence that perfect should be the last form to be acquired by both children \isi{learning} \isi{English} and speakers of an \isi{English} creole in the course of \isi{decreolization} (see Bickerton [1975:126ff.] for details on the latter process)? If, as suggested later in this chapter, \isi{decreolization} and the later stages of acquisition are processes which show a principled relationship, then there is no coincidence, but rather a joint reflection of one of the difficulties involved in getting from the bioprogram to \isi{English}.


\item For instance, ``double pasts\isi{'}\isi{'} of the kind discussed in Note 5 above a:re assumed in orthodox generative accounts (e.g., Hurford 197 5) to stem from a process which copies the past-tense marker in Aux onto the verb-stem, as in the familiar ``Aux-Hopping\isi{'}\isi{'} rules, but then fails to delete the original occurrence of \isi{past tense} under the Aux node (but see Matatsos and Kuczaj [1978] for criticism of this proposal). The fact that ``double pasts\isi{'}\isi{'} occur so frequently while ``double-WHs\isi{'}\isi{'} don't occur at all casts strong doubt on the assumption that children's mistakes stem from incomplete applications of standard transformational processes. 
%\originalpage{3}

8, In other words, creolization and \isi{decreolization} correspond to the two (overlapping) halves of the acquisition process proposed at the beginning of this chapter. The first half, dominated by the bio\-program, corresponds to creolization, but the second half, dominated by other components, in which the child bridges the gap between bioprogram and target language, corresponds to \isi{decreolization}. The only significant difference would seem to be that creolization and \isi{decreolization} cannot overlap, while the \isi{evolution} of the bioprogram and the pressure from the target language can, do, and indeed must overlap. However, since this difference stems directly from purely pragmatic differences between the circumstances of the ``normal\isi{'}\isi{'} child and the circumstances of the creole-creating child, it can in no way invalidate the correspondence.
 
\item See also the theoretical discussion of this process in \citet{Bickerton1980}.
\item Why children don't do what they don't is often even more 
mysterious (for the conventional wisdom) than why they do what they do, so that \isi{questions} such as the one at the beginning of this paragraph are studiously avoided. However, there is no need to avoid such \isi{questions} with the present model; why they don't do what they don't is in fact loaded with clues as to why they do do what they do.

\item It should not need to be emphasized that, fast, there is no evidence for ``hyperstrategic\isi{'}\isi{'} devices as such, beyond the problems whose solution might seem to call for them, and second, that if they did exist, they would constitute an innate component no less surely (although with far less justification) than does the bioprogram proposed here.
\end{enumerate}

\textbf{CHAPTER} \textbf{4}


\begin{enumerate}
\item The exchange, which took place at the New York Academy of Sciences Conference on Language Origins in 1975, should be quoted at length; it demonstrates the orthogonal approaches and seemingly 
invincible mutual incomprehensibility that have bedeviled glottogenetic
% {\textbackslash}
%\originalpage{315}
studies better than could countless pages of exegesis:

\textit{Hamad:} Let me just ask a question which everyone else who has been faithfully attending these sessions is surely burning to ask. If some rules you have described constitute universal constraints on all languages, yet they are not learned, nor are they somehow logically necessary \textit{a} \textit{priori,} how did language get that way?

\textit{Chomsky:} Well, it seems to me that would be like asking the question how does the heart get that way? I mean, we don't learn to have a heart, we don't learn to have arms rather than wings. What is interesting to me is that the question should be asked. It seems to be a natural question; everyone asks it. And I think we should ask why people ask it.

The question ``Why do you ask that question?\isi{'}\isi{'} is of course a stalling ploy familiar to psychoanalysts; indeed, it was programmed into the ``robot psychiatrist\isi{'}\isi{'} with which some ingenious psychologists were able to simulate, with surprising plausibility, a therapeutic session, The present writer believes, as firmly as Chomsky, that we get language like
we get a heart and arms, yet I entirely fail to see why Harnad's question
was an illegitimate one or why it does not deserve, or rather demand, an answer. How we first got arms or a heart are \isi{questions} so phylo\-genetically remote and so unrelated to the mental life of our species that Chomsky is right to dismiss them as not worth asking (except, presumably, for those whose professional specialism they are). But the \isi{evolution} of language is so recent that we may reasonably suppose that its present nature is still conditioned by those origins, and its crucial role in distinguishing between us and other species (while any number of other species have arms and hearts) ls such that it must strongly influence, even if it does not wholly determine, all that we think and do. Thus, to put the determination of its origins on a par
with the determination of the \isi{origins of} physical organs seems to me a piece of evasive perversity,

%\originalpage{316}
 
\item \citet{Hewes1975} provides a fairly exhaustive account of these theories.
\item In the Hockett and Ascher ``Flintstone,\isi{'}\isi{'} the key development 
is a hominid who, in encountering food and danger at the same \isi{time}, gives half the call for food and half the call for danger. Not one shred of even the most oblique evidence from ethological or other studies, or even the authors\isi{'} own ratiocinations, is adduced in support of this inherently unlikely development, beyond their admission that they can't think of any other way language could have begun. 
\item However, I have some (admittedly anecdotal) evidence that 
dogs use \isi{cognitive mapping} in recognition. Our dog, Rufus, will rush from the opposite end of the apartment to greet my wife when she comes home, but on meeting her on campus he ignores or even recoils from her until she is just a couple of feet from him, whereupon he performs his usual acts of greeting. It is not easy to account for such behavior unless (as is the case with us) part of the way he recognizes people has to do with a network of particular associations. He recog\-nizes her where he expects her to be, and fails to recognize her else\-where, in the same way (and why not for the same reason? ) that we fail to recognize, on the beach or in a restaurant, the clerk or cashier we may have met dozens of times in a work setting. 
\item Nothing Blake ever wrote should be taken lightly. In the 
broad brush-strokes with which we have to draw our cognitive maps, 
is worse, we locked into stereotypic \textit{(} \textit{k} \textit{ike,} \textit{freak} , \textit{faggot} are some pernicious examples) which lead us to deny one another's individuality. A creature that could compute from \isi{percepts} rather than concepts would out\-shine us as the sun outshines the moon (more on this in \textit{Language and Species}.
 
\item Some scholars remain unimpressed by the evidence that apes have concepts. For instance, Seidenberg and \citet{Petitto1979} seem to
need reassurance that before and after Washoe signed \textit{water-bird} he did not also sign \textit{banana-bird} , \textit{water-berry,} \textit{banana-berry-in} other words, they at least envisage the possibility that signing apes proceed like
%\originalpage{317}
demented computers, throwing off random strings of signs (they have, after all, been reinforced for signing) from which biased experimenters simply pick out the rare one which happens, by pure chance, to be contextually appropriate. Leaving aside the unmerited slur which this casts on the morals and/or wide-awakeness of many dedicated re\-searchers, the approach adds a Cartesian twist to the old behaviorist\-nativist controversy: scholars who. are behaviorists with regard to animals and nativists with regard to people. It is more parsimonious as well as more fruitful to suppose that when animals similar to our\-selves evince behavior like ours, similar mechanisms underlie both sets of phenomena. 
\item That the nature of linguistic facts can be determined by the order in which they necessarily occur and/or originally occurred has already been suggested in the contrast between the development of tense that takes place in learners of \isi{English} and that which takes place in learners of \isi{Italian}. Those who continue to believe (see Note 1, this chapter) that there is nothing to be learned from \isi{learning} how language developed should read and compare these two cases and then ask themselves whether their attitude is not one of simple obscurantism.
\item This is not, of course, to say that older structures do not undergo changes, adaptations, and linkages. The neural dysfunction known as Gilles \isi{de la Tourette's syndrome} is one that affects the limbic area, yet its victims shout lexical obscenities as well as more animal-like cries, ln general, lexical utterances are under cortical con trol, but in the case of those which express strong emotion, like non\-verbal vocal utterances, linkage between the speech areas of the neo- cortex and the limbic area must have been forged at some stage sub\-sequent to the farmer's development.
\item In fact, discussion of semantics would be clearer if \textit{semantic} \textit{prime} were reserved exclusively for category distinctions of potentially universal application (like the SNSD, the PNPD, etc.) and if what are sometimes referred to as ``\isi{semantic primes}\isi{'}\isi{'} were referred to as \textit{primi\-} \textit{tive concepts.} However, note that \isi{primitive concepts} are not necessarily constructed out of \isi{semantic primes}. 
%\originalpage{318} 
\item The reading ``The answer is \textit{ther}\textit{e}\textit{/{\textquotedbl}} is of course not intended.
\item It is an open question whether any language could make the past-present-future distinction before the \isi{culture} that used it produced any kind of \isi{time}-measuring device. The fact that \isi{time}-enslaved linguists may have analyzed preliterate languages as having such a distinction is, of course, no proof of anything-they have consistently done the same for creoles and they have been consistently wrong in so doing. In fact, there already exist more careful studies of such languages (e.g., Arnott 1970, Welmers 1973) which explicitly recognize the absence of the characteristic Western temporal framework. Analysis of TMA systems is too subtle to be left to logicians. 
\item In Bickerton 1974.
\item Another common source is (phonologically salient) auxiliary verb forms in the superstrate. However, since there could not have been auxiliaries before there were auxiliaries, the situations of creole and primordial languages will differ in at least this respect.
\item Order in terms of distance from the verb is of course intended, 
and not the left-to-right ordering of surface constituents. 
\item It was observed in Chapter 2 that the similarities between creole languages were in many cases closer and more consistent in the semantic component than they were in the syntactic component. This result would issue very naturally if the semantics of language depended on relatively old neural structnres while syntax depended partly on relatively new neural structures but also partly on extraneural factors intrinsic to the task of building a linear vocal language. These latter factors might in a number of cases permit more than one possible solution to a given structural problem, whereas with semantic struc\-tures, single solutions would be imposed in almost all cases.
\end{enumerate}

\textbf{CHAPTER} \textbf{5}
\begin{enumerate}
\item Indeed, one objection to the hypothetical history oflanguage given in the preceding chapter might be that many essential prerequi\-sites of language, such as the development of the neural and physio%
% {\textbackslash}
%\originalpage{319}
logical mechanisms required for vocalization , the lateralization of the brain, and the growth of auditory processing mechanisms. or ``tem\-plates\isi{'}\isi{'} which , as suggested in some fascinating work by Marler and associates (Marler 1977, 1980; Marler and Peters 1979, etc.), show striking parallels to those of avian species, have simply been ignored. However, these omissions in no way reflect my estimate of the impor\-tance of such developments. The reasons for them are threefold. First, reasons of space (and the overall purpose of this volume) pre\-vented me from describing everything that went into the makeup of language; second, these other developments have been excellently treated elsewhere; and third, I wanted to deal precisely with those aspects of language development which have been most systematically ignored or misunderstood. Certainly, such omissions were not for the purpose of strengthening my case since all the omitted developments are much more obviously the product of the genetic code than the developments discussed in this volume. 
\item I certainly do not wish to suggest by this that no sooner had language reached the creole level than agri\isi{culture} began. There may well have been an interval of tens of thousands of years between these two even ts, years during which cognitive maps became only gradually more complex; or the interval may have been quite short. There is no way, at present, that we can choose between these alternatives-or even prove that language in its present form did not exist two million years ago, although the latter possibility seems intrinsically unlikely.
\item I write ``seem to be\isi{'}\isi{'} because only empirical investigation will reveal whether such languages are indeed as far from the bio\-program as our intuitions would suggest. One test will be the \isi{time} taken by children to acquire the main grammatical structures of given 
languages. rt was often claimed (at a \isi{time} when acquisition had hardly
been studied!) that all languages were equally easy for children to learn. This belief was, of course, simply deduced from the ``all-languages-are-developmentally-equal\isi{'}\isi{'} dogma. Work by Slobin and his associates already suggests this may be quite far from the truth.
\end{enumerate}





  


  
 